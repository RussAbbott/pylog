%% bare_conf.tex
%% V1.4b
%% 2015/08/26
%% by Michael Shell
%% See:
%% http://www.michaelshell.org/
%% for current contact information.
%%
%% This is a skeleton file demonstrating the use of IEEEtran.cls
%% (requires IEEEtran.cls version 1.8b or later) with an IEEE
%% conference paper.
%%
%% Support sites:
%% http://www.michaelshell.org/tex/ieeetran/
%% http://www.ctan.org/pkg/ieeetran
%% and
%% http://www.ieee.org/

%%*************************************************************************
%% Legal Notice:
%% This code is offered as-is without any warranty either expressed or
%% implied; without even the implied warranty of MERCHANTABILITY or
%% FITNESS FOR A PARTICULAR PURPOSE! 
%% User assumes all risk.
%% In no event shall the IEEE or any contributor to this code be liable for
%% any damages or losses, including, but not limited to, incidental,
%% consequential, or any other damages, resulting from the use or misuse
%% of any information contained here.
%%
%% All comments are the opinions of their respective authors and are not
%% necessarily endorsed by the IEEE.
%%
%% This work is distributed under the LaTeX Project Public License (LPPL)
%% ( http://www.latex-project.org/ ) version 1.3, and may be freely used,
%% distributed and modified. A copy of the LPPL, version 1.3, is included
%% in the base LaTeX documentation of all distributions of LaTeX released
%% 2003/12/01 or later.
%% Retain all contribution notices and credits.
%% ** Modified files should be clearly indicated as such, including  **
%% ** renaming them and changing author support contact information. **
%%*************************************************************************


% *** Authors should verify (and, if needed, correct) their LaTeX system  ***
% *** with the testflow diagnostic prior to trusting their LaTeX platform ***
% *** with production work. The IEEE's font choices and paper sizes can   ***
% *** trigger bugs that do not appear when using other class files.       ***                          ***
% The testflow support page is at:
% http://www.michaelshell.org/tex/testflow/



\documentclass[conference]{IEEEtran}
\usepackage{pythonhl}
\usepackage{multicol,multirow,listings,fontaxes}
\usepackage{epsfig,times,graphics,amssymb,amsbsy,url,setspace,float,xcolor,tcolorbox,mdframed,listings,hyphenat,enumitem,capt-of,amsmath,url, hyperref,textcomp}
\usepackage[all]{xy}
\usepackage[style=numeric,backend=biber,seconds=true]{biblatex}
\addbibresource{pylog.bib}
\lstdefinelanguage[programming]{TeX}[AlLaTeX]{TeX}{%
  deletetexcs={title,author,bibliography},%
  deletekeywords={tabular},
  morekeywords={abstract},%
  moretexcs={chapter},%
  moretexcs=[2]{title,author,subtitle,keywords,maketitle,titlerunning,authorinfo,affiliation,authorrunning,paperdetails,acks,email},
  moretexcs=[3]{addbibresource,printbibliography,bibliography},%
}%
\lstset{%
  language={[programming]TeX},%
  keywordstyle=\firamedium,
  stringstyle=\color{Brown},%
  texcsstyle=*{\color{Purple}\mdseries},%
  texcsstyle=*[2]{\color{Blue1}},%
  texcsstyle=*[3]{\color{ForestGreen}},%
  commentstyle={\color{FireBrick}},%
  escapechar=`,}
\newcommand*{\CTAN}[1]{\href{http://ctan.org/tex-archive/#1}{\nolinkurl{CTAN:#1}}}
%%


\def\inv{\vspace*{-6pt}}
\def\sinv{\vspace*{-3pt}}
\def\smallinv{\vspace*{-3pt}}
\def\smallinh{\hspace*{-3pt}}
\def\smallv{\vspace*{2pt}}
\def\largev{\vspace*{6pt}}
\def\smallh{\hspace*{2pt}}


\newcommand{\textbftt}[1]{\textbf{\texttt{#1}}}
\newcommand{\textittt}[1]{\textit{\texttt{#1}}}

\usepackage{hyperref}
\hypersetup{
    colorlinks=true,
    linkcolor=blue,
    filecolor=magenta,      
    urlcolor=blue,
}

% \newenvironment{prettypython}
% {\begin{center}\begin{minipage}[c]{0.45\textwidth}\begin{python1}}
% {\end{python1}\end{minipage}\end{center}}

% Some Computer Society conferences also require the compsoc mode option,
% but others use the standard conference format.
%
% If IEEEtran.cls has not been installed into the LaTeX system files,
% manually specify the path to it like:
% \documentclass[conference]{../sty/IEEEtran}

% Some very useful LaTeX packages include:
% (uncomment the ones you want to load)


% *** MISC UTILITY PACKAGES ***
%
%\usepackage{ifpdf}
% Heiko Oberdiek's ifpdf.sty is very useful if you need conditional
% compilation based on whether the output is pdf or dvi.
% usage:
% \ifpdf
%   % pdf code
% \else
%   % dvi code
% \fi
% The latest version of ifpdf.sty can be obtained from:
% http://www.ctan.org/pkg/ifpdf
% Also, note that IEEEtran.cls V1.7 and later provides a builtin
% \ifCLASSINFOpdf conditional that works the same way.
% When switching from latex to pdflatex and vice-versa, the compiler may
% have to be run twice to clear warning/error messages.


% *** CITATION PACKAGES ***
%
%\usepackage{cite}
% cite.sty was written by Donald Arseneau
% V1.6 and later of IEEEtran pre-defines the format of the cite.sty package
% \cite{} output to follow that of the IEEE. Loading the cite package will
% result in citation numbers being automatically sorted and properly
% "compressed/ranged". e.g., [1], [9], [2], [7], [5], [6] without using
% cite.sty will become [1], [2], [5]--[7], [9] using cite.sty. cite.sty's
% \cite will automatically add leading space, if needed. Use cite.sty's
% noadjust option (cite.sty V3.8 and later) if you want to turn this off
% such as if a citation ever needs to be enclosed in parenthesis.
% cite.sty is already installed on most LaTeX systems. Be sure and use
% version 5.0 (2009-03-20) and later if using hyperref.sty.
% The latest version can be obtained at:
% http://www.ctan.org/pkg/cite
% The documentation is contained in the cite.sty file itself.



% *** GRAPHICS RELATED PACKAGES ***
%
\ifCLASSINFOpdf
  % \usepackage[pdftex]{graphicx}
  % declare the path(s) where your graphic files are
  % \graphicspath{{../pdf/}{../jpeg/}}
  % and their extensions so you won't have to specify these with
  % every instance of \includegraphics
  % \DeclareGraphicsExtensions{.pdf,.jpeg,.png}
\else
  % or other class option (dvipsone, dvipdf, if not using dvips). graphicx
  % will default to the driver specified in the system graphics.cfg if no
  % driver is specified.
  % \usepackage[dvips]{graphicx}
  % declare the path(s) where your graphic files are
  % \graphicspath{{../eps/}}
  % and their extensions so you won't have to specify these with
  % every instance of \includegraphics
  % \DeclareGraphicsExtensions{.eps}
\fi
% graphicx was written by David Carlisle and Sebastian Rahtz. It is
% required if you want graphics, photos, etc. graphicx.sty is already
% installed on most LaTeX systems. The latest version and documentation
% can be obtained at: 
% http://www.ctan.org/pkg/graphicx
% Another good source of documentation is "Using Imported Graphics in
% LaTeX2e" by Keith Reckdahl which can be found at:
% http://www.ctan.org/pkg/epslatex
%
% latex, and pdflatex in dvi mode, support graphics in encapsulated
% postscript (.eps) format. pdflatex in pdf mode supports graphics
% in .pdf, .jpeg, .png and .mps (metapost) formats. Users should ensure
% that all non-photo figures use a vector format (.eps, .pdf, .mps) and
% not a bitmapped formats (.jpeg, .png). The IEEE frowns on bitmapped formats
% which can result in "jaggedy"/blurry rendering of lines and letters as
% well as large increases in file sizes.
%
% You can find documentation about the pdfTeX application at:
% http://www.tug.org/applications/pdftex





% *** MATH PACKAGES ***
%
%\usepackage{amsmath}
% A popular package from the American Mathematical Society that provides
% many useful and powerful commands for dealing with mathematics.
%
% Note that the amsmath package sets \interdisplaylinepenalty to 10000
% thus preventing page breaks from occurring within multiline equations. Use:
%\interdisplaylinepenalty=2500
% after loading amsmath to restore such page breaks as IEEEtran.cls normally
% does. amsmath.sty is already installed on most LaTeX systems. The latest
% version and documentation can be obtained at:
% http://www.ctan.org/pkg/amsmath





% *** SPECIALIZED LIST PACKAGES ***
%
%\usepackage{algorithmic}
% algorithmic.sty was written by Peter Williams and Rogerio Brito.
% This package provides an algorithmic environment fo describing algorithms.
% You can use the algorithmic environment in-text or within a figure
% environment to provide for a floating algorithm. Do NOT use the algorithm
% floating environment provided by algorithm.sty (by the same authors) or
% algorithm2e.sty (by Christophe Fiorio) as the IEEE does not use dedicated
% algorithm float types and packages that provide these will not provide
% correct IEEE style captions. The latest version and documentation of
% algorithmic.sty can be obtained at:
% http://www.ctan.org/pkg/algorithms
% Also of interest may be the (relatively newer and more customizable)
% algorithmicx.sty package by Szasz Janos:
% http://www.ctan.org/pkg/algorithmicx




% *** ALIGNMENT PACKAGES ***
%
%\usepackage{array}
% Frank Mittelbach's and David Carlisle's array.sty patches and improves
% the standard LaTeX2e array and tabular environments to provide better
% appearance and additional user controls. As the default LaTeX2e table
% generation code is lacking to the point of almost being broken with
% respect to the quality of the end results, all users are strongly
% advised to use an enhanced (at the very least that provided by array.sty)
% set of table tools. array.sty is already installed on most systems. The
% latest version and documentation can be obtained at:
% http://www.ctan.org/pkg/array


% IEEEtran contains the IEEEeqnarray family of commands that can be used to
% generate multiline equations as well as matrices, tables, etc., of high
% quality.




% *** SUBFIGURE PACKAGES ***
%\ifCLASSOPTIONcompsoc
%  \usepackage[caption=false,font=normalsize,labelfont=sf,textfont=sf]{subfig}
%\else
%  \usepackage[caption=false,font=footnotesize]{subfig}
%\fi
% subfig.sty, written by Steven Douglas Cochran, is the modern replacement
% for subfigure.sty, the latter of which is no longer maintained and is
% incompatible with some LaTeX packages including fixltx2e. However,
% subfig.sty requires and automatically loads Axel Sommerfeldt's caption.sty
% which will override IEEEtran.cls' handling of captions and this will result
% in non-IEEE style figure/table captions. To prevent this problem, be sure
% and invoke subfig.sty's "caption=false" package option (available since
% subfig.sty version 1.3, 2005/06/28) as this is will preserve IEEEtran.cls
% handling of captions.
% Note that the Computer Society format requires a larger sans serif font
% than the serif footnote size font used in traditional IEEE formatting
% and thus the need to invoke different subfig.sty package options depending
% on whether compsoc mode has been enabled.
%
% The latest version and documentation of subfig.sty can be obtained at:
% http://www.ctan.org/pkg/subfig




% *** FLOAT PACKAGES ***
%
%\usepackage{fixltx2e}
% fixltx2e, the successor to the earlier fix2col.sty, was written by
% Frank Mittelbach and David Carlisle. This package corrects a few problems
% in the LaTeX2e kernel, the most notable of which is that in current
% LaTeX2e releases, the ordering of single and double column floats is not
% guaranteed to be preserved. Thus, an unpatched LaTeX2e can allow a
% single column figure to be placed prior to an earlier double column
% figure.
% Be aware that LaTeX2e kernels dated 2015 and later have fixltx2e.sty's
% corrections already built into the system in which case a warning will
% be issued if an attempt is made to load fixltx2e.sty as it is no longer
% needed.
% The latest version and documentation can be found at:
% http://www.ctan.org/pkg/fixltx2e


%\usepackage{stfloats}
% stfloats.sty was written by Sigitas Tolusis. This package gives LaTeX2e
% the ability to do double column floats at the bottom of the page as well
% as the top. (e.g., "\begin{figure*}[!b]" is not normally possible in
% LaTeX2e). It also provides a command:
%\fnbelowfloat
% to enable the placement of footnotes below bottom floats (the standard
% LaTeX2e kernel puts them above bottom floats). This is an invasive package
% which rewrites many portions of the LaTeX2e float routines. It may not work
% with other packages that modify the LaTeX2e float routines. The latest
% version and documentation can be obtained at:
% http://www.ctan.org/pkg/stfloats
% Do not use the stfloats baselinefloat ability as the IEEE does not allow
% \baselineskip to stretch. Authors submitting work to the IEEE should note
% that the IEEE rarely uses double column equations and that authors should try
% to avoid such use. Do not be tempted to use the cuted.sty or midfloat.sty
% packages (also by Sigitas Tolusis) as the IEEE does not format its papers in
% such ways.
% Do not attempt to use stfloats with fixltx2e as they are incompatible.
% Instead, use Morten Hogholm'a dblfloatfix which combines the features
% of both fixltx2e and stfloats:
%
% \usepackage{dblfloatfix}
% The latest version can be found at:
% http://www.ctan.org/pkg/dblfloatfix




% *** PDF, URL AND HYPERLINK PACKAGES ***
%
%\usepackage{url}
% url.sty was written by Donald Arseneau. It provides better support for
% handling and breaking URLs. url.sty is already installed on most LaTeX
% systems. The latest version and documentation can be obtained at:
% http://www.ctan.org/pkg/url
% Basically, \url{my_url_here}.




% *** Do not adjust lengths that control margins, column widths, etc. ***
% *** Do not use packages that alter fonts (such as pslatex).         ***
% There should be no need to do such things with IEEEtran.cls V1.6 and later.
% (Unless specifically asked to do so by the journal or conference you plan
% to submit to, of course. )


% correct bad hyphenation here
\hyphenation{op-tical net-works semi-conduc-tor}


\begin{document}
%
% paper title
% Titles are generally capitalized except for words such as a, an, and, as,
% at, but, by, for, in, nor, of, on, or, the, to and up, which are usually
% not capitalized unless they are the first or last word of the title.
% Linebreaks \\ can be used within to get better formatting as desired.
% Do not put math or special symbols in the title.
\title{Constraint programming as the most reliable platform for Web Intelligence}


% author names and affiliations
% use a multiple column layout for up to three different
% affiliations
\author{\IEEEauthorblockN{Russ Abbott, Jungsoo Lim}
\IEEEauthorblockA{Department of Computer Science\\
California State University, Los Angeles\\
Los Angeles, California 90032\\
Email: rabbott@calstatela.edu, jlim34@calstatela.edu}
\and
\IEEEauthorblockN{Jay Patel}
\IEEEauthorblockA{Visa \\ Foster City\\
USA\\
Email: imjaypatel12@gmail.com}
}

% conference papers do not typically use \thanks and this command
% is locked out in conference mode. If really needed, such as for
% the acknowledgment of grants, issue a \IEEEoverridecommandlockouts
% after \documentclass

% for over three affiliations, or if they all won't fit within the width
% of the page, use this alternative format:
% 
%\author{\IEEEauthorblockN{Michael Shell\IEEEauthorrefmark{1},
%Homer Simpson\IEEEauthorrefmark{2},
%James Kirk\IEEEauthorrefmark{3}, 
%Montgomery Scott\IEEEauthorrefmark{3} and
%Eldon Tyrell\IEEEauthorrefmark{4}}
%\IEEEauthorblockA{\IEEEauthorrefmark{1}School of Electrical and Computer Engineering\\
%Georgia Institute of Technology,
%Atlanta, Georgia 30332--0250\\ Email: see http://www.michaelshell.org/contact.html}
%\IEEEauthorblockA{\IEEEauthorrefmark{2}Twentieth Century Fox, Springfield, USA\\
%Email: homer@thesimpsons.com}
%\IEEEauthorblockA{\IEEEauthorrefmark{3}Starfleet Academy, San Francisco, California 96678-2391\\
%Telephone: (800) 555--1212, Fax: (888) 555--1212}
%\IEEEauthorblockA{\IEEEauthorrefmark{4}Tyrell Inc., 123 Replicant Street, Los Angeles, California 90210--4321}}




% use for special paper notices
%\IEEEspecialpapernotice{(Invited Paper)}




% make the title area
\maketitle

% For peer review papers, you can put extra information on the cover
% page as needed:
% \ifCLASSOPTIONpeerreview
% \begin{center} \bfseries EDICS Category: 3-BBND \end{center}
% \fi
%

\begin{abstract}
Pylog explores the integration of two distinct programming language paradigms: (i) the modern, general purpose programming paradigm which, in its Python implementation, has been broadened to include procedural, object-oriented, and functional programming and (ii) the logic programming paradigm, most notably including logic variables (with unification) and depth-first, backtracking search. 
          
Pylog illustrates how the core logic programming features can be implemented in and integrated into Python. 

Simultaneously, Pylog demonstrates Python's breadth. Python is used in situations ranging introductory programming classes to the development of very sophisticated software. Pylog demonstrates two interestingly distinct uses for Python's \textbf{for}-loop construct.

Pylog exemplifies programming at its best, using Python features in innovative yet clear ways to integrate features of a non-Python programming paradigm into its range of capabilities. The overall result is software worth reading.
          
          The Pylog code is available at \href{https://github.com/RussAbbott/pylog}{\underline{this GitHub repository}}.
\end{abstract}
% For peerreview papers, this IEEEtran command inserts a page break and
% creates the second title. It will be ignored for other modes.
\IEEEpeerreviewmaketitle


%%%%%%%%%%%%%%%%%%%%%%%%%%%%%%%%%%%%%%%%%%%%%%
\section{Introduction}
%%%%%%%%%%%%%%%%%%%%%%%%%%%%%%%%%%%%%%%%%%%%%%

\noindent\textbf{Symbolic artificial intelligence}. The birth announcement for Artificial Intelligence took the form of a workshop proposal. The proposal predicted that \textit{every aspect of learning---or any other feature of intelligence---can in principle be so precisely described that a machine can be made to simulate it.}\cite{mccarthy2006proposal}  

At the workshop, held in 1956, Newell and Simon claimed that their Logic Theorist not only took a giant step toward that goal but even \textit{solved the mind-body problem}.\cite{russell2010artificial} A year later Simon doubled-down.
\begin{quote}
    [T]here are now machines that can think, that can learn, and that can create. Moreover, their ability to do these things is going to increase rapidly until---in a visible future---the range of problems they can handle will be coextensive with the range to which the human mind has been applied.\cite{simon1957models}
\end{quote}

Perhaps not unexpectedly, such extreme optimism about the power of symbolic AI, as this work was (and is still) known, faded into the gloom of what has been labelled the AI winter. 

%%%%%%%%%%%%%%%%%%%%%%%%%%%%%%%%%%%%%%%%%%%%%%
\smallv\noindent\textbf{Deep learning}. All was not lost. Winter was followed by spring and the green shoots of (non-symbolic) deep neural networks sprang forth. Andrew Ng said of that development, 
\begin{quote}
Just as electricity transformed almost everything 100 years ago, today I actually have a hard time thinking of an industry that I don’t think AI will transform in the next several years.\cite{ng2018ai}
\end{quote}

But another disappointment followed. Deep neural nets 
\begin{quote}
are surprisingly susceptible to what are known as \textit{adversarial} attacks. Small perturbations to images that are (almost) imperceptible to human vision can cause a neural network to completely change its prediction. When minimally modified, a correctly classified image of a school bus is reclassified as an ostrich. Even worse, the classifiers report high confidence in this wrong prediction.\cite{akhtar2018threat}
\end{quote}
% (Adversarial images did not kill off deep learning. They have been co-opted, and their use is now built into  deep neural network training methodologies.\cite{shrivastava2017learning})

%%%%%%%%%%%%%%%%%%%%%%%%%%%%%%%%%%%%%%%%%%%%%%
\smallv\noindent\textbf{Deep learning: the current state}. Deep learning has achieved extraordinary success in fields such as image captioning and natural language translation.\cite{garnelo2019reconciling} But other than its remarkable achievements in game-playing via reinforcement learning\cite{silver2018general}, it's triumphs have often been superficial. 

By that we don't mean that the work is trivial. We suggest that many deep learning systems learn little more than surface patterns. The patterns may be both subtle and complex, but they are surface patterns nevertheless.

Lacker\cite{lacker-gpt3} elicits many examples of such superficial (but very sophisticated) patterns from GPT-3\cite{brown2020language}, a highly acclaimed natural language system. In one, GPT-3, acting as a personal assistant, offers to read to its interlocutor his latest email. The problem is that GPT-3 has no access to that person's email---and doesn't "know" that without access it can't read the email. (Much of the excitement surrounding GPT-3 derives from its skill as a fiction author.) 

Both the conversational interaction and the made-up email sound plausible and natural. In reality, each consists of words strung together based simply on co-occurrences that GPT-3 found in the billions upon billions of word sequences it had scanned. Although what GPT-3 produces sounds like coherent English, it's all surface patterns with no underlying semantics.

Recent work\cite{geirhos2018imagenet} (see \cite{Cepelewicz-textures-2020} for a popular discussion) suggests that much of the success of deep learning, at least when applied to image categorization, derives from the tendency of deep learning systems to focus on textures---the ultimate surface feature---rather than shapes.  This insight offers an explanation for some of deep learning's brittleness and superficiality as well as a possible mitigation strategy.

%%%%%%%%%%%%%%%%%%%%%%%%%%%%%%%%%%%%%%%%%%%%%%
\smallv\noindent\textbf{The Holy Grail: constraint programming}. In the mean time, work on symbolic AI continued. Constraint programming was born in the 1980s as an outgrowth of the interest in logic programming triggered by the Japanese Fifth Generation initiative.\cite{shapiro1983fifth} Logic Programming led to Constraint Logic Programming, which evolved into Constraint Programming. (A familiar constraint programming example is the well-known n-queens problem: how can you place n queens on an n \textit{x} n chess board so that no queen threatens any other queen? There are, of course, many practical constraint programming applications as well.)

In 1997, Eugene Frueder characterized constraint programming as \textit{the Holy Grail of computer science: the user simply states the problem and the computer solves it.}\cite{freuder1997pursuit}  Software that solves constraint programming problems is known as a solver. Constraint programming has many desirable properties. 
\begin{itemize}
    \item Solutions found by constraint programming solvers actually solve the given problem. There is no issue of how ``confident'' the solver is in the solutions it finds.
    
    \item One can understand how the solver arrived at the solution. This contrasts with the frustrating feature of neural nets that the solutions they find are generally hidden within a maze of parameters, unintelligible to human beings. 

    \item  The structure and limits of constraint programming are well understood: there will be no grand disappointments similar to those that followed the birth of artificial intelligence---unless quantum computing, once implemented, turns out to be a bust. 
    
    \item Constraint programming is closely related to computational complexity, which provides a well-studied theoretical framework for it. 
    
    \item There will be no surprises such as adversarial images. 

    \item Solver technology is easy to characterize. It is an exercise in search: find values for uninstantiated variables that satisfy the constraints.

    \item Improvements are generally incremental and consist primarily of new heuristics and better search strategies. For example, in the n-queens problem one can propagate solution steps by marking as unavailable board squares that are threatened by newly placed pieces. This reduces search times. We will see  example heuristics below.

\end{itemize}

Constraint programming solvers are now available in multiple forms. MiniZinc\cite{wallace2020problem} allows users to express constraints in what is essentially executable predicate calculus.

Solvers are also available as package add-ons to many programming languages: Choco\cite{prud2019choco} and JaCoP\cite{kuchcinski2013jacop} (two Java libraries), OscaR/CBLS\cite{Oscar} and Yuck\cite{Yuck} (two Scala libraries), and Google's OR-tools\cite{Google-OR-tools} (a collection of C++ libraries, which sport Python, Java, and .NET front ends).

In the systems just mentioned, the solver is a black box. One sets up a problem, either directly in predicate calculus or in the host language, and then calls on the solver to solve it. 

This can be frustrating for those who want more insight into the internal workings of the solvers they use. Significantly more insight is available when working either (a) in a system like Picat\cite{zhou2015constraint}, a language that combines features of logic programming and imperative programming, or (b) with Prolog (say either SICStus Prolog\cite{carlsson2014sicstus} or SWI Prolog\cite{swi-prolog}) to which a Finite Domain package has been added. But neither option helps those without a logic programming background. 

%%%%%%%%%%%%%%%%%%%%%%%%%%%%%%%%%%%%%%%%%%%%%%
\smallv\noindent\textbf{Shallow embeddings}.  Solver capabilities may be implemented directly in a host language and made available to programs in that language.\cite{hoare1998unifying, gibbons2014folding} Recent examples include Kanren\cite{Rocklin2019}, a Python embedding, and Muli\cite{dageforde2018constraint}, a Java embedding.

Most shallow embeddings have well-defined APIs; but like libraries, their inner workings are not visible. This is the case with both Kanren and Muli. Kanren is open source, but it offers no implementation documentation. Dagef{\"o}rde and Kuchen describe the Muli virtual machine\cite{dageforde2019compiler}, but the documentation is quite technical. Many who would like to understand its internal functioning may find it difficult going. 

%%%%%%%%%%%%%%%%%%%%%%%%%%%%%%%%%%%%%%%%%%%%%%
\smallv\noindent\textbf{Back to basics}. This brings us to our goal for the rest of this paper: to offer an under-the-covers tutorial about how a fully functioning embedded solver works. 

One can think of Prolog as the skeleton of a constraint satisfaction solver. Consequently, we focus on Prolog as a basic paradigmatic solver. We describe Pylog, a Python shallow embedding of Prolog's core capabilities. 

Our primary focus will be on helping readers understand how Prolog's two fundamental features, backtracking and logic variables, can be implemented \textit{simply and cleanly}. We also show how two of the heuristics common to Finite Domain packages can be added. 

Pylog should be accessible to anyone reasonable fluent in Python. In addition, the techniques used in the implementation are easily transferred to many other languages. 

We stress \textit{simply and cleanly}. There are many ways to implement backtracking and logic variables, some quite complex. Our approach is straightforward and easy to understand. 

An advantage we have over earlier Prolog embeddings is Python generators. Without generators, one is pushed to use more complex backtracking implementations, such as continuation passing\cite{amin2019lightweight} or monads\cite{seres1999embedding}. Generators, which are now widespread\cite{wikipedia-generators}, eliminate the need for such complexity. 

To be clear, we did not invent the use of generators for implementing backtracking. It has a nearly two-decade history: \cite{berger2004, Bolz2007, Delford2009, Frederiksen2011, Meyers2015, Thompson2017, Santini2018, Cesar2019, Miljkovic2019}. We would like especially to thank Ian Piumarta\cite{Piumarta2017}; Pylog began as a fork of his efforts. 

The preceding are sketches and prototypes. We offer a cleanly coded, well-explained, and fully operational solver.


%%%%%%%%%%%%%%%%%%%%%%%%%%%%%%%%%%%%%%%%%%%%%%
\section{Solver basics and heuristics} \label{sec:solver-basics}
%%%%%%%%%%%%%%%%%%%%%%%%%%%%%%%%%%%%%%%%%%%%%%

As an example problem we will use the computation of a transversal. Given a sequence of sets, a transversal is a non-repeating sequence of elements with the property that the \textit{n\textsuperscript{th}} element of the traversal belongs to the \textit{n\textsuperscript{th}} set in the sequence. For example, the set sequence \{1, 2, 3\}, \{1, 2, 4\}, \{1\} has three transversals: [2, 4, 1], [3, 2, 1], and [3, 4, 1]. 

This problem can be solved with a simple depth-first search. Here's a high level description. 
\begin{itemize}
    \item Look for transversal elements from left to right.
    \item Select an element from the first set and (tentatively) assign that as the first element of the transversal.
    \item Recursively look for a transversal for the rest of the sets---being sure not to reuse any already selected elements.
    \item If, at any point, we cannot proceed, say because we have reached a set all of whose elements have already been used, go back to an earlier set, select a different element from that set, and proceed forward.
\end{itemize}

First a utility function (Listing \ref{lis:uninstantiated-indices}) and then \textit{tnvsl\_dfs} (Listing \ref{lis:tnvsl-dfs}), the solver. (Please pardon our Python style deficiencies. The column width and page limit compelled compromises.) 


\begin{center}
\begin{minipage}[c]{0.45\textwidth}
\begin{python1}  
unassigned = '_'
def uninstantiated_indices(transversal):
  """ Find indices of uninstantiated components. """
  return [indx for indx in range(len(transversal)) 
               if transversal[indx] is unassigned]
\end{python1}\linv
\begin{lstlisting} [caption={\textit{uninstantiated\_indices}}, label={lis:uninstantiated-indices}]
\end{lstlisting}
\end{minipage}
\end{center}

\begin{figure}[htb]
    \centering
\begin{minipage}[c]{0.45\textwidth}
\begin{python1}  
def tnvsl_dfs(sets, tnvsl):
  remaining_indices = uninstantiated_indices(tnvsl)
  if not remaining_indices: return tnvsl

  nxt_indx = min(remaining_indices)
  for elmt in sets[nxt_indx]:
    if elmt not in tnvsl:
      new_tnvsl = tnvsl[:nxt_indx] \
                  + (elmt, ) \
                  + tnvsl[nxt_indx+1:]
      full_tnvsl = tnvsl_dfs(sets, new_tnvsl)
      if full_tnvsl is not None: return full_tnvsl
\end{python1}\linv
\begin{lstlisting} [caption={\textit{tnvsl\_dfs}}, label={lis:tnvsl-dfs}]
\end{lstlisting}
\end{minipage}\linv
\end{figure}

Here's an explanation of the search engine in some detail.
\begin{itemize}
    \item The function \textit{tnvsl\_dfs} takes two parameters: 
        \begin{enumerate}
            \item \textit{sets}: a list of sets
            \item \textit{tnvsl}: a tuple with as many positions as there are sets, but initialized to undefined.
        \end{enumerate}
    \item \textit{line 2}. \textit{remaining\_indices} is a list of the indices of uninstantiated elements of \textit{tnvsl}. Initially this will be all of them. Since \textit{tnvsl\_dfs} generates values from left to right, the first element of \textit{remaining\_indices} will always be the leftmost undefined index position.
    \item \textit{line 3}. If \textit{remaining\_indices} is null, we have a complete transversal. Return it. Otherwise, go on to \textit{line 5}.
    \item \textit{line 5}. Set \textit{nxt\_indx} to the first undefined index position.
    \item \textit{line 6}. Begin a loop that looks at the elements of \textit{sets[nxt\_indx]}, the set at position  \textit{nxt\_indx}. We want an element from that set to represent it in the transversal.
    \item \textit{line 7}. If the currently selected \textit{elmt} of \textit{sets[nxt\_indx]} is not already in \textit{tnvsl}:
    \begin{enumerate}
        \item \textit{lines 8-10}. Put \textit{elmt} at position \textit{nxt\_indx}.
        \item \textit{line 11}. Call \textit{tnvsl\_dfs} recursively to complete the transversal, passing \textit{new\_tnvsl}, the extended \textit{tnvsl}. Assign the returned result to \textit{full\_tnvsl}.
        \item \textit{line 12}. If \textit{full\_tnvsl} is not \textbf{None}, we have found a transversal. Return it to the caller. If \textit{full\_tnvsl} is \textbf{None}, the \textit{elmt} we selected from \textit{sets[nxt\_indx]} did not lead to a complete transversal. Return to \textit{line 6} to select another element from \textit{ sets[nxt\_indx]}.
    \end{enumerate}
\end{itemize}

This is standard depth first search. \textit{tnvsl\_dfs} will either find the first transversal, if there are any, or return \textbf{None}.

Here's a trace of the recursive calls.

\mediumv
\begin{minipage}[c]{0.45\textwidth}
\begin{python1}  
sets: [{1,2,3}, {1,2,4}, {1}], tnvsl: (_,_,_)
  sets: [{1,2,3}, {1,2,4}, {1}], tnvsl: (1,_,_)
    sets: [{1,2,3}, {1,2,4}, {1}], tnvsl: (1,2,_)
    sets: [{1,2,3}, {1,2,4}, {1}], tnvsl: (1,4,_)
  sets: [{1,2,3}, {1,2,4}, {1}], tnvsl: (2,_,_)
    sets: [{1,2,3}, {1,2,4}, {1}], tnvsl: (2,1,_)
    sets: [{1,2,3}, {1,2,4}, {1}], tnvsl: (2,4,_)
      sets: [{1,2,3}, {1,2,4}, {1}], tnvsl: (2,4,1)
\end{python1}\linv
\begin{lstlisting} [caption={\textit{tnvsl\_dfs trace}}]
\end{lstlisting}
\end{minipage}

\begin{itemize}
    \item \textit{line 1}. Initially (and on each call) the \textit{sets} are \[\{1, 2, 3\}, \{1, 2, 4\}, \{1\}\] Initially \textit{tnvsl} is completely undefined: \textit{(\_, \_, \_)}
    \item  \textit{line 2}. \textit{1} is selected as the first element of \textit{trvs}.
    \item  \textit{line 3}. \textit{1}  and \textit{2} are selected as the first two elements.
    \item \textit{line 4}. But now we are stuck. Since \textit{1} is already in \textit{trvs}, we can't use it as the third element of \textit{trvs}. Depth first search operates blindly. Instead of selecting an alternative for the first set, it backs up to the most recent selection and selects \textit{4} to represent the second set. 
    \item \textit{lines 5}. Of course, that doesn't solve the problem. So we back up again. Since we have already tried all elements of the second set, we back up to the first set and select \textit{2} as its representative. 
    \item \textit{lines 6}. Going forward, we select \textit{1} for second set.
    \item \textit{lines 7}. Again, we cannot use \textit{1} for the third set. So we back up and select \text{4} to represent the second set. (We can't use \textit{2} since it is already taken.)
    \item \textit{lines 8}. Finally, we can select \textit{1} as the third element of \textit{trvs}, and we're done.
\end{itemize}

\noindent\textbf{How recursively nested for-loops implement choicepoints and backtracking}. This simple depth-first search appears to incorporate backtracking. In fact, there is no backtracking. Recursively nested \textbf{for}-loops produce a backtracking effect.  

It is common to use the term \textit{choicepoint} for a place in a program where (a) multiple choices are possible and (b) one wants to try them all, if necessary. Our simple solver implements choicepoints via (recursively) nested \textbf{for}-loops. 

The \textbf{for}-loop on line 6 generates options until either we find one for which the remainder of the program completes the traversal, or, if the options available have been exhausted, the program fails out of that recursive call and ``backtracks'' to a choicepoint at a higher/earlier level of the recursion.

In this context, backtracking means popping an element from the call stack and restoring the program at the next higher level. As with any function call, the calling function continues at the point after the function call---in this case, line 12. 

If the function called on line 11 returns a complete transversal, we return it to the \textit{next} higher level, which continues to return it up the stack until we reach the original caller. 

If what was returned on line 11 was not a complete transversal, we go around the \textbf{for}-loop again, bind \textit{element} to the next member of \textit{sets[nxt\_indx]}, and try again. 

The call stack serves as a record of earlier, pending choicepoints. We resume them in reverse order as needed. That's exactly what depth-first search is all about.

\smallv
\noindent We now turn to two heuristics that improve solver efficiency. 

\smallv
\noindent\textit{Propagate}. When we select an element for \textit{trvs} we can \textit{propagate} that selection by removing that element from the remaining sets. We can do that with the following changes. (Of course, a real solver would not hard-code heuristics. This is just to show how it works.)
\begin{enumerate}
    \item Before \textit{line 11}, insert this line.
  
\begin{minipage}[c]{0.45\textwidth}
\begin{python1}
new_sets = [set - {elmt} for set in sets]
\end{python1}
\end{minipage}

Then replace \textit{sets} with \textit{new\_sets} in \textit{line 11}.
This will remove \textit{elmt} from the remaining sets.

    \item Before \textit{line 5}, insert

\begin{minipage}[c]{0.45\textwidth}
\begin{python1}
if any(not sets[idx] for idx in remaining_indices):
  return None
\end{python1}
\end{minipage}


This tests whether any of our unrepresented sets are now empty. If so, we can't continue. (Recall that Python style recommends treating a set as a boolean when testing for emptiness. An empty set is considered \textbf{False}.)


\end{enumerate}

Because the empty sets in lines 2 and 4 of the trace trigger backtracking, the execution takes 6 steps rather than 8.

\begin{flushright}
\begin{minipage}[c]{0.45\textwidth}
\begin{python1}  
sets: [{1,2,3}, {1,2,4}, {1}], tnvsl: (_,_,_)
  sets: [{2,3}, {2,4}, set()], tnvsl: (1,_,_)
  sets: [{1,3}, {1,4}, {1}], tnvsl: (2,_,_)
    sets: [{3}, {4}, set()], tnvsl: (2,1 _)
    sets: [{1,3}, {1}, {1}], tnvsl: (2,4,_)
      sets: [{3}, set(), set()], tnvsl: (2,4,1)
\end{python1}\linv
\begin{lstlisting} [caption={\textit{tnvsl\_dfs\_prop trace}}]
\end{lstlisting}
\end{minipage}
\end{flushright}

The \textit{Propagate} heuristic is a partial implementation of the \textit{all-different} constraint. It can be applied to this problem because we know that the transversal elements must all be different from each other.

\noindent\textit{Smallest first}. When selecting which \textit{tnvsl} index to fill next, pick the position associated with the smallest remaining set. 

In the original code, replace line 5 with
\begin{center}
\begin{minipage}[c]{0.45\textwidth}
\begin{python1}
 nxt_indx = min(remaining_indices,
                key=lambda indx: len(sets[indx]))
\end{python1}
\end{minipage}
\end{center}
The resulting trace (Listing \ref{lis:dfs-4-lines}) is only 4 lines. (At line 3, the first two sets are the same size. \textit{min} selects the first.) 

\begin{figure}[htb]
    \centering\begin{minipage}[c]{0.45\textwidth}
\begin{python1}  
sets: [{1,2,3}, {1,2,4}, {1}], tnvsl: (_,_,_)
  sets: [{1,2,3}, {1,2,4}, {1}], tnvsl: (_,_,1)
    sets: [{1,2,3}, {1,2,4}, {1}], tnvsl: (2,_,1)
      sets: [{1,2,3}, {1,2,4}, {1}, tnvsl: (2,4,1)
\end{python1}\linv
\begin{lstlisting} [caption={\textit{tnvsl\_dfs\_smallest trace}}, label={lis:dfs-4-lines}]
\end{lstlisting}\inv
\end{minipage}\linv
\end{figure}

One could apply both heuristics. Since \textit{smallest first} eliminated backtracking, adding the \textit{propagate} heuristic makes no effective difference. But, one can watch the sets shrink.

\mediumv
\begin{minipage}[c]{0.45\textwidth}
\begin{python1} 
sets: [{1,2,3}, {1,2,4}, {1}], tnvsl: (_,_,_)
  sets: [{2,3}, {2,4}, {}], tnvsl: (_,_,1)
    sets: [{3}, {4}, {}], tnvsl: (2,_,1)
      sets: [{3}, {}, {}, tnvsl: (2,4,1)
\end{python1}\linv
\begin{lstlisting} [caption={\textit{tnvsl\_dfs\_both\_heuristics trace}}]
\end{lstlisting}
\end{minipage}

This concludes our discussion of a basic depth-first solver and two useful heuristics. We have yet to mention generators.

%%%%%%%%%%%%%%%%%%%%%%%%%%%%%%%%%%%%%%%%%%%%%%
\section{Generators} \label{sec:generators}
%%%%%%%%%%%%%%%%%%%%%%%%%%%%%%%%%%%%%%%%%%%%%%
In our previous examples, we have been happy to stop once we found a transversal,  any transversal. But what if the problem were a bit harder and we were looking for a transversal whose elements added to a given sum. The solvers we have seen so far wouldn't help---unless we added the new constraint to the solver itself. But we don't want to do that. We want to keep the transversal solvers independent of other constraints. (Adding heuristics don't violate this principle. Heuristics only make solvers more efficient.)

One approach would be to modify the solver to find and return all transversals. We could then select the one(s) that satisfied our additional constraints. But what if there were many transversals? Generating them all before looking at any of them would waste a colossal amount of time. 

We need a solver than can return results while keeping track of where it is with respect to its choicepoints so that it can continue from there if necessary. That's what a generator does. 

Listing \ref{lis:dfs-gen} shows a generator version of our solver, including both heuristics. When called as in Listing \ref{lis:dfs-gen-call}, it produces the trace in Listing \ref{lis:dfs-gen-trace}. 

\begin{figure}[htb]
    \centering
\begin{minipage}[c]{0.45\textwidth}
\begin{python1}  
def tnvsl_dfs_gen(sets, tnvsl):
  remaining_indices = uninstantiated_indices(tnvsl)

  if not remaining_indices: yield tnvsl
  else:
    if any(not sets[i] for i in remaining_indices):
      return None
      
    nxt_indx = min(remaining_indices,
                   key=lambda indx: len(sets[indx]))
    for elmt in sets[nxt_indx]:
      if elmt not in tnvsl:
        new_tnvsl = tnvsl[:nxt_indx] \
                    + (elmt, ) \
                    + tnvsl[nxt_indx+1:]
        new_sets = [set - {elmt} for set in sets]
        for full_tnvsl in tnvsl_dfs_gen(new_sets, 
                                        new_tnvsl):
          yield full_tnvsl
\end{python1}\linv
\begin{lstlisting} [caption={\textit{tnvsl\_dfs\_gen}}, label={lis:dfs-gen}]
\end{lstlisting}
\end{minipage}\linv
\end{figure}


\begin{figure}[htb]
    \centering
\begin{minipage}[c]{0.45\textwidth}
\begin{python1}  
for tnvsl in tnvsl_dfs_gen(sets, ('_','_','_')):
    print('=> ', tnvsl)
\end{python1}\linv
\begin{lstlisting} [caption={\textit{tnvsl\_dfs\_gen}}, label={lis:dfs-gen-call}]
\end{lstlisting}
\end{minipage}\linv
\end{figure}


\begin{figure}[!ht]
    \centering
\begin{minipage}[c]{0.45\textwidth}
\begin{python1}  
sets: [{1,2,3}, {1,2,4}, {1}], tnvsl: (_,_,_)
  sets: [{2,3}, {2,4}, {}], tnvsl: (_,_,1)
    sets: [{3}, {4}, {}], tnvsl: (2,_,1)
      sets: [{3}, {}, {}], tnvsl: (2,4,1)
=>  (2, 4, 1)
    sets: [{2}, {2,4}, {}], tnvsl: (3,_,1)
      sets: [{}, {4}, {}], tnvsl: (3,2,1)
=>  (3, 2, 1)
      sets: [{2}, {2}, {}], tnvsl: (3,4,1)
=>  (3, 4, 1)
\end{python1}\linv
\begin{lstlisting} [caption={\textit{tnvsl\_dfs\_gen trace}}, label={lis:dfs-gen-trace}]
\end{lstlisting}\inv
\end{minipage}\linv
\end{figure}

Some comments on Listing \ref{lis:dfs-gen}.  % \textit{tnvsl\_dfs\_gen}.
\begin{itemize}
    \item The newly added \textbf{else} on line 5 is necessary. Previously, if there were no \textit{remaining\_indices}, we returned \textit{tnvsl}. That was the end of execution for this recursive call. But if we \textbf{yield} instead of \textbf{return}, when \textit{tnvsl\_dfs\_gen} is asked for more results, \textit{it continues with the line after the \textbf{yield}}. But if have already found a transversal, we don't want to continue. The \textbf{else} divides the code into two mutually exclusive components. \textbf{return} had done that implicitly.
    
    \item Lines 17-20 call \textit{tnvsl\_dfs\_gen} recursively and ask for all the transversals that can be constructed from the current state. Each one is then \textbf{yield}ed. No need to exclude \textbf{None}. \textit{tnvsl\_dfs\_gen} will \textbf{yield} only complete transversals. 
    
    \smallv
Lines 17-20 can be replaced by this single line.
\end{itemize}
\begin{center}
\begin{minipage}[c]{0.45\textwidth}
\begin{python1}
    yield from tnvsl_dfs_gen(new_sets, new_tnvsl)
\end{python1}
\end{minipage}   
\end{center}

Let's use \textit{tnvsl\_dfs\_gen} (Listing \ref{lis:dfs-gen}) to solve our initial problem: find a transversal whose elements sum to, say, 6.

\begin{center}
\begin{minipage}[c]{0.45\textwidth}
\begin{python1}
n = 6
for tnvsl in tnvsl_dfs_gen(sets, ('_','_','_')):
  sum_string = ' + '.join(str(i) for i in tnvsl)
  equals = '==' if sum(tnvsl) == n else '!='
  print(f'{sum_string} {equals} {n}')
  if sum(tnvsl) == n: break
\end{python1}\linv
\begin{lstlisting} [caption={\textit{running tnvsl\_dfs\_gen}}, label={lis:dfs-gen-call2}]
\end{lstlisting}
\end{minipage}
\end{center}

The output (without trace) will be as follows.
\begin{center}
\begin{minipage}[c]{0.45\textwidth}
\begin{python1}  
    2 + 4 + 1 != 6
    3 + 2 + 1 == 6
\end{python1}\linv
\begin{lstlisting} [caption={\textit{tnvsl\_dfs\_gen trace}}]
\end{lstlisting}
\end{minipage}
\end{center}
We generated transversals until we found one whose elements summed to 6. Then we stopped.







%%%%%%%%%%%%%%%%%%%%%%%%%%%%%%%%%%%%%%%%%%%%%%
\section{Logic variables} \label{sec:logic-variables}
%%%%%%%%%%%%%%%%%%%%%%%%%%%%%%%%%%%%%%%%%%%%%%
This section discusses logic variables and their realization. 

\subsection{Instantiation}
Logic variables are either instantiated, i.e., have a value, or uninstantiated. The instantiation operation is called \textit{unify}.   

\textit{unify} is a \textit{generator}---even though the act of instantiation \textit{does not} \textbf{yield} a value. Activating \textit{unify} establishes a context within which unification holds. Leaving that context undoes the unification. 

Consider the following sequence of short code segments. 
\begin{center}
\begin{minipage}[c]{0.45\textwidth}
\begin{python1}
A = Var()
\end{python1}
\end{minipage}
\end{center}
\textit{A} is a standard Python identifier. We use an initial capital letter to distinguish logic variables from regular Python variables. \textit{Var} is the constructor for logic variables.

\textit{A} is now an uninstantiated logic variable. When an uninstantiated logic variables is printed, an internal value is shown to distinguish among logic variables. As the first logic variable in this program, \textit{A}'s internal value is \textit{\_1}.

\begin{center}
\begin{minipage}[c]{0.45\textwidth}
\begin{python1}
print(A)  # => _1
\end{python1}
\end{minipage}
\end{center}

Now we \textit{unify A} with \textit{abc}, i.e., instantiate \textit{A} to \textit{abc}. Since \textit{unify} does not \textbf{yield} a value, the \textbf{for}-loop variable is not used. 

The \textbf{for}-loop establishes a context for \textit{unify}. Within the \textbf{for}-loop body \textit{A} is instantiated to  \textit{abc}.

\begin{center}
\begin{minipage}[c]{0.45\textwidth}
\begin{python1}
for _ in unify(A, 'abc'):
    print(A)  # => abc
print(A)  # => _1
\end{python1}
\end{minipage}
\end{center}

Since there is only one way to \textit{unify A} with \textit{abc}, the  \textbf{for}-loop body runs only once.  Leaving the \textit{unify} context undoes the instantiation.

Within a \textit{unify} context, logic variables are immutable. Once a logic variable has a value, it cannot change within its context.

\begin{center}
\begin{minipage}[c]{0.45\textwidth}
\begin{python1}
A = Var()
print(A)  # => _1
for _ in unify(A, 'abc'):
    print(A)  # => abc
    # This unify fails. Its body never runs.
    for _ in unify(A 'def'):
      print(A)  # Never executed
    print(A)  # => abc
print(A)  # => _1
\end{python1}
\end{minipage}
\end{center}

\subsection{The power of \textit{unify}}
\textit{unify} can also identify logic variables with each other. After two uninstantiated logic variables are unified, whenever either gets a value, the other gets that same value.

Unification is surprisingly straightforward. Each \textit{Var} includes a \textit{next} field, which is initially \textbf{None}. When two \textit{Var}s are unified, the result depends on their states of instantiation.  
\begin{itemize}
    \item If both are uninstantiated the \textit{next} field of one points to the other. It makes no difference which points to which. A chain of linked  \textit{Var}s unifies all the \textit{Var}s in the chain. 
    \item If only one is uninstantiated, the uninstantiated one points to the other.  
    \item If both are instantiated to the same value, they are effectively unified. \textit{unify succeeds} but nothing changes.
    \item If both are instantiated but to different values, \textit{unify fails}.
\end{itemize}

A note on terminology. When called (as part of a \textbf{for}-loop) a generator will either \textbf{yield} or \textbf{return}. When a generator \textbf{yield}s, it is said to \textit{succeed}; the \textbf{for}-loop body runs. When a generator \textbf{return}s, it is said to fail; the \textbf{for}-loop body does not run. Instead we exit the \textbf{for}-loop.

We can trace the unifications in Listing \ref{unif-example}.  

% \begin{figure}[hbt]
% \centering
\begin{center}
\begin{minipage}[c]{0.45\textwidth}
\begin{python1}
(A, B, C, D) = (Var(), Var(), Var(), Var())
print(A, B, C, D) # => _1 _2 _3 _4
for _ in unify(A, B):
  for _ in unify(D, C):
    print(A, B, C, D) # => _2 _2 _3 _3
    for _ in unify(A, 'abc'):
      print(A, B, C, D) # => abc abc _3 _3
      for _ in unify(A, D):
        print(A, B, C, D) # => abc abc abc abc
      print(A, B, C, D) # => abc abc _3 _3
    print(A, B, C, D) # => _2 _2 _3 _3
  print(A, B, C, D) # => _2 _2 _3 _4
print(A, B, C, D) # => _1 _2 _3 _4
\end{python1}\linv
\begin{lstlisting} [caption={\textit{Unification example}},  label={unif-example}]
\end{lstlisting}
\end{minipage}
\end{center}
% \end{figure}

The first unifications, lines 3 and 4, produce the following. 
\begin{equation}\label{eq:one}
\begin{array}{c c c c c c c c }
A & \to & B \\
D & \to & C 
\end{array}
\end{equation}

Line 6 unifies \textit{A} and \textit{'abc'}. The first step is to go to the ends of the relevant unification chains. In this case, \textit{B} (the end of \textit{A}'s unification chain) is pointed to \textit{'abc'}. Since  \textit{'abc'} is instantiated, the arrow can only go from \textit{B} to \textit{'abc'}. 

\begin{equation}\label{eq:two}
\begin{array}{c c c c c c c c c c c}
A & \to & B            & \to & 'abc'    \\ 
  &     & D            & \to & C        
\end{array}
\end{equation}

Finally, line 8  unifies \textit{A} with \textit{D}. \textit{C} (the end of \textit{D}'s unification chain) is set to point to \textit{'abc'} (the end of \textit{A}'s unification chain). % The arrow can go only from \textit{C} to \textit{'abc'}.

\begin{equation}\label{eq:three}
\begin{array}{c c c c c c c c c c c}
A & \to & B            & \to & 'abc'      \\ 
  &     &              &     & \uparrow   \\ 
  &     & D            & \to & C        
\end{array}
\end{equation}


% To determine the value of a logical variable, one goes to the end of its unification chain. If the end element is instantiated, that is the (current) value of the variable. That's why all the variables have \textit{abc} as their values after line 8. 

% If the end of a unification chain is unintantiated, the internal value associated with that end element is a place-holder value.

\smallv
\subsection{A logic-variable version of \textit{tnvsl\_dfs\_gen}}
Listing \ref{lis:dfs-with-gen-and-logic-variables} adapts Listing \ref{lis:dfs-gen} for logic variables. The strategy is for \textit{trnsvl} to start as a tuple of uninstantiated \textit{Var}s, which become instantiated as the program runs.

First, an adapted \textit{uninstan\_indices\_lv} returns the indices of the uninstantiated \textit{Var}s in \textit{trnsvl}.
\begin{center}
\begin{minipage}[c]{0.45\textwidth}
\begin{python1}
def uninstan_indices_lv(tnvsl):
  return [indx for indx in range(len(tnvsl)) 
               if not tnvsl[indx].is_instantiated()]
\end{python1}
\end{minipage}
\end{center}

Note that \textit{tnvsl[indx]} retrieves the \textit{indx\textsuperscript{th}} \textit{tnvsl} element. If it's instantiated, it represents the value associated with the \textit{indx\textsuperscript{th}} set. If not, we don't yet have a value for the  \textit{indx\textsuperscript{th}} set.

\begin{figure}[htb]
\centering
\begin{minipage}[c]{0.45\textwidth}
\begin{python1}
def tnvsl_dfs_gen_lv(sets, tnvsl):
  var_indxs = uninstan_indices_lv(tnvsl)
    
  if not var_indxs: yield tnvsl
  else:
    empty_sets = [sets[indx].is_empty() 
                  for indx in var_indxs]
    if any(empty_sets): return None

    nxt_indx = min(var_indxs,
                   key=lambda indx: len(sets[indx]))
    used_values = PyList([tnvsl[i] 
                          for i in range(len(tnvsl)) 
                          if i not in var_indxs])
    T_Var = tnvsl[nxt_indx]
      for _ in member(T_Var, sets[nxt_indx]):
        for _ in fails(member)(T_Var, used_values):
          new_sets = [set.discard(T_Var) 
                      for set in sets]
          yield from tnvsl_dfs_gen_lv(new_sets, 
                                      tnvsl)
\end{python1}\linv
\begin{lstlisting} [caption={\textit{dfs-with-gen-and-logic-variables}},  label={lis:dfs-with-gen-and-logic-variables}]
\end{lstlisting}
\end{minipage}\linv
\end{figure}

Some comments on Listing \ref{lis:dfs-with-gen-and-logic-variables}. (We reformatted some of the lines and changed some of the names from \textit{tnvsl\_dfs\_gen} (Listing \ref{lis:dfs-gen}) so that the program will fit the width of a column.)

\begin{itemize}
    \item \textit{line 6}. The parameter \textit{sets} is a list of \textit{PySet}s. These are logic variable versions of sets. An \textit{is\_empty} method is defined for them.
    \item \textit{lines 12-14}. \textit{used\_values} are the values of the instantiated \textit{tnvsl} elements.
    \item \textit{line 15}. \textit{T\_Var} is the element at the \textit{nxt\_indx\textsuperscript{th}} position of \textit{tnvsl}. Since \textit{nxt\_indx} was selected from the uninstantiated variables, \textit{T\_Var} is an uninstantited \textit{Var}.
    \item \textit{line 16}. \textit{member} successively unifies its first argument with the elements of its second argument. It's equivalent to \textit{\textbf{for} T\_Var \textbf{in} sets[nxt\_indx]} but using unification.
    \item  \textit{line 17}. \textit{fails} takes a predicate as its argument. It converts the predicate to its negation. So \textit{fails(member)} succeeds if and only if \textit{member} fails.
    \item  \textit{line 18}. \textit{PySet}s have a \textit{discard} method that returns a copy of the \textit{PySet} without the argument.
\end{itemize}

When run, we get the same result as before---except that the uninstantiated transversal variables appear as we saw above.
\begin{center}
\begin{minipage}[c]{0.45\textwidth}
\begin{python1}
sets: [{1,2,3}, {1,2,4}, {1}], tnvsl: (_1, _2, _3)
  sets: [{2,3}, {2,4}, {}], tnvsl: (_1, _2, 1)
    sets: [{3}, {4}, {}], tnvsl: (2, _2, 1)
      sets: [{3}, {}, {}], tnvsl: (2, 4, 1)
=> (2, 4, 1)
    sets: [{2}, {2,4}, {}], tnvsl: (3, _2, 1)
      sets: [{}, {4}, {}], tnvsl: (3, 2, 1)
=> (3, 2, 1)
      sets: [{2}, {2}, {}], tnvsl: (3, 4, 1)
=> (3, 4, 1)
\end{python1}
\end{minipage}
\end{center}

The following logic variable version of Listing \ref{lis:dfs-gen-call} will run \textit{tnvsl\_dfs\_gen\_lv} and produce the same result.

\begin{center}
\begin{minipage}[c]{0.45\textwidth}
\begin{python1}
(A, B, C) = (Var(), Var(), Var())
Py_Sets = [PySet(set) for set in sets]
# PyValue creates a logic variable constant.
N = PyValue(6)
for _ in tnvsl_dfs_gen_lv(Py_Sets, (A, B, C)):
  sum_string = ' + '.join(str(i) for i in (A, B, C))
  equals = '==' if A + B + C == N else '!='
  print(f'{sum_string} {equals} {N}')
  if A + B + C == N: break
\end{python1}
\end{minipage}
\end{center}

Here we created three logic variables,  \textit{A}, \textit{B}, and \textit{C} and passed them to \textit{tnvsl\_dfs\_gen\_lv} on line 5. Each time a transversal is found, the body of the \textbf{for}-loop is executed with the values to which \textit{A}, \textit{B}, and \textit{C} have been instantiated. 

The preceding offers some sense of what one can do with logic variables. The next section really puts them to work.

%%%%%%%%%%%%%%%%%%%%%%%%%%%%%%%%%%%%%%%%%%%%%%
\section{A logic puzzle}\label{sec:logic-puzzle}
%%%%%%%%%%%%%%%%%%%%%%%%%%%%%%%%%%%%%%%%%%%%%%

At this point, one might expect a complex logic puzzle like the Zebra Puzzle\cite{ZebraPuzzle}. Instead we present a similar but simpler puzzle. The techniques are the same, but the following Scholarship Puzzle\cite{ScholarshipPuzzle} fits the available space better. 


\begin{itemize}
    \item There are four students: Ada, Emmy, Lynn, and Marie. Each has a scholarship and a major.  No two students have the same scholarship or the same major. 
    \item The scholarships and majors are \$25,000, \$30,000, \$35,000 and \$40,000 and Bio, CS, Math, and Phys. 
\end{itemize}

From the clues listed below, determine which student studies which major and the amount of each student's scholarship.

%\largev
We create a \textbf{class} \textit{Stdnt}, each of whose instances has two fields: \textit{name} and \textit{major}. (We do \textit{not} keep track of the students' scholarships!) For example, a \textit{Stdnt} object that represents \textit{Ada} studying \textit{Phys} is constructed like this \textit{Stdnt(name='Ada', major='Phys)} and printed as \textit{Ada/Phys}. 

Objects are not always fully instantiated. Missing information is represented by an underscore (\_). An object that represents some person studying \textit{Bio} would look like this \textit{\_/Bio}. It would be constructed as: \textit{Stdnt(major='Bio')}.

Our \textit{world} consists of a list of \textit{Stdnt} objects with scholarships of increasing size. (Although we do not record scholarship amounts, we know their relative sizes!) This list is passed to the clues and will be fully instantiated as the answer.

A number of utility methods are defined.
\begin{itemize}
    \item \textit{is\_contiguous\_in(list1, list2)} unifies the elements of \textit{list1} with those of \textit{list2} if the elements of \textit{list1} appear together in \textit{list2} in the same order as in \textit{list1}. On backtracking, yields all possible matches. 
    
    \smallv
    Unification fails between objects with instantiated fields having different values. For example \textit{Marie/Physics} would not unify with \textit{\_/Math}.
    
    \smallv
    But \textit{Marie/\_} would unify with \textit{\_/Phys}. After unification, the two objects would each have both fields identically instantiated: \textit{Mia/Physics}.
    
    \item \textit{is\_subseq(list1, list2)} is the same as \textit{is\_contiguous\_in}, but the elements of \textit{list1} may appear in \textit{list2} with gaps between them.
    \item \textit{member(student, list)} unifies \textit{student} with eligible elements of \textit{list}. On backtracking, yields all matches. This is the same method we used in the transversal problem.
\end{itemize}

Listing \ref{lis:clues} contains the clues as simple methods. Listing \ref{lis:search-engine} contains a list of method names and the search engine.  (The \textit{run\_clue} method runs the \textit{all-different} heuristic to prevent the same field value from being used more than once.) 

\begin{figure*}[htb]
\flushright
\begin{minipage}[c]{0.95\textwidth}
\begin{python1}
def clue_1(self, Stdnts):
  """ The student who studies Phys gets a smaller scholarship than Emmy. """
  yield from is_subseq([Stdnt(major='Phys'),Stdnt(name='Emmy')],Stdnts)

def clue_2(self, Stdnts):
  """ Emmy studies either Math or Bio. """
  # Create Major as a local logic variable.
  Major = Var()
  for _ in member(Stdnt(name='Emmy', major=Major),Stdnts):
    yield from member(Major, PyList(['Math','Bio'])) 
  
def clue_3(self, Stdnts):
  """ The Stdnt who studies CS has a $5,000 larger scholarship than Lynn. """
  yield from is_contiguous_in([Stdnt(name='Lynn'), Stdnt(major='CS')], Stdnts)
  
def clue_4(self, Stdnts):
  """ Marie gets $10,000 more than Lynn. """
  yield from is_contiguous_in([Stdnt(name='Lynn'),Var(),Stdnt(name='Marie')],Stdnts)
  
def clue_5(self, Stdnts):
  """ Ada has a larger scholarship than the Stdnt who studies Bio. """
  yield from is_subseq([Stdnt(major='Bio'), Stdnt(name='Ada')],Stdnts)
\end{python1}\linv
\begin{lstlisting} [caption={\textit{sample}},  label={lis:clues}]
\end{lstlisting}
\end{minipage}\linv
\end{figure*}

% self.clues = [clue_1, clue_2, clue_3, clue_4, clue_5] 

% def run_all_clues(self, clue_number):
%   # Ran all the clues. Succeed.
%   if clue_number >= len(self.clues): yield
%   else:
%     # Run the current clue and then the rest.
%     for _ in self.run_clue(clue_number):
%       yield from self.run_all_clues(clue_number + 1)

% \begin{center}
% \begin{minipage}[c]{0.45\textwidth}
% \begin{python1}
% def clue_1(self, Stdnts):
%   """ The student who studies Phys gets a 
%       smaller scholarship than Emmy. """
%   yield from is_subseq(
%     [Stdnt(major='Phys'),Stdnt(name='Emmy')],Stdnts)
% \end{python1}
% \end{minipage}

% \begin{minipage}[c]{0.45\textwidth}
% \begin{python1}
% def clue_2(self, Stdnts):
%   """ Emmy studies either Math or Bio. """
%   # Create Major as a local logic variable.
%   Major = Var()
%   for _ in member(Stdnt(name='Emmy', major=Major), 
%                   Stdnts):
%     yield from member(Major, PyList(['Math','Bio']))
% \end{python1}
% \end{minipage}

% \begin{minipage}[c]{0.45\textwidth}
% \begin{python1}
% def clue_3(self, Stdnts):
%   """ The Stdnt who studies CS has 
%       a $5,000 larger scholarship than Lynn. """
%   yield from is_contiguous_in(
%     [Stdnt(name='Lynn'), Stdnt(major='CS')], Stdnts)
% \end{python1}
% \end{minipage}

% \begin{minipage}[c]{0.45\textwidth}
% \begin{python1}
% def clue_4(self, Stdnts):
%   """ Marie gets $10,000 more than Lynn. """
%   yield from is_contiguous_in(
%     [Stdnt(name='Lynn'),Var(),Stdnt(name='Marie')], 
%     Stdnts)
% \end{python1}
% \end{minipage}

% \begin{minipage}[c]{0.45\textwidth}
% \begin{python1}
% def clue_5(self, Stdnts):
%   """ Ada has a larger scholarship than the Stdnt 
%       who studies Bio. """
%   yield from is_subseq(
%     [Stdnt(major='Bio'), Stdnt(name='Ada')], 
%     Stdnts)
% \end{python1}
% \end{minipage}
% \end{center}

\begin{center}
\begin{minipage}[c]{0.46\textwidth}
\begin{python1}
self.clues = [clue_1, clue_2, clue_3, clue_4, clue_5] 

def run_all_clues(self, clue_number):
  # Ran all the clues. Succeed.
  if clue_number >= len(self.clues): yield
  else:
    # Run the current clue and then the rest.
    for _ in self.run_clue(clue_number):
      yield from self.run_all_clues(clue_number + 1)
\end{python1}\linv
\begin{lstlisting} [caption={\textit{sample}},  label={lis:search-engine}]
\end{lstlisting}
\end{minipage}
\end{center}

The execution trace in Listing \ref{scholarship-problem} shows the order (including backtracking) in which the clues were executed. Each line shows the then-current list of partially instantiated students. 

The trace shows that we requested further backtracking to look for other solutions. (There aren't any.) 

The total compute time on a 3-year-old laptop was 0.01 sec. 


\begin{figure}[hb]
    \flushright
\begin{minipage}[c]{0.45\textwidth}
\begin{python1}
Initially: _/_, _/_, _/_, _/_
Clue 1: _/Phys, Emmy/_, _/_, _/_
Clue 2: _/Phys, Emmy/Math, _/_, _/_
Clue 3: _/Phys, Emmy/Math, Lynn/_, _/CS
Clue 2: _/Phys, Emmy/Bio, _/_, _/_
Clue 3: _/Phys, Emmy/Bio, Lynn/_, _/CS
Clue 1: _/Phys, _/_, Emmy/_, _/_
Clue 2: _/Phys, _/_, Emmy/Math, _/_
Clue 3: Lynn/Phys, _/CS, Emmy/Math, _/_
Clue 2: _/Phys, _/_, Emmy/Bio, _/_
Clue 3: Lynn/Phys, _/CS, Emmy/Bio, _/_
Clue 1: _/Phys, _/_, _/_, Emmy/_
Clue 2: _/Phys, _/_, _/_, Emmy/Math
Clue 3: Lynn/Phys, _/CS, _/_, Emmy/Math
Clue 4: Lynn/Phys, _/CS, Marie/_, Emmy/Math
Clue 3: _/Phys, Lynn/_, _/CS, Emmy/Math
Clue 2: _/Phys, _/_, _/_, Emmy/Bio
Clue 3: Lynn/Phys, _/CS, _/_, Emmy/Bio
Clue 4: Lynn/Phys, _/CS, Marie/_, Emmy/Bio
Clue 3: _/Phys, Lynn/_, _/CS, Emmy/Bio
Clue 1: _/_, _/Phys, Emmy/_, _/_
Clue 2: _/_, _/Phys, Emmy/Math, _/_
Clue 2: _/_, _/Phys, Emmy/Bio, _/_
Clue 1: _/_, _/Phys, _/_, Emmy/_
Clue 2: _/_, _/Phys, _/_, Emmy/Math
Clue 3: _/_, Lynn/Phys, _/CS, Emmy/Math
Clue 2: _/_, _/Phys, _/_, Emmy/Bio
Clue 3: _/_, Lynn/Phys, _/CS, Emmy/Bio
Clue 1: _/_, _/_, _/Phys, Emmy/_
Clue 2: _/_, _/_, _/Phys, Emmy/Math
Clue 3: Lynn/_, _/CS, _/Phys, Emmy/Math
Clue 4: Lynn/_, _/CS, Marie/Phys, Emmy/Math
Clue 5: Lynn/Bio, Ada/CS, Marie/Phys, Emmy/Math

After 33 rule applications,
Solution: 
	1. Lynn/Bio	($25,000 scholarship)
	2. Ada/CS	($30,000 scholarship)
	3. Marie/Phys	($35,000 scholarship)
	4. Emmy/Math	($40,000 scholarship)

More? (y, or n)? > y
Clue 2: _/_, _/_, _/Phys, Emmy/Bio
Clue 3: Lynn/_, _/CS, _/Phys, Emmy/Bio
Clue 4: Lynn/_, _/CS, Marie/Phys, Emmy/Bio
\end{python1}\linv
\begin{lstlisting} [caption={\textit{Trace of the scholarship problem}}, label={scholarship-problem}]
\end{lstlisting}
\end{minipage} \linv
\end{figure}

% More? (y, or n)? > y

% Clue 2: _/_, _/_, _/Phys, Emmy/Bio
% Clue 3: Lynn/_, _/CS, _/Phys, Emmy/Bio
% Clue 4: Lynn/_, _/CS, Marie/Phys, Emmy/Bio

% After 3 final rule applications, no more solutions.

% The total compute time was: 0.01 sec

%%%%%%%%%%%%%%%%%%%%%%%%%%%%%%%%%%%%%%%%%%%%%%
\section{Conclusion} \label{sec:conclusion}
%%%%%%%%%%%%%%%%%%%%%%%%%%%%%%%%%%%%%%%%%%%%%%

We have explained how solvers for constraint problems work and how they can be integrated into Python programs. 

It's difficult to imagine a neural net (of any depth!) solving the problems discussed here---although preliminary work toward that end has been reported. \cite{xu2018towards, amel2019shallow, dubois2019towards}

The Pylog code is available on \href{https://github.com/RussAbbott/pylog/tree/master/pylog}{GitHub}:

\href{https://github.com/RussAbbott/pylog/tree/master/pylog}{github.com/RussAbbott/pylog/tree/master/pylog}.


% \section{Conclusion}\label{sec:conclusion}
Embedding rule chaining in the clues as in the previous section suggests a general template.

\begin{minipage}{\linewidth}
\begin{python}
   def some_clause(...):
     for _ in <generate options>:
       <local conditions>
       yield from next_clause(...)
\end{python}
\end{minipage}

More generally, Pylog offers a way to integrate logic programming features into a Python environment.

\begin{itemize}
  \item The magic of unification requires little more than linked chains.
  \item Prolog's control structures, including "backtracking," can be implemented as nested \textbftt{for}-loops (for both choicepoints and scope setting), with \textbftt{yield} and \textbftt{yield from} gluing the pieces together.
\end{itemize}

\appendix 
\section{The Trace decorator}

The \textit{Trace} decorator is defined as a class rather than a function. 

@Trace logs parameter values for both regular functions and generators. 

@Trace does not handle keyword parameters.

\begin{minipage}{\linewidth} \largev \hrulefill
\begin{python}[numbers=left]
from inspect import isgeneratorfunction, signature

class Trace:

    def __init__(self, f):
        self.param_names = [param.name for param in signature(f).parameters.values()]
        self.f = f
        self.depth = 0

    def __call__(self, *args):
        print(self.trace_line(args))
        self.depth += 1
        if isgeneratorfunction(self.f):
            return self.yield_from(*args)
        else:
            f_return = self.f(*args)
            self.depth -= 1
            return f_return

    def yield_from(self, *args):
        yield from self.f(*args)
        self.depth -= 1

    @staticmethod
    def to_str(xs):
        xs_string = f'[{", ".join(Trace.to_str(x) for x in xs)}]' if isinstance(xs, list) else str(xs)
        return xs_string

    def trace_line(self, args):
        # The quoted string on the next line is two spaces.
        prefix = "  " * self.depth
        params = ", ".join([f'{param_name}: {Trace.to_str(arg)}'
                            for (param_name, arg) in zip(self.param_names, args)])
        # Special case for the transversal functions
        termination = ' <=' if not args[0] else ''
        return prefix + params + termination

\end{python}

\begin{lstlisting} [caption={The Trace decorator},  label={lis:Trace}]
\end{lstlisting}
\end{minipage}

\printbibliography 
% \section{Introduction}
% % no \IEEEPARstart
% This demo file is intended to serve as a ``starter file''
% for IEEE conference papers produced under \LaTeX\ using
% IEEEtran.cls version 1.8b and later.
% % You must have at least 2 lines in the paragraph with the drop letter
% % (should never be an issue)
% I wish you the best of success.

% \hfill mds
 
% \hfill August 26, 2015

% \subsection{Subsection Heading Here}
% Subsection text here.


% \subsubsection{Subsubsection Heading Here}
% Subsubsection text here.


% An example of a floating figure using the graphicx package.
% Note that \label must occur AFTER (or within) \caption.
% For figures, \caption should occur after the \includegraphics.
% Note that IEEEtran v1.7 and later has special internal code that
% is designed to preserve the operation of \label within \caption
% even when the captionsoff option is in effect. However, because
% of issues like this, it may be the safest practice to put all your
% \label just after \caption rather than within \caption{}.
%
% Reminder: the "draftcls" or "draftclsnofoot", not "draft", class
% option should be used if it is desired that the figures are to be
% displayed while in draft mode.
%
%\begin{figure}[!t]
%\centering
%\includegraphics[width=2.5in]{myfigure}
% where an .eps filename suffix will be assumed under latex, 
% and a .pdf suffix will be assumed for pdflatex; or what has been declared
% via \DeclareGraphicsExtensions.
%\caption{Simulation results for the network.}
%\label{fig_sim}
%\end{figure}

% Note that the IEEE typically puts floats only at the top, even when this
% results in a large percentage of a column being occupied by floats.


% An example of a double column floating figure using two subfigures.
% (The subfig.sty package must be loaded for this to work.)
% The subfigure \label commands are set within each subfloat command,
% and the \label for the overall figure must come after \caption.
% \hfil is used as a separator to get equal spacing.
% Watch out that the combined width of all the subfigures on a 
% line do not exceed the text width or a line break will occur.
%
%\begin{figure*}[!t]
%\centering
%\subfloat[Case I]{\includegraphics[width=2.5in]{box}%
%\label{fig_first_case}}
%\hfil
%\subfloat[Case II]{\includegraphics[width=2.5in]{box}%
%\label{fig_second_case}}
%\caption{Simulation results for the network.}
%\label{fig_sim}
%\end{figure*}
%
% Note that often IEEE papers with subfigures do not employ subfigure
% captions (using the optional argument to \subfloat[]), but instead will
% reference/describe all of them (a), (b), etc., within the main caption.
% Be aware that for subfig.sty to generate the (a), (b), etc., subfigure
% labels, the optional argument to \subfloat must be present. If a
% subcaption is not desired, just leave its contents blank,
% e.g., \subfloat[].


% An example of a floating table. Note that, for IEEE style tables, the
% \caption command should come BEFORE the table and, given that table
% captions serve much like titles, are usually capitalized except for words
% such as a, an, and, as, at, but, by, for, in, nor, of, on, or, the, to
% and up, which are usually not capitalized unless they are the first or
% last word of the caption. Table text will default to \footnotesize as
% the IEEE normally uses this smaller font for tables.
% The \label must come after \caption as always.
%
%\begin{table}[!t]
%% increase table row spacing, adjust to taste
%\renewcommand{\arraystretch}{1.3}
% if using array.sty, it might be a good idea to tweak the value of
% \extrarowheight as needed to properly center the text within the cells
%\caption{An Example of a Table}
%\label{table_example}
%\centering
%% Some packages, such as MDW tools, offer better commands for making tables
%% than the plain LaTeX2e tabular which is used here.
%\begin{tabular}{|c||c|}
%\hline
%One & Two\\
%\hline
%Three & Four\\
%\hline
%\end{tabular}
%\end{table}


% Note that the IEEE does not put floats in the very first column
% - or typically anywhere on the first page for that matter. Also,
% in-text middle ("here") positioning is typically not used, but it
% is allowed and encouraged for Computer Society conferences (but
% not Computer Society journals). Most IEEE journals/conferences use
% top floats exclusively. 
% Note that, LaTeX2e, unlike IEEE journals/conferences, places
% footnotes above bottom floats. This can be corrected via the
% \fnbelowfloat command of the stfloats package.




% \section{Conclusion}
% The conclusion goes here.




% conference papers do not normally have an appendix


% use section* for acknowledgment
% \section*{Acknowledgment}


% The authors would like to thank...





% trigger a \newpage just before the given reference
% number - used to balance the columns on the last page
% adjust value as needed - may need to be readjusted if
% the document is modified later
%\IEEEtriggeratref{8}
% The "triggered" command can be changed if desired:
%\IEEEtriggercmd{\enlargethispage{-5in}}

% references section

% can use a bibliography generated by BibTeX as a .bbl file
% BibTeX documentation can be easily obtained at:
% http://mirror.ctan.org/biblio/bibtex/contrib/doc/
% The IEEEtran BibTeX style support page is at:
% http://www.michaelshell.org/tex/ieeetran/bibtex/
%\bibliographystyle{IEEEtran}
% argument is your BibTeX string definitions and bibliography database(s)
%\bibliography{IEEEabrv,../bib/paper}
%
% <OR> manually copy in the resultant .bbl file
% set second argument of \begin to the number of references
% (used to reserve space for the reference number labels box)
% \begin{thebibliography}{1}

% \bibitem{IEEEhowto:kopka}
% H.~Kopka and P.~W. Daly, \emph{A Guide to \LaTeX}, 3rd~ed.\hskip 1em plus
%   0.5em minus 0.4em\relax Harlow, England: Addison-Wesley, 1999.

% \end{thebibliography}




% that's all folks
\end{document}


