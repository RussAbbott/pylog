\section{Unification and control}\label{subsec:unification}
This section examines the Pylog implementation of unification along with a number of control functions.

\subsection{\textittt{Unification}}
As it turns out, unification is relatively straightforward. 

The \textittt{unify} function is called, \textittt{unify(Left, Right)}, where \textittt{Left} and \textittt{Right} are the Pylog objects to be unified. (Argument order is immaterial.) 

The first step (line 4) ensures that the arguments are Pylog objects. If either is an immutable Python element, such as a string or int, it is wrapped in a \textittt{PyValue}. This allows us to call, e.g, \textittt{unify(X, 'abc')}.
   
There are four \textittt{unify} cases.

\begin{enumerate}
    \item \textittt{Left} and \textittt{Right} are already the same. Since Pylog objects are immutable, neither can change, and there's nothing to do. Succeed quietly via \textbftt{yield}.

    \item \textittt{Left} and \textittt{Right} are both \textittt{PyValue}s, and exactly one of them has a value. Assign the uninstantiated \textittt{PyValue} the value of the instantiated one.
    \smallv \\
    An important step is to set the assignment back to \textbftt{None} after the \textbftt{yield} statement. (line 18) This undoes the unification on backtracking.

    \item \textittt{Left} and \textittt{Right} are both \textittt{Structure}s, and they have the same functor. Unification consists of unifying the respective arguments. 

    \item Either \textittt{Left} or \textittt{Right} is a \textittt{Var}. Point the \textittt{Var} to the other element at the end of the \textittt{Var} element's unification chain. As line 1 shows \textittt{unify} has a decorator. \textittt{euc} ensures that if either argument is a \textittt{Var} it is replaced by the element at the end of its unification chain. ("euc" stands for "end of unification chain".) Again, the unification must be undone on backtracking. (line 30) 

\end{enumerate} 

\begin{minipage}{\linewidth}  \largev \hrulefill
\begin{python}[numbers=left]
@euc
def unify(Left: Any, Right: Any):

  (Left, Right) = map(ensure_is_logic_variable, (Left, Right))

  # Case 1.
  if Left == Right:
    yield

  # Case 2.
  elif isinstance(Left, PyValue) and isinstance(Right, PyValue) and \
       (not Left.is_instantiated( ) or not Right.is_instantiated( )) and \
       (Left.is_instantiated( ) or Right.is_instantiated( )):
    (assignedTo, assignedFrom) = (Left, Right) if Right.is_instantiated( ) else (Right, Left)
    assignedTo._set_py_value(assignedFrom.get_py_value( ))
    yield

    assignedTo._set_py_value(None)

  # Case 3.
  elif isinstance(Left, Structure) and isinstance(Right, Structure) and Left.functor == Right.functor:
    yield from unify_sequences(Left.args, Right.args)

  # Case 4.
  elif isinstance(Left, Var) or isinstance(Right, Var):
    (pointsFrom, pointsTo) = (Left, Right) if isinstance(Left, Var) else (Right, Left)
    pointsFrom.unification_chain_next = pointsTo
    yield

    pointsFrom.unification_chain_next = None

\end{python}
\begin{lstlisting} [caption={unify},  label={lis:unify}]
\end{lstlisting}
\end{minipage}

\subsection{\textittt{Control functions}}
The following control functions are defined. We leave it to the doc-strings to explain what they do. It's striking how straight-forward they are.

\begin{minipage}{\linewidth}  \largev \hrulefill
\begin{python}[numbers=left]
def fails(f):
  """
  Applied to a function so that the resulting function succeeds iff the original fails.
  Note that it is applied to the function itself, not to a function call.
  Similar to a decorator but applied explicitly when used.
  """
  def fails_wrapper(*args, **kwargs):
    for _ in f(*args, **kwargs):
      # Fail, i.e., don't yield, if f succeeds
      return  
    # Succeed if f fails.
    yield     

  return fails_wrapper
\end{python}
\begin{lstlisting} [caption={Control functions},  label={lis:control}]
\end{lstlisting}
\end{minipage}

\begin{minipage}{\linewidth}  \largev \hrulefill
\begin{python}[numbers=left]
def forall(gens):
  """
  Succeeds if all generators in the gens list succeed. The elements in the gens list
  are embedded in lambda functions to avoid premature evaluation.
  """
  if not gens:
    # They have all succeeded.
    yield
  else:
    # Get gens[0] and evaluate the lambda expression to get a fresh iterator.
    # If it succeeds, run the rest of the generators in the list.
    for _ in gens[0]( ):
      yield from forall(gens[1:])
\end{python}
\begin{lstlisting} [caption={Control functions},  label={lis:control}]
\end{lstlisting}
\end{minipage}

\begin{minipage}{\linewidth}  \largev \hrulefill
\begin{python}[numbers=left]
def forany(gens):
  """
  Succeeds if any of the generators in the gens list succeed. On "back-up," tries them all. 
  The gen elements must be embedded in lambda functions.
  """
  for gen in gens:
    yield from gen( )

\end{python}
\begin{lstlisting} [caption={Control functions},  label={lis:control}]
\end{lstlisting}
\end{minipage}

\begin{minipage}{\linewidth}  \largev \hrulefill
\begin{python}[numbers=left]
def trace(x, succeed=True, show_trace=True):
  """
  Can be included in a list of generators (as in forall and forany) to see where we are.
  The second argument determines whether it succeeds or fails.
  When included in a list of forall generators, succeed should be set to True so that
  it doesn't prevent forall from succeeding.
  When included in a list of forany generators, succeed should be set to False so that forany
  will go on the the next generator and won't take this as an extraneous successes.
  """
  if show_trace:
    print(x)
  if succeed:
    yield

\end{python}
\begin{lstlisting} [caption={Control functions},  label={lis:control}]
\end{lstlisting}
\end{minipage}

\begin{minipage}{\linewidth}  \largev \hrulefill
\begin{python}[numbers=left]
def would_succeed(f):
  """
  Applied to a function so that the resulting function succeeds/fails if and only if 
  the original function succeeds/fails.
  If the original function succeeds, this also succeeds but without binding any variables.
  Similar to a decorator but applied explicitly when used.
  """
  def would_succeed_wrapper(*args, **kwargs):
    succeeded = False
    for _ in f(*args, **kwargs):
      succeeded = True
      
    # Do not yield in the context of f succeeding.
    # Un-unify any unifications that occurred in f.
    if succeeded:
      # Succeed if f succeeded.
      yield  
    # else:
    #   Fail if f failed.
    #   pass   

  return would_succeed_wrapper

\end{python}
\begin{lstlisting} [caption={Control functions},  label={lis:control}]
\end{lstlisting}
\end{minipage}


Finally, there is a more complex control structure, which is defined as an instance of the \textittt{Bool\_Yield\_Wrapper} class. It effectively converts \textbftt{yield} to a boolean. \textittt{Bool\_Yield\_Wrapper} instances are created via a \textittt{bool\_yield\_wrapper} decorator function. The decorator returns a function that instantiates \textittt{Bool\_Yield\_Wrapper} with the decorted function.

\begin{minipage}{\linewidth}  \largev \hrulefill
\begin{python}[numbers=left]
def bool_yield_wrapper(gen):
  """
  A decorator. Produces a function that generates a Bool_Yield_Wrapper object. 
  """
  def wrapped_func(*args, **kwargs):
    return Bool_Yield_Wrapper(gen(*args, **kwargs))

  return wrapped_func
\end{python}
\begin{lstlisting} [caption={Control functions},  label={lis:control}]
\end{lstlisting}
\end{minipage}

Suppose the \textittt{append} function discussed earlier is decorated with \textittt{@bool\_yield\_wrapper}. The resulting function may be used as follows. (Note that line 1 is required to generate the \textittt{append}-based instance of \textittt{@bool\_yield\_wrapper}, which is then used in the \textbftt{while} loop.)

\begin{minipage}{\linewidth}  \largev \hrulefill
\begin{python}[numbers=left]
with append(Xs, Ys, Zs) as gen:
  while gen.has_more():
    print(f'Xs = {Xs}\nYs = {Ys}\nZs = {Zs}\n')
\end{python}
\begin{lstlisting} [caption={Control functions},  label={lis:control}]
\end{lstlisting}
\end{minipage}
The output will be all the solutions to the append operation, depending on the initial values and state of instantiation of \textittt{Xs}, \textittt{Ys}, and \textittt{Zs}.

An advantage of this approach is that it avoids the \textbftt{for} loop. \textbftt{for} loops just don't feel like the right structure to use for Prolog backtracking. The disadvantage of this approach is its additional wordiness and increased indentation, i.e., two nested lines, \textbftt{with} and \textbftt{while}, instead of the simpler \textbftt{for} construct. In practice, we found ourselves using the \textbftt{for} construct most of the time.