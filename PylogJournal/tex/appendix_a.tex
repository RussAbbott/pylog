\appendix 

\newpage
%%%%%%%%%%%%%%%%%%%%%%%%%%%%%%%%%%%%%%%%%%%%%%
\section{Introduction} 
%%%%%%%%%%%%%%%%%%%%%%%%%%%%%%%%%%%%%%%%%%%%%%
There are no listings from the \textit{Introduction}.

%%%%%%%%%%%%%%%%%%%%%%%%%%%%%%%%%%%%%%%%%%%%%%
\section{Related work} 
%%%%%%%%%%%%%%%%%%%%%%%%%%%%%%%%%%%%%%%%%%%%%%
There are no listings from \textit{Related work}.

%%%%%%%%%%%%%%%%%%%%%%%%%%%%%%%%%%%%%%%%%%%%%%
\section{From Python to Prolog (Listings from Section \ref{sec:pylog})} \label{appsec:pylog}
%%%%%%%%%%%%%%%%%%%%%%%%%%%%%%%%%%%%%%%%%%%%%%

%%%%%%%%%%%%%%%%%%%%%%%%%%%%%%%%%%%%%%%%%%%%%%
\subsection{tvsl\_dfs\_first (Listings from Section \ref{subsec:tvsl_dfs_first})} \label{appsubsec:tvsl_dfs_first}
%%%%%%%%%%%%%%%%%%%%%%%%%%%%%%%%%%%%%%%%%%%%%%

\begin{minipage}{\linewidth}   \hrulefill
\begin{python}[numbers=left]
@Trace
def tvsl_dfs_first(sets: List[List[int]], partial_transversal: Tuple = ()) -> Optional[Tuple]:
  if not sets:
    return partial_transversal
  else:
    for element in sets[0]:
      if element not in partial_transversal:
        complete_transversal = tvsl_dfs_first(sets[1:], partial_transversal + (element, ))
        if complete_transversal is not None:
          return complete_transversal 
    return None
\end{python}
\begin{lstlisting} [caption={\textit{tvsl\_dfs\_first}}, label={lis:dfsfirst}]
\end{lstlisting}
\end{minipage}

\noindent
\begin{minipage}{\linewidth}
\largev   
\begin{python}[numbers=left]
sets: [[1, 2, 3], [2, 4], [1]]
  sets: [[2, 4], [1]], partial_transversal: (1,)
    sets: [[1]], partial_transversal: (1, 2)
    sets: [[1]], partial_transversal: (1, 4)
  sets: [[2, 4], [1]], partial_transversal: (2,)
    sets: [[1]], partial_transversal: (2, 4)
      sets: [], partial_transversal: (2, 4, 1) <=
\end{python}
\begin{lstlisting} [caption={\textit{transversal\_dfs\_first trace}},  label={lis:dfs_first_trace}]
\end{lstlisting}
\end{minipage}

%%%%%%%%%%%%%%%%%%%%%%%%%%%%%%%%%%%%%%%%%%%%%%
\subsection{\textbf{for}-loops as choice points and as computational aggregators (Listings from Section \ref{subsec:forloops})} \label{appsubsec:forloops}
%%%%%%%%%%%%%%%%%%%%%%%%%%%%%%%%%%%%%%%%%%%%%%

\begin{minipage}{\linewidth}  \largev \begin{python}[numbers=left]
def find_largest(lst):
    largest = lst[0]
    for element in lst[1:]:
        largest = max(largest, element)
    return largest

a_list = [3, 5, 2, 7, 4]
print(f'Largest of {a_list} is {find_largest(a_list)}.')
\end{python}
\begin{lstlisting} [caption={find largest},  label={lis:find_largest}]
\end{lstlisting}
\end{minipage}

%%%%%%%%%%%%%%%%%%%%%%%%%%%%%%%%%%%%%%%%%%%%%%
\subsection{tvsl\_dfs\_all (Listings from Section \ref{subsec:tvsl_dfs_all})} \label{appsubsec:tvsl_dfs_all}
%%%%%%%%%%%%%%%%%%%%%%%%%%%%%%%%%%%%%%%%%%%%%%

\begin{minipage}{\linewidth} \largev   \hrulefill
\begin{python}[numbers=left]
@Trace
def tvsl_dfs_all(sets: List[List[int]], partial_transversal: Tuple = ()) -> List[Tuple]:
  if not sets:
    return [partial_transversal]
  else:
    all_transversals = []
    for element in sets[0]:
      if element not in partial_transversal:
        all_transversals += tvsl_dfs_all(sets[1:], partial_transversal + (element, ))
    return all_transversals
\end{python}
\begin{lstlisting} [caption={transversal\_dfs\_all},  label={lis:dfsall}]
\end{lstlisting}
\end{minipage}

\noindent
\begin{minipage}{\linewidth}   \hrulefill  
\begin{python}
sets: [[1, 2, 3], [2, 4], [1]]
  sets: [[2, 4], [1]], partial_transversal: (1,)
    sets: [[1]], partial_transversal: (1, 2)
    sets: [[1]], partial_transversal: (1, 4)
  sets: [[2, 4], [1]], partial_transversal: (2,)
    sets: [[1]], partial_transversal: (2, 4)
      sets: [], partial_transversal: (2, 4, 1) <=
  sets: [[2, 4], [1]], partial_transversal: (3,)
    sets: [[1]], partial_transversal: (3, 2)
      sets: [], partial_transversal: (3, 2, 1) <=
    sets: [[1]], partial_transversal: (3, 4)
      sets: [], partial_transversal: (3, 4, 1) <=
\end{python}
\begin{lstlisting} [caption={\textit{transversal\_dfs\_all trace}},  label={lis:dfsalltrace}]
\end{lstlisting}
\end{minipage}


%%%%%%%%%%%%%%%%%%%%%%%%%%%%%%%%%%%%%%%%%%%%%%
\subsection{tvsl\_yield (Listings from Section \ref{subsec:tvsl_yield})} \label{appsubsec:tvsl_yield}
%%%%%%%%%%%%%%%%%%%%%%%%%%%%%%%%%%%%%%%%%%%%%%

\begin{minipage}{\linewidth}   \hrulefill
\begin{python}[numbers=left]
@Trace
def tvsl_yield(sets: List[List[int]], partial_transversal: Tuple = ()) -> Generator[Tuple, None, None]:
  if not sets:
    yield partial_transversal
  else:
    for element in sets[0]:
      if element not in partial_transversal:
        yield from tvsl_yield(sets[1:], partial_transversal + (element, ))
\end{python}
\begin{lstlisting} [caption={transversal\_dfs\_yield},  label={lis:dfsyield}]
\end{lstlisting}
\end{minipage}

\noindent
\begin{minipage}{\linewidth} \largev
\begin{python}
sets: [[1, 2, 3], [2, 4], [1]]
  sets: [[2, 4], [1]], partial_transversal: (1,)
    sets: [[1]], partial_transversal: (1, 2)
    sets: [[1]], partial_transversal: (1, 4)
  sets: [[2, 4], [1]], partial_transversal: (2,)
    sets: [[1]], partial_transversal: (2, 4)
      sets: [], partial_transversal: (2, 4, 1) <=
Transversal: (2, 4, 1)
  sets: [[2, 4], [1]], partial_transversal: (3,)
    sets: [[1]], partial_transversal: (3, 2)
      sets: [], partial_transversal: (3, 2, 1) <=
Transversal: (3, 2, 1)
    sets: [[1]], partial_transversal: (3, 4)
      sets: [], partial_transversal: (3, 4, 1) <=
Transversal: (3, 4, 1)
\end{python}
\begin{lstlisting} [caption={tvrsl\_yield trace},  label={lis:tvrsl_yield_trace}]
\end{lstlisting}
\end{minipage}

%%%%%%%%%%%%%%%%%%%%%%%%%%%%%%%%%%%%%%%%%%%%%%
\subsection{tvsl\_yield\_lv (Listings from Section \ref{subsec:tvsl_yield_lv})} \label{appsubsec:tvsl_yield_lv}
%%%%%%%%%%%%%%%%%%%%%%%%%%%%%%%%%%%%%%%%%%%%%%

\begin{minipage}{\linewidth}   \hrulefill
\begin{python}[numbers=left]
@Trace
def tvsl_yield_lv(Sets: List[PyList], Partial_Transversal: PyTuple, Complete_Tvsl: Var):
  if not Sets:
    yield from unify(Partial_Transversal, Complete_Tvsl)
  else:
    Element = Var()
    for _ in member(Element, Sets[0]):
      for _ in fails(member)(Element, Partial_Transversal):
        yield from tvsl_yield_lv(Sets[1:], Partial_Transversal + PyList([Element]), Complete_Tvsl)
\end{python}
\begin{lstlisting} [caption={tvsl\_yield\_lv},  label={lis:yieldlv}]
\end{lstlisting}
\end{minipage}

\noindent
\begin{minipage}{\linewidth} \largev 
\begin{python}
Sets: [[1, 2, 3], [2, 4], [1]], Partial_Transversal: (), Complete_Transversal: _10
  Sets: [[2, 4], [1]], Partial_Transversal: (1, ), Complete_Transversal: _10
    Sets: [[1]], Partial_Transversal: (1, 2), Complete_Transversal: _10
    Sets: [[1]], Partial_Transversal: (1, 4), Complete_Transversal: _10
  Sets: [[2, 4], [1]], Partial_Transversal: (2, ), Complete_Transversal: _10
    Sets: [[1]], Partial_Transversal: (2, 4), Complete_Transversal: _10
      Sets: [], Partial_Transversal: (2, 4, 1), Complete_Transversal: _10 <=
Transversal: (2, 4, 1)

  Sets: [[2, 4], [1]], Partial_Transversal: (3, ), Complete_Transversal: _10
    Sets: [[1]], Partial_Transversal: (3, 2), Complete_Transversal: _10
      Sets: [], Partial_Transversal: (3, 2, 1), Complete_Transversal: _10 <=
Transversal: (3, 2, 1)

    Sets: [[1]], Partial_Transversal: (3, 4), Complete_Transversal: _10
      Sets: [], Partial_Transversal: (3, 4, 1), Complete_Transversal: _10 <=
Transversal: (3, 4, 1)
\end{python}
\begin{lstlisting} [caption={Trace of tvsl\_yield\_lv},  label={lis:tvsl_yield_lv_output}]
\end{lstlisting}
\end{minipage}


%%%%%%%%%%%%%%%%%%%%%%%%%%%%%%%%%%%%%%%%%%%%%%
\subsection{tvsl\_prolog (Listings from Section \ref{subsec:tvsl_prolog})} \label{appsubsec:tvsl_prolog}
%%%%%%%%%%%%%%%%%%%%%%%%%%%%%%%%%%%%%%%%%%%%%%

\begin{minipage}{\linewidth} \largev
\begin{python}[numbers=left]
tvsl_prolog(Sets, Partial_Transversal, _Complete_Transversal) :-
    writeln('Sets': Sets;'  Partial_Transversal': Partial_Transversal), 
    fail.

tvsl_prolog([], Complete_Transversal, Complete_Transversal) :-
    format(' => '),
    writeln(Complete_Transversal).

tvsl_prolog([S|Ss], Partial_Transversal, Complete_Transversal_X) :-
    member(X, S),
    \+ member(X, Partial_Transversal),
    append(Partial_Transversal, [X], Partial_Transversal_X),
    tvsl_prolog(Ss, Partial_Transversal_X, Complete_Transversal_X).

\end{python}
\begin{lstlisting} [caption={transversal\_prolog},  label={lis:transversalprolog1}]
\end{lstlisting}
\end{minipage}


\noindent
\begin{minipage}{\linewidth} \largev 
\begin{python}
?- tvsl_prolog([[1, 2, 3], [2, 4], [1]], [], Complete_Transversal).

Sets:[[1, 2, 3], [2, 4], [1]]; Partial_Transversal:[]
Sets:[[2, 4], [1]]; Partial_Transversal:[1]
Sets:[[1]]; Partial_Transversal:[1, 2]
Sets:[[1]]; Partial_Transversal:[1, 4]
Sets:[[2, 4], [1]]; Partial_Transversal:[2]
Sets:[[1]]; Partial_Transversal:[2, 4]
Sets:[]; Partial_Transversal:[2, 4, 1]
 => [2, 4, 1]
 Complete_Transversal = [2, 4, 1]

Sets:[[2, 4], [1]]; Partial_Transversal:[3]
Sets:[[1]]; Partial_Transversal:[3, 2]
Sets:[]; Partial_Transversal:[3, 2, 1]
 => [3, 2, 1]
 Complete_Transversal = [3, 2, 1]

Sets:[[1]]; Partial_Transversal:[3, 4]
Sets:[]; Partial_Transversal:[3, 4, 1]
 => [3, 4, 1]
 Complete_Transversal = [3, 4, 1]
\end{python}
\begin{lstlisting} [caption={transversal\_prolog trace},  label={lis:transversal_prolog_trace}]
\end{lstlisting}
\end{minipage}


%%%%%%%%%%%%%%%%%%%%%%%%%%%%%%%%%%%%%%%%%%%%%%
\section{Control Functions (Listings From Section \ref{sec:control_functions})} \label{appsec:control_functions}
%%%%%%%%%%%%%%%%%%%%%%%%%%%%%%%%%%%%%%%%%%%%%%

%%%%%%%%%%%%%%%%%%%%%%%%%%%%%%%%%%%%%%%%%%%%%%
\subsection{Control flow in Prolog (Listings from Section \ref{subsec:control_flow_prolog})} \label{appsubsec:control_flow_prolog}
%%%%%%%%%%%%%%%%%%%%%%%%%%%%%%%%%%%%%%%%%%%%%%

\begin{minipage}{\linewidth} \hrulefill
\begin{python}[numbers=left]
solve([]).
solve([Term|Terms]):-
  clause(Term, Body), 
  append(Body, Terms, New_Terms), 
  solve(New_Terms).
\end{python}
\begin{lstlisting} [caption={A prolog interpreter in prolog},  label={lis:prologInterpreter}]
\end{lstlisting}
\end{minipage}


%%%%%%%%%%%%%%%%%%%%%%%%%%%%%%%%%%%%%%%%%%%%%%
\subsection{Prolog control flow in Pylog (Listings from Section \ref{subsec:control_flow_pylog})} \label{appsubsec:control_flow_pylog}
%%%%%%%%%%%%%%%%%%%%%%%%%%%%%%%%%%%%%%%%%%%%%%

\begin{minipage}{\linewidth}   \hrulefill
\begin{python}[numbers=left]
    for element in sets[0]:
      if element not in partial_transversal:
        complete_transversal = tvsl_dfs_first(sets[1:], partial_transversal + (element, ))
        if complete_transversal is not None:
          return complete_transversal 
    return None
\end{python}
\begin{lstlisting} [caption={The \textbf{else} branch of \textittt{tvsl\_dfs\_first}}, label={lis:dfsfirstelse}]
\end{lstlisting}
\end{minipage}



\noindent
\begin{minipage}{\linewidth} \largev  
\begin{python}[numbers=left]
    for element in sets[0]:
      if element not in partial_transversal:
        yield from tvsl_yield(sets[1:], partial_transversal + (element, ))
\end{python}
\begin{lstlisting} [caption={The \textbf{else} branch of \textittt{tvsl\_yield}}, label={lis:yieldelse}]
\end{lstlisting}
\end{minipage}

%%%%%%%%%%%%%%%%%%%%%%%%%%%%%%%%%%%%%%%%%%%%%%
\subsection{A review of Python generators (Listings from Section \ref{subsec:generators})} \label{appsubsec:generators}
%%%%%%%%%%%%%%%%%%%%%%%%%%%%%%%%%%%%%%%%%%%%%%


\begin{minipage}{\linewidth}  \hrulefill  
\begin{python}[numbers=left]
def find_number(search_number):
    i = 0
    while True:
        i += 1
        if i == search_number:
            print("\nFound the number:", search_number)
            return
        else:
            yield i

search_number = 5
find_number_object = find_number(search_number)
while True:
    k = next(find_number_object)
    print(f'{k} is not {search_number}')
\end{python}
\begin{lstlisting} [caption={\textittt{Generator example}},  label={lis:generatorExample1}]
\end{lstlisting}
\end{minipage}


\noindent
\begin{minipage}{\linewidth}  \largev   
\begin{verbatim}
1 is not 5
2 is not 5
3 is not 5
4 is not 5

Found the number: 5

Traceback (most recent call last):
  <line number where error occurred> 
    k = next(find_number_object)
StopIteration

Process finished with exit code 1
\end{verbatim}
\begin{lstlisting} [caption={\textittt{Generator example output}},  label={lis:generatorExample2}]
\end{lstlisting}
\end{minipage}

\noindent
\begin{minipage}{\linewidth}  \largev   
\begin{python}
def use_yield_from():
    yield from find_number_object
    print('find_number failed, but "yield from" caught the Stop Iteration exception.')
    return

for k in use_yield_from():
    print(f'{k} is not 5')
\end{python}
\begin{lstlisting} [caption={\textittt{yield from example}},  label={lis:yieldfromExample}]
\end{lstlisting}
\end{minipage}

\noindent
\begin{minipage}{\linewidth}  \largev   
\begin{verbatim}
1 is not 5
2 is not 5
3 is not 5
4 is not 5
Found the number: 5
find_number failed, but "yield from" caught the Stop Iteration exception.

Process finished with exit code 0
\end{verbatim}
\begin{lstlisting} [caption={\textittt{yield from example output}},  label={lis:yieldFromExampleOutput}]
\end{lstlisting}
\end{minipage}

%%%%%%%%%%%%%%%%%%%%%%%%%%%%%%%%%%%%%%%%%%%%%%
\subsection{\textbf{yield} : \textit{succeed} :: \textit{return} : \textit{fail} (Listings from Section \ref{subsec:yield_succeed})} \label{appsubsec:yield_succeed}
%%%%%%%%%%%%%%%%%%%%%%%%%%%%%%%%%%%%%%%%%%%%%%

\begin{minipage}{\linewidth}   \hrulefill  
\begin{python}
head :- body_1.
head :- body_2.
\end{python}
\begin{lstlisting} [caption={Prolog multiple clauses},  label={lis:prologmultipleclauses}]
\end{lstlisting}
\end{minipage}

\noindent
\begin{minipage}{\linewidth}  \largev   
\begin{python}
def head():
    <some code>
    yield
    
    <other code>
    yield
\end{python}
\begin{lstlisting} [caption={Pylog multiple sequential yields},  label={lis:pylogmultipleyields}]
\end{lstlisting}
\end{minipage}

\noindent
\begin{minipage}{\linewidth}  \largev   
\begin{python}
head :- !, body_1.
head :- body_2.
\end{python}
\begin{lstlisting} [caption={Prolog multiple clauses with a cut},  label={lis:prologmultipleclauseswithcut}]
\end{lstlisting}
\end{minipage}

\noindent
\begin{minipage}{\linewidth}  \largev   
\begin{python}
def head():
    if <condition>:
      <some code>
      yield
    else
      <other code>
      yield
\end{python}
\begin{lstlisting} [caption={Multiple Pylog \textbf{yield}s in separate \textbf{if}-\textbf{else} arms},  label={lis:pylogmultipleclauseswithifelse}]
\end{lstlisting}
\end{minipage}

%%%%%%%%%%%%%%%%%%%%%%%%%%%%%%%%%%%%%%%%%%%%%%
\subsection{Control functions (Listings from Section \ref{subsec:controlfunctions})} \label{appsubsec:controlfunctions}
%%%%%%%%%%%%%%%%%%%%%%%%%%%%%%%%%%%%%%%%%%%%%%

\begin{minipage}{\linewidth}  \largev 
\begin{python}[numbers=left]
def fails(f):
  """
  Applied to a function so that the resulting function succeeds if and only if the original fails.
  Note that fails is applied to the function itself, not to a function call.
  Similar to a decorator but applied explicitly when used.
  """
  def fails_wrapper(*args, **kwargs):
    for _ in f(*args, **kwargs):
      # Fail, i.e., don't yield, if f succeeds
      return  
    # Succeed if f fails.
    yield     

  return fails_wrapper
\end{python}
\begin{lstlisting} [caption={fails},  label={lis:fails}]
\end{lstlisting}
\end{minipage}

\noindent
\begin{minipage}{\linewidth}  \largev 
\begin{python}[numbers=left]
def forall(gens):
  """
  Succeeds if all generators in the gens list succeed. The elements in the gens list
  are embedded in lambda functions to avoid premature evaluation.
  """
  if not gens:
    # They have all succeeded.
    yield
  else:
    # Get gens[0] and evaluate the lambda expression to get a fresh iterator.
    # The parentheses after gens[0] evaluates the lambda expression.
    # If it succeeds, run the rest of the generators in the list.
    for _ in gens[0]( ):
      yield from forall(gens[1:])
\end{python}
\begin{lstlisting} [caption={forall},  label={lis:forall}]
\end{lstlisting}
\end{minipage}


\noindent
\begin{minipage}{\linewidth}  \largev 
\begin{python}[numbers=left]
def forany(gens):
  """
  Succeeds if any of the generators in the gens list succeed. On backtracking, tries them all. 
  The gens elements must be embedded in lambda functions.
  """
  for gen in gens:
    yield from gen( )

\end{python}
\begin{lstlisting} [caption={forany},  label={lis:forany}]
\end{lstlisting}
\end{minipage}

\noindent
\begin{minipage}{\linewidth}  \largev 
\begin{python}[numbers=left]
def trace(x, succeed=True, show_trace=True):
  """
  Can be included in a list of generators (as in forall and forany) to see where we are.
  The second argument determines whether trace succeeds or fails. The third turns printing on or off.
  When included in a list of forall generators, succeed should be set to True so that
  it doesn't prevent forall from succeeding.
  When included in a list of forany generators, succeed should be set to False so that forany
  will go on the the next generator and won't take trace as an extraneous successes.
  """
  if show_trace:
    print(x)
  if succeed:
    yield

\end{python}
\begin{lstlisting} [caption={trace},  label={lis:trace}]
\end{lstlisting}
\end{minipage}

\noindent
\begin{minipage}{\linewidth}  \largev 
\begin{python}[numbers=left]
def would_succeed(f):
  """
  Applied to a function so that the resulting function succeeds/fails if and only if the original
  function succeeds/fails. If the original function succeeds, this also succeeds but without 
  binding any variables. Similar to a decorator but applied explicitly when used.
  """
  def would_succeed_wrapper(*args, **kwargs):
    succeeded = False
    for _ in f(*args, **kwargs):
      succeeded = True
      # Do not yield in the context of f succeeding.
      
    # Exit the for-loop so that unification will be undone.
    if succeeded:
      # Succeed if f succeeded.
      yield  
    # The else clause is redundant. It is included here for clarity.
    # else:
    #   Fail if f failed.
    #   pass   

  return would_succeed_wrapper

\end{python}
\begin{lstlisting} [caption={would\_succeed},  label={lis:wouldsucceed}]
\end{lstlisting}
\end{minipage}

\noindent
\begin{minipage}{\linewidth}  \largev 
\begin{python}[numbers=left]
def bool_yield_wrapper(gen):
  """
  A decorator. Produces a function that generates a Bool_Yield_Wrapper object. 
  """
  def wrapped_func(*args, **kwargs):
    return Bool_Yield_Wrapper(gen(*args, **kwargs))

  return wrapped_func
\end{python}
\begin{lstlisting} [caption={bool\_yield\_wrapper},  label={lis:boolYieldWrapper}]
\end{lstlisting}
\end{minipage}


\noindent
\begin{minipage}{\linewidth}  \largev 
\begin{python}[numbers=left]
  @bool_yield_wrapper
  def squares(n: int, X2: Var) -> Bool_Yield_Wrapper:
    for i in range(n):
      unify_gen = bool_yield_wrapper(unify)(X2, i**2)
      while unify_gen.has_more():
        yield

  Square = Var()
  squares_gen = squares(5, Square)
  while squares_gen.has_more():
    print(Square)
\end{python}
\begin{lstlisting} [caption={bool\_yield\_wrapper example},  label={lis:boolYieldWrapperExample}]
\end{lstlisting}
\end{minipage}


%%%%%%%%%%%%%%%%%%%%%%%%%%%%%%%%%%%%%%%%%%%%%%
\section{Logic variables (Listings from Section \ref{sec:logic_variables})} \label{appsec:logic_variables} 
%%%%%%%%%%%%%%%%%%%%%%%%%%%%%%%%%%%%%%%%%%%%%%

%%%%%%%%%%%%%%%%%%%%%%%%%%%%%%%%%%%%%%%%%%%%%%
\subsection{PyValue (Listings from Section \ref{subsec:pyvalue})} \label{appsubsec:pyvalue}
%%%%%%%%%%%%%%%%%%%%%%%%%%%%%%%%%%%%%%%%%%%%%%
No listings from this section.

%%%%%%%%%%%%%%%%%%%%%%%%%%%%%%%%%%%%%%%%%%%%%%
\subsection{Var (Listings from Section \ref{subsec:var})} \label{appsubsec:var}
%%%%%%%%%%%%%%%%%%%%%%%%%%%%%%%%%%%%%%%%%%%%%%

\begin{minipage}{\linewidth}   \hrulefill
\begin{python}[numbers=left]
def print_ABCDE(A, B, C, D, E):
    print(f'A: {A}, B: {B}, C: {C}, D: {D}, E: {E}')

(A, B, C, D, E) = (Var(), Var(), Var(), Var(), 'abc')
print_ABCDE(A, B, C, D, E) 
for _ in unify(A, B):
  print_ABCDE(A, B, C, D, E) 
  for _ in unify(D, C):
    print_ABCDE(A, B, C, D, E) 
    for _ in unify(A, C):
      print_ABCDE(A, B, C, D, E) 
      for _ in unify(E, D):
        print_ABCDE(A, B, C, D, E) 
      print_ABCDE(A, B, C, D, E) 
    print_ABCDE(A, B, C, D, E) 
  print_ABCDE(A, B, C, D, E) 
print_ABCDE(A, B, C, D, E) 
\end{python}
\begin{lstlisting} [caption={Unifying logic variables},  label={lis:unifylogicvars1}]
\end{lstlisting}
\end{minipage}

\noindent
\begin{minipage}{\linewidth} \largev  
\begin{python}[numbers=left]
A: _195, B: _196, C: _197, D: _198, E: abc
A: _196, B: _196, C: _197, D: _198, E: abc
A: _196, B: _196, C: _197, D: _197, E: abc
A: _197, B: _197, C: _197, D: _197, E: abc
A: abc, B: abc, C: abc, D: abc, E: abc
A: _197, B: _197, C: _197, D: _197, E: abc
A: _196, B: _196, C: _197, D: _197, E: abc
A: _196, B: _196, C: _197, D: _198, E: abc
A: _195, B: _196, C: _197, D: _198, E: abc
\end{python}
\begin{lstlisting} [caption={Unifying logic variables},  label={lis:unifylogicvars2}]
\end{lstlisting}
\end{minipage}

\noindent
\begin{minipage}{\linewidth} \largev  
\begin{python}
(A, B, C, D, E) = (*n_Vars(4), 'abc')
for _ in unify_pairs([(A, B), (D, C), (A, C), (E, D)]):
\end{python}
\begin{lstlisting} [caption={Unifying logic variables shortened},  label={lis:unifylogicvarsshortened}]
\end{lstlisting}
\end{minipage}

%%%%%%%%%%%%%%%%%%%%%%%%%%%%%%%%%%%%%%%%%%%%%%
\subsection{Structure (Listings from Section \ref{subsec:structure})} \label{appsubsec:structure}
%%%%%%%%%%%%%%%%%%%%%%%%%%%%%%%%%%%%%%%%%%%%%%
No listings from this section.

%%%%%%%%%%%%%%%%%%%%%%%%%%%%%%%%%%%%%%%%%%%%%%
\subsection{Lists (Listings from Section \ref{subsec:lists})} \label{appsubsec:lists}
%%%%%%%%%%%%%%%%%%%%%%%%%%%%%%%%%%%%%%%%%%%%%%
No listings from this section.

%%%%%%%%%%%%%%%%%%%%%%%%%%%%%%%%%%%%%%%%%%%%%%
\subsection{\textit{append} (Listings from Section \ref{subsec:append})} \label{appsubsec:append}
%%%%%%%%%%%%%%%%%%%%%%%%%%%%%%%%%%%%%%%%%%%%%%

\begin{minipage}{\linewidth} \hrulefill
\begin{python}
(Xs, Ys, Zs) = (Var(), Var(), LinkedList([1, 2, 3]))
for _ in append(Xs, Ys, Zs):
  print(f'Xs = {Xs}\nYs = {Ys}\n')
\end{python}
\begin{lstlisting} [caption={append},  label={lis:append}]
\end{lstlisting}
\end{minipage}

\noindent
\begin{minipage}{\linewidth}  \largev 
\begin{python}
Xs = []
Ys = [1, 2, 3]

Xs = [1]
Ys = [2, 3]

Xs = [1, 2]
Ys = [3]

Xs = [1, 2, 3]
Ys = []
\end{python}
\begin{lstlisting} [caption={append output},  label={lis:append_output}]
\end{lstlisting}
\end{minipage}
\smallv

\noindent
\begin{minipage}{\linewidth}  \largev 
\begin{python}
append([], Ys, Ys).
append([XZ|Xs], Ys, [XZ|Zs]) :- append(Xs, Ys, Zs).
\end{python}
\begin{lstlisting} [caption={prolog append code},  label={lis:prolog_append_code}]
\end{lstlisting}
\end{minipage}

% Now the wordier but isomorphic Pylog version.


\noindent
\begin{minipage}{\linewidth}  \largev 
\begin{python}[numbers=left]
# For a cleaner presentation, declarations are dropped. All variables are Union[LinkedList, Var].
def append(Xs, Ys, Zs):

  # Corresponds to: append([], Ys, Ys).
  yield from unify_pairs([(Xs, LinkedList([])), (Ys, Zs)])

  # Corresponds to: append([XZ|Xs], Ys, [XZ|Zs]) :- append(Xs, Ys, Zs).
  (XZ_Head, Xs_Tail, Zs_Tail) = n_Vars(3)
  for _ in unify_pairs([(Xs, LinkedList(XZ_Head, Xs_Tail)),
                       (Zs, LinkedList(XZ_Head, Zs_Tail))]):
    yield from append(Xs_Tail, Ys, Zs_Tail)

\end{python}
\begin{lstlisting} [caption={Pylog append code},  label={lis:append_code}]
\end{lstlisting}
\end{minipage}

%%%%%%%%%%%%%%%%%%%%%%%%%%%%%%%%%%%%%%%%%%%%%%
\subsection{Unification (Listings from Section \ref{subsec:unify})} \label{appsubsec:unify}
%%%%%%%%%%%%%%%%%%%%%%%%%%%%%%%%%%%%%%%%%%%%%%

\begin{minipage}{\linewidth} \hrulefill
\begin{python}[numbers=left]
@euc
def unify(Left: Any, Right: Any):

  (Left, Right) = map(ensure_is_logic_variable, (Left, Right))

  # Case 1.
  if Left == Right:
    yield

  # Case 2.
  elif isinstance(Left, PyValue) and isinstance(Right, PyValue) and \
       (not Left.is_instantiated( ) or not Right.is_instantiated( )) and \
       (Left.is_instantiated( ) or Right.is_instantiated( )):
    (assignedTo, assignedFrom) = (Left, Right) if Right.is_instantiated( ) else (Right, Left)
    assignedTo._set_py_value(assignedFrom.get_py_value( ))
    yield

    assignedTo._set_py_value(None)

  # Case 3.
  elif isinstance(Left, Structure) and isinstance(Right, Structure) and Left.functor == Right.functor:
    yield from unify_sequences(Left.args, Right.args)

  # Case 4.
  elif isinstance(Left, Var) or isinstance(Right, Var):
    (pointsFrom, pointsTo) = (Left, Right) if isinstance(Left, Var) else (Right, Left)
    pointsFrom.unification_chain_next = pointsTo
    yield

    pointsFrom.unification_chain_next = None

\end{python}
\begin{lstlisting} [caption={unify},  label={lis:unify}]
\end{lstlisting}
\end{minipage}

%%%%%%%%%%%%%%%%%%%%%%%%%%%%%%%%%%%%%%%%%%%%%%
\subsection{Back to tvsl\_yield\_lv (Listings from Section \ref{subsec:more_tvsl_yield_lv})} \label{appsubsec:more_tvsl_yield_lv}
%%%%%%%%%%%%%%%%%%%%%%%%%%%%%%%%%%%%%%%%%%%%%%

No listing from this section.

%%%%%%%%%%%%%%%%%%%%%%%%%%%%%%%%%%%%%%%%%%%%%%
\section{The Zebra Puzzle (Listings From Section \ref{sec:zebra})} \label{appsec:zebra}
%%%%%%%%%%%%%%%%%%%%%%%%%%%%%%%%%%%%%%%%%%%%%%

%%%%%%%%%%%%%%%%%%%%%%%%%%%%%%%%%%%%%%%%%%%%%%
\subsection{The clues and a Prolog solution (Listings from Section \ref{subsec:clues})} \label{appsubsec:clues}
%%%%%%%%%%%%%%%%%%%%%%%%%%%%%%%%%%%%%%%%%%%%%%

\begin{minipage}{\linewidth}
\hrulefill
\begin{python}
zebra_problem(Houses) :-
    Houses = [house(_, _, _, _, _), house(_, _, _, _, _), house(_, _, _, _, _), 
              house(_, _, _, _, _), house(_, _, _, _, _)], 

    % 1. The English live in the red house.
    member(house(english, _, _, _, red), Houses), 

    % 2. The Spanish have a dog.
    member(house(spanish, _, dog, _, _), Houses), 

    % 3. They drink coffee in the green house.
    member(house(_, _, _, coffee, green), Houses), 

    % 4. The Ukrainians drink tea.
    member(house(ukranians, _, _, tea, _), Houses), 

    % 5. The green house is immediately to the right of the white house.
    nextto(house(_, _, _, _, white), house(_, _, _, _, green), Houses), 

    % 6. The Old Gold smokers have snails.
    member(house(_, old_gold, snails, _, _), Houses), 

    % 7. They smoke Kool in the yellow house.
    member(house(_, kool, _, _, yellow), Houses), 

    % 8. They drink milk in the middle house.
    Houses = [_, _, house(_, _, _, milk, _), _, _], 

    % 9. The Norwegians live in the first house on the left.
    Houses = [house(norwegians, _, _, _, _) | _], 

    % 10. The Chesterfield smokers live next to the fox.
    next_to(house(_, chesterfield, _, _, _), house(_, _, fox, _, _), Houses), 

    % 11. They smoke Kool in the house next to the horse.
    next_to(house(_, kool, _, _, _), house(_, _, horse, _, _), Houses), 

    % 12. The Lucky smokers drink juice.
    member(house(_, lucky, _, juice, _), Houses), 

    % 13. The Japanese smoke Parliament.
    member(house(japanese, parliament, _, _, _), Houses), 

    % 14. The Norwegians live next to the blue house.
    next_to(house(norwegians, _, _, _, _), house(_, _, _, _, blue), Houses), 
\end{python}
\begin{lstlisting} [caption={Zebra puzzle in Prolog},  label={lis:zebra_prolog}]
\end{lstlisting}
\end{minipage}

\noindent
\begin{minipage}{\linewidth}
\begin{python}
?- zebra_problem(Houses).
[    
    house(norwegians, kool, fox, water, yellow), 
    house(ukranians, chesterfield, horse, tea, blue), 
    house(english, old_gold, snails, milk, red), 
    house(spanish, lucky, dog, juice, white), 
    house(japanese, parliament, zebra, coffee, green)     
]
\end{python}
\begin{lstlisting} [caption={Zebra puzzle in Prolog},  label={lis:zebra_solution}]
\end{lstlisting}
\end{minipage}

%%%%%%%%%%%%%%%%%%%%%%%%%%%%%%%%%%%%%%%%%%%%%%
\subsection{A Pylog solution (Listings from Section \ref{subsec:pylog_solution})} \label{appsubsec:pylog_solution}
%%%%%%%%%%%%%%%%%%%%%%%%%%%%%%%%%%%%%%%%%%%%%%

\begin{minipage}{\linewidth}
\hrulefill
\begin{python}
  def clue_1(self, Houses: SuperSequence):
    """ 1. The English live in the red house.  """
    yield from member(House(nationality='English', color='red'), Houses)

  ...

  def clue_8(self, Houses: SuperSequence):
    """ 8. They drink milk in the middle house. """
    yield from unify(House(drink='milk'), Houses[2])

  ...
\end{python}
\begin{lstlisting} [caption={Clues as Pylog functions},  label={lis:clues_as_pylog_functions}]
\end{lstlisting}
\end{minipage}

\noindent
\begin{minipage}{\linewidth} \largev
\begin{python}
After 1392 rule applications, 
	1. Norwegians(Kool, fox, water, yellow)
	2. Ukrainians(Chesterfield, horse, tea, blue)
	3. English(Old Gold, snails, milk, red)
	4. Spanish(Lucky, dog, juice, white)
	5. Japanese(Parliament, zebra, coffee, green)
The Japanese own a zebra, and the Norwegians drink water.
\end{python}
\begin{lstlisting} [caption={Pylog solution},  label={lis:pylog_solution}]
\end{lstlisting}
\end{minipage}

\noindent
\begin{minipage}{\linewidth} \largev
\begin{python}
def zebra_problem(Houses) :-
    for _ in forall{[
        # 1. The English live in the red house.
        lambda: member(house(english, _, _, _, red), Houses), 
        # 2. The Spanish have a dog.
        lambda: member(house(spanish, _, dog, _, _), Houses), 
        # ...
        ]}
\end{python}
\begin{lstlisting} [caption={Pylog solution},  label={lis:zebra_forall}]
\end{lstlisting}
\end{minipage}

\noindent
\begin{minipage}{\linewidth} \largev
\begin{python}
def run_all_clues(World_List: List[Term], clues: List[Callable]):
    if not clues:
      # Ran all the clues. Succeed.
      yield
    else:
      # Run the current clue and then the rest of the clues.
      for _ in clues[0](World_List):
        yield from run_all_clues(World_List, clues[1:])
\end{python}
\begin{lstlisting} [caption={Pylog solution},  label={lis:zebra_ run_all_clues}]
\end{lstlisting}
\end{minipage}

\noindent
\begin{minipage}{\linewidth} \largev
\begin{python}
  def clue_1(Houses: SuperSequence):
    """ 1. The English live in the red house.  """
    for _ in member(House(nationality='English', color='red'), Houses):
      yield from clue_2(Houses)

  def clue_2(Houses: SuperSequence):
    """ 2. The Spanish have a dog. """
    for _ in member(House(nationality='Spanish', pet='dog'), Houses):
      yield from clue_3(Houses)
      
  ...
\end{python}
\begin{lstlisting} [caption={Pylog solution},  label={lis:zebra_rule_chaining}]
\end{lstlisting}
\end{minipage}

%%%%%%%%%%%%%%%%%%%%%%%%%%%%%%%%%%%%%%%%%%%%%%
\section{Conclusion (Listings from Section \ref{sec:conclusion})} \label{appsec:conclusion}
%%%%%%%%%%%%%%%%%%%%%%%%%%%%%%%%%%%%%%%%%%%%%%

\begin{minipage}{\linewidth}
\hrulefill
\begin{python}
   def some_clause(...):
     for _ in <generate options>:
       <local conditions>
       yield from next_clause(...)
\end{python}
\begin{lstlisting} [caption={A Pylog/Prolog template},  label={lis:template}]
\end{lstlisting}
\end{minipage}
