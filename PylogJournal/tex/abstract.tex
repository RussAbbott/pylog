\begin{abstract}
% Pylog inhabits three programming contexts. 
% \begin{enumerate}
  % \begin{itemize}
      % \item 
      Pylog explores the integration of two distinct programming language paradigms: (i) the modern, general purpose programming paradigm which, in its Python implementation, has been broadened to include procedural, object-oriented, and functional programming and (ii) the logic programming paradigm, most notably including logic variables (with unification) and depth-first, backtracking search. 
      
      Pylog illustrates how the core logic programming features can be implemented in and integrated into Python. 
      
      % In addition, a Python template for the standard Prolog control mechanism is presented.
      
      % \item 
      Simultaneously, Pylog demonstrates the breadth and broad applicability of Python, one of the most widely used programming languages. 
      
      % \item 
      Pylog exemplifies programming at its best, using Python features in innovative yet clear ways to integrate features of a non-Python programming paradigm into its range of capabilities. The overall result is software worth reading.
      
      % \item 
      This paper fits into \textit{The Programming Journal's} Art-of-Programming category: general-purpose programming.
      
      The Python code is available at: \textittt{https://github.com/RussAbbott/pylog}.
     % \end{itemize}
    % \end{enumerate}
\end{abstract}