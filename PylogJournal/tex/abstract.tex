\begin{abstract}
Pylog explores the integration of two distinct programming language paradigms: (i) the modern, general purpose programming paradigm which, in its Python implementation, has been broadened to include procedural, object-oriented, and functional programming and (ii) the logic programming paradigm, most notably including logic variables (with unification) and depth-first, backtracking search. 
          
Pylog illustrates how the core logic programming features can be implemented in and integrated into Python. 

Simultaneously, Pylog demonstrates Python's breadth. Python is used in situations ranging introductory programming classes to the development of very sophisticated software. Pylog demonstrates two interestingly distinct uses for Python's \textbf{for}-loop construct.

Pylog exemplifies programming at its best, using Python features in innovative yet clear ways to integrate features of a non-Python programming paradigm into its range of capabilities. The overall result is software worth reading.
          
          The Pylog code is available at \href{https://github.com/RussAbbott/pylog}{\underline{this GitHub repository}}.
\end{abstract}