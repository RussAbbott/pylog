%%%%%%%%%%%%%%%%%%%%%%%%%%%%%%%%%%%%%%%%%%%%%%
\section{The Zebra Puzzle (Listings in Appendix \ref{appsec:zebra})}\label{sec:zebra}
%%%%%%%%%%%%%%%%%%%%%%%%%%%%%%%%%%%%%%%%%%%%%%

The Zebra Puzzle is a well known logic puzzle.

\begin{quotation}
There are five houses in a row. Each has a unique color and is occupied by a family of unique nationality. Each family has a unique favorite smoke, a unique pet, and a unique favorite drink. Fourteen clues (Listing \ref{lis:zebra_prolog}) provide additional constraints. \textit{Who has a zebra and who drinks water?}
\end{quotation}

%%%%%%%%%%%%%%%%%%%%%%%%%%%%%%%%%%%%%%%%%%%%%%
\subsection{The clues and a Prolog solution (Listings in Appendix \ref{appsubsec:clues})} \label{subsec:clues}
%%%%%%%%%%%%%%%%%%%%%%%%%%%%%%%%%%%%%%%%%%%%%%

One can easily write Prolog programs to solve this and similar puzzles.
\begin{itemize}
\item Represent a house as a Prolog \textit{house} term with the parameters corresponding to the indicated properties:
\begin{python}
   house(<nationality>, <cigarette brand>, <pet>, <drink>, <house color>)
\end{python}
\item Define the world as a list of five \textit{house} terms, with all fields initially uninstantiated.

\item Write the clues (Listing \ref{lis:zebra_prolog}) as more-or-less direct translations of the English.
\end{itemize}

% \begin{minipage}{\linewidth}
% \begin{python}
% zebra_problem(Houses) :-
%     Houses = [house(_, _, _, _, _), house(_, _, _, _, _), house(_, _, _, _, _), 
%               house(_, _, _, _, _), house(_, _, _, _, _)], 

%     % 1. The English live in the red house.
%     member(house(english, _, _, _, red), Houses), 

%     % 2. The Spanish have a dog.
%     member(house(spanish, _, dog, _, _), Houses), 

%     % 3. They drink coffee in the green house.
%     member(house(_, _, _, coffee, green), Houses), 

%     % 4. The Ukrainians drink tea.
%     member(house(ukranians, _, _, tea, _), Houses), 

%     % 5. The green house is immediately to the right of the white house.
%     nextto(house(_, _, _, _, white), house(_, _, _, _, green), Houses), 

%     % 6. The Old Gold smokers have snails.
%     member(house(_, old_gold, snails, _, _), Houses), 

%     % 7. They smoke Kool in the yellow house.
%     member(house(_, kool, _, _, yellow), Houses), 

%     % 8. They drink milk in the middle house.
%     Houses = [_, _, house(_, _, _, milk, _), _, _], 

%     % 9. The Norwegians live in the first house on the left.
%     Houses = [house(norwegians, _, _, _, _) | _], 

%     % 10. The Chesterfield smokers live next to the fox.
%     next_to(house(_, chesterfield, _, _, _), house(_, _, fox, _, _), Houses), 

%     % 11. They smoke Kool in the house next to the horse.
%     next_to(house(_, kool, _, _, _), house(_, _, horse, _, _), Houses), 

%     % 12. The Lucky smokers drink juice.
%     member(house(_, lucky, _, juice, _), Houses), 

%     % 13. The Japanese smoke Parliament.
%     member(house(japanese, parliament, _, _, _), Houses), 

%     % 14. The Norwegians live next to the blue house.
%     next_to(house(norwegians, _, _, _, _), house(_, _, _, _, blue), Houses), 
% \end{python}
% \begin{lstlisting} [caption={Zebra puzzle in Prolog},  label={lis:zebra_prolog}]
% \end{lstlisting}
% \end{minipage}

\largev
After the following adjustments, we can run this program online using SWI-Prolog. 
\begin{itemize}
    \item SWI-Prolog includes \textit{member} and \textit{nextto} predicates. SWI-Prolog's \textit{nextto} means in the order given, as in clue 5.

    \item SWI-Prolog does not include a predicate for \textit{next to} in the sense of clues 10, 11, and 14 in which the order is unspecified. But we can write our own, say, \textit{next\_to}.
    
\begin{python}
next_to(A, B, List) :- nextto(A, B, List).
next_to(A, B, List) :- nextto(B, A, List).
\end{python}

    \item Since none of the clues mentions either a zebra or water, we add the following.

\begin{minipage}{\linewidth}
\begin{python}

    % 15. (implicit). 
    member(house(_, _, zebra, _, _), Houses), 
    member(house(_, _, _, water, _), Houses).
\end{python}
\end{minipage}

\end{itemize}

When this program is run, we get an almost instantaneous answer---shown manually formatted in Listing \ref{lis:zebra_solution}. We can conclude that 
\begin{quote} 
\begin{quote} 
\textit{The Japanese have a zebra, and the Norwegians drink water}.
\end{quote} 
\end{quote} 

% \begin{minipage}{\linewidth}
% \begin{python}
% ?- zebra_problem(Houses).
% [    
%     house(norwegians, kool, fox, water, yellow), 
%     house(ukranians, chesterfield, horse, tea, blue), 
%     house(english, old_gold, snails, milk, red), 
%     house(spanish, lucky, dog, juice, white), 
%     house(japanese, parliament, zebra, coffee, green)     
% ]
% \end{python}
% \begin{lstlisting} [caption={Zebra puzzle in Prolog},  label={lis:zebra_solution}]
% \end{lstlisting}
% \end{minipage}

\smallv

% We can conclude that 
% \begin{quote} 
% \begin{quote} 
% \textit{The Japanese have a zebra, and the Norwegians drink water}.
% \end{quote} 
% \end{quote} 

%%%%%%%%%%%%%%%%%%%%%%%%%%%%%%%%%%%%%%%%%%%%%%
\subsection{A Pylog solution (Listings in Appendix \ref{appsubsec:pylog_solution})} \label{subsec:pylog_solution}
%%%%%%%%%%%%%%%%%%%%%%%%%%%%%%%%%%%%%%%%%%%%%%

To write and run the Zebra problem in Pylog we built the following framework.
\begin{itemize}
    \item \sloppy We created a \textit{House} class as a subclass of \textit{Structure}. Users may select a house property as a pseudo-functor for displaying houses. We selected \textit{nationality}.

    \item Each clue is expressed as a Pylog function. (See Listing \ref{lis:clues_as_pylog_functions}.) 

        % \begin{minipage}{\linewidth}
        % \begin{python}
        %   def clue_1(self, Houses: SuperSequence):
        %     """ 1. The English live in the red house.  """
        %     yield from member(House(nationality='English', color='red'), Houses)

        %   ...
  
        %   def clue_8(self, Houses: SuperSequence):
        %     """ 8. They drink milk in the middle house. """
        %     yield from unify(House(drink='milk'), Houses[2])

        %   ...
        % \end{python}
        % \begin{lstlisting} [caption={Clues as Pylog functions},  label={lis:clues_as_pylog_functions}]
        % \end{lstlisting}
        % \end{minipage}

    \item The \textit{Houses} list may be any form of \textit{SuperSequence}.
    
    \item We added some simple constraint checking.
\end{itemize}
When run, the answer is the same as in the Prolog version. (See listing \ref{lis:pylog_solution}.)

% \begin{minipage}{\linewidth}
% \begin{python}
% After 1392 rule applications, 
% 	1. Norwegians(Kool, fox, water, yellow)
% 	2. Ukrainians(Chesterfield, horse, tea, blue)
% 	3. English(Old Gold, snails, milk, red)
% 	4. Spanish(Lucky, dog, juice, white)
% 	5. Japanese(Parliament, zebra, coffee, green)
% The Japanese own a zebra, and the Norwegians drink water.
% \end{python}
% \begin{lstlisting} [caption={Pylog solution},  label={lis:pylog_solution}]
% \end{lstlisting}
% \end{minipage}

\smallv
Let's compare the underlying Prolog and Pylog mechanisms. 
\smallv

\textbf{Prolog}. It's trivial to write a Prolog interpreter in Prolog. See Listing \ref{lis:prologInterpreter} \cite{Bartak1998}.

\smallv
\textbf{Pylog}. We developed \textit{three} Pylog approaches to rule interpretation. 
\begin{enumerate}

\item \textit{forall}. Use the \textit{forall} construct as in Listing \ref{lis:zebra_forall}.

% \begin{minipage}{\linewidth}
% \begin{python}
% def zebra_problem(Houses) :-
%     for _ in forall{[
%         # 1. The English live in the red house.
%         lambda: member(house(english, _, _, _, red), Houses), 
%         # 2. The Spanish have a dog.
%         lambda: member(house(spanish, _, dog, _, _), Houses), 
%         # ...
%         ]}
% \end{python}
% \begin{lstlisting} [caption={Pylog solution},  label={lis:zebra_forall}]
% \end{lstlisting}
% \end{minipage}

\textit{forall} succeeds if and only if all members of the list it is passed succeed. Each list element is protected within a \textbf{lambda} construct to prevent evaluation.

\item \textit{run\_all\_rules}. We developed a Python function that accepts a list, e.g., of houses, reflecting the state of the world, along with a list of functions. It succeeds if and only if the functions all succeed. Listing \ref{lis:zebra_ run_all_clues} is a somewhat simplified version.

% \begin{minipage}{\linewidth}
% \begin{python}
% def run_all_clues(World_List: List[Term], clues: List[Callable]):
%     if not clues:
%       # Ran all the clues. Succeed.
%       yield
%     else:
%       # Run the current clue and then the rest of the clues.
%       for _ in clues[0](World_List):
%         yield from run_all_clues(World_List, clues[1:])
% \end{python}
% \begin{lstlisting} [caption={Pylog solution},  label={lis:zebra_ run_all_clues}]
% \end{lstlisting}
% \end{minipage}

\item \textit{Embed rule chaining in the rules.} For example, see Listing \ref{lis:zebra_rule_chaining}.

% \begin{minipage}{\linewidth}
% \begin{python}
%   def clue_1(Houses: SuperSequence):
%     """ 1. The English live in the red house.  """
%     for _ in member(House(nationality='English', color='red'), Houses):
%       yield from clue_2(Houses)

%   def clue_2(Houses: SuperSequence):
%     """ 2. The Spanish have a dog. """
%     for _ in member(House(nationality='Spanish', pet='dog'), Houses):
%       yield from clue_3(Houses)
%   ...
% \end{python}
% \begin{lstlisting} [caption={Pylog solution},  label={lis:zebra_rule_chaining}]
% \end{lstlisting}
% \end{minipage}

% In this organization, unification propagates forward and backward, and backtracking occurs naturally.

Call \textit{clue\_1} with a list of uninstantiated houses, and the problem runs itself.
\end{enumerate}
\smallv 

The three approaches produce the same solution.

