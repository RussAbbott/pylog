
%%%%%%%%%%%%%%%%%%%%%%%%%%%%%%%%%%%%%%%%%%%%%%
\section{Logic variables (Listings in Appendix \ref{appsec:logic_variables})} \label{sec:logic_variables}
%%%%%%%%%%%%%%%%%%%%%%%%%%%%%%%%%%%%%%%%%%%%%%

Figure \ref{fig:class_tree} shows Pylog's primary logic variable classes. This section discusses \textit{PyValue}, \textit{Var}, \textit{Structure}, and the three types of sequences.  (\textit{Term} is an abstract class.)

% \section{Diagram Test}
\begin{figure}
\centering
 \setlength{\unitlength}{0.12cm}
\begin{picture}(75,75)
    \put(29, 70){$\footnotesize{Term}$}
    \put(10, 65){\line(1,0){45}}
    \put(10, 65){\line(0,-1){5}}
    \put(33, 70){\line(0,-1){10}}
    \put(55, 65){\line(0,-1){5}}
    \put(5, 57){$\footnotesize{PyValue}$}
    \put(30, 57){$\footnotesize{Var}$}
    \put(50, 57){$\footnotesize{Structure}$}
    \put(10, 51){\line(0, 1){5}}
    \put(2, 51){\line(1,0){27}}  
    \put(55, 51){\line(0,1){5}}
    \put(2, 47){$\footnotesize{int, float, string, etc.}$}
    \put(48, 48){$\footnotesize{SuperSequence}$}
    \put(55, 42){\line(0,1){5}}
    \put(35, 33){$\footnotesize{LinkedList}$}
    \put(60, 33){$\footnotesize{PySequence}$}
    \put(40, 42){\line(1,0){30}}
    \put(40, 42){\line(0,-1){5}}
    \put(70, 42){\line(0,-1){5}}
    \put(55, 17){$\footnotesize{PyList}$}
    \put(72, 17){$\footnotesize{PyTuple}$}
    \put(70, 31){\line(0,-1){5}}
    \put(60, 25){\line(1,0){20}}
    \put(60, 25){\line(0,-1){5}}
    \put(80, 25){\line(0,-1){5}}
\end{picture}
\sinv\sinv\sinv\sinv\sinv\sinv\sinv\sinv\sinv
\caption{This diagram shows a more complete list of Pylog classes.}
\label{fig:class_tree}
\end{figure}


% \begin{itemize}

%%%%%%%%%%%%%%%%%%%%%%%%%%%%%%%%%%%%%%%%%%%%%%
\subsection{PyValue (Listings in Appendix \ref{appsubsec:pyvalue})} \label{subsec:pyvalue}
%%%%%%%%%%%%%%%%%%%%%%%%%%%%%%%%%%%%%%%%%%%%%%

A \textit{PyValue} provides a bridge between logic variables and Python values. A \textit{PyValue} may hold any immutable Python value, e.g., a number, a string, or a tuple. Tuples are allowed as \textit{PyValue} values only if their components are also immutable. % In the example of the preceding section, the logic variable \textit{E} with value \textit{'abc'} was actually \textit{PyValue('abc')} behind the scenes.
\smallv

%%%%%%%%%%%%%%%%%%%%%%%%%%%%%%%%%%%%%%%%%%%%%%
\subsection{Var (Listings in Appendix \ref{appsubsec:var})} \label{subsec:var}
%%%%%%%%%%%%%%%%%%%%%%%%%%%%%%%%%%%%%%%%%%%%%%

A \textit{Var} functions as a traditional logic variable: it supports unification. 

Unification is surprisingly easy to implement. Each \textit{Var} object includes a \textit{next} field, which is initially \textbf{None}. When two \textit{Var}s are unified, the \textit{next} field of one is set to point to the other. (It makes no difference, which points to which.) A chain of linked  \textit{Var}s unify all the \textit{Var}s in the chain. 

% Consider the example in Listing \ref{lis:unifylogicvars1}.

% \begin{minipage}{\linewidth} \largev   \hrulefill
% \begin{python}[numbers=left]
% def print_ABCDE(A, B, C, D, E):
%     print(f'A: {A}, B: {B}, C: {C}, D: {D}, E: {E}')

% (A, B, C, D, E) = (Var(), Var(), Var(), Var(), 'abc')
% print_ABCDE(A, B, C, D, E) 
% for _ in unify(A, B):
%   print_ABCDE(A, B, C, D, E) 
%   for _ in unify(D, C):
%     print_ABCDE(A, B, C, D, E) 
%     for _ in unify(A, C):
%       print_ABCDE(A, B, C, D, E) 
%       for _ in unify(E, D):
%         print_ABCDE(A, B, C, D, E) 
%       print_ABCDE(A, B, C, D, E) 
%     print_ABCDE(A, B, C, D, E) 
%   print_ABCDE(A, B, C, D, E) 
% print_ABCDE(A, B, C, D, E) 
% \end{python}
% \begin{lstlisting} [caption={Unifying logic variables},  label={lis:unifylogicvars1}]
% \end{lstlisting}
% \end{minipage}

% As we discussed earlier (Section \ref{subsec:forloops}), \textbf{for}-loops can serve as combination choicepoints and scope definitions. We elaborate that discussion here.

Consider Listing \ref{lis:unifylogicvars1}. It's important not to be confused by \textbf{for}-loops. Even though the nested \textbf{for}-loops look like nested iteration, that's not the case. \textit{There is no iteration!} In this example, the \textbf{for}-loops serve solely as choicepoints and scope definitions. 

 Since  \textit{unify} succeeds at most once, each \textbf{for}-loop offers only a single choice. There is never any backtracking. The only function of the \textbf{for}-loops is (a) to call the various \textit{unify} operations and (b) to define the scope over which they hold.  
 
 The output (Listing \ref{lis:unifylogicvars2}) should make this clear. 

% \begin{minipage}{\linewidth} \largev   \hrulefill
% \begin{python}[numbers=left]
% A: _195, B: _196, C: _197, D: _198, E: abc
% A: _196, B: _196, C: _197, D: _198, E: abc
% A: _196, B: _196, C: _197, D: _197, E: abc
% A: _197, B: _197, C: _197, D: _197, E: abc
% A: abc, B: abc, C: abc, D: abc, E: abc
% A: _197, B: _197, C: _197, D: _197, E: abc
% A: _196, B: _196, C: _197, D: _197, E: abc
% A: _196, B: _196, C: _197, D: _198, E: abc
% A: _195, B: _196, C: _197, D: _198, E: abc
% \end{python}
% \begin{lstlisting} [caption={Unifying logic variables},  label={lis:unifylogicvars2}]
% \end{lstlisting}
% \end{minipage}

Numbers with leading underscores indicate uninstantiated logic variables. 

Line 1. All the logic variables are distinct. Each has its own identification number.

Line 2. \textit{A} and \textit{B} have been unified. They have the same identification number.

Line 3. \textit{C} and \textit{D} have also been unified. They have the same identification number, but different from that of \textit{A} and \textit{B}.

Line 4. All the logic variables have been unified---with a single identifier.

Line 5. All the logic variables have \textit{abc} as their value.

Lines 6 - 9. Exit the unification scopes as defined by the \textbf{for}-loops and undo the respective unifications.

We can trace through the unifications diagrammatically. The first two unifications produce the following. (The arrows may be reversed.)
\begin{equation}\label{eq:one}
\begin{split}
A \,\to\, B \\
D \,\to\, C 
\end{split}
\end{equation}
The next unification is \textit{A} with \textit{C}. The first step in unification is to go to the end of the unification chains of the elements to be unified. In this case, \textit{B} (at the end of \textit{A}'s unification chain) is unified with \textit{C}. The result is either of the following.

\begin{equation}\label{eq:two}
\begin{array}{c c c c c c c c }
A & \to & B            & \qquad \qquad \qquad \qquad \qquad &   A & \to  & B   \\
  &     & \downarrow   & \qquad \qquad \qquad \qquad \qquad &     &      & \uparrow \\
D & \to & C            & \qquad \qquad \qquad \qquad \qquad &   D & \to  & C
\end{array}
\end{equation}
Finally, to unify \textit{E} with \textit{D}, we go the the end of \textit{D}'s unification chain---\textit{B} or \textit{C}.
\begin{equation}\label{eq:three}
\begin{array}{c c c c c c c c c c c}
A & \to & B            &     &          & \quad \quad  \: &   A & \to  & B & \to & E('abc')   \\
  &     & \downarrow   &     &          & \quad \quad \: &     &      &\uparrow & &  \\
D & \to & C            & \to & E('abc') & \quad \quad \: &   D & \to  & C & & 
\end{array}
\end{equation}
Different as they appear, these two structures are equivalent for unification purposes.

To determine a \textit{Var}'s value, follow its unification chain. If the end is a \textit{PyValue}, the \textit{PyValue}'s value is the \textit{Var}'s value. In (3), all \textit{Var}s have value \textit{'abc'}. If the end of a unification chain is an uninstantiated \textit{Var} (as in (2) for all \textit{Var}s), the \textit{Var}'s in the tributary chains are mutually unified, but uninstantiated. When the end \textit{Var} gets a value, it will be the value for all \textit{Var}'s leading to it.

The following convenience methods make it possible to write the preceding code more concisely---but without the \textit{print} statements. See Listing \ref{lis:unifylogicvarsshortened}.

\begin{itemize}
    \item \textit{n\_Vars} takes an integer argument and generates that many \textit{Var} objects.
    \item \textit{unify\_pairs} takes a list of pairs (as tuples) and unifies the elements of each pair.
\end{itemize}

% \begin{minipage}{\linewidth} \largev   \hrulefill
% \begin{python}
% (A, B, C, D, E) = (*n_Vars(4), 'abc')
% for _ in unify_pairs([(A, B), (D, C), (A, C), (E, D)]):
% \end{python}
% \begin{lstlisting} [caption={Unifying logic variables shortened},  label={lis:unifylogicvarsshortened}]
% \end{lstlisting}
% \end{minipage}

%%%%%%%%%%%%%%%%%%%%%%%%%%%%%%%%%%%%%%%%%%%%%%
\subsection{Structure (Listings in Appendix \ref{appsubsec:structure})} \label{subsec:structure}
%%%%%%%%%%%%%%%%%%%%%%%%%%%%%%%%%%%%%%%%%%%%%%

The \textit{Structure} class enables the construction of Prolog terms. A \textit{Structure} object consists of a functor along with a tuple of values. The Zebra puzzle (Section \ref{sec:zebra}) uses \textit{Structure}s to build \textit{house} terms. \textit{house} is the functor; the tuple contains the house attributes. 

\centerline{\textit{house(\textless nationality\textgreater,~\textless cigarette\textgreater,~\textless pet\textgreater,~\textless drink\textgreater,~\textless house~color\textgreater)}}
\smallv

\textit{Structure} objects can be unified---but, as in Prolog, only if they have the same functor and the same number of tuple elements. To unify two \textit{Structure} objects their corresponding tuple components must unify. 
\smallv

Let \textit{N} and \textit{P} be uninstantiated \textit{Var}s and consider unifying the following objects.\footnote{The underscores represent don't-care elements.} 
\begin{python}
   house(japanese, _, P, coffee, _)
   house(N, _, zebra, coffee, _)
\end{python}

Unification would leave both \textit{house} objects like this.
\begin{python}
   house(japanese, _, zebra, coffee, _)
\end{python}

Unification would have failed if the \textit{house} objects had different \textit{drink} attributes. 

Prolog's unification functionality is central to how it solves such puzzles so easily. We discuss the \textit{unify} function in Section \ref{subsec:unify}. 

%%%%%%%%%%%%%%%%%%%%%%%%%%%%%%%%%%%%%%%%%%%%%%
\subsection{Lists (Listings in Appendix \ref{appsubsec:lists})} \label{subsec:lists}
%%%%%%%%%%%%%%%%%%%%%%%%%%%%%%%%%%%%%%%%%%%%%%

Pylog includes two \textit{list} classes. \textit{PySequence} objects mimic Python lists and tuples. They are fixed in size; they are immutable; and their components are (recursively) required to be immutable. The only difference between \textit{PyList} and \textit{PyTuple} objects is that the former are displayed with square brackets, the latter with parentheses.
\smallv

More interestingly, Pylog also offers a \textit{LinkedList} class. Its functionality is similar to Prolog lists. In particular, a \textit{LinkedList} may have an uninstantiated tail, which is not possible with standard Python lists or tuples or with \textit{PySequence} objects.

\smallv
\textit{LinkedList}s may be created in two ways.
\begin{itemize}
    \item Pass the \textit{LinkedList} class the desired head and tail, e.g., \newline\textit{Xs = LinkedList(Xs\_Head, Xs\_Tail)}.
    \item Pass the  \textit{LinkedList} class a Python list. For example,
    \textit{LinkedList([])} is an empty \textit{LinkedList}. 
\end{itemize}

The next section (on \textit{append}) illustrates the power of Linked Lists.

%%%%%%%%%%%%%%%%%%%%%%%%%%%%%%%%%%%%%%%%%%%%%%
\subsection{append (Listings in Appendix \ref{appsubsec:append})} \label{subsec:append}
%%%%%%%%%%%%%%%%%%%%%%%%%%%%%%%%%%%%%%%%%%%%%%

The paradigmatic Prolog list function, and one that illustrates the power of logic variables, is \textit{append/3}. 

Pylog's \textit{append} has Prolog functionality for both \textit{LinkedList}s and \textit{PySequence}s. For example, running the code in Listing \ref{lis:append} produces the output in Listing \ref{lis:append_output}.\footnote{The output is the same whether we use \textit{PySequence}s or \textit{LinkedList}s.}

% \begin{minipage}{\linewidth}  \largev \hrulefill
% \begin{python}
% (Xs, Ys, Zs) = (Var(), Var(), LinkedList([1, 2, 3]))
% for _ in append(Xs, Ys, Zs):
%   print(f'Xs = {Xs}\nYs = {Ys}\n')
% \end{python}
% \begin{lstlisting} [caption={append},  label={lis:append}]
% \end{lstlisting}
% \end{minipage}
% produces this output.\footnote{The output is the same whether we use \textit{PySequence}s or \textit{LinkedList}s.}
% \smallv

% \begin{minipage}{\linewidth}  \largev \hrulefill
% \begin{python}
% Xs = []
% Ys = [1, 2, 3]

% Xs = [1]
% Ys = [2, 3]

% Xs = [1, 2]
% Ys = [3]

% Xs = [1, 2, 3]
% Ys = []
% \end{python}
% \begin{lstlisting} [caption={append output},  label={lis:append_output}]
% \end{lstlisting}
% \end{minipage}


\smallv

Pylog's \textit{append} function for \textit{LinkedList}s parallels Prolog's \textit{append/3}. The Prolog code is in Listing \ref{lis:prolog_append_code}; the Pylog code is in  Listing \ref{lis:append_code}.

% \begin{minipage}{\linewidth}  \largev \hrulefill
% \begin{python}
% append([], Ys, Ys).
% append([XZ|Xs], Ys, [XZ|Zs]) :- append(Xs, Ys, Zs).
% \end{python}
% \begin{lstlisting} [caption={prolog append},  label={lis:prolog_append_code}]
% \end{lstlisting}
% \end{minipage}
% Now the wordier but isomorphic Pylog version.

% % (For a cleaner presentation, declarations are dropped. All variables are: \textit{Union[LinkedList, Var]}.)

% \begin{minipage}{\linewidth}  \largev \hrulefill
% \begin{python}[numbers=left]
% def append(Xs, Ys, Zs):
%   # Corresponds to: append([], Ys, Ys).
%   yield from unify_pairs([(Xs, LinkedList([])), (Ys, Zs)])

%   # Corresponds to: append([XZ|Xs], Ys, [XZ|Zs]) :- append(Xs, Ys, Zs).
%   (XZ_Head, Xs_Tail, Zs_Tail) = n_Vars(3)
%   for _ in unify_pairs([(Xs, LinkedList(XZ_Head, Xs_Tail)),
%                       (Zs, LinkedList(XZ_Head, Zs_Tail))]):
%     yield from append(Xs_Tail, Ys, Zs_Tail)

% \end{python}
% \begin{lstlisting} [caption={append code},  label={lis:append_code}]
% \end{lstlisting}
% \end{minipage}

Note that \textbf{yield from} appears twice. If after execution of the first \textbf{yield from} (line 3), \textit{append} is called for another result, e.g., as a result of backtracking, it continues on to the second \textbf{yield from} (line 9). (As discussed in Section \ref{sec:control_functions}, this is standard behavior for Python generators.) The second part of the function calls itself recursively. Results are returned to the original caller from the first \textbf{yield from}---as in the Prolog version. 

%%%%%%%%%%%%%%%%%%%%%%%%%%%%%%%%%%%%%%%%%%%%%%
\subsection{Unification (Listings in Appendix \ref{appsubsec:unify})} \label{subsec:unify}
%%%%%%%%%%%%%%%%%%%%%%%%%%%%%%%%%%%%%%%%%%%%%%

To complete the discussion of logic variables, this section discusses the \textit{unify} function---which, like so many Pylog functions, is surprisingly straightforward. (Listing \ref{lis:unify}.)

The \textit{unify} function is called, \textit{unify(Left, Right)}, where \textit{Left} and \textit{Right} are the Pylog objects to be unified. (Argument order is immaterial.) 

% \begin{minipage}{\linewidth}  \largev \hrulefill
% \begin{python}[numbers=left]
% @euc
% def unify(Left: Any, Right: Any):

%   (Left, Right) = map(ensure_is_logic_variable, (Left, Right))

%   # Case 1.
%   if Left == Right:
%     yield

%   # Case 2.
%   elif isinstance(Left, PyValue) and isinstance(Right, PyValue) and \
%       (not Left.is_instantiated( ) or not Right.is_instantiated( )) and \
%       (Left.is_instantiated( ) or Right.is_instantiated( )):
%     (assignedTo, assignedFrom) = (Left, Right) if Right.is_instantiated( ) else (Right, Left)
%     assignedTo._set_py_value(assignedFrom.get_py_value( ))
%     yield

%     assignedTo._set_py_value(None)

%   # Case 3.
%   elif isinstance(Left, Structure) and isinstance(Right, Structure) and Left.functor == Right.functor:
%     yield from unify_sequences(Left.args, Right.args)

%   # Case 4.
%   elif isinstance(Left, Var) or isinstance(Right, Var):
%     (pointsFrom, pointsTo) = (Left, Right) if isinstance(Left, Var) else (Right, Left)
%     pointsFrom.unification_chain_next = pointsTo
%     yield

%     pointsFrom.unification_chain_next = None

% \end{python}
% \begin{lstlisting} [caption={unify},  label={lis:unify}]
% \end{lstlisting}
% \end{minipage}

The first step (line 4) ensures that the arguments are Pylog objects. If either is an immutable Python element, such as a string or int, it is wrapped in a \textit{PyValue}. This allows us to call, e.g, \textit{unify(X, `abc')} and  \textit{unify(`abc', X)}.
   
There are four \textit{unify} cases.

\begin{enumerate}
    \item \textit{Left} and \textit{Right} are already the same. Since Pylog objects are immutable, neither can change, and there's nothing to do. Succeed quietly via \textbf{yield}.

    \item \textit{Left} and \textit{Right} are both \textit{PyValue}s, and exactly one of them has a value. Assign the uninstantiated \textit{PyValue} the value of the instantiated one.
    \smallv \\
    An important step is to set the assignment back to \textbf{None} after the \textbf{yield} statement. (line 18) This undoes the unification on backtracking.

    \item \textit{Left} and \textit{Right} are both \textit{Structure}s, and they have the same functor. Unification consists of unifying the respective arguments. 

    \item Either \textit{Left} or \textit{Right} is a \textit{Var}. Point the \textit{Var} to the  element at the end of the other element's unification chain. As line 1 shows, \textit{unify} has a decorator. \textit{euc} ensures that if either argument is a \textit{Var} it is replaced by the element at the end of its unification chain. (\textit{euc} stands for \underline{e}nd of \underline{u}nification \underline{c}hain.) Again, unification must be undone on backtracking. (line 30) 

\end{enumerate} 

%%%%%%%%%%%%%%%%%%%%%%%%%%%%%%%%%%%%%%%%%%%%%%
\subsection{Back to tvsl\_yield\_lv (Listings in Appendix \ref{appsubsec:more_tvsl_yield_lv})} \label{subsec:more_tvsl_yield_lv}
%%%%%%%%%%%%%%%%%%%%%%%%%%%%%%%%%%%%%%%%%%%%%%

We are now able to add more detail to our discussion of \textit{tvsl\_yield\_lv} (Listing \ref{lis:yieldlv}). We will step through the code line by line. We will see that \textit{tvsl\_yield\_lv} is essentially a Pylog translation of  \textit{tvsl\_prolog} (Listing \ref{lis:transversalprolog1}).

Line 2. \textit{tvsl\_yield\_lv} has three parameters, as does \textit{tvsl\_prolog}. (The other Python transversal programs had two.) The parameters of  \textit{tvsl\_yield\_lv} and \textit{tvsl\_prolog} match up. In both cases. The third parameter is used to return the transversal to the caller.

Lines 3 and 4. These lines correspond to the second clause of \textit{tvsl\_prolog} . (The first clause generates a log.) If we have reached the end of the sets, \textit{Partial\_Transversal} is a complete transversal. Unify it with \textit{Complete\_Tvsl}.

Lines 6-9. These lines correspond to the third clause of \textit{tvsl\_prolog}.

\begin{quote}
\begin{quote}
Line 6 defines \textit{Element} as a new \textit{Var}.

\smallv
Line 7 unifies \textit{Element} with a member of \textit{Sets[0]}. The Pylog \textit{member} function is like the Prolog  \textit{member} function. On backtracking it unifies its first argument with successive members of its second argument. (This corresponds to line 10 of \textit{tvsl\_prolog}.)

\smallv
Line 8 ensures that the current value of  \textit{Element} is not already a member of \textit{Partial\_Transversal}. (See the \textit{fails} function in Section \ref{subsec:controlfunctions}.)  (This corresponds to line 11 of  \textit{tvsl\_prolog}.)

\smallv
Line 9 calls \textit{tvsl\_yield\_lv} recursively (via \textbf{yield from}). (This corresponds to lines 12 and 13 of \textit{tvsl\_prolog}.)
\end{quote}
\end{quote}

