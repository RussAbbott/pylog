\section{Logic variables}\label{subsec:logic_variables}
The following diagram shows Pylog's primary logic variable classes. This section discusses \textittt{PyValue}, \textittt{Var}, \textittt{Structure}, and the three types of lists.  (\textittt{Term} is an abstract class.)

% \section{Diagram Test}
\begin{figure}
    \centering
 \setlength{\unitlength}{0.12cm}
\begin{picture}(75,75)
    \put(29, 70){$\footnotesize{Term}$}
    \put(10, 65){\line(1,0){45}}
    \put(10, 65){\line(0,-1){5}}
    \put(33, 70){\line(0,-1){10}}
    \put(55, 65){\line(0,-1){5}}
    \put(5, 57){$\footnotesize{PyValue}$}
    \put(30, 57){$\footnotesize{Var}$}
    \put(50, 57){$\footnotesize{Structure}$}
    \put(10, 51){\line(0, 1){5}}
    \put(2, 51){\line(1,0){27}}  
    \put(55, 51){\line(0,1){5}}
    \put(2, 47){$\footnotesize{int, float, string, etc.}$}
    \put(48, 48){$\footnotesize{SuperSequence}$}
    \put(55, 42){\line(0,1){5}}
    \put(35, 33){$\footnotesize{LinkedList}$}
    \put(60, 33){$\footnotesize{PySequence}$}
    \put(40, 42){\line(1,0){30}}
    \put(40, 42){\line(0,-1){5}}
    \put(70, 42){\line(0,-1){5}}
    \put(55, 17){$\footnotesize{PyList}$}
    \put(72, 17){$\footnotesize{PyTuple}$}
    \put(70, 31){\line(0,-1){5}}
    \put(60, 25){\line(1,0){20}}
    \put(60, 25){\line(0,-1){5}}
    \put(80, 25){\line(0,-1){5}}
\end{picture}
\sinv\sinv\sinv\sinv\sinv\sinv\sinv\sinv\sinv
\caption{This diagram shows a more complete list.}
% \label{fig:class_tree}
\end{figure}

% Pylog includes the following logic variables classes.\footnote{This diagram shows a more complete list.
% \begin{displaymath}
% \xymatrixcolsep{1pc}
%     \xymatrix{& & Term \ar[dl] \ar[d] \ar[drr]  \\
%     & PyValue \ar[d] & Var & & Structure \ar[d] \\
%     & \overline{(numbers, strings, ...)} & & & SuperSequence \ar[dl] \ar[dr] \\
%      &  &  &  PySequence \ar[dl] \ar[dr] & & LinkedList \\
%      & & PyList & & PyTuple}
% \end{displaymath}
% }

\begin{itemize}

\item\textbftt{PyValue}. A \textittt{PyValue} provides a bridge between logic variables and Python values. A \textittt{PyValue} may hold any immutable Python value, e.g., a number, a string, or a tuple. Tuples are allowed as \textbftt{PyValue} values only if their components are also immutable. In the example of the preceding section, the logic variable \textittt{E} with value \textittt{'abc'} was actually \textittt{PyValue('abc')} behind the scenes.
\smallv

\item\textbftt{Var}. A \textittt{Var} functions as a traditional logic variable: it supports unification. 

Unification is surprisingly easy. Each \textittt{Var} object includes a \textittt{next} field, which is initially \textbftt{None}. When two \textittt{Var}s are unified, the \textittt{next} field of one is set to point to the other object. (It makes no difference, which points to which.) A chain of linked  \textittt{Var}s unify all the \textittt{Var}s in the chain. Consider the earlier example.

\begin{python}
(A, B, C, D, E) = (*n_Vars(4), 'abc')
for _ in unify_pairs([(A, B), (D, C), (A, C), (E, D)]):
\end{python}

The first two unifications produce the following. (The arrows may be reversed.)
\begin{equation}\label{eq:one}
\begin{split}
A \,\to\, B \\
D \,\to\, C 
\end{split}
\end{equation}
The next unification is \textittt{A} with \textittt{C}. The first step in unification is to go to the end of the unification chains of the elements to be unified. In this case, \textittt{B} (at the end of \textittt{A}'s unification chain) is unified with \textittt{C}. The result is either of the following.
\begin{equation}\label{eq:two}
\begin{split}
A \,\to\, B \: \: \qquad \qquad \qquad \qquad A \,\to\, B \\
\downarrow \qquad \qquad \qquad \quad \qquad  \: \qquad \uparrow\\
D \,\to\, C \: \: \qquad \qquad  \qquad \qquad D \,\to\, C 
\end{split}
\end{equation}
Finally, to unify \textittt{E} with \textittt{D} we go the the end of \textittt{D}'s unification chain---\textittt{B} or \textittt{C}.
\begin{equation}\label{eq:three}
\begin{split}
A \,\to\, B \: \:  \qquad \qquad  \qquad \qquad \qquad A \,\to\, B \,\to\, E('abc') \\
\downarrow \qquad \qquad \qquad \qquad \qquad \qquad \quad \uparrow  \:  \:  \quad \qquad  \qquad \\
D \,\to\, C  \,\to\, E('abc') \: \quad \qquad \qquad D \,\to\, C \:  \quad   \qquad \qquad 
\end{split}
\end{equation}
Different as they appear, these two structures are equivalent for unification purposes. To determine a \textittt{Var}'s value, follow its unification chain. If the end is a \textittt{PyValue}, the \textittt{PyValue}'s value is the \textittt{Var}'s value. In (3), all \textittt{Var}s have value \textittt{'abc'}. If the end of a unification chain is an uninstantiated \textittt{Var} (as in (2) for all \textittt{Var}s), the \textittt{Var}'s in the tributary chains are mutually unified, but uninstantiated. When the end  \textittt{Var} gets a value, it will be the value for all  \textittt{Var}'s leading to it.

\item\textbftt{Structure}. The \textittt{Structure} class enables the construction of Prolog terms. A \textittt{Structure} object consists of a functor along with a tuple of values. The Zebra puzzle (see section 5) uses \textittt{Structure}s to build \textittt{house} terms. The functor is \textittt{house}, and the tuple contains the house attributes. 

\centerline{\textit{house(\textless nationality\textgreater,~\textless cigarette\textgreater,~\textless pet\textgreater,~\textless drink\textgreater,~\textless house~color\textgreater)}}
\smallv

\textittt{Structure} objects can be unified---but, as in Prolog, only if they have the same functor and the same number of tuple elements. To unify two \textittt{Structure} objects their corresponding tuple components must unify. 
\smallv

Let \textittt{N} and \textittt{P} be uninstantiated \textittt{Var}s and consider unifying the following two \textittt{house} objects.\footnote{The underscores represent don't-care elements.} 
\begin{python}
   house(japanese, _, P, coffee, _)
   house(N, _, zebra, coffee, _)
\end{python}

Unification would leave both \textittt{house} objects like this.
\begin{python}
   house(japanese, _, zebra, coffee, _)
\end{python}
Unification would have failed if, for example, the \textittt{house} objects had different \textittt{drink} attributes. 

Prolog's unification functionality is central to how it solves such puzzles so easily.

\item\textbftt{Lists}. Pylog includes two \textittt{list} classes. \textittt{PySequence} objects mimic Python lists and tuples. They are fixed in size; they are immutable; and their components are (recursively) required to be immutable. The only difference between \textittt{PyList} and \textittt{PyTuple} objects is that the former are displayed with square brackets, the latter with parentheses.
\smallv

More interestingly, Pylog also offers a \textittt{LinkedList} class. Its functionality is similar to Prolog lists. In particular, a \textittt{LinkedList} may have an uninstantiated tail. Uninstantiated tails are not possible with standard Python lists or tuples or with \textittt{PySequence} objects.
\smallv

The paradigmatic Prolog list function is \textittt{append/3}. Pylog's \textittt{append/3} has Prolog functionality for both \textittt{LinkedList}s and \textittt{PySequence}s. For example, running:

\begin{minipage}{\linewidth}  \largev \hrulefill
\begin{python}
(Xs, Ys, Zs) = (Var(), Var(), LinkedList([1, 2, 3]))
for _ in append(Xs, Ys, Zs):
  print(f'Xs = {Xs}\nYs = {Ys}\n')
\end{python}
\begin{lstlisting} [caption={append},  label={lis:append}]
\end{lstlisting}
\end{minipage}
produces this output.\footnote{The output is the same whether we use \textittt{PySequence}s or \textittt{LinkedList}s.}
\smallv

\begin{minipage}{\linewidth}  \largev \hrulefill
\begin{python}
Xs = []
Ys = [1, 2, 3]

Xs = [1]
Ys = [2, 3]

Xs = [1, 2]
Ys = [3]

Xs = [1, 2, 3]
Ys = []
\end{python}
\begin{lstlisting} [caption={append output},  label={lis:append_output}]
\end{lstlisting}
\end{minipage}
\smallv

Pylog's \textittt{append/3} for \textittt{LinkedList}s parallels Prolog's \textittt{append/3}.
\smallv

First the Prolog version.

\begin{minipage}{\linewidth}  \largev \hrulefill
\begin{python}
append([], Ys, Ys).
append([XZ|Xs], Ys, [XZ|Zs]) :- append(Xs, Ys, Zs).
\end{python}
\begin{lstlisting} [caption={prolog append},  label={lis:prolog_append_code}]
\end{lstlisting}
\end{minipage}
Now the wordier but isomorphic Pylog version.\footnote{To understand the code, you should know that \textittt{LinkedList}s may be created in two ways.
\begin{itemize}
    \item Pass the \textittt{LinkedList} class the desired head and tail, e.g., \newline\textittt{Xs = LinkedList(Xs\_Head, Xs\_Tail)}.
    \item Pass the  \textittt{LinkedList} class a Python list. For example,
    \textittt{LinkedList([])} is an empty \textittt{LinkedList}. 
\end{itemize}
} (For a cleaner presentation, declarations are dropped. All parameters are: \textittt{Union[LinkedList, Var]}.)

\begin{minipage}{\linewidth}  \largev \hrulefill
\begin{python}
def append(Xs, Ys, Zs):
  # Corresponds to: append([], Ys, Ys).
  yield from unify_pairs([(Xs, LinkedList([])), (Ys, Zs)])

  # Corresponds to: append([XZ|Xs], Ys, [XZ|Zs]) :- append(Xs, Ys, Zs).
  (XZ_Head, Xs_Tail, Zs_Tail) = n_Vars(3)
  for _ in unify_pairs([(Xs, LinkedList(XZ_Head, Xs_Tail)),
                       (Zs, LinkedList(XZ_Head, Zs_Tail))]):
    yield from append(Xs_Tail, Ys, Zs_Tail)

\end{python}
\begin{lstlisting} [caption={append code},  label={lis:append_code}]
\end{lstlisting}
\end{minipage}

Note that \textbftt{yield from} appears twice. If after execution of the first \textbftt{yield from}, \textittt{append/3} is called for another result, e.g., as a result of "backtracking," it continues on to the second \textbftt{yield from}. (This is standard behavior for Python generators.) The second part of the function calls itself recursively. Results are returned to the caller from the first \textbftt{yield from}---as in the Prolog version. 
\end{itemize}
