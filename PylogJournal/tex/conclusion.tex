%%%%%%%%%%%%%%%%%%%%%%%%%%%%%%%%%%%%%%%%%%%%%%
\section{Conclusion (Listings in Appendix \ref{appsec:conclusion})} \label{sec:conclusion}
%%%%%%%%%%%%%%%%%%%%%%%%%%%%%%%%%%%%%%%%%%%%%%

Embedding rule chaining in the clues suggests the template for Pylog as in Listing \ref{lis:template}.

% \begin{minipage}{\linewidth}
% \begin{python}
%   def some_clause(...):
%      for _ in <generate options>:
%       <local conditions>
%       yield from next_clause(...)
% \end{python}
% \begin{lstlisting} [caption={The Trace decorator},  label={lis:template}]
% \end{lstlisting}
% \end{minipage}

More generally, Pylog offers a way to integrate logic programming into Python.

\begin{itemize}
  \item The magic of unification requires little more than linked chains.
  \item Prolog's control structures can be implemented as nested \textbf{for}-loops (for both choicepoints and scope setting), with \textbf{yield} and \textbf{yield from} gluing the pieces together.
\end{itemize}

% \appendix 
% \section{The Trace decorator}

% The \textit{Trace} decorator is defined as a class rather than a function. \textit{Trace} logs parameter values for both regular functions and generators, but \textit{Trace} does not handle keyword parameters.

% \begin{minipage}{\linewidth} \largev \hrulefill
% \begin{python}[numbers=left]
% from inspect import isgeneratorfunction, signature

% class Trace:

%     def __init__(self, f):
%         self.param_names = [param.name for param in signature(f).parameters.values()]
%         self.f = f
%         self.depth = 0

%     def __call__(self, *args):
%         print(self.trace_line(args))
%         self.depth += 1
%         if isgeneratorfunction(self.f):
%             return self.yield_from(*args)
%         else:
%             f_return = self.f(*args)
%             self.depth -= 1
%             return f_return

%     def yield_from(self, *args):
%         yield from self.f(*args)
%         self.depth -= 1

%     @staticmethod
%     def to_str(xs):
%         xs_string = f'[{", ".join(Trace.to_str(x) for x in xs)}]' if isinstance(xs, list) else str(xs)
%         return xs_string

%     def trace_line(self, args):
%         # The quoted string on the next line is two spaces.
%         prefix = "  " * self.depth
%         params = ", ".join([f'{param_name}: {Trace.to_str(arg)}'
%                             for (param_name, arg) in zip(self.param_names, args)])
%         # Special case for the transversal functions
%         termination = ' <=' if not args[0] else ''
%         return prefix + params + termination
% \end{python}

% \begin{lstlisting} [caption={The Trace decorator},  label={lis:Trace}]
% \end{lstlisting}
% \end{minipage}
