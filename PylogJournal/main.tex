% -*- coding: utf-8; -*-
% vim: set fileencoding=utf-8 :
\documentclass[english,submission]{programming}
\usepackage{mdframed, framed, rotating}
\usepackage[style=numeric,backend=biber,seconds=true]{biblatex}
\addbibresource{example.bib}
\addbibresource{pylog.bib}
\usepackage{pythonhl}
%\DeclareUnicodeCharacter{21B3}{\drarrow}

\lstdefinelanguage[programming]{TeX}[AlLaTeX]{TeX}{%
  deletetexcs={title,author,bibliography},%
  deletekeywords={tabular},
  morekeywords={abstract},%
  moretexcs={chapter},%
  moretexcs=[2]{title,author,subtitle,keywords,maketitle,titlerunning,authorinfo,affiliation,authorrunning,paperdetails,acks,email},
  moretexcs=[3]{addbibresource,printbibliography,bibliography},%
}%
\lstset{%
  language={[programming]TeX},%
  keywordstyle=\firamedium,
  stringstyle=\color{RosyBrown},%
  texcsstyle=*{\color{Purple}\mdseries},%
  texcsstyle=*[2]{\color{Blue1}},%
  texcsstyle=*[3]{\color{ForestGreen}},%
  commentstyle={\color{FireBrick}},%
  escapechar=`,}
\newcommand*{\CTAN}[1]{\href{http://ctan.org/tex-archive/#1}{\nolinkurl{CTAN:#1}}}
%%


\def\inv{\vspace*{-6pt}}
\def\sinv{\vspace*{-3pt}}
\def\smallinv{\vspace*{-3pt}}
\def\smallinh{\hspace*{-3pt}}
\def\smallv{\vspace*{2pt}}
\def\largev{\vspace*{6pt}}
\def\smallh{\hspace*{2pt}}


\newcommand{\textbftt}[1]{\textbf{\texttt{#1}}}
\newcommand{\textittt}[1]{\textit{\texttt{#1}}}

\usepackage{hyperref}
\hypersetup{
    colorlinks=true,
    linkcolor=blue,
    filecolor=magenta,      
    urlcolor=blue,
}


\begin{document}

\title{Pylog$\colon$ Prolog in Python \\ \small{\textbf{yield} :: \textbf{succeed} == \textbf{return} :: \textbf{fail}}}
%\subtitle{Preparing Articles for Programming}% optional
%\titlerunning{Preparing Articles for Programming} %optional, in case that the title is too long; the running title should fit into the top page column

\author[a]{Russ Abbott}
\authorinfo{is the author of this paper. Contact him at \email{rabbott@calstatela.edu}.}
\affiliation[a]{California State University, Los Angeles, USA}

\author[a]{Jungsoo Lim}
\authorinfo{is a co-author of this paper. Contact her at \email{jlim34@calstatela.edu}.}
% \affiliation[a]{California State University, Los Angeles, USA}

\author[b]{Jay Patel}
\authorinfo{is a co-author of this paper. Contact him at \email{imjaypatel12@gmail.com}.}
\affiliation[b]{Visa, Foster City, USA}

\keywords{backtracking, logic programming, logic variables, programming paradigms, Prolog, Pylog, Python, unification} 
% please provide 1--5 keywords


%%%%%%%%%%%%%%%%%%
%% These data MUST be filled for your submission. (see 5.3)
\paperdetails{
  %% perspective options are: art, sciencetheoretical, scienceempirical, engineering.
  %% Choose exactly the one that best describes this work. (see 2.1)
  perspective=art,
  %% State one or more areas, separated by a comma. (see 2.2)
  %% Please see list of areas in http://programming-journal.org/cfp/
  %% The list is open-ended, so use other areas if yours is/are not listed.
  area={Social Coding, General-purpose programming},
  %% You may choose the license for your paper (see 3.)
  %% License options include: cc-by (default), cc-by-nc
  %% license=cc-by,
}
%%%%%%%%%%%%%%%%%%

%%%%%%%%%%%%%%%%%%%%%%%%%%%%%
% Please go to https://dl.acm.org/ccs/ccs.cfm and generate your Classification
% System [view CCS TeX Code] stanz and copy _all of it_ to this place.
%% From HERE
\begin{CCSXML}
<ccs2012>
   <concept>
       <concept_id>10011007.10011006.10011008.10011009.10011021</concept_id>
       <concept_desc>Software and its engineering~Multiparadigm languages</concept_desc>
       <concept_significance>500</concept_significance>
    </concept>
 </ccs2012>
\end{CCSXML}

\ccsdesc[500]{Software and its engineering~Multiparadigm languages}

% \begin{CCSXML}
% <ccs2012>
% <concept>
% <concept_id>10011007.10010940</concept_id>
% <concept_desc>Software and its engineering~Software organization and properties</concept_desc>
% <concept_significance>500</concept_significance>
% </concept>
% </ccs2012>
% \end{CCSXML}

% \ccsdesc[500]{Software and its engineering~Software organization and properties}
% To HERE
%%%%%%%%%%%%%%%%%%%%%%%


\maketitle
\begin{abstract}

\smallv
\noindent
% \begin{quote}
\textbf{Context: What is the broad context of the work? What is the importance of the general research area?}

Pylog inhabits three programming contexts.
\begin{itemize}
    \item Pylog explores the integration of two distinct programming language paradigms: (i) the modern general purpose programming paradigm, including features of procedural programming, object-oriented programming, functional programming, and meta-programming, here represented by Python, and (ii) logic programming, whose primary features are logic variables (and unification) and built-in depth-first backtracking search, here represented by Prolog. These logic programming feature are generally missing from modern general purpose languages. Pylog illustrates how these two features can be implemented in and integrated into Python.

    \item Pylog demonstrates the breadth and broad applicability of Python. Although Python is one of the most widely used programming language for teaching introductory programming, it has also become very widely used for sophisticated programming tasks. One of the reasons for its popularity is the range of capabilities it offers—most of which are not used in elementary programming classes. Pylog makes effective use of many of those capabilities.

    \item Pylog exemplifies programming at its best. Pylog is first-of-all a programming exercise: How can the primary features of logic programming be integrated with Python? Secondly, Pylog uses features of Python in ways that are both intended and innovative. These include distinguishing between two uses of Python’s for-loop structure—as choicepoints and as aggregating constructs. The overall result is software worth reading. 
\end{itemize}
% \end{quote}

\smallv
\noindent
% \begin{quote}
\textbf{Inquiry: What problem or question does the paper address? How has this problem or question been addressed by others (if at all)?}

The primary issue addressed is how logic variables and backtracking can be integrated cleanly into a Python framework. Although significant work has been done in this area, much of it well done, most has been incomplete. Pylog is the first complete system (as far as we know) to achieve the goal of full integration. Also, as far as we know, this paper offers the first thorough explanation for how such integration can be accomplished.
% \end{quote}

\smallv
\noindent
% \begin{quote}
\textbf{Approach: What was done that unveiled new knowledge?}

Pylog demonstrates how logic variables and backtracking can be interwoven with standard Python data structures and control structures.
% \end{quote}

\smallv
\noindent
% \begin{quote}
\textbf{Knowledge: What new facts were uncovered? If the research was not results oriented, what new capabilities are enabled by the work?}

Pylog is available as a library for use in Python software. Pylog’s implementation techniques and insights may be used in Python programs not limited to logic programming.
% \end{quote}

\smallv
\noindent
% \begin{quote}
\textbf{Grounding: What argument, feasibility proof, artifacts, or results and evaluation support this work?}

By its existence Pylog demonstrates that logic variables and backtracking can be integrated into Python.
% \end{quote}

\smallv
\noindent
% \begin{quote}
\textbf{Importance: Why does this work matter?}

Python is known to be compatible with functional programming and other paradigms. This work shows that it is also compatible with logic programming. This work demonstrates the power and elegance of well-designed software.
% \end{quote}

\smallv
\noindent
% \begin{quote}
The Pylog code is available at \href{https://github.com/RussAbbott/pylog}{\underline{this GitHub repository}}.
% \end{quote}

\largev
ACM CCS 2012
\begin{itemize}
    \item Software and its engineering $\sim$ Multiparadigm languages;
\end{itemize}
\end{abstract}
\section{Introduction}

Prolog, a programming language derived from logic, was developed in the early 1970s. It became very popular during the 1980s as an AI language, especially as part of the Japanese $5^{th}$ generation project.

Prolog went out of favor because it was difficult to trace the execution of Prolog programs---which made debugging very challenging. But Prolog didn't die and has been making something of a comeback. 
\begin{itemize}
    \item SWI Prolog (free), GNU Prolog (free), and Sictus Prolog (commercial) have kept the Prolog flame burning and have large and active communities.
    \item  Although many of the following are from the popular media, these recent articles illustrate the extent to which Prolog has retained its lustre. Dhruv \cite{Dhruv2018}, Mathur \cite{Mathur2018}, Mehta \cite{Mehta2018},  Raturi \cite{Raturi2019}, and Sagar \cite{Sagar2019} all list Prolog as a top AI programming language. 
\end{itemize}

Prolog is both one of the syntactically simplest—--you can learn the syntax very quickly--—and at the same time most sophisticated of all programming languages. It is strongly declarative. One first declares facts and rules. (The rules are the Prolog programs.) One then constructs what are called queries, typically with embedded variables, and asks the system to find values for those variables so that the query satisfies the facts and rules. 

Prolog's most distinctive features are (i) logic variables (unification in particular) and (ii) built-in backtracking search. Prolog was also one of the first programming languages with immutable variables. 

Prolog feels like a different world--spare, unfamiliar, seductively powerful, elegant, and often frustratingly confusing when one can't visualize how the running program arrived at a particular point in the code. 

Python, in contrast, is a very well-known and widely used language. Python is one of the easiest programming languages to learn and is used in a great many introductory programming courses. 

Although easy to learn, Python includes many powerful computational and meta-level capabilities, which facilitate the development of quite sophisticated programs. Python's NumPy library\footnote{\url{https://numpy.org/}} for numerical programming can seem like magic even to experienced Python programmers. 

Because of its flexibility and ease-of use Python is in first place in almost all lists of AI languages---although primarily for its role as a scripting language---rather than a processing language---to tie together functions in its neural net and data science libraries. 

Python supports the procedural, object-oriented, and functional programming paradigms. It does not support Prolog's logic programming paradigm. This paper shows how logic programming features such as unification and backtracking can be integrated into standard Python.

The remainder of this paper is organized as follows.
 \begin{itemize}

 \item Section 2 discusses related work, much of which is quite recent.  

\item Section 3 provides a guided tour through Pylog. The tour is organized around five programs, which illustrate, step by step, how prolog features may be integrated into Python programs. This section explores Pylog's implementation of Prolog backtracking in some detail.

\item Section 4 discusses Pylog's logic variables and unification. 

\item Section 5 describes the widely-cited Zebra problem. We discuss Prolog and Python solutions along with the programming meta-structures, in particular depth-first backtracking search, that enable them. 

\item Section 6 is a brief conclusion.  
\end{itemize} 

A warning for readers interested in programming language theory. Virtually all our results are presented in terms of examples. There is very little theoretical discussion.

\section{Related work}

Quite a bit of work has been done in implementing Prolog features in Python, much of it fairly recently. As far as we can tell, none of it is as complete and as fully thought through as Pylog. But nearly all make important contributions. Following, in chronological order, are the authors' own descriptions, lightly edited for clarity and brevity.

\begin{itemize} %[label=$~$]
\item Berger (2004) \cite{berger2004}. Pythologic 
    \begin{quote}
    Python's meta-programming features are used to enable the writing of functions that include Prolog-like features.
    \end{quote}
\item Bolz (2007) \cite{Bolz2007} A Prolog Interpreter in Python.  
    \begin{quote}
    A proof-of-concept implementation of a Prolog interpreter in RPython, a restricted subset of the Python language intended for system programming. Performance compares reasonably well with other embedded Prologs.
    \end{quote}
\item Delford (2009) \cite{Delford2009} PyLog. 
    \begin{quote}
    A proof-of-concept implementation of a Prolog interpreter in RPython.
    \end{quote}
\item Frederiksen (2011) \cite{Frederiksen2011} Pike
    \begin{quote}
    A form of Logic Programming that integrates with Python.
    \end{quote}
\item Meyers (2015) \cite{Meyers2015} Prolog in Python. 
    \begin{quote}A hobby project developed over a number of years.\end{quote}
\item Maxime (2016) \cite{Maxime2016} Prology: Logic programming for Python3.
    \begin{quote}
    A minimal library that brings Logic Programming to Python.
    \end{quote}
\item Piumarta (2017) \cite{Piumarta2017} Notes and slides from a course on programming paradigms.
\item Thompson (2017) \cite{Thompson2017} Yield Prolog.
    \begin{quote} 
    Enables the embedding of Prolog-style predicates directly in Python. \end{quote}
\item Santini (2018) \cite{Santini2018} The pattern matching for python you always dreamed of.
    \begin{quote}
    Pampy is small, reasonably fast, and often makes code more readable.
    \end{quote}
\item Cesar (2019) \cite{Cesar2019} Prol: a minimal Prolog interpreter in a few lines of Python. 
\item Kopec (2019) \cite{Kopec2019} Constraint-Satisfaction Problems in Python.  
    \begin{quote} Chapter 3 of Kopec (2019) \textit{Classic Computer Science Problems in Python}.
    \end{quote}
\item Miljkovic (2019)\cite{Miljkovic2019} A simple Prolog Interpreter in a few lines of Python 3.
\item Niemeyer and Celles (2019) \cite{Niemeyer2019} A python-constraint library.
    \begin{quote}
    Pure Python solvers for Constraint Satisfaction Problems.
    \end{quote}
\item Rocklin (2019) \cite{Rocklin2019} kanren: Logic Programming in Python.
    \begin{quote}
    Enables the expression of relations and the search for values that satisfy them. (Rocklin is the author of the widely used Python toolz library.)
    \end{quote}
\end{itemize}

As this short survey suggests, most of the important ideas for embedding Prolog-like capabilities in Python have been known for a while. Pylog offers a more fully developed, more fully explained, and more integrated version of these ideas.

%%%%%%%%%%%%%%%%%%%%%%%%%%%%%%%%%%%%%%%%%%%%%%
\section{Solver basics and heuristics} \label{sec:solver-basics}
%%%%%%%%%%%%%%%%%%%%%%%%%%%%%%%%%%%%%%%%%%%%%%

To discuss solvers, it helps to refer to an example problem. We will use the computation of a transversal. Given a sequence of sets, a transversal is a non-repeating sequence of elements with the property that the \textit{n\textsuperscript{th}} element of the traversal belongs to the \textit{n\textsuperscript{th}} set in the sequence.  For example, the sets \[\{1, 2, 3\}, \{1, 2, 4\}, \{1\}\] has three transversals: [2, 4, 1], [3, 2, 1], and [3, 4, 1]. 

This is clearly a search problem. It can be solved with a simple depth-first search. First a utility function.

\begin{minipage}[c]{0.45\textwidth}
\begin{python1}  
unassigned = '_'
def uninstantiated_indices(transversal):
  """ Find indices of uninstantiated components. """
  return [indx for indx in range(len(transversal)) 
               if transversal[indx] is unassigned]
\end{python1}\linv
\begin{lstlisting} [caption={\textit{uninstantiated\_indices}}]
\end{lstlisting}
\end{minipage}

We apologize in advance for our Python code style deficiencies. This seemed to be the only way to include our code in the article given the column width and page limit.

First, a high level description of how \textit{tnvsl\_dfs} works. 
\begin{itemize}
    \item \textit{tnvsl\_dfs} looks for a transversal by finding transversal elements from left to right.
    \item It selects an element from the first set and (tentatively) assigns that as the first element of the transversal.
    \item It then recursively looks for a transversal for the rest of the sets---making sure that the element it selected from the first set is not repeated.
    \item If, at any point, it cannot proceed, say because it has reached a set all of whose elements have already been used, it ``fails,'' goes back to an earlier set, selects a different element from that set, and proceeds forward.
\end{itemize}

Now the code.

\begin{minipage}[c]{0.45\textwidth}
\begin{python1}  
def tnvsl_dfs(sets, tnvsl):
  remaining_indices = uninstantiated_indices(tnvsl)
  if not remaining_indices: return tnvsl

  nxt_indx = min(remaining_indices)
  for elmt in sets[nxt_indx]:
    if elmt not in tnvsl:
      new_tnvsl = tnvsl[:nxt_indx] \
                  + (elmt, ) \
                  + tnvsl[nxt_indx+1:]
      full_tnvsl = tnvsl_dfs(sets, new_tnvsl)
      if full_tnvsl is not None: return full_tnvsl
\end{python1}\linv
\begin{lstlisting} [caption={\textit{tnvsl\_dfs}}]
\end{lstlisting}
\end{minipage}

Here's an explanation of the code in some detail.
\begin{itemize}
    \item The function \textit{tnvsl\_dfs} takes two parameters: 
        \begin{enumerate}
            \item \textit{sets}: a list of sets
            \item \textit{tnvsl}: a tuple with as many positions as there are sets, but initialized to undefined.
        \end{enumerate}
    \item \textit{line 2}. Let \textit{remaining\_indices} be the indices of uninstantiated elements of \textit{tnvsl}. Initially this will be all of them. Since this first version of \textit{tnvsl\_dfs} generates values from left to right, the first element of \textit{remaining\_indices} will always be the leftmost undefined index position.
    \item \textit{line 3}. If \textit{remaining\_indices} is null, we have a complete transversal. Return it. Otherwise, go on to \textit{line 5}.
    \item \textit{line 5}. Set \textit{nxt\_indx} to the first undefined index position.
    \item \textit{line 6}. Begin a loop that looks at the elements of \textit{sets[nxt\_indx]}, the set at position  \textit{nxt\_indx}. We want an element from that set to represent it in the transversal.
    \item \textit{line 7}. If the currently selected \textit{elmt} of \textit{sets[nxt\_indx]} is not already in \textit{tnvsl}:
    \begin{enumerate}
        \item \textit{lines 8-10}. Put \textit{elmt} at position \textit{nxt\_indx}.
        \item \textit{line 11}. Call \textit{tnvsl\_dfs} recursively to complete the transversal, passing \textit{new\_tnvsl}, the extended \textit{tnvsl}. Assign the returned result to \textit{full\_tnvsl}.
        \item \textit{line 12}. If the returned result is not \textbf{None}, we have found a transversal. Return it to the caller. If the  returned result is \textbf{None}, the \textit{elmt} we selected from \textit{sets[nxt\_indx]} did not lead to a complete transversal. Return to \textit{line 6} to select another element.
    \end{enumerate}
\end{itemize}

This is standard depth first search. If there is at least one transversal, \textit{tnvsl\_dfs} will find the first one. Otherwise \textit{tnvsl\_dfs} will run off the end and return \textbf{None}.

Here's a trace of the recursive calls.

\smallv
\begin{minipage}[c]{0.45\textwidth}
\begin{python1}  
sets: [{1,2,3}, {1,2,4}, {1}], tnvsl: (_,_,_)
  sets: [{1,2,3}, {1,2,4}, {1}], tnvsl: (1,_,_)
    sets: [{1,2,3}, {1,2,4}, {1}], tnvsl: (1,2,_)
    sets: [{1,2,3}, {1,2,4}, {1}], tnvsl: (1,4,_)
  sets: [{1,2,3}, {1,2,4}, {1}], tnvsl: (2,_,_)
    sets: [{1,2,3}, {1,2,4}, {1}], tnvsl: (2,1,_)
    sets: [{1,2,3}, {1,2,4}, {1}], tnvsl: (2,4,_)
      sets: [{1,2,3}, {1,2,4}, {1}], tnvsl: (2,4,1)
\end{python1}\linv
\begin{lstlisting} [caption={\textit{tnvsl\_dfs trace}}]
\end{lstlisting}
\end{minipage}

\begin{itemize}
    \item \textit{line 1}. Initially (and on each call) the \textit{sets} are \[\{1, 2, 3\}, \{1, 2, 4\}, \{1\}\] Initially \textit{tnvsl} is completely undefined: \textit{(\_, \_, \_)}
    \item  \textit{line 2}. \textit{1} is selected as the first element of \textit{trvs}.
    \item  \textit{line 3}. \textit{1}  and \textit{2} are selected as the first two elements.
    \item \textit{line 4}. But now we are stuck. Since \textit{1} is already in \textit{trvs}, we can't use it as the third element of \textit{trvs}. Depth first search operates blindly. Instead of selecting an alternative for the first set, it backs up to the most recent selection and selects \textit{4} to represent the second set. 
    \item \textit{lines 5}. Of course, that doesn't solve the problem. So we back up again. Since we have already tried all elements of the second set, we back up to the first set and select \textit{2} as its representative. 
    \item \textit{lines 6}. Going forward, we select \textit{1} for second set.
    \item \textit{lines 7}. Again, we cannot use \textit{1} for the third set. So we back up and select \text{4} to represent the second set. (We can't use \textit{2} since it is already taken.)
    \item \textit{lines 8}. Finally, we can select \textit{1} as the third element of \textit{trvs}, and we're done.
\end{itemize}

Even though this is a simple depth-first search, it incorporates (what appears to be and what we have been referring to as) backtracking. In fact, there is no backtracking. The recursively nested \textbf{for}-loops produce a backtracking effect.  

It is common to use the term \textit{choicepoint} for places when (a) multiple choices are possible and (b) one wants to try them all, if necessary. Our simple solver implements choicepoints via (recursively) nested \textbf{for}-loops. 

We can see, though, that there is lots of room for improvement. We'll discuss two heuristics. 

\noindent\textbf{Propagate}. When we select an element for \textit{trvs} we can \textit{propagate} that selection by removing that element from the remaining sets. We can do that with the following changes to our original code. (Of course, a real solver would not hard-code heuristics. This is just to show how it works.)
\begin{enumerate}
    \item Before \textit{line 11}, insert this line.
    
\begin{minipage}[c]{0.45\textwidth}
\begin{python1}
new_sets = [set - {elmt} for set in sets]
\end{python1}
\end{minipage}

Then replace \textit{sets} with \textit{new\_sets} in \textit{line 11}.
This will remove \textit{elmt} from the remaining sets.

    \item Before \textit{line 5}, insert
    
\begin{python1}
if any(not sets[idx] for idx in remaining_indices):
  return None
\end{python1}

This tests whether any of our unrepresented sets are now empty. If so, we can't continue. (Recall that Python style recommends treating a set as a boolean when testing for emptiness. An empty set is considered \textbf{False}.)


\end{enumerate}

Because the empty sets in lines 2 and 4  trigger immediate backtracking, the execution takes 6 steps rather than 8.

\smallv
\begin{minipage}[c]{0.45\textwidth}
\begin{python1}  
sets: [{1,2,3}, {1,2,4}, {1}], tnvsl: (_,_,_)
  sets: [{2,3}, {2,4}, set()], tnvsl: (1,_,_)
  sets: [{1,3}, {1,4}, {1}], tnvsl: (2,_,_)
    sets: [{3}, {4}, set()], tnvsl: (2,1 _)
    sets: [{1,3}, {1}, {1}], tnvsl: (2,4,_)
      sets: [{3}, set(), set()], tnvsl: (2,4,1)
\end{python1}\linv
\begin{lstlisting} [caption={\textit{tnvsl\_dfs\_prop trace}}]
\end{lstlisting}
\end{minipage}

\noindent\textbf{Smallest first}. When selecting which \textit{tnvsl} index to fill next, pick the position associated with the smallest remaining set. 

In the original code, replace line 5 with
\begin{center}
\begin{minipage}[c]{0.45\textwidth}
\begin{python1}
 nxt_indx = min(remaining_indices,
                key=lambda indx: len(sets[indx]))
\end{python1}
\end{minipage}
\end{center}
The resulting trace is only 4 lines. (At line 3, the first two sets are the same size. By convention, \textit{min} selects the first.)

\begin{minipage}[c]{0.45\textwidth}
\begin{python1}  
sets: [{1,2,3}, {1,2,4}, {1}], tnvsl: (_,_,_)
  sets: [{1,2,3}, {1,2,4}, {1}], tnvsl: (_,_,1)
    sets: [{1,2,3}, {1,2,4}, {1}], tnvsl: (2,_,1)
      sets: [{1,2,3}, {1,2,4}, {1}, tnvsl: (2,4,1)
\end{python1}\linv
\begin{lstlisting} [caption={\textit{tnvsl\_dfs\_smallest trace}}]
\end{lstlisting}
\end{minipage}

One could apply both heuristics. Since in this case there was no backtracking, adding the \textit{propagate} heuristic makes no effective difference. Since you can see the pending sets shrinking however, the trace is a bit prettier.

\begin{minipage}[c]{0.45\textwidth}
\begin{python1} 
sets: [{1,2,3}, {1,2,4}, {1}], tnvsl: (_,_,_)
  sets: [{2,3}, {2,4}, {}], tnvsl: (_,_,1)
    sets: [{3}, {4}, {}], tnvsl: (2,_,1)
      sets: [{3}, {}, {}, tnvsl: (2,4,1)
\end{python1}\linv
\begin{lstlisting} [caption={\textit{tnvsl\_dfs\_both\_heuristics trace}}]
\end{lstlisting}
\end{minipage}

This concludes our discussion of a basic depth-first solver and two useful heuristics. We have yet to mention generators.

%%%%%%%%%%%%%%%%%%%%%%%%%%%%%%%%%%%%%%%%%%%%%%
\section{Generators} \label{sec:generators}
%%%%%%%%%%%%%%%%%%%%%%%%%%%%%%%%%%%%%%%%%%%%%%
In our previous examples, we have been happy to stop once we found a transversal,  any transversal. But what if the problem were a bit harder and we were looking for a transversal whose elements added to a given sum. The solvers we have seen so far wouldn't help---unless we added the new constraint to the solver itself. But we don't want to do that. We want to keep the transversal solvers independent of other constraints. (Adding heuristics don't violate this principle. Heuristics only make solvers more efficient.)

One approach would be to modify our solver to find and return all transversals. We could then select the one(s) that satisfied our additional constraints. But what if there were a great many transversals? Generating them all before looking at any of them would waste a colossal amount of time. 

We need a solver than can return results while keeping track of where it is with respect to its choice points so that it can continue from there if necessary. That's what a generator does. 

Here is our solver, \textit{including both heuristics}, as a generator.

\begin{minipage}[c]{0.45\textwidth}
\begin{python1}  
def tnvsl_dfs_gen(sets, tnvsl):
  remaining_indices = uninstantiated_indices(tnvsl)

  if not remaining_indices: yield tnvsl
  else:
    if any(not sets[i] for i in remaining_indices):
      return None
      
    nxt_indx = min(remaining_indices,
                   key=lambda indx: len(sets[indx]))
    for elmt in sets[nxt_indx]:
      if elmt not in tnvsl:
        new_tnvsl = tnvsl[:nxt_indx] \
                    + (elmt, ) \
                    + tnvsl[nxt_indx+1:]
        new_sets = [set - {elmt} for set in sets]
        for full_tnvsl in tnvsl_dfs_gen(sets, 
                                        new_tnvsl):
          yield full_tnvsl
\end{python1}\linv
\begin{lstlisting} [caption={\textit{tnvsl\_dfs\_gen}}, label={lis:dfs-gen}]
\end{lstlisting}
\end{minipage}

When called as in the following,

\begin{minipage}[c]{0.45\textwidth}
\begin{python1}  
for tnvsl in tnvsl_dfs_gen(sets, ('_','_','_')):
    print('=> ', tnvsl)
\end{python1}
\end{minipage}

the Trace looks like this.

\begin{minipage}[c]{0.45\textwidth}
\begin{python1}  
sets: [{1,2,3}, {1,2,4}, {1}], tnvsl: (_,_,_)
  sets: [{2,3}, {2,4}, {}], tnvsl: (_,_,1)
    sets: [{3}, {4}, {}], tnvsl: (2,_,1)
      sets: [{3}, {}, {}], tnvsl: (2,4,1)
=>  (2, 4, 1)
    sets: [{2}, {2,4}, {}], tnvsl: (3,_,1)
      sets: [{}, {4}, {}], tnvsl: (3,2,1)
=>  (3, 2, 1)
      sets: [{2}, {2}, {}], tnvsl: (3,4,1)
=>  (3, 4, 1)
\end{python1}\linv
\begin{lstlisting} [caption={\textit{tnvsl\_dfs\_gen trace}}]
\end{lstlisting}
\end{minipage}

All transversals are generated with no unnecessary backtracking.

Some comments on \textit{tnvsl\_dfs\_gen}.
\begin{itemize}
    \item The newly added \textbf{else} on line 5 is necessary. Previously, if there were no \textit{remaining\_indices}, we returned \textit{tnvsl}. That was the end of execution for this recursive call. But if we \textbf{yield} instead of \textbf{return}, when \textit{tnvsl\_dfs\_gen} is asked for more results, it continues with the line after the \textbf{yield}. But if have already found a transversal, we are done. We don't want to continue. The \textbf{else} divides the code into two mutually exclusive components. In effect \textbf{return} had done that implicitly.
    
    \item Lines 17-20 call \textit{tnvsl\_dfs\_gen} recursively and ask for all the transversals that can be constructed from the current state. Each one is then \textbf{yield}ed. No need to exclude \textbf{None}.  \textit{tnvsl\_dfs\_gen} will \textbf{yield} only complete transversals. 
    
    \smallv
Using \textbf{yield from}, lines 17-20 can be replaced by a line that \textbf{yield}s, one at a time, what \textit{tnvsl\_dfs\_gen} \textbf{yield}s to it.
\end{itemize}
\begin{center}
\begin{minipage}[c]{0.45\textwidth}
\begin{python1}
yield from tnvsl_dfs_gen(new_sets, new_tnvsl)
\end{python1}
\end{minipage}   
\end{center}

Let's turn off \textit{trace} and solve our initial problem: find a transversal whose elements sum to 6.

\begin{minipage}[c]{0.45\textwidth}
\begin{python1}
for tnvsl in tnvsl_dfs_gen(sets, ('_','_','_')):
  sum_string = ' + '.join(str(i) for i in tnvsl)
  equals = '==' if sum(tnvsl) == 6 else '!='
  print(f'{sum_string} {equals} 6')
  if sum(tnvsl) == 6: break
\end{python1}\linv
\begin{lstlisting} [caption={\textit{running tnvsl\_dfs\_gen}}, label={lis:dfs-gen-call}]
\end{lstlisting}
\end{minipage}
\smallv
The output will be as follows.

\begin{minipage}[c]{0.45\textwidth}
\begin{python1}  
2 + 4 + 1 != 6
3 + 2 + 1 == 6
\end{python1}\linv
\begin{lstlisting} [caption={\textit{tnvsl\_dfs\_gen trace}}]
\end{lstlisting}
\end{minipage}

We generated transversals until we found one whose elements summed to 6. Then we stopped.

\section{Control functions}\label{sec:control}
This section discusses Prolog's control flow and explains how Pylog implements it. It also presents a number of Pylog control-flow functions.

%%%%%%%%%%%%%%%%%%%%%%%%%%%%%%%%%%%%%%%%%%%%%%
\subsection{Control flow in Prolog}
%%%%%%%%%%%%%%%%%%%%%%%%%%%%%%%%%%%%%%%%%%%%%%
Prolog, or at least so-called "pure" Prolog, is a satisfiability theorem prover turned into a programming language. One supplies a Prolog execution engine with (a) a "query" or "goal" term along with (b) a database of terms and clauses and asks whether values for variables in the query/goal term can be found that are consistent with the database. The engine conducts a depth-first search looking for such values. 

Once Prolog was released as a programming language, programmers used it in a wide variety of applications, not necessarily limited to establishing satisfiability. 

An important feature of prolog as a programming language is that it distinguishes data flow from control flow far more sharply than most (if not all) other programming languages. 
\begin{enumerate}
    \item By \textittt{control flow} we mean the mechanisms that control the order in which program elements are executed or evaluated. This section discusses Pylog control flow.

    \item By \textit{dataflow} we mean the mechanisms that move data around within a program. Section \ref{sec:logic_variables} discusses how data flows through a Prolog program via logic variables and how Pylog implements logic variables.
\end{enumerate}

The fundamental control flow control mechanisms in most programming languages involve (a) sequential execution, i.e., one statement or expression following another in the order in which they appear in the source code, (b) conditional execution, e.g., \textbf{if} and related statements or expressions, (c) repeated execution, e.g., \textbf{while} statements or similar constructs, and (d) the execution/evaluation of sub-portions of a program such as functions and procedures via method calls and returns. 

Even programming languages known as declarative, such a Prolog, include explicit or implicit means to control the order of execution. That's even the case when the language includes lazy evaluation, in which an expression is evaluated only when its value is needed. 

Whether or not the language designers intended this to happen, programmers can generally learn how the execution/evaluation engine of a programming language works and write code to take advantage of that knowledge. This is not meant as a criticism. It's a simple consequence of the fact that computers---at least traditional, single-core computers---do one thing at a time, and programmers can design their code to exploit that ordering. 

Prolog, especially the basic Prolog this paper is considering, offers a straight-forward control-flow framework: lazy, backtracking, depth-first search. Listing \ref{lis:prologInterpreter}\footnote{See Bartak \cite{Bartak1998}.} shows a simple Prolog interpreter written in Prolog. (The code is so simple because unification and backtracking, the heart of Prolog, can both be taken for granted. There is hardly any work to do!)

\begin{minipage}{\linewidth}  \largev \hrulefill
\begin{python}[numbers=left]
solve([]).
solve([Term|Terms]):-
  clause(Term, Body), 
  append(Body, Terms, New_Terms), 
  solve(New_Terms).
\end{python}
\begin{lstlisting} [caption={A prolog interpreter in prolog},  label={lis:prologInterpreter}]
\end{lstlisting}
\end{minipage}

The execution engine, here represented by the \textittt{solve} predicate, starts with a list containing the query/goal term, typically with one or more uninstantiated variables. It then looks up and unifies, if possible, that term with a compatible term in the database (line 3). If unification is successful, the possibly empty body of the clause is appended to the list of unexamined terms (line 4), and the engine continues to work its way through that list. Should the list ever become empty (line 1), \textittt{solve} terminates successfully. The typically newly instantiated variables in the query contain the information returned by the program's execution of the original query.

If unification with a term in the database (line 3) is not possible, the program is said to have \textit{fail}ed (for the current execution path). The engine then backs up to the most recent point where it had made a choice. This typically occurs at line 3 where we are looking for a clause in the database with which to unify a term. If there are multiple such clauses, one is selected for further processing. If that term leads to a dead end, \textittt{solve} tries another of the unifiable terms.

In short, terms either \textittt{succeed} in unifying with a database term, which may extend the list of terms to be processed, or they \textittt{fail}, in which case the engine backtracks to the most recent choicepoint.\footnote{Operations internal to the program, such as arithmetic, may also fail and result in backtracking.} This is standard depth-first search, which we saw in \textit{trvsl\_dfs\_first}. 

In addition, when the engine, i.e., \textit{solve}, makes a selection at a choicepoint, it keeps track of mechanisms to produce other possible selections---as we saw with \textit{tvsl\_yield}. Not all the possible selections are produced at once. The engine may be \textit{lazy} in that it generates possible selections as needed. 

Even when \textit{solve} finds a path through the database that empties its list of terms, it retains the ability to backtrack and explore other paths. This capability enables Prolog to generate multiple answers to a query (but one at a time), just as \textit{tvsl\_yield} is able to generate multiple transversals, but again, one at a time when requested.

Prolog often seems strange to newcomers in that lazy backtracking search is the one and only mechanism Prolog (at least pure prolog) offers for controlling the flow of program execution. Although backtracking depth-first search itself is familiar to most programmers, lazy backtracking search may be less familiar. When writing Prolog code, one must get used to a world in which program flow is defined by lazy backtracking search.

%%%%%%%%%%%%%%%%%%%%%%%%%%%%%%%%%%%%%%%%%%%%%%
\subsection{Prolog control flow in Pylog}
%%%%%%%%%%%%%%%%%%%%%%%%%%%%%%%%%%%%%%%%%%%%%%
Prolog's lazy backtracking depth-first search is built on a mechanism that keeps track of unused choicepoint elements \textit{even after a successful element has been found}. Let's compare the relevant lines of \textit{tvsl\_dfs\_first} and \textit{tvsl\_yield}. In both cases we are interested in the \textbf{else} branches of these programs.

\begin{minipage}{\linewidth} \largev   \hrulefill
\begin{python}[numbers=left]
    for element in sets[0]:
      if element not in partial_transversal:
        complete_transversal = tvsl_dfs_first(sets[1:], partial_transversal + (element, ))
        if complete_transversal is not None:
          return complete_transversal 
    return None
\end{python}
\begin{lstlisting} [caption={The \textbf{else} branch of \textittt{tvsl\_dfs\_first}}, label={lis:dfsfirstelse}]
\end{lstlisting}
\end{minipage}


\begin{minipage}{\linewidth} \largev   \hrulefill
\begin{python}[numbers=left]
    for element in sets[0]:
      if element not in partial_transversal:
        yield from tvsl_yield(sets[1:], partial_transversal + (element, ))
\end{python}
\begin{lstlisting} [caption={The \textbf{else} branch of \textittt{tvsl\_yield}}, label={lis:yieldelse}]
\end{lstlisting}
\end{minipage}

In both cases, the choicepoint elements are the members of \textit{sets[0]}. (Recall that \textit{sets} is a list of sets; \textit{sets[0]} is the first set in that list. The choicepoint elements are the members of \textit{sets[0]}.) 

The first two lines of the two code segments are identical: define a \textbf{for}-loop over \textit{sets[0]}; establish that the selected element is not already in the partial transversal.

The third line adds that element to the partial transversal and asks the transversal program (\textit{tvsl\_dfs\_first} or \textit{tvsl\_yield}) to continue looking for the rest of the transversal. 

Here's where the two programs diverge.
\begin{itemize}
    \item In \textit{tvsl\_dfs\_first}, if a complete transversal is found, i.e., if something other than \textbf{None} is returned, that result is returned to the caller. The loop over the choicepoints terminates when the program exits the function via \textbf{return} on line 5.
    
    \item In \textit{tvsl\_yield}, if a complete transversal is found, i.e., if \textbf{yield from} returns a result, that result is \textbf{yield}ed back to the caller. But \textit{tvsl\_yield} does \textit{not} exit the loop over the choicepoints. The structure of the code suggests that perhaps the loop might somehow continue, i.e., that \textbf{yield} might not terminate the loop and exit the function the way \textbf{return} does. How can one return a value found in a loop but allow for the possibility that the loop might resume? That's the magic of Python generators---and the subject of the next section. 
\end{itemize}

%%%%%%%%%%%%%%%%%%%%%%%%%%%%%%%%%%%%%%%%%%%%%%
\subsection{Review of Python generators}
%%%%%%%%%%%%%%%%%%%%%%%%%%%%%%%%%%%%%%%%%%%%%%

This paper is not about Python generators. We assume readers are already familiar with them. Even so, because generators are so central to Pylog, this section provides a brief review.

\largev
Any Python function that contains \textbf{yield} or \textbf{yield from} is considered a generator. This is a black-and-white decision made by the Python compiler. Nothing is required to create a generator other than to include \textbf{yield} or \textbf{yield from} in the code.

So the question is: how do generators work operationally?

Using a generator requires two steps.
\begin{enumerate}
    \item Initialize the generator, essentially by calling it as a function. Initialization does \textit{not} run the generator. Instead when a generator function is called, a generator object is returned. That generator object can be activated (or reactivated) as in the next step.
    
    \item Activate (or reactivate) a generator object by calling \textit{next} with the generator object as a parameter. 
    
    \smallv
    When a generator is activated by \textit{next}, it runs until it reaches a \textbf{yield} or \textbf{yield from} statement. Like \textbf{return}, a \textbf{yield} statement may optionally include a value to be returned to the \textit{next}-caller. Whether or not a value is sent back to the \textit{next}-caller, a generator that encounters a \textbf{yield} stops running (much like a traditional function does when it encounters \textbf{return}). 
\end{enumerate}
        
    A significant difference between generators and traditional functions is that when a generator encounters \textbf{yield} \textit{it retains its state}. On a subsequent \textit{next} call, the generator resumes execution at the line after the \textbf{yield} statement.
    
    In other words, unlike functions, which may be understood to be associated with a stack frame---and which may be understood to have their stack frame discarded when the function encounters \textbf{return}---generator frames are maintained independently of the stack of the program that executes the  \textit{next} call.
    
    This structure allows generators to be activated/reactivated repeatedly via multiple \textit{next} calls. Consider the following simple example.
    
\begin{minipage}{\linewidth}  \largev  \hrulefill  
\begin{python}
def find_number(search_number):
    i = 0
    while True:
        i += 1
        if i == search_number:
            print("\nFound the number:", search_number)
            return
        else:
            yield i

search_number = 5
find_number_object = find_number(search_number)
while True:
    k = next(find_number_object)
    print(f'{k} is not {search_number}')
\end{python}
\begin{lstlisting} [caption={\textittt{Generator example}},  label={lis:generatorExample}]
\end{lstlisting}
\end{minipage}

When executed, the result will be as follows.

\begin{minipage}{\linewidth}  \largev  \hrulefill  
%\begin{python}
\begin{verbatim}
1 is not 5
2 is not 5
3 is not 5
4 is not 5

Found the number: 5

Traceback (most recent call last):
  <line number where error occurred> 
    k = next(find_number_object)
StopIteration

Process finished with exit code 1
\end{verbatim}
%\end{python}
\begin{lstlisting} [caption={\textittt{Generator example}},  label={lis:generatorExample}]
\end{lstlisting}
\end{minipage}

As \textit{find\_number} runs through 1 .. 4 it \textbf{yield}s them to the \textit{next}-caller at the top level, which prints that they are not the search number. Note what happens when \textit{find\_number} finds the search number. It executes \textbf{return} instead of \textbf{yield}. This produces a \textit{StopIteration} exception---because as a generator \textit{find\_number} is expected to \textbf{yield} instead of \textbf{return}. If the \textit{next}-caller does not handle that exception, as in this example, the exception propagates to the top level of the overall program, and the program terminates with an error code. 

Python's \textbf{for}-loop catches \textit{StopIteration} exceptions and simply terminates. If we replaced the \textbf{while}-loop above with 
\begin{python}
for k in find_number(search_number):
    print(f'{k} is not 5')
\end{python}
the output would be identical except that instead of terminating with a \textit{StopIteration} exception, we would terminate normally.

Notice also that the \textbf{for}-loop generates the generator object. The step that produces \textit{find\_number\_object} occurs when \textit{find\_number(search\_number)} runs when the  \textbf{for}-loop begins execution.

\textbf{yield from} also catches \textit{StopIteration} exceptions. Consider adding an intermediate function that uses \textbf{yield from}.\footnote{An intermediate function is required because \textbf{yield} and \textbf{yield from} may be used only within a function. We can't just put \textbf{yield from} inside the top-level \textbf{for}-loop.}\footnote{This example was adapted from \href{https://www.python-course.eu/python3_generators.php}{\underline{this generator tutorial}}.} 

\begin{minipage}{\linewidth}  \largev  \hrulefill  
\begin{python}
def use_yield_from():
    yield from find_number_object
    print('find_number failed, but "yield from" caught the Stop Iteration exception.')
    return

for k in use_yield_from():
    print(f'{k} is not 5')
\end{python}
\begin{lstlisting} [caption={\textittt{yield from example}},  label={lis:yieldfromExample}]
\end{lstlisting}
\end{minipage}
The result is similar to the previous---with no uncaught exceptions. 

\begin{minipage}{\linewidth}  \largev  \hrulefill  
\begin{verbatim}
1 is not 5
2 is not 5
3 is not 5
4 is not 5
Found the number: 5
find_number failed, but "yield from" caught the Stop Iteration exception.

Process finished with exit code 0
\end{verbatim}
\begin{lstlisting} [caption={\textittt{yield from example output}},  label={lis:yieldFromExampleOutput}]
\end{lstlisting}
\end{minipage}

Note that when \textit{find\_number} fails, i.e., when it does not perform a \textbf{yield}, the \textbf{yield from} line in \textit{call\_yield\_from} does not perform a yield. Instead \textit{use\_yield\_from} goes on to its next line and prints the \textit{find\_number failed} message. It then terminates without performing a \textbf{yield}, thereby causing the top-level \textbf{for}-loop to (catch the \textit{StopInteration} exception and) to terminate. 

In short, because Python generators maintain state after performing a \textit{yield}, they can be used to model Prolog backtracking.

%%%%%%%%%%%%%%%%%%%%%%%%%%%%%%%%%%%%%%%%%%%%%%
\subsection{\textbf{yield} :: \textit{succeed} == \textit{return} :: \textit{fail}}
%%%%%%%%%%%%%%%%%%%%%%%%%%%%%%%%%%%%%%%%%%%%%%
Generators perform an additional service. Recall that Prolog predicates either \textit{succeed} or \textit{fail}. In particular when a Prolog predicate fails, it does not return a negative result---recall how \textit{tvsl\_dfs\_first} returned \textbf{None} when it failed to complete a transversal. Instead, a failed predicate simply terminates the current execution path. The Prolog engine then backtracks to the most recent choicepoint.

Similarly, if a generator terminates, i.e., \textbf{return}s, before encountering a \textbf{yield}, it generates a \textit{StopIteration} exception. The \textit{next}-caller typically interprets that to indicate the equivalent of failure. In this way Prolog's succeed and fail map onto generator \textbf{yield} and \textbf{return}. This makes it fairly straightforward to write generators that mimic Prolog predicates.

\begin{itemize}
    \item A Pylog generator \textit{succeeds} when it performs a \textbf{yield}. 
    \item A Pylog generator \textit{fails} when it \textbf{return}s without performing a \textbf{yield}. 
\end{itemize}

Generators provide a second parallel construct. Multiple-clause Prolog predicates map onto a Pylog function with multiple \textbf{yield}s in a single control path.

\begin{minipage}{\linewidth}  \largev  \hrulefill  
\begin{python}
head :- body_1.

head :- body_2.
\end{python}
\begin{lstlisting} [caption={Prolog multiple clauses},  label={lis:prologmultipleclauses}]
\end{lstlisting}
\end{minipage}

can be implemented as follows.

\begin{minipage}{\linewidth}  \largev  \hrulefill  
\begin{python}
def head():
    <some code>
    yield
    
    <other code>
    yield
\end{python}
\begin{lstlisting} [caption={Pylog multiple sequential yields},  label={lis:pylogmultipleyields}]
\end{lstlisting}
\end{minipage}

Prolog's \textbf{cut} ('!') corresponds to a Python \textbf{if}-\textbf{else} structure.

\begin{minipage}{\linewidth}  \largev  \hrulefill  
\begin{python}
head :- !, body_1.

head :- body_2.
\end{python}
\begin{lstlisting} [caption={Prolog multiple clauses with a cut},  label={lis:prologmultipleclauseswithcut}]
\end{lstlisting}
\end{minipage}

can be implemented as follows. The two \textbf{yield}s are in separate arms of an \textbf{if}-\textbf{else} construct.

\begin{minipage}{\linewidth}  \largev  \hrulefill  
\begin{python}
def head():
    if <condition>:
      <some code>
      yield
    else
      <other code>
      yield
\end{python}
\begin{lstlisting} [caption={Multiple Pylog \textbf{yield}s in separate \textbf{if}-\textbf{else} arms},  label={lis:pylogmultipleclauseswithifelse}]
\end{lstlisting}
\end{minipage}

The control-flow functions discussed in Section \ref{subsec:controlfunctions} along with the \textit{append} function discussed in Section \ref{subsec:append} offer numerous examples.

\largev
Python's generator system has many more features than those covered above. But these are the ones on which Pylog depends. 

%%%%%%%%%%%%%%%%%%%%%%%%%%%%%%%%%%%%%%%%%%%%%%
\subsection{Control functions} \label{subsec:controlfunctions}
%%%%%%%%%%%%%%%%%%%%%%%%%%%%%%%%%%%%%%%%%%%%%%

The following control functions are defined. We leave it to the doc-strings to explain what they do.  Section \ref{sec:zebra} contains additional control functions examples. It's striking the extent to which generators make implementation straight-forward.

\begin{minipage}{\linewidth}  \largev \hrulefill
\begin{python}[numbers=left]
def fails(f):
  """
  Applied to a function so that the resulting function succeeds if and only if the original fails.
  Note that fails is applied to the function itself, not to a function call.
  Similar to a decorator but applied explicitly when used.
  """
  def fails_wrapper(*args, **kwargs):
    for _ in f(*args, **kwargs):
      # Fail, i.e., don't yield, if f succeeds
      return  
    # Succeed if f fails.
    yield     

  return fails_wrapper
\end{python}
\begin{lstlisting} [caption={fails},  label={lis:fails}]
\end{lstlisting}
\end{minipage}

\begin{minipage}{\linewidth}  \largev \hrulefill
\begin{python}[numbers=left]
def forall(gens):
  """
  Succeeds if all generators in the gens list succeed. The elements in the gens list
  are embedded in lambda functions to avoid premature evaluation.
  """
  if not gens:
    # They have all succeeded.
    yield
  else:
    # Get gens[0] and evaluate the lambda expression to get a fresh iterator.
    # The parentheses after gens[0] evaluates the lambda expression.
    # If it succeeds, run the rest of the generators in the list.
    for _ in gens[0]( ):
      yield from forall(gens[1:])
\end{python}
\begin{lstlisting} [caption={forall},  label={lis:forall}]
\end{lstlisting}
\end{minipage}

\begin{minipage}{\linewidth}  \largev \hrulefill
\begin{python}[numbers=left]
def forany(gens):
  """
  Succeeds if any of the generators in the gens list succeed. On backtracking, tries them all. 
  The gens elements must be embedded in lambda functions.
  """
  for gen in gens:
    yield from gen( )

\end{python}
\begin{lstlisting} [caption={forany},  label={lis:forany}]
\end{lstlisting}
\end{minipage}

\begin{minipage}{\linewidth}  \largev \hrulefill
\begin{python}[numbers=left]
def trace(x, succeed=True, show_trace=True):
  """
  Can be included in a list of generators (as in forall and forany) to see where we are.
  The second argument determines whether trace succeeds or fails. The third turns printing on or off.
  When included in a list of forall generators, succeed should be set to True so that
  it doesn't prevent forall from succeeding.
  When included in a list of forany generators, succeed should be set to False so that forany
  will go on the the next generator and won't take trace as an extraneous successes.
  """
  if show_trace:
    print(x)
  if succeed:
    yield

\end{python}
\begin{lstlisting} [caption={trace},  label={lis:trace}]
\end{lstlisting}
\end{minipage}


\begin{minipage}{\linewidth}  \largev \hrulefill
\begin{python}[numbers=left]
def would_succeed(f):
  """
  Applied to a function so that the resulting function succeeds/fails if and only if the original
  function succeeds/fails. If the original function succeeds, this also succeeds but without 
  binding any variables. Similar to a decorator but applied explicitly when used.
  """
  def would_succeed_wrapper(*args, **kwargs):
    succeeded = False
    for _ in f(*args, **kwargs):
      succeeded = True
      # Do not yield in the context of f succeeding.
      
    # Exit the for-loop so that unification will be undone.
    if succeeded:
      # Succeed if f succeeded.
      yield  
    # The else clause is redundant. It is included here for clarity.
    # else:
    #   Fail if f failed.
    #   pass   

  return would_succeed_wrapper

\end{python}
\begin{lstlisting} [caption={would\_succeed},  label={lis:wouldsucceed}]
\end{lstlisting}
\end{minipage}

\section{Logic variables}\label{sec:logic_variables}

%%%%%%%%%%%%%%%%%%%%%%%%%%%%%%%%%%%%%%%%%%%%%%
% \subsection{Pylog classes}
%%%%%%%%%%%%%%%%%%%%%%%%%%%%%%%%%%%%%%%%%%%%%%

Figure \ref{fig:class_tree} shows Pylog's primary logic variable classes. This section discusses \textittt{PyValue}, \textittt{Var}, \textittt{Structure}, and the three types of sequences.  (\textittt{Term} is an abstract class.)

% \section{Diagram Test}
\begin{figure}
    \centering
 \setlength{\unitlength}{0.12cm}
\begin{picture}(75,75)
    \put(29, 70){$\footnotesize{Term}$}
    \put(10, 65){\line(1,0){45}}
    \put(10, 65){\line(0,-1){5}}
    \put(33, 70){\line(0,-1){10}}
    \put(55, 65){\line(0,-1){5}}
    \put(5, 57){$\footnotesize{PyValue}$}
    \put(30, 57){$\footnotesize{Var}$}
    \put(50, 57){$\footnotesize{Structure}$}
    \put(10, 51){\line(0, 1){5}}
    \put(2, 51){\line(1,0){27}}  
    \put(55, 51){\line(0,1){5}}
    \put(2, 47){$\footnotesize{int, float, string, etc.}$}
    \put(48, 48){$\footnotesize{SuperSequence}$}
    \put(55, 42){\line(0,1){5}}
    \put(35, 33){$\footnotesize{LinkedList}$}
    \put(60, 33){$\footnotesize{PySequence}$}
    \put(40, 42){\line(1,0){30}}
    \put(40, 42){\line(0,-1){5}}
    \put(70, 42){\line(0,-1){5}}
    \put(55, 17){$\footnotesize{PyList}$}
    \put(72, 17){$\footnotesize{PyTuple}$}
    \put(70, 31){\line(0,-1){5}}
    \put(60, 25){\line(1,0){20}}
    \put(60, 25){\line(0,-1){5}}
    \put(80, 25){\line(0,-1){5}}
\end{picture}
\sinv\sinv\sinv\sinv\sinv\sinv\sinv\sinv\sinv
\caption{This diagram shows a more complete list of Pylog classes.}
\label{fig:class_tree}
\end{figure}

% \begin{itemize}

%%%%%%%%%%%%%%%%%%%%%%%%%%%%%%%%%%%%%%%%%%%%%%
\subsection{PyValue}
%%%%%%%%%%%%%%%%%%%%%%%%%%%%%%%%%%%%%%%%%%%%%%
A \textittt{PyValue} provides a bridge between logic variables and Python values. A \textittt{PyValue} may hold any immutable Python value, e.g., a number, a string, or a tuple. Tuples are allowed as \textbftt{PyValue} values only if their components are also immutable. % In the example of the preceding section, the logic variable \textittt{E} with value \textittt{'abc'} was actually \textittt{PyValue('abc')} behind the scenes.
\smallv

%%%%%%%%%%%%%%%%%%%%%%%%%%%%%%%%%%%%%%%%%%%%%%
\subsection{Var} \label{subsec:var}
%%%%%%%%%%%%%%%%%%%%%%%%%%%%%%%%%%%%%%%%%%%%%%
A \textittt{Var} functions as a traditional logic variable: it supports unification. 

Unification is surprisingly easy. Each \textittt{Var} object includes a \textittt{next} field, which is initially \textbftt{None}. When two \textittt{Var}s are unified, the \textittt{next} field of one is set to point to the other. (It makes no difference, which points to which.) A chain of linked  \textittt{Var}s unify all the \textittt{Var}s in the chain. 

Consider this example.

\begin{minipage}{\linewidth} \largev   \hrulefill
\begin{python}[numbers=left]
def print_ABCDE(A, B, C, D, E):
    print(f'A: {A}, B: {B}, C: {C}, D: {D}, E: {E}')

(A, B, C, D, E) = (Var(), Var(), Var(), Var(), 'abc')
print_ABCDE(A, B, C, D, E) 
for _ in unify(A, B):
  print_ABCDE(A, B, C, D, E) 
  for _ in unify(D, C):
    print_ABCDE(A, B, C, D, E) 
    for _ in unify(A, C):
      print_ABCDE(A, B, C, D, E) 
      for _ in unify(E, D):
        print_ABCDE(A, B, C, D, E) 
      print_ABCDE(A, B, C, D, E) 
    print_ABCDE(A, B, C, D, E) 
  print_ABCDE(A, B, C, D, E) 
print_ABCDE(A, B, C, D, E) 
\end{python}
\begin{lstlisting} [caption={Unifying logic variables},  label={lis:unifylogicvars}]
\end{lstlisting}
\end{minipage}

As we discussed earlier (Section \ref{subsec:forloops}), \textbf{for}-loops can serve as combination choicepoints and scope definitions. We elaborate that discussion here.

It's important not to be confused by \textbf{for}-loops. In normal Python a \textbf{for}-loop signals the repetition of the loop body. Even though the nested \textbf{for}-loops above look like nested iteration, that's not the case. \textit{There is no iteration!} In this example, the \textbf{for}-loops all serve solely as choicepoints and scope definitions. 

As choicepoints, however, each \textbf{for}-loop offers only a single choice. Since  \textit{unify} succeeds at most once, there is never any backtracking. So the only function of the \textbf{for}-loops is to define the scope over which the various \textit{unify} operations hold.  The output from running the preceding should make this clear.

\begin{minipage}{\linewidth} \largev   \hrulefill
\begin{python}[numbers=left]
A: _195, B: _196, C: _197, D: _198, E: abc
A: _196, B: _196, C: _197, D: _198, E: abc
A: _196, B: _196, C: _197, D: _197, E: abc
A: _197, B: _197, C: _197, D: _197, E: abc
A: abc, B: abc, C: abc, D: abc, E: abc
A: _197, B: _197, C: _197, D: _197, E: abc
A: _196, B: _196, C: _197, D: _197, E: abc
A: _196, B: _196, C: _197, D: _198, E: abc
A: _195, B: _196, C: _197, D: _198, E: abc
\end{python}
\begin{lstlisting} [caption={Unifying logic variables},  label={lis:unifylogicvars}]
\end{lstlisting}
\end{minipage}

Numbers with leading underscores indicate uninstantiated logic variables. 

Line 1. All the logic variables are distinct. Each has its own identification number.

Line 2. \textit{A} and \textit{B} have been unified. They have the same identification number.

Line 3. \textit{C} and \textit{D} have also been unified. They have the same identification number, but different from that of \textit{A} and \textit{B}.

Line 4. All the logic variables have been unified. They all have the same identification number.

Line 5. All the logic variables have \textit{abc} as their value.

Lines 6 - 9. Exit the unification scopes as defined by the \textbf{for}-loops and undo the respective unifications.

We can trace through the unifications diagrammatically. The first two unifications produce the following. (The arrows may be reversed.)
\begin{equation}\label{eq:one}
\begin{split}
A \,\to\, B \\
D \,\to\, C 
\end{split}
\end{equation}
The next unification is \textittt{A} with \textittt{C}. The first step in unification is to go to the end of the unification chains of the elements to be unified. In this case, \textittt{B} (at the end of \textittt{A}'s unification chain) is unified with \textittt{C}. The result is either of the following.

\begin{equation}\label{eq:two}
\begin{array}{c c c c c c c c }
A & \to & B            & \qquad \qquad \qquad \qquad \qquad &   A & \to  & B   \\
  &     & \downarrow   & \qquad \qquad \qquad \qquad \qquad &     &      & \uparrow \\
D & \to & C            & \qquad \qquad \qquad \qquad \qquad &   D & \to  & C
\end{array}
\end{equation}
Finally, to unify \textittt{E} with \textittt{D} we go the the end of \textittt{D}'s unification chain---\textittt{B} or \textittt{C}.
\begin{equation}\label{eq:three}
\begin{array}{c c c c c c c c c c c}
A & \to & B            &     &          & \quad \quad  \: &   A & \to  & B & \to & E('abc')   \\
  &     & \downarrow   &     &          & \quad \quad \: &     &      &\uparrow & &  \\
D & \to & C            & \to & E('abc') & \quad \quad \: &   D & \to  & C & & 
\end{array}
\end{equation}
Different as they appear, these two structures are equivalent for unification purposes.

To determine a \textittt{Var}'s value, follow its unification chain. If the end is a \textittt{PyValue}, the \textittt{PyValue}'s value is the \textittt{Var}'s value. In (3), all \textittt{Var}s have value \textittt{'abc'}. If the end of a unification chain is an uninstantiated \textittt{Var} (as in (2) for all \textittt{Var}s), the \textittt{Var}'s in the tributary chains are mutually unified, but uninstantiated. When the end \textittt{Var} gets a value, it will be the value for all \textittt{Var}'s leading to it.

The following convenience methods make it possible to write the preceding code more concisely---but without the \textit{print} statements..
\begin{itemize}
    \item \textittt{n\_Vars} takes an integer argument and generates that many \textittt{Var} objects.
    \item \textittt{unify\_pairs} takes a list of pairs (as tuples) and unifies the elements of each pair.
\end{itemize}

\begin{minipage}{\linewidth} \largev   \hrulefill
\begin{python}
(A, B, C, D, E) = (*n_Vars(4), 'abc')
for _ in unify_pairs([(A, B), (D, C), (A, C), (E, D)]):
\end{python}
\begin{lstlisting} [caption={Unifying logic variables shortened},  label={lis:unifylogicvarsshortened}]
\end{lstlisting}
\end{minipage}

%%%%%%%%%%%%%%%%%%%%%%%%%%%%%%%%%%%%%%%%%%%%%%
\subsection{Structure}
%%%%%%%%%%%%%%%%%%%%%%%%%%%%%%%%%%%%%%%%%%%%%%
The \textittt{Structure} class enables the construction of Prolog terms. A \textittt{Structure} object consists of a functor along with a tuple of values. The Zebra puzzle (see section \ref{sec:zebra}) uses \textittt{Structure}s to build \textittt{house} terms. The functor is \textittt{house}, and the tuple contains the house attributes. 

\centerline{\textit{house(\textless nationality\textgreater,~\textless cigarette\textgreater,~\textless pet\textgreater,~\textless drink\textgreater,~\textless house~color\textgreater)}}
\smallv

\textittt{Structure} objects can be unified---but, as in Prolog, only if they have the same functor and the same number of tuple elements. To unify two \textittt{Structure} objects their corresponding tuple components must unify. 
\smallv

Let \textittt{N} and \textittt{P} be uninstantiated \textittt{Var}s and consider unifying the following two \textittt{house} objects.\footnote{The underscores represent don't-care elements.} 
\begin{python}
   house(japanese, _, P, coffee, _)
   house(N, _, zebra, coffee, _)
\end{python}

Unification would leave both \textittt{house} objects like this.
\begin{python}
   house(japanese, _, zebra, coffee, _)
\end{python}

Unification would have failed if, for example, the \textittt{house} objects had different \textittt{drink} attributes. 

Prolog's unification functionality is central to how it solves such puzzles so easily. We discuss the \textit{unify} function in Section \ref{subsec:unify}. 

%%%%%%%%%%%%%%%%%%%%%%%%%%%%%%%%%%%%%%%%%%%%%%
\subsection{Lists}
%%%%%%%%%%%%%%%%%%%%%%%%%%%%%%%%%%%%%%%%%%%%%%
Pylog includes two \textittt{list} classes. \textittt{PySequence} objects mimic Python lists and tuples. They are fixed in size; they are immutable; and their components are (recursively) required to be immutable. The only difference between \textittt{PyList} and \textittt{PyTuple} objects is that the former are displayed with square brackets, the latter with parentheses.
\smallv

More interestingly, Pylog also offers a \textittt{LinkedList} class. Its functionality is similar to Prolog lists. In particular, a \textittt{LinkedList} may have an uninstantiated tail. Uninstantiated tails are not possible with standard Python lists or tuples or with \textittt{PySequence} objects.
\smallv

\textittt{LinkedList}s may be created in two ways.
\begin{itemize}
    \item Pass the \textittt{LinkedList} class the desired head and tail, e.g., \newline\textittt{Xs = LinkedList(Xs\_Head, Xs\_Tail)}.
    \item Pass the  \textittt{LinkedList} class a Python list. For example,
    \textittt{LinkedList([])} is an empty \textittt{LinkedList}. 
\end{itemize}

The following section on \text{append} illustrates the power of Linked Lists.
%%%%%%%%%%%%%%%%%%%%%%%%%%%%%%%%%%%%%%%%%%%%%%
\subsection{append} \label{subsec:append}
%%%%%%%%%%%%%%%%%%%%%%%%%%%%%%%%%%%%%%%%%%%%%%

The paradigmatic Prolog list function, and one that illustrates the power of logic variables, is \textittt{append/3}. 

Pylog's \textittt{append/3} has Prolog functionality for both \textittt{LinkedList}s and \textittt{PySequence}s. For example, running:

\begin{minipage}{\linewidth}  \largev \hrulefill
\begin{python}
(Xs, Ys, Zs) = (Var(), Var(), LinkedList([1, 2, 3]))
for _ in append(Xs, Ys, Zs):
  print(f'Xs = {Xs}\nYs = {Ys}\n')
\end{python}
\begin{lstlisting} [caption={append},  label={lis:append}]
\end{lstlisting}
\end{minipage}
produces this output.\footnote{The output is the same whether we use \textittt{PySequence}s or \textittt{LinkedList}s.}
\smallv

\begin{minipage}{\linewidth}  \largev \hrulefill
\begin{python}
Xs = []
Ys = [1, 2, 3]

Xs = [1]
Ys = [2, 3]

Xs = [1, 2]
Ys = [3]

Xs = [1, 2, 3]
Ys = []
\end{python}
\begin{lstlisting} [caption={append output},  label={lis:append_output}]
\end{lstlisting}
\end{minipage}
\smallv

Pylog's \textittt{append} for \textittt{LinkedList}s parallels Prolog's \textittt{append/3}.
\smallv

First the Prolog version.

\begin{minipage}{\linewidth}  \largev \hrulefill
\begin{python}
append([], Ys, Ys).
append([XZ|Xs], Ys, [XZ|Zs]) :- append(Xs, Ys, Zs).
\end{python}
\begin{lstlisting} [caption={prolog append},  label={lis:prolog_append_code}]
\end{lstlisting}
\end{minipage}
Now the wordier but isomorphic Pylog version.

% (For a cleaner presentation, declarations are dropped. All variables are: \textittt{Union[LinkedList, Var]}.)

\begin{minipage}{\linewidth}  \largev \hrulefill
\begin{python}[numbers=left]
def append(Xs, Ys, Zs):
  # Corresponds to: append([], Ys, Ys).
  yield from unify_pairs([(Xs, LinkedList([])), (Ys, Zs)])

  # Corresponds to: append([XZ|Xs], Ys, [XZ|Zs]) :- append(Xs, Ys, Zs).
  (XZ_Head, Xs_Tail, Zs_Tail) = n_Vars(3)
  for _ in unify_pairs([(Xs, LinkedList(XZ_Head, Xs_Tail)),
                       (Zs, LinkedList(XZ_Head, Zs_Tail))]):
    yield from append(Xs_Tail, Ys, Zs_Tail)

\end{python}
\begin{lstlisting} [caption={append code},  label={lis:append_code}]
\end{lstlisting}
\end{minipage}

Note that \textbftt{yield from} appears twice. If after execution of the first \textbftt{yield from} (line 3), \textittt{append} is called for another result, e.g., as a result of backtracking, it continues on to the second \textbftt{yield from} (line 9). (As discussed in Section \ref{sec:control}, this is standard behavior for Python generators.) The second part of the function calls itself recursively. Results are returned to the original caller from the first \textbftt{yield from}---as in the Prolog version. 

%%%%%%%%%%%%%%%%%%%%%%%%%%%%%%%%%%%%%%%%%%%%%%
\subsection{Unification} \label{subsec:unify}
%%%%%%%%%%%%%%%%%%%%%%%%%%%%%%%%%%%%%%%%%%%%%%

To complete the discussion of logic variables, this section discusses the \textittt{unify} function---which, like so many Pylog functions, is surprisingly straightforward. 

The \textittt{unify} function is called, \textittt{unify(Left, Right)}, where \textittt{Left} and \textittt{Right} are the Pylog objects to be unified. (Argument order is immaterial.) 

\begin{minipage}{\linewidth}  \largev \hrulefill
\begin{python}[numbers=left]
@euc
def unify(Left: Any, Right: Any):

  (Left, Right) = map(ensure_is_logic_variable, (Left, Right))

  # Case 1.
  if Left == Right:
    yield

  # Case 2.
  elif isinstance(Left, PyValue) and isinstance(Right, PyValue) and \
       (not Left.is_instantiated( ) or not Right.is_instantiated( )) and \
       (Left.is_instantiated( ) or Right.is_instantiated( )):
    (assignedTo, assignedFrom) = (Left, Right) if Right.is_instantiated( ) else (Right, Left)
    assignedTo._set_py_value(assignedFrom.get_py_value( ))
    yield

    assignedTo._set_py_value(None)

  # Case 3.
  elif isinstance(Left, Structure) and isinstance(Right, Structure) and Left.functor == Right.functor:
    yield from unify_sequences(Left.args, Right.args)

  # Case 4.
  elif isinstance(Left, Var) or isinstance(Right, Var):
    (pointsFrom, pointsTo) = (Left, Right) if isinstance(Left, Var) else (Right, Left)
    pointsFrom.unification_chain_next = pointsTo
    yield

    pointsFrom.unification_chain_next = None

\end{python}
\begin{lstlisting} [caption={unify},  label={lis:unify}]
\end{lstlisting}
\end{minipage}

The first step (line 4) ensures that the arguments are Pylog objects. If either is an immutable Python element, such as a string or int, it is wrapped in a \textittt{PyValue}. This allows us to call, e.g, \textittt{unify(X, `abc')} and  \textittt{unify(`abc', X)}, which are functionally the same.
   
There are four \textittt{unify} cases.

\begin{enumerate}
    \item \textittt{Left} and \textittt{Right} are already the same. Since Pylog objects are immutable, neither can change, and there's nothing to do. Succeed quietly via \textbftt{yield}.

    \item \textittt{Left} and \textittt{Right} are both \textittt{PyValue}s, and exactly one of them has a value. Assign the uninstantiated \textittt{PyValue} the value of the instantiated one.
    \smallv \\
    An important step is to set the assignment back to \textbftt{None} after the \textbftt{yield} statement. (line 18) This undoes the unification on backtracking.

    \item \textittt{Left} and \textittt{Right} are both \textittt{Structure}s, and they have the same functor. Unification consists of unifying the respective arguments. 

    \item Either \textittt{Left} or \textittt{Right} is a \textittt{Var}. Point the \textittt{Var} to the  element at the end of the other element's unification chain. As line 1 shows, \textittt{unify} has a decorator. \textittt{euc} ensures that if either argument is a \textittt{Var} it is replaced by the element at the end of its unification chain. (\textit{euc} stands for \underline{e}nd of \underline{u}nification \underline{c}hain.) Again, unification must be undone on backtracking. (line 30) 

\end{enumerate} 

%%%%%%%%%%%%%%%%%%%%%%%%%%%%%%%%%%%%%%%%%%%%%%
\subsection{Back to tvsl\_yield\_lv}
%%%%%%%%%%%%%%%%%%%%%%%%%%%%%%%%%%%%%%%%%%%%%%
Of the four Python transversal programs, we have yet to discuss \textit{tvsl\_yield\_lv} (Listing \ref{lis:yieldlv}). Given the preceding discussion of logic variables we are now able to do so. We will step through the code line by line. But look ahead to the Prolog transversal program (Listing \ref{lis:transversalprolog}. You will find that \textit{tvsl\_yield\_lv} is essentially a Pylog translation of that program.

Line 2. \textit{tvsl\_yield\_lv} has three parameters.(The other Python transversal programs had two.) The Prolog transversal program has three parameters. The parameters of \textit{tvsl\_yield\_lv} and the prolog transversal function match up. In both cases, the third parameter is used to return the transversal to the caller.

Lines 3 and 4. These lines correspond to the second clause of the Prolog transversal program. (The first clause of that program generates a log. It plays no role in finding a transversal.) If we have reached the end of the sets, \textit{Partial\_Transversal} is a complete transversal. Unify it with \textit{Complete\_Tvsl}.

Lines 6 - 9. These lines correspond to the third clause of the Prolog transversal program.

\begin{quote}
\begin{quote}
Line 6 defines the variable \textit{Element} as a new \textit{Var}.

\smallv
Line 7 unifies \textit{Element} with the members of \textit{Sets[0]}. The Pylog \textit{member} function is like the Prolog  \textit{member} function. On backtracking it unifies its first argument with successive members of its second argument. (This corresponds to line 10 of the Prolog transversal program.)

\smallv
Line 8 ensures that the current value of  \textit{Element} is not already a member of \textit{Partial\_Transversal}. (See the \textit{fails} function in Section \ref{subsec:controlfunctions}.)  (This corresponds to line 11 of the Prolog transversal program.)

\smallv
Line 9 calls \textit{tvsl\_yield\_lv} recursively (via \textbftt{yield from}). (This corresponds to lines 12 and 13 of the Prolog transversal program.)
\end{quote}
\end{quote}


\section{The Zebra Puzzle}\label{subsec:zebra}
The Zebra Puzzle, also known as the Einstein's Riddle, is a well known logic puzzle.
\begin{quotation}
There are five houses in a row. Each has a unique color and is occupied by a family of unique nationality. Each family has a unique favorite smoke, a unique pet, and a unique favorite drink. Fourteen clues (see below) provide additional constraints. \textit{Who has a zebra and who drinks water?}
\end{quotation}

One can easily write Prolog programs to solve this and similar puzzles.
\begin{itemize}
\item We represent a house as a Prolog \textittt{house} term as discussed above.
\item Our world is a list of 5 \textittt{house} terms, with all fields initially uninstantiated.
\smallv

\item The clues can be written as more-or-less direct translations of the English.
\end{itemize}

\begin{python}
zebra_problem(Houses) :-
    Houses = [house(_,_,_,_,_),house(_,_,_,_,_),house(_,_,_,_,_),
              house(_,_,_,_,_),house(_,_,_,_,_)],
    % 1. The English live in the red house.
    member(house(english,_,_,_,red), Houses),
    % 2. The Spanish have a dog.
    member(house(spanish,_,dog,_,_), Houses),
    % 3. They drink coffee in the green house.
    member(house(_,_,_,coffee,green), Houses),
    % 4. The Ukrainians drink tea.
    member(house(ukranians,_,_,tea,_), Houses),
    % 5. The green house is immediately to the right of the white house.
    nextto(house(_,_,_,_,white), house(_,_,_,_,green), Houses),
    % 6. The Old Gold smokers have snails.
    member(house(_,old_gold,snails,_,_), Houses),
    % 7. They smoke Kool in the yellow house.
    member(house(_,kool,_,_,yellow), Houses),
    % 8. They drink milk in the middle house.
    Houses = [_, _, house(_,_,_,milk,_), _, _],
    % 9. The Norwegians live in the first house on the left.
    Houses = [house(norwegians,_,_,_,_) | _],
    % 10. The Chesterfield smokers live next to the fox.
    next_to(house(_,chesterfield,_,_,_), house(_,_,fox,_,_), Houses),
    % 11. They smoke Kool in the house next to the horse.
    next_to(house(_,kool,_,_,_), house(_,_,horse,_,_), Houses),
    % 12. The Lucky smokers drink juice.
    member(house(_,lucky,_,juice,_), Houses),
    % 13. The Japanese smoke Parliament.
    member(house(japanese,parliament,_,_,_), Houses),
    % 14. The Norwegians live next to the blue house.
    next_to(house(norwegians,_,_,_,_), house(_,_,_,_,blue), Houses),
\end{python}

We can run this using SWI-Prolog online. SWI-Prolog includes \textittt{member} and \textittt{nextto} predicates. SWI-Prolog's \textittt{nextto} means in the order given as in clue 5.
\smallv 

SWI-Prolog does not include a predicate for "next to" in the sense of clues 10, 11, and 14: the order is unspecified. But it's easy enough to write our own, say, \textittt{next\_to}.
\begin{python}
next_to(A, B, List) :- nextto(A, B, List).
next_to(A, B, List) :- nextto(B, A, List).
\end{python}
A final issue needs resolution. None of the clues mention either a zebra or water. The following implicit clue solves that problem.

\begin{minipage}{\linewidth}
\begin{python}
    % 15. (implicit). 
    member(house(_,_,zebra,_,_), Houses),
    member(house(_,_,_,water,_), Houses).
\end{python}
\end{minipage}
When run, we get an almost instantaneous answer---shown here manually formatted.

\begin{minipage}{\linewidth}
\begin{python}
?- zebra_problem(Houses).
[    
    house(norwegians, kool, fox, water, yellow), 
    house(ukranians, chesterfield, horse, tea, blue), 
    house(english, old_gold, snails, milk, red), 
    house(spanish, lucky, dog, juice, white), 
    house(japanese, parliament, zebra, coffee, green)     
]
\end{python}
\end{minipage}
\smallv

\textit{The Japanese have a zebra, and the Norwegians drink water.}
\smallv
\smallv
\smallv

To write and run the Zebra problem in Pylog we built a more general framework. 
\begin{itemize}
    \item \sloppy We created a \textittt{House} class with named arguments. \textittt{House} is a subclass of \textittt{PrologStructure} (not pictured above) which is a subclass of  \textittt{Structure}.
    \item Each clue is expressed as a Python function. For example,
\begin{python}
  def clue_1(self, Houses: SuperSequence):
    """ 1. The English live in the red house.  """
    yield from member(House(nationality='English', color='red'), Houses)

  def clue_8(self, Houses: SuperSequence):
    """ 8. They drink milk in the middle house. 
           (Note the use of slice notation.)     """
    yield from unify(House(drink='milk'), Houses[2])
\end{python}
    \item The \textittt{Houses} list may be either a \textittt{LinkedList} or a \textittt{PyList}, i.e., \textittt{SuperSequence}.
    \item Users may select a house property as a pseudo-functor for displaying houses. We selected \textittt{nationality}. (See solution list below.)
    \item We added some simple constraint checking.
\end{itemize}
When run, the answer is the same. (The system helpfully counts clue evaluations.)

\begin{minipage}{\linewidth}
\begin{python}
After 1392 rule applications,
	1. Norwegians(Kool, fox, water, yellow)
	2. Ukrainians(Chesterfield, horse, tea, blue)
	3. English(Old Gold, snails, milk, red)
	4. Spanish(Lucky, dog, juice, white)
	5. Japanese(Parliament, zebra, coffee, green)
The Japanese own a zebra, and the Norwegians drink water.
\end{python}
\end{minipage}
\smallv
Consider the underlying Prolog/Python mechanisms that run these programs. 
\smallv

\textbftt{Prolog}. It's trivial to write a Prolog interpreter in Prolog. The following \textittt{solve} predicate \cite{Bartak1998} is given a list with a single \textittt{goal}, which it is asked to satisfy. Each Prolog clause is available as \textittt{clause(Head, Body)}, where  \textittt{Body} is a list of terms. (If \textittt{Head} is a Prolog fact, its \textittt{Body} is the empty list.)

\begin{minipage}{\linewidth}
\begin{python}
solve([]).
solve([Term|Terms]):-
  clause(Term, Body),
  append(Body, Terms, New_Terms),
  solve(New_Terms).
\end{python}
\end{minipage}

It feels like cheating to use Prolog to write a Prolog interpreter. Unification and backtracking can both be taken for granted. There is hardly any work to do! 
\smallv

\textbftt{Python}. We developed \textit{three} Python approaches to rule interpretation. 
\begin{enumerate}

\item \textit{forall}. We developed a \textit{forall} construct.\footnote{We also developed a parallel \textit{forany} construct} The Zebra problem is coded as follows.

\begin{python}
def zebra_problem(Houses) :-
    for _ in forall{[
        # 1. The English live in the red house.
        lambda: member(house(english,_,_,_,red), Houses),
        # 2. The Spanish have a dog.
        lambda: member(house(spanish,_,dog,_,_), Houses),
        # ...
        ]}
\end{python}

 \textit{forall} succeeds if and only if all members of the list it is passed succeed. Each list element is protected within a \textit{lambda} construct to prevent premature evaluation.
 \smallv
 
 \textit{forall} itself is coded as follows.
 
\begin{python}
def forall(lambdas: List[Callable]):
  if not lambdas:
    # They have all succeeded.
    yield
  else:
    # Execute lambdas[0]--note the parentheses after lambdas[0].
    # and then the rest. 
    for _ in lambdas[0]( ):
      yield from forall(lambdas[1:])
 
 \end{python}

\item \textittt{run\_all\_rules}. We developed a Python function that accepts a list of functions along with a list, e.g., of houses, reflecting the state of the world. It succeeds if and only if the functions all succeed. Following is a somewhat simplified version.

\begin{python}
def run_all_clues(World_List: List[Term], clues: List[Callable]):
    if not clues:
      # Ran all the clues. Succeed.
      yield
    else:
      # Run the current clue and then the rest of the clues.
      for _ in clues[0](World_List):
        yield from run_all_clues(World_List, clues[1:])
\end{python}

\item \textittt{Embed rule chaining in the rules.} For example,

\begin{python}
  def clue_1(Houses: SuperSequence):
    """ 1. The English live in the red house.  """
    for _ in member(House(nationality='English', color='red'), Houses):
      yield from clue_2(Houses)
\end{python}


\begin{python}
  def clue_2(Houses: SuperSequence):
    """ 2. The Spanish have a dog. """
    for _ in member(House(nationality='Spanish', pet='dog'), Houses):
      yield from clue_3(Houses)
\end{python}

Call \textit{clue\_1} with a list of uninstantiated houses as above, and the problem runs itself.
\end{enumerate}

The three approaches produce the same solution.
\smallv
\smallv
\smallv

Embedding rule chaining in the clues suggests a general template.

\begin{python}
      def some_term(...):
        for _ in <generate options>:
          yield from next_term(...)
\end{python}

This template illustrates how
\begin{itemize}
\item \textbftt{for} loops can implement backtracking, while
\item \textbftt{yield} and \textbftt{yield from} can serve as glue between clauses. 
\end{itemize}

\section{Conclusion}
  Pylog offers a way to integrate logic programming features into a Python environment.
  
  \begin{itemize}
  \item The magic of unification requires little more than linked chains.
  \item Prolog's control structures, including "backtracking," can be implemented as nested \textbftt{for} loops, with \textbftt{yield} and \textbftt{yield from} gluing the pieces together.
\end{itemize}

\newpage
\printbibliography 
\end{document}

