% -*- coding: utf-8; -*-
% vim: set fileencoding=utf-8 :
\documentclass[english,submission]{programming}
\usepackage{pythonhl}
\usepackage{mdframed, framed, rotating}
\usepackage[style=numeric,backend=biber,seconds=true]{biblatex}
\addbibresource{example.bib}
\addbibresource{pylog.bib}


\lstdefinelanguage[programming]{TeX}[AlLaTeX]{TeX}{%
  deletetexcs={title,author,bibliography},%
  deletekeywords={tabular},
  morekeywords={abstract},%
  moretexcs={chapter},%
  moretexcs=[2]{title,author,subtitle,keywords,maketitle,titlerunning,authorinfo,affiliation,authorrunning,paperdetails,acks,email},
  moretexcs=[3]{addbibresource,printbibliography,bibliography},%
}%
\lstset{%
  language={[programming]TeX},%
  keywordstyle=\firamedium,
  stringstyle=\color{RosyBrown},%
  texcsstyle=*{\color{Purple}\mdseries},%
  texcsstyle=*[2]{\color{Blue1}},%
  texcsstyle=*[3]{\color{ForestGreen}},%
  commentstyle={\color{FireBrick}},%
  escapechar=`,}
\newcommand*{\CTAN}[1]{\href{http://ctan.org/tex-archive/#1}{\nolinkurl{CTAN:#1}}}
%%


\def\inv{\vspace*{-6pt}}
\def\sinv{\vspace*{-3pt}}
\def\smallinv{\vspace*{-3pt}}
\def\smallinh{\hspace*{-3pt}}
\def\smallv{\vspace*{2pt}}
\def\largev{\vspace*{6pt}}
\def\smallh{\hspace*{2pt}}


\newcommand{\textbftt}[1]{\textbf{\texttt{#1}}}
\newcommand{\textittt}[1]{\textit{\texttt{#1}}}

\usepackage{hyperref}
\hypersetup{
    colorlinks=true,
    linkcolor=blue,
    filecolor=magenta,      
    urlcolor=blue,
}


\begin{document}

\title{Pylog: Prolog in Python}
\subtitle{yield : succeed :: return : fail} % optional
% \titlerunning{Pylog: Prolog in Python} % optional, in case that the title is too long; the running title should fit into the top page column

\author[a]{Russ Abbott}
\authorinfo{is the author of this paper. Contact him at \email{rabbott@calstatela.edu}.}
\affiliation[a]{California State University, Los Angeles, USA}

\author[a]{Jungsoo Lim}
\authorinfo{is a co-author of this paper. Contact her at \email{jlim34@calstatela.edu}.}

\author[b]{Jay Patel}
\authorinfo{is a co-author of this paper. Contact him at \email{imjaypatel12@gmail.com}.}
\affiliation[b]{Visa, Foster City, USA}

\keywords{backtracking, logic programming, logic variables, programming paradigms, Prolog, Pylog, Python, unification} 
% please provide 1--5 keywords


%%%%%%%%%%%%%%%%%%
%% These data MUST be filled for your submission. (see 5.3)
\paperdetails{
  %% perspective options are: art, sciencetheoretical, scienceempirical, engineering.
  %% Choose exactly the one that best describes this work. (see 2.1)
  perspective=art,
  %% State one or more areas, separated by a comma. (see 2.2)
  %% Please see list of areas in http://programming-journal.org/cfp/
  %% The list is open-ended, so use other areas if yours is/are not listed.
  area={Multiparadigm languages, General-purpose programming},
  %% You may choose the license for your paper (see 3.)
  %% License options include: cc-by (default), cc-by-nc
  %% license=cc-by,
}
%%%%%%%%%%%%%%%%%%

%%%%%%%%%%%%%%%%%%%%%%%%%%%%%
% Please go to https://dl.acm.org/ccs/ccs.cfm and generate your Classification
% System [view CCS TeX Code] stanz and copy _all of it_ to this place.
%% From HERE
\begin{CCSXML}
<ccs2012>
   <concept>
       <concept_id>10011007.10011006.10011008.10011009.10011021</concept_id>
       <concept_desc>Software and its engineering~Multiparadigm languages</concept_desc>
       <concept_significance>500</concept_significance>
    </concept>
 </ccs2012>
\end{CCSXML}   
% \ccsdesc[500]{Software and its engineering~Multiparadigm languages}
% \ccsdesc[500]{Software and its engineering~Software organization and properties}
% To HERE
%%%%%%%%%%%%%%%%%%%%%%%
% Formerly
% \begin{CCSXML}
% <ccs2012>
% <concept>
% <concept_id>10011007.10010940</concept_id>
% <concept_desc>Software and its engineering~Software organization and properties</concept_desc>
% <concept_significance>500</concept_significance>
% </concept>
% </ccs2012>
% \end{CCSXML}




\maketitle
\begin{abstract}
Pylog explores the integration of two distinct programming language paradigms: (i) the modern, general purpose programming paradigm which, in its Python implementation, has been broadened to include procedural, object-oriented, and functional programming and (ii) the logic programming paradigm, most notably including logic variables (with unification) and depth-first, backtracking search. 
          
Pylog illustrates how the core logic programming features can be implemented in and integrated into Python. 

Simultaneously, Pylog demonstrates Python's breadth. Python is used in situations ranging introductory programming classes to the development of very sophisticated software. Pylog demonstrates two interestingly distinct uses for Python's \textbf{for}-loop construct.

Pylog exemplifies programming at its best, using Python features in innovative yet clear ways to integrate features of a non-Python programming paradigm into its range of capabilities. The overall result is software worth reading.
          
          The Pylog code is available at \href{https://github.com/RussAbbott/pylog}{\underline{this GitHub repository}}.
\end{abstract}


%%%%%%%%%%%%%%%%%%%%%%%%%%%%%%%%%%%%%%%%%%%%%%
\section{Introduction}
%%%%%%%%%%%%%%%%%%%%%%%%%%%%%%%%%%%%%%%%%%%%%%

\noindent\textbf{Symbolic artificial intelligence}. The birth announcement for Artificial Intelligence took the form of a workshop proposal. The proposal predicted that \textit{every aspect of learning---or any other feature of intelligence---can in principle be so precisely described that a machine can be made to simulate it.}\cite{mccarthy2006proposal}  

At the workshop, held in 1956, Newell and Simon claimed that their Logic Theorist not only took a giant step toward that goal but even \textit{solved the mind-body problem}.\cite{russell2010artificial} A year later Simon doubled-down.
\begin{quote}
    [T]here are now machines that can think, that can learn, and that can create. Moreover, their ability to do these things is going to increase rapidly until---in a visible future---the range of problems they can handle will be coextensive with the range to which the human mind has been applied.\cite{simon1957models}
\end{quote}

Perhaps not unexpectedly, such extreme optimism about the power of symbolic AI, as this work was (and is still) known, faded into the gloom of what has been labelled the AI winter. 

%%%%%%%%%%%%%%%%%%%%%%%%%%%%%%%%%%%%%%%%%%%%%%
\smallv\noindent\textbf{Deep learning}. All was not lost. Winter was followed by spring and the green shoots of (non-symbolic) deep neural networks sprang forth. Andrew Ng said of that development, 
\begin{quote}
Just as electricity transformed almost everything 100 years ago, today I actually have a hard time thinking of an industry that I don’t think AI will transform in the next several years.\cite{ng2018ai}
\end{quote}

But another disappointment followed. Deep neural nets 
\begin{quote}
are surprisingly susceptible to what are known as \textit{adversarial} attacks. Small perturbations to images that are (almost) imperceptible to human vision can cause a neural network to completely change its prediction. When minimally modified, a correctly classified image of a school bus is reclassified as an ostrich. Even worse, the classifiers report high confidence in this wrong prediction.\cite{akhtar2018threat}
\end{quote}
% (Adversarial images did not kill off deep learning. They have been co-opted, and their use is now built into  deep neural network training methodologies.\cite{shrivastava2017learning})

%%%%%%%%%%%%%%%%%%%%%%%%%%%%%%%%%%%%%%%%%%%%%%
\smallv\noindent\textbf{Deep learning: the current state}. Deep learning has achieved extraordinary success in fields such as image captioning and natural language translation.\cite{garnelo2019reconciling} But other than its remarkable achievements in game-playing via reinforcement learning\cite{silver2018general}, it's triumphs have often been superficial. 

By that we don't mean that the work is trivial. We suggest that many deep learning systems learn little more than surface patterns. The patterns may be both subtle and complex, but they are surface patterns nevertheless.

Lacker\cite{lacker-gpt3} elicits many examples of such superficial (but very sophisticated) patterns from GPT-3\cite{brown2020language}, a highly acclaimed natural language system. In one, GPT-3, acting as a personal assistant, offers to read to its interlocutor his latest email. The problem is that GPT-3 has no access to that person's email---and doesn't "know" that without access it can't read the email. (Much of the excitement surrounding GPT-3 derives from its skill as a fiction author.) 

Both the conversational interaction and the made-up email sound plausible and natural. In reality, each consists of words strung together based simply on co-occurrences that GPT-3 found in the billions upon billions of word sequences it had scanned. Although what GPT-3 produces sounds like coherent English, it's all surface patterns with no underlying semantics.

Recent work\cite{geirhos2018imagenet} (see \cite{Cepelewicz-textures-2020} for a popular discussion) suggests that much of the success of deep learning, at least when applied to image categorization, derives from the tendency of deep learning systems to focus on textures---the ultimate surface feature---rather than shapes.  This insight offers an explanation for some of deep learning's brittleness and superficiality as well as a possible mitigation strategy.

%%%%%%%%%%%%%%%%%%%%%%%%%%%%%%%%%%%%%%%%%%%%%%
\smallv\noindent\textbf{The Holy Grail: constraint programming}. In the mean time, work on symbolic AI continued. Constraint programming was born in the 1980s as an outgrowth of the interest in logic programming triggered by the Japanese Fifth Generation initiative.\cite{shapiro1983fifth} Logic Programming led to Constraint Logic Programming, which evolved into Constraint Programming. (A familiar constraint programming example is the well-known n-queens problem: how can you place n queens on an n \textit{x} n chess board so that no queen threatens any other queen? There are, of course, many practical constraint programming applications as well.)

In 1997, Eugene Frueder characterized constraint programming as \textit{the Holy Grail of computer science: the user simply states the problem and the computer solves it.}\cite{freuder1997pursuit}  Software that solves constraint programming problems is known as a solver. Constraint programming has many desirable properties. 
\begin{itemize}
    \item Solutions found by constraint programming solvers actually solve the given problem. There is no issue of how ``confident'' the solver is in the solutions it finds.
    
    \item One can understand how the solver arrived at the solution. This contrasts with the frustrating feature of neural nets that the solutions they find are generally hidden within a maze of parameters, unintelligible to human beings. 

    \item  The structure and limits of constraint programming are well understood: there will be no grand disappointments similar to those that followed the birth of artificial intelligence---unless quantum computing, once implemented, turns out to be a bust. 
    
    \item Constraint programming is closely related to computational complexity, which provides a well-studied theoretical framework for it. 
    
    \item There will be no surprises such as adversarial images. 

    \item Solver technology is easy to characterize. It is an exercise in search: find values for uninstantiated variables that satisfy the constraints.

    \item Improvements are generally incremental and consist primarily of new heuristics and better search strategies. For example, in the n-queens problem one can propagate solution steps by marking as unavailable board squares that are threatened by newly placed pieces. This reduces search times. We will see  example heuristics below.

\end{itemize}

Constraint programming solvers are now available in multiple forms. MiniZinc\cite{wallace2020problem} allows users to express constraints in what is essentially executable predicate calculus.

Solvers are also available as package add-ons to many programming languages: Choco\cite{prud2019choco} and JaCoP\cite{kuchcinski2013jacop} (two Java libraries), OscaR/CBLS\cite{Oscar} and Yuck\cite{Yuck} (two Scala libraries), and Google's OR-tools\cite{Google-OR-tools} (a collection of C++ libraries, which sport Python, Java, and .NET front ends).

In the systems just mentioned, the solver is a black box. One sets up a problem, either directly in predicate calculus or in the host language, and then calls on the solver to solve it. 

This can be frustrating for those who want more insight into the internal workings of the solvers they use. Significantly more insight is available when working either (a) in a system like Picat\cite{zhou2015constraint}, a language that combines features of logic programming and imperative programming, or (b) with Prolog (say either SICStus Prolog\cite{carlsson2014sicstus} or SWI Prolog\cite{swi-prolog}) to which a Finite Domain package has been added. But neither option helps those without a logic programming background. 

%%%%%%%%%%%%%%%%%%%%%%%%%%%%%%%%%%%%%%%%%%%%%%
\smallv\noindent\textbf{Shallow embeddings}.  Solver capabilities may be implemented directly in a host language and made available to programs in that language.\cite{hoare1998unifying, gibbons2014folding} Recent examples include Kanren\cite{Rocklin2019}, a Python embedding, and Muli\cite{dageforde2018constraint}, a Java embedding.

Most shallow embeddings have well-defined APIs; but like libraries, their inner workings are not visible. This is the case with both Kanren and Muli. Kanren is open source, but it offers no implementation documentation. Dagef{\"o}rde and Kuchen describe the Muli virtual machine\cite{dageforde2019compiler}, but the documentation is quite technical. Many who would like to understand its internal functioning may find it difficult going. 

%%%%%%%%%%%%%%%%%%%%%%%%%%%%%%%%%%%%%%%%%%%%%%
\smallv\noindent\textbf{Back to basics}. This brings us to our goal for the rest of this paper: to offer an under-the-covers tutorial about how a fully functioning embedded solver works. 

One can think of Prolog as the skeleton of a constraint satisfaction solver. Consequently, we focus on Prolog as a basic paradigmatic solver. We describe Pylog, a Python shallow embedding of Prolog's core capabilities. 

Our primary focus will be on helping readers understand how Prolog's two fundamental features, backtracking and logic variables, can be implemented \textit{simply and cleanly}. We also show how two of the heuristics common to Finite Domain packages can be added. 

Pylog should be accessible to anyone reasonable fluent in Python. In addition, the techniques used in the implementation are easily transferred to many other languages. 

We stress \textit{simply and cleanly}. There are many ways to implement backtracking and logic variables, some quite complex. Our approach is straightforward and easy to understand. 

An advantage we have over earlier Prolog embeddings is Python generators. Without generators, one is pushed to use more complex backtracking implementations, such as continuation passing\cite{amin2019lightweight} or monads\cite{seres1999embedding}. Generators, which are now widespread\cite{wikipedia-generators}, eliminate the need for such complexity. 

To be clear, we did not invent the use of generators for implementing backtracking. It has a nearly two-decade history: \cite{berger2004, Bolz2007, Delford2009, Frederiksen2011, Meyers2015, Thompson2017, Santini2018, Cesar2019, Miljkovic2019}. We would like especially to thank Ian Piumarta\cite{Piumarta2017}; Pylog began as a fork of his efforts. 

The preceding are sketches and prototypes. We offer a cleanly coded, well-explained, and fully operational solver.


%%%%%%%%%%%%%%%%%%%%%%%%%%%%%%%%%%%%%%%%%%%%%%
\section{Solver basics and heuristics} \label{sec:solver-basics}
%%%%%%%%%%%%%%%%%%%%%%%%%%%%%%%%%%%%%%%%%%%%%%

As an example problem we will use the computation of a transversal. Given a sequence of sets, a transversal is a non-repeating sequence of elements with the property that the \textit{n\textsuperscript{th}} element of the traversal belongs to the \textit{n\textsuperscript{th}} set in the sequence. For example, the set sequence \{1, 2, 3\}, \{1, 2, 4\}, \{1\} has three transversals: [2, 4, 1], [3, 2, 1], and [3, 4, 1]. 

This problem can be solved with a simple depth-first search. Here's a high level description. 
\begin{itemize}
    \item Look for transversal elements from left to right.
    \item Select an element from the first set and (tentatively) assign that as the first element of the transversal.
    \item Recursively look for a transversal for the rest of the sets---being sure not to reuse any already selected elements.
    \item If, at any point, we cannot proceed, say because we have reached a set all of whose elements have already been used, go back to an earlier set, select a different element from that set, and proceed forward.
\end{itemize}

First a utility function (Listing \ref{lis:uninstantiated-indices}) and then \textit{tnvsl\_dfs} (Listing \ref{lis:tnvsl-dfs}), the solver. (Please pardon our Python style deficiencies. The column width and page limit compelled compromises.) 


\begin{center}
\begin{minipage}[c]{0.45\textwidth}
\begin{python1}  
unassigned = '_'
def uninstantiated_indices(transversal):
  """ Find indices of uninstantiated components. """
  return [indx for indx in range(len(transversal)) 
               if transversal[indx] is unassigned]
\end{python1}\linv
\begin{lstlisting} [caption={\textit{uninstantiated\_indices}}, label={lis:uninstantiated-indices}]
\end{lstlisting}
\end{minipage}
\end{center}

\begin{figure}[htb]
    \centering
\begin{minipage}[c]{0.45\textwidth}
\begin{python1}  
def tnvsl_dfs(sets, tnvsl):
  remaining_indices = uninstantiated_indices(tnvsl)
  if not remaining_indices: return tnvsl

  nxt_indx = min(remaining_indices)
  for elmt in sets[nxt_indx]:
    if elmt not in tnvsl:
      new_tnvsl = tnvsl[:nxt_indx] \
                  + (elmt, ) \
                  + tnvsl[nxt_indx+1:]
      full_tnvsl = tnvsl_dfs(sets, new_tnvsl)
      if full_tnvsl is not None: return full_tnvsl
\end{python1}\linv
\begin{lstlisting} [caption={\textit{tnvsl\_dfs}}, label={lis:tnvsl-dfs}]
\end{lstlisting}
\end{minipage}\linv
\end{figure}

Here's an explanation of the search engine in some detail.
\begin{itemize}
    \item The function \textit{tnvsl\_dfs} takes two parameters: 
        \begin{enumerate}
            \item \textit{sets}: a list of sets
            \item \textit{tnvsl}: a tuple with as many positions as there are sets, but initialized to undefined.
        \end{enumerate}
    \item \textit{line 2}. \textit{remaining\_indices} is a list of the indices of uninstantiated elements of \textit{tnvsl}. Initially this will be all of them. Since \textit{tnvsl\_dfs} generates values from left to right, the first element of \textit{remaining\_indices} will always be the leftmost undefined index position.
    \item \textit{line 3}. If \textit{remaining\_indices} is null, we have a complete transversal. Return it. Otherwise, go on to \textit{line 5}.
    \item \textit{line 5}. Set \textit{nxt\_indx} to the first undefined index position.
    \item \textit{line 6}. Begin a loop that looks at the elements of \textit{sets[nxt\_indx]}, the set at position  \textit{nxt\_indx}. We want an element from that set to represent it in the transversal.
    \item \textit{line 7}. If the currently selected \textit{elmt} of \textit{sets[nxt\_indx]} is not already in \textit{tnvsl}:
    \begin{enumerate}
        \item \textit{lines 8-10}. Put \textit{elmt} at position \textit{nxt\_indx}.
        \item \textit{line 11}. Call \textit{tnvsl\_dfs} recursively to complete the transversal, passing \textit{new\_tnvsl}, the extended \textit{tnvsl}. Assign the returned result to \textit{full\_tnvsl}.
        \item \textit{line 12}. If \textit{full\_tnvsl} is not \textbf{None}, we have found a transversal. Return it to the caller. If \textit{full\_tnvsl} is \textbf{None}, the \textit{elmt} we selected from \textit{sets[nxt\_indx]} did not lead to a complete transversal. Return to \textit{line 6} to select another element from \textit{ sets[nxt\_indx]}.
    \end{enumerate}
\end{itemize}

This is standard depth first search. \textit{tnvsl\_dfs} will either find the first transversal, if there are any, or return \textbf{None}.

Here's a trace of the recursive calls.

\mediumv
\begin{minipage}[c]{0.45\textwidth}
\begin{python1}  
sets: [{1,2,3}, {1,2,4}, {1}], tnvsl: (_,_,_)
  sets: [{1,2,3}, {1,2,4}, {1}], tnvsl: (1,_,_)
    sets: [{1,2,3}, {1,2,4}, {1}], tnvsl: (1,2,_)
    sets: [{1,2,3}, {1,2,4}, {1}], tnvsl: (1,4,_)
  sets: [{1,2,3}, {1,2,4}, {1}], tnvsl: (2,_,_)
    sets: [{1,2,3}, {1,2,4}, {1}], tnvsl: (2,1,_)
    sets: [{1,2,3}, {1,2,4}, {1}], tnvsl: (2,4,_)
      sets: [{1,2,3}, {1,2,4}, {1}], tnvsl: (2,4,1)
\end{python1}\linv
\begin{lstlisting} [caption={\textit{tnvsl\_dfs trace}}]
\end{lstlisting}
\end{minipage}

\begin{itemize}
    \item \textit{line 1}. Initially (and on each call) the \textit{sets} are \[\{1, 2, 3\}, \{1, 2, 4\}, \{1\}\] Initially \textit{tnvsl} is completely undefined: \textit{(\_, \_, \_)}
    \item  \textit{line 2}. \textit{1} is selected as the first element of \textit{trvs}.
    \item  \textit{line 3}. \textit{1}  and \textit{2} are selected as the first two elements.
    \item \textit{line 4}. But now we are stuck. Since \textit{1} is already in \textit{trvs}, we can't use it as the third element of \textit{trvs}. Depth first search operates blindly. Instead of selecting an alternative for the first set, it backs up to the most recent selection and selects \textit{4} to represent the second set. 
    \item \textit{lines 5}. Of course, that doesn't solve the problem. So we back up again. Since we have already tried all elements of the second set, we back up to the first set and select \textit{2} as its representative. 
    \item \textit{lines 6}. Going forward, we select \textit{1} for second set.
    \item \textit{lines 7}. Again, we cannot use \textit{1} for the third set. So we back up and select \text{4} to represent the second set. (We can't use \textit{2} since it is already taken.)
    \item \textit{lines 8}. Finally, we can select \textit{1} as the third element of \textit{trvs}, and we're done.
\end{itemize}

\noindent\textbf{How recursively nested for-loops implement choicepoints and backtracking}. This simple depth-first search appears to incorporate backtracking. In fact, there is no backtracking. Recursively nested \textbf{for}-loops produce a backtracking effect.  

It is common to use the term \textit{choicepoint} for a place in a program where (a) multiple choices are possible and (b) one wants to try them all, if necessary. Our simple solver implements choicepoints via (recursively) nested \textbf{for}-loops. 

The \textbf{for}-loop on line 6 generates options until either we find one for which the remainder of the program completes the traversal, or, if the options available have been exhausted, the program fails out of that recursive call and ``backtracks'' to a choicepoint at a higher/earlier level of the recursion.

In this context, backtracking means popping an element from the call stack and restoring the program at the next higher level. As with any function call, the calling function continues at the point after the function call---in this case, line 12. 

If the function called on line 11 returns a complete transversal, we return it to the \textit{next} higher level, which continues to return it up the stack until we reach the original caller. 

If what was returned on line 11 was not a complete transversal, we go around the \textbf{for}-loop again, bind \textit{element} to the next member of \textit{sets[nxt\_indx]}, and try again. 

The call stack serves as a record of earlier, pending choicepoints. We resume them in reverse order as needed. That's exactly what depth-first search is all about.

\smallv
\noindent We now turn to two heuristics that improve solver efficiency. 

\smallv
\noindent\textit{Propagate}. When we select an element for \textit{trvs} we can \textit{propagate} that selection by removing that element from the remaining sets. We can do that with the following changes. (Of course, a real solver would not hard-code heuristics. This is just to show how it works.)
\begin{enumerate}
    \item Before \textit{line 11}, insert this line.
  
\begin{minipage}[c]{0.45\textwidth}
\begin{python1}
new_sets = [set - {elmt} for set in sets]
\end{python1}
\end{minipage}

Then replace \textit{sets} with \textit{new\_sets} in \textit{line 11}.
This will remove \textit{elmt} from the remaining sets.

    \item Before \textit{line 5}, insert

\begin{minipage}[c]{0.45\textwidth}
\begin{python1}
if any(not sets[idx] for idx in remaining_indices):
  return None
\end{python1}
\end{minipage}


This tests whether any of our unrepresented sets are now empty. If so, we can't continue. (Recall that Python style recommends treating a set as a boolean when testing for emptiness. An empty set is considered \textbf{False}.)


\end{enumerate}

Because the empty sets in lines 2 and 4 of the trace trigger backtracking, the execution takes 6 steps rather than 8.

\begin{flushright}
\begin{minipage}[c]{0.45\textwidth}
\begin{python1}  
sets: [{1,2,3}, {1,2,4}, {1}], tnvsl: (_,_,_)
  sets: [{2,3}, {2,4}, set()], tnvsl: (1,_,_)
  sets: [{1,3}, {1,4}, {1}], tnvsl: (2,_,_)
    sets: [{3}, {4}, set()], tnvsl: (2,1 _)
    sets: [{1,3}, {1}, {1}], tnvsl: (2,4,_)
      sets: [{3}, set(), set()], tnvsl: (2,4,1)
\end{python1}\linv
\begin{lstlisting} [caption={\textit{tnvsl\_dfs\_prop trace}}]
\end{lstlisting}
\end{minipage}
\end{flushright}

The \textit{Propagate} heuristic is a partial implementation of the \textit{all-different} constraint. It can be applied to this problem because we know that the transversal elements must all be different from each other.

\noindent\textit{Smallest first}. When selecting which \textit{tnvsl} index to fill next, pick the position associated with the smallest remaining set. 

In the original code, replace line 5 with
\begin{center}
\begin{minipage}[c]{0.45\textwidth}
\begin{python1}
 nxt_indx = min(remaining_indices,
                key=lambda indx: len(sets[indx]))
\end{python1}
\end{minipage}
\end{center}
The resulting trace (Listing \ref{lis:dfs-4-lines}) is only 4 lines. (At line 3, the first two sets are the same size. \textit{min} selects the first.) 

\begin{figure}[htb]
    \centering\begin{minipage}[c]{0.45\textwidth}
\begin{python1}  
sets: [{1,2,3}, {1,2,4}, {1}], tnvsl: (_,_,_)
  sets: [{1,2,3}, {1,2,4}, {1}], tnvsl: (_,_,1)
    sets: [{1,2,3}, {1,2,4}, {1}], tnvsl: (2,_,1)
      sets: [{1,2,3}, {1,2,4}, {1}, tnvsl: (2,4,1)
\end{python1}\linv
\begin{lstlisting} [caption={\textit{tnvsl\_dfs\_smallest trace}}, label={lis:dfs-4-lines}]
\end{lstlisting}\inv
\end{minipage}\linv
\end{figure}

One could apply both heuristics. Since \textit{smallest first} eliminated backtracking, adding the \textit{propagate} heuristic makes no effective difference. But, one can watch the sets shrink.

\mediumv
\begin{minipage}[c]{0.45\textwidth}
\begin{python1} 
sets: [{1,2,3}, {1,2,4}, {1}], tnvsl: (_,_,_)
  sets: [{2,3}, {2,4}, {}], tnvsl: (_,_,1)
    sets: [{3}, {4}, {}], tnvsl: (2,_,1)
      sets: [{3}, {}, {}, tnvsl: (2,4,1)
\end{python1}\linv
\begin{lstlisting} [caption={\textit{tnvsl\_dfs\_both\_heuristics trace}}]
\end{lstlisting}
\end{minipage}

This concludes our discussion of a basic depth-first solver and two useful heuristics. We have yet to mention generators.

%%%%%%%%%%%%%%%%%%%%%%%%%%%%%%%%%%%%%%%%%%%%%%
\section{Generators} \label{sec:generators}
%%%%%%%%%%%%%%%%%%%%%%%%%%%%%%%%%%%%%%%%%%%%%%
In our previous examples, we have been happy to stop once we found a transversal,  any transversal. But what if the problem were a bit harder and we were looking for a transversal whose elements added to a given sum. The solvers we have seen so far wouldn't help---unless we added the new constraint to the solver itself. But we don't want to do that. We want to keep the transversal solvers independent of other constraints. (Adding heuristics don't violate this principle. Heuristics only make solvers more efficient.)

One approach would be to modify the solver to find and return all transversals. We could then select the one(s) that satisfied our additional constraints. But what if there were many transversals? Generating them all before looking at any of them would waste a colossal amount of time. 

We need a solver than can return results while keeping track of where it is with respect to its choicepoints so that it can continue from there if necessary. That's what a generator does. 

Listing \ref{lis:dfs-gen} shows a generator version of our solver, including both heuristics. When called as in Listing \ref{lis:dfs-gen-call}, it produces the trace in Listing \ref{lis:dfs-gen-trace}. 

\begin{figure}[htb]
    \centering
\begin{minipage}[c]{0.45\textwidth}
\begin{python1}  
def tnvsl_dfs_gen(sets, tnvsl):
  remaining_indices = uninstantiated_indices(tnvsl)

  if not remaining_indices: yield tnvsl
  else:
    if any(not sets[i] for i in remaining_indices):
      return None
      
    nxt_indx = min(remaining_indices,
                   key=lambda indx: len(sets[indx]))
    for elmt in sets[nxt_indx]:
      if elmt not in tnvsl:
        new_tnvsl = tnvsl[:nxt_indx] \
                    + (elmt, ) \
                    + tnvsl[nxt_indx+1:]
        new_sets = [set - {elmt} for set in sets]
        for full_tnvsl in tnvsl_dfs_gen(new_sets, 
                                        new_tnvsl):
          yield full_tnvsl
\end{python1}\linv
\begin{lstlisting} [caption={\textit{tnvsl\_dfs\_gen}}, label={lis:dfs-gen}]
\end{lstlisting}
\end{minipage}\linv
\end{figure}


\begin{figure}[htb]
    \centering
\begin{minipage}[c]{0.45\textwidth}
\begin{python1}  
for tnvsl in tnvsl_dfs_gen(sets, ('_','_','_')):
    print('=> ', tnvsl)
\end{python1}\linv
\begin{lstlisting} [caption={\textit{tnvsl\_dfs\_gen}}, label={lis:dfs-gen-call}]
\end{lstlisting}
\end{minipage}\linv
\end{figure}


\begin{figure}[!ht]
    \centering
\begin{minipage}[c]{0.45\textwidth}
\begin{python1}  
sets: [{1,2,3}, {1,2,4}, {1}], tnvsl: (_,_,_)
  sets: [{2,3}, {2,4}, {}], tnvsl: (_,_,1)
    sets: [{3}, {4}, {}], tnvsl: (2,_,1)
      sets: [{3}, {}, {}], tnvsl: (2,4,1)
=>  (2, 4, 1)
    sets: [{2}, {2,4}, {}], tnvsl: (3,_,1)
      sets: [{}, {4}, {}], tnvsl: (3,2,1)
=>  (3, 2, 1)
      sets: [{2}, {2}, {}], tnvsl: (3,4,1)
=>  (3, 4, 1)
\end{python1}\linv
\begin{lstlisting} [caption={\textit{tnvsl\_dfs\_gen trace}}, label={lis:dfs-gen-trace}]
\end{lstlisting}\inv
\end{minipage}\linv
\end{figure}

Some comments on Listing \ref{lis:dfs-gen}.  % \textit{tnvsl\_dfs\_gen}.
\begin{itemize}
    \item The newly added \textbf{else} on line 5 is necessary. Previously, if there were no \textit{remaining\_indices}, we returned \textit{tnvsl}. That was the end of execution for this recursive call. But if we \textbf{yield} instead of \textbf{return}, when \textit{tnvsl\_dfs\_gen} is asked for more results, \textit{it continues with the line after the \textbf{yield}}. But if have already found a transversal, we don't want to continue. The \textbf{else} divides the code into two mutually exclusive components. \textbf{return} had done that implicitly.
    
    \item Lines 17-20 call \textit{tnvsl\_dfs\_gen} recursively and ask for all the transversals that can be constructed from the current state. Each one is then \textbf{yield}ed. No need to exclude \textbf{None}. \textit{tnvsl\_dfs\_gen} will \textbf{yield} only complete transversals. 
    
    \smallv
Lines 17-20 can be replaced by this single line.
\end{itemize}
\begin{center}
\begin{minipage}[c]{0.45\textwidth}
\begin{python1}
    yield from tnvsl_dfs_gen(new_sets, new_tnvsl)
\end{python1}
\end{minipage}   
\end{center}

Let's use \textit{tnvsl\_dfs\_gen} (Listing \ref{lis:dfs-gen}) to solve our initial problem: find a transversal whose elements sum to, say, 6.

\begin{center}
\begin{minipage}[c]{0.45\textwidth}
\begin{python1}
n = 6
for tnvsl in tnvsl_dfs_gen(sets, ('_','_','_')):
  sum_string = ' + '.join(str(i) for i in tnvsl)
  equals = '==' if sum(tnvsl) == n else '!='
  print(f'{sum_string} {equals} {n}')
  if sum(tnvsl) == n: break
\end{python1}\linv
\begin{lstlisting} [caption={\textit{running tnvsl\_dfs\_gen}}, label={lis:dfs-gen-call2}]
\end{lstlisting}
\end{minipage}
\end{center}

The output (without trace) will be as follows.
\begin{center}
\begin{minipage}[c]{0.45\textwidth}
\begin{python1}  
    2 + 4 + 1 != 6
    3 + 2 + 1 == 6
\end{python1}\linv
\begin{lstlisting} [caption={\textit{tnvsl\_dfs\_gen trace}}]
\end{lstlisting}
\end{minipage}
\end{center}
We generated transversals until we found one whose elements summed to 6. Then we stopped.






%%%%%%%%%%%%%%%%%%%%%%%%%%%%%%%%%%%%%%%%%%%%%%
\section{Control functions (Listings in Appendix \ref{appsec:control_functions})} \label{sec:control_functions}  %\label{sec:control}
%%%%%%%%%%%%%%%%%%%%%%%%%%%%%%%%%%%%%%%%%%%%%%

This section discusses Prolog's control flow and explains how Pylog implements it. It also presents a number of Pylog control-flow functions.

%%%%%%%%%%%%%%%%%%%%%%%%%%%%%%%%%%%%%%%%%%%%%%
\subsection{Control flow in Prolog (Listings in Appendix \ref{appsubsec:control_flow_prolog})} \label{subsec:control_flow_prolog}
%%%%%%%%%%%%%%%%%%%%%%%%%%%%%%%%%%%%%%%%%%%%%%

Prolog, or at least so-called "pure" Prolog, is a satisfiability theorem prover turned into a programming language. One supplies a Prolog execution engine with (a) a "query" or "goal" term along with (b) a database of terms and clauses and asks whether values for variables in the query/goal term can be found that are consistent with the database. The engine conducts a depth-first search looking for such values. 

Once released as a programming language, programmers used Prolog in a wide variety of applications, not necessarily limited to establishing satisfiability. 

An important feature of Prolog is that it distinguishes far more sharply than most programming languages between data flow and control flow. 
\begin{enumerate}
    \item By \textit{control flow} we mean the mechanisms that determine the order in which program elements are executed or evaluated. This section discusses Pylog control flow.

    \item By \textit{dataflow} we mean the mechanisms that move data around within a program. Section \ref{sec:logic_variables} discusses how data flows through a Prolog program via logic variables and how Pylog implements logic variables.
\end{enumerate}

The fundamental control flow control mechanisms in most programming languages involve (a) sequential execution, i.e., one statement or expression following another in the order in which they appear in the source code, (b) conditional execution, e.g., \textbf{if} and related statements or expressions, (c) repeated execution, e.g., \textbf{while} statements or similar constructs, and (d) the execution/evaluation of sub-portions of a program such as functions and procedures via method calls and returns. 

Even declarative programming languages, such a Prolog, include explicit or implicit means to control the order of execution. That holds even when the language includes lazy evaluation, in which an expression is evaluated only when its value is needed. 

Whether or not the language designers intended this to happen, programmers can generally learn how the execution/evaluation engine of a programming language works and write code to take advantage of that knowledge. This is not meant as a criticism. It's a simple consequence of the fact that computers---at least traditional, single-core computers---do one thing at a time, and programmers can design their code to exploit that ordering. 

Prolog, especially the basic Prolog this paper is considering, offers a straight-forward control-flow framework: lazy, backtracking, depth-first search. Listing \ref{lis:prologInterpreter} (See Bartak \cite{Bartak1998}) shows a simple Prolog interpreter written in Prolog. The code is so simple because unification and backtracking can be taken for granted!

% \begin{minipage}{\linewidth}  \largev \hrulefill
% \begin{python}[numbers=left]
% solve([]).
% solve([Term|Terms]):-
%   clause(Term, Body), 
%   append(Body, Terms, New_Terms), 
%   solve(New_Terms).
% \end{python}
% \begin{lstlisting} [caption={A prolog interpreter in prolog},  label={lis:prologInterpreter}]
% \end{lstlisting}
% \end{minipage}

The execution engine, here represented by the \textit{solve} predicate, starts with a list containing the query/goal term, typically with one or more uninstantiated variables. It then looks up and unifies, if possible, that term with a compatible term in the database (line 3). If unification is successful, the possibly empty body of the clause is appended to the list of unexamined terms (line 4), and the engine continues to work its way through that list. Should the list ever become empty (line 1), \textit{solve} terminates successfully. The typically newly instantiated variables in the query contain the information returned by the program's execution.

If unification with a term in the database (line 3) is not possible, the program is said to have \textit{fail}ed (for the current execution path). The engine then backs up to the most recent point where it had made a choice. This typically occurs at line 3 where we are looking for a clause in the database with which to unify a term. If there are multiple such clauses, another one is selected. If that term leads to a dead end, \textit{solve} tries another of the unifiable terms.

In short, terms either \textit{succeed} in unifying with a database term,\footnote{Operations such as arithmetic, may also fail and result in backtracking.} or they \textit{fail}, in which case the engine backtracks to the most recent choicepoint. This is standard depth-first search---as in \textit{trvsl\_dfs\_first}. In addition, when the engine makes a selection at a choicepoint, it retains the ability to produce other possible selections---as in \textit{tvsl\_yield}. The engine may be \textit{lazy} in that it generates possible selections as needed. 

Even when \textit{solve} empties its list of terms, it retains the ability to backtrack and explore other paths. This capability enables Prolog to generate multiple answers to a query (but one at a time), just as \textit{tvsl\_yield} is able to generate multiple transversals, but again, one at a time when requested.

Prolog often seems strange in that lazy backtracking search is the one and only mechanism Prolog (at least pure prolog) offers for controlling program  flow. Although backtracking depth-first search itself is familiar to most programmers, lazy backtracking search may be less familiar. When writing Prolog code, one must get used to a world in which program flow is defined by lazy backtracking search.

%%%%%%%%%%%%%%%%%%%%%%%%%%%%%%%%%%%%%%%%%%%%%%
\subsection{Prolog control flow in Pylog (Listings in Appendix \ref{appsubsec:control_flow_pylog})} \label{subsec:control_flow_pylog}
%%%%%%%%%%%%%%%%%%%%%%%%%%%%%%%%%%%%%%%%%%%%%%

Prolog's lazy, backtracking, depth-first search is built on a mechanism that keeps track of unused choicepoint elements \textit{even after a successful element has been found}. Let's compare the relevant lines of \textit{tvsl\_dfs\_first} (Listing \ref{lis:dfsfirstelse}) and \textit{tvsl\_yield} (Listing \ref{lis:yieldelse}). We are interested in the \textbf{else} arms of these programs.

% \begin{minipage}{\linewidth} \largev   \hrulefill
% \begin{python}[numbers=left]
%     for element in sets[0]:
%       if element not in partial_transversal:
%         complete_transversal = tvsl_dfs_first(sets[1:], partial_transversal + (element, ))
%         if complete_transversal is not None:
%           return complete_transversal 
%     return None
% \end{python}
% \begin{lstlisting} [caption={The \textbf{else} branch of \textittt{tvsl\_dfs\_first}}, label={lis:dfsfirstelse}]
% \end{lstlisting}
% \end{minipage}


% \begin{minipage}{\linewidth} \largev   \hrulefill
% \begin{python}[numbers=left]
%     for element in sets[0]:
%       if element not in partial_transversal:
%         yield from tvsl_yield(sets[1:], partial_transversal + (element, ))
% \end{python}
% \begin{lstlisting} [caption={The \textbf{else} branch of \textittt{tvsl\_yield}}, label={lis:yieldelse}]
% \end{lstlisting}
% \end{minipage}

In both cases, the choicepoint elements are the members of \textit{sets[0]}. (Recall that \textit{sets} is a list of sets; \textit{sets[0]} is the first set in that list. The choicepoint elements are the members of \textit{sets[0]}.) 

The first two lines of the two code segments are identical: define a \textbf{for}-loop over \textit{sets[0]}; establish that the selected element is not already in the partial transversal.

The third line adds that element to the partial transversal and asks the transversal program (\textit{tvsl\_dfs\_first} or \textit{tvsl\_yield}) to continue looking for the rest of the transversal. 

Here's where the two programs diverge.
\begin{itemize}
    \item In \textit{tvsl\_dfs\_first}, if a complete transversal is found, i.e., if something other than \textbf{None} is returned, that result is returned to the caller. The loop over the choicepoints terminates when the program exits the function via \textbf{return} on line 5.
    
    \item In \textit{tvsl\_yield}, if a complete transversal is found, i.e., if \textbf{yield from} returns a result, that result is \textbf{yield}ed back to the caller. But \textit{tvsl\_yield} does \textit{not} exit the loop over the choicepoints. The visible structure of the code suggests that perhaps the loop might somehow continue, i.e., that \textbf{yield} might not terminate the loop and exit the function the way \textbf{return} does. How can one return a value but allow for the possibility that the loop might resume? That's the magic of Python generators, the subject of the next section. 
\end{itemize}

%%%%%%%%%%%%%%%%%%%%%%%%%%%%%%%%%%%%%%%%%%%%%%
\subsection{A review of Python generators (Listings in Appendix \ref{appsubsec:generators})} \label{subsec:generators}
%%%%%%%%%%%%%%%%%%%%%%%%%%%%%%%%%%%%%%%%%%%%%%

This paper is not about Python generators. We assume readers are already familiar with them. Even so, because they are so central to Pylog, we offer a brief review.

\largev
Any Python function that contains \textbf{yield} or \textbf{yield from} is considered a generator. This is a black-and-white decision made by the Python compiler. Nothing is required to create a generator other than to include \textbf{yield} or \textbf{yield from} in the code.

So the question is: how do generators work operationally?

Using a generator requires two steps.
\begin{enumerate}
    \item Initialize the generator, essentially by calling it as a function. Initialization does \textit{not} run the generator. Instead, the generator function returns a generator object. That generator object can be activated (or reactivated) as in the next step.
    
    \item Activate (or reactivate) a generator object by calling \textit{next} with the generator object as a parameter. When a generator is activated by \textit{next}, it runs until it reaches a \textbf{yield} or \textbf{yield from} statement. Like \textbf{return}, a \textbf{yield} statement may optionally include a value to be returned to the \textit{next}-caller. Whether or not a value is sent back to the \textit{next}-caller, a generator that encounters a \textbf{yield} stops running (much like a traditional function does when it encounters \textbf{return}). 
    
    \smallv
    Generators differ from traditional functions in that when a generator encounters \textbf{yield} \textit{it retains its state}. On a subsequent \textit{next} call, the generator resumes execution at the line after the \textbf{yield} statement.
    
    \smallv
    In other words, unlike functions, which may be understood to be associated with a stack frame---and which may be understood to have their stack frame discarded when the function encounters \textbf{return}---generator frames are maintained independently of the stack of the program that executes the \textit{next} call.
    
    \smallv
    This allows generators to be (re-)activated repeatedly via multiple \textit{next} calls.
\end{enumerate}
        
    
% \begin{minipage}{\linewidth}  \largev  \hrulefill  
% \begin{python}
% def find_number(search_number):
%     i = 0
%     while True:
%         i += 1
%         if i == search_number:
%             print("\nFound the number:", search_number)
%             return
%         else:
%             yield i

% search_number = 5
% find_number_object = find_number(search_number)
% while True:
%     k = next(find_number_object)
%     print(f'{k} is not {search_number}')
% \end{python}
% \begin{lstlisting} [caption={\textittt{Generator example}},  label={lis:generatorExample1}]
% \end{lstlisting}
% \end{minipage}

Consider the simple example shown in Listing \ref{lis:generatorExample1}. When executed, the result will be as shown in Listing \ref{lis:generatorExample2}.

% \begin{minipage}{\linewidth}  \largev  \hrulefill  
% %\begin{python}
% \begin{verbatim}
% 1 is not 5
% 2 is not 5
% 3 is not 5
% 4 is not 5

% Found the number: 5

% Traceback (most recent call last):
%   <line number where error occurred> 
%     k = next(find_number_object)
% StopIteration

% Process finished with exit code 1
% \end{verbatim}
% %\end{python}
% \begin{lstlisting} [caption={\textittt{Generator example}},  label={lis:generatorExample2}]
% \end{lstlisting}
% \end{minipage}

As \textit{find\_number} runs through 1 .. 4 it \textbf{yield}s them to the \textit{next}-caller at the top level, which prints that they are not the search number. But note what happens when \textit{find\_number} finds the search number. It executes \textbf{return} instead of \textbf{yield}. This produces a \textit{StopIteration} exception---because as a generator, \textit{find\_number} is expected to \textbf{yield}, not \textbf{return}. If the \textit{next}-caller does not handle that exception, as in this example, the exception propagates to the top level, and the program terminates with an error code. 

Python's \textbf{for}-loop catches \textit{StopIteration} exceptions and simply terminates. If we replaced the \textbf{while}-loop in Listing \ref{lis:generatorExample1} with 
\begin{python}
for k in find_number(search_number):
    print(f'{k} is not 5')
\end{python}
the output would be identical except that instead of terminating with a \textit{StopIteration} exception, we would terminate normally.

Notice also that the \textbf{for}-loop generates the generator object. The step that produces \textit{find\_number\_object} (originally line 12) occurs when the  \textbf{for}-loop begins execution.

\smallv
\textbf{yield from} also catches \textit{StopIteration} exceptions. Consider adding an intermediate function that uses \textbf{yield from} as in Listing \ref{lis:yieldfromExample}.\footnote{An intermediate function is required because \textbf{yield} and \textbf{yield from} may be used only within a function. We can't just put \textbf{yield from} inside the top-level \textbf{for}-loop.}\footnote{This example was adapted from \href{https://www.python-course.eu/python3_generators.php}{\underline{this generator tutorial}}.} The result is similar to the previous---but with no uncaught exceptions. See Listing\ref{lis:yieldFromExampleOutput}. 

% \begin{minipage}{\linewidth}  \largev  \hrulefill  
% \begin{python}
% def use_yield_from():
%     yield from find_number_object
%     print('find_number failed, but "yield from" caught the Stop Iteration exception.')
%     return

% for k in use_yield_from():
%     print(f'{k} is not 5')
% \end{python}
% \begin{lstlisting} [caption={\textittt{yield from example}},  label={lis:yieldfromExample}]
% \end{lstlisting}
% \end{minipage}
% The result is similar to the previous---with no uncaught exceptions. See Listing\ref{lis:yieldFromExampleOutput}. 

% \begin{minipage}{\linewidth}  \largev  \hrulefill  
% \begin{verbatim}
% 1 is not 5
% 2 is not 5
% 3 is not 5
% 4 is not 5
% Found the number: 5
% find_number failed, but "yield from" caught the Stop Iteration exception.

% Process finished with exit code 0
% \end{verbatim}
% \begin{lstlisting} [caption={\textittt{yield from example output}},  label={lis:yieldFromExampleOutput}]
% \end{lstlisting}
% \end{minipage}

Note that when \textit{find\_number} fails in Listing \ref{lis:yieldfromExample}, i.e., when  \textit{find\_number} does not perform a \textbf{yield}, the \textbf{yield from} line in \textit{use\_yield\_from} does not perform a yield. Instead it goes on to its next line and prints the \textit{find\_number failed} message. It then terminates without performing a \textbf{yield}, producing a \textit{StopInteration} exception. The top-level \textbf{for}-loop catches that exception and terminates normally. 

In short, because Python generators maintain state after performing a \textit{yield}, they can be used to model Prolog backtracking.

%%%%%%%%%%%%%%%%%%%%%%%%%%%%%%%%%%%%%%%%%%%%%%
\subsection{\textbf{yield} : \textit{succeed} :: \textit{return} : \textit{fail} (Listings in Appendix \ref{appsubsec:yield_succeed})} \label{subsec:yield_succeed}
%%%%%%%%%%%%%%%%%%%%%%%%%%%%%%%%%%%%%%%%%%%%%%
Generators perform an additional service. Recall that Prolog predicates either \textit{succeed} or \textit{fail}. In particular when a Prolog predicate fails, it does not return a negative result---recall how \textit{tvsl\_dfs\_first} returned \textbf{None} when it failed to complete a transversal. Instead, a failed predicate simply terminates the current execution path. The Prolog engine then backtracks to the most recent choicepoint.

Similarly, if a generator terminates, i.e., \textbf{return}s, before encountering a \textbf{yield}, it generates a \textit{StopIteration} exception. The \textit{next}-caller typically interprets that to indicate the equivalent of failure. In this way Prolog's succeed and fail map onto generator \textbf{yield} and \textbf{return}. This makes it fairly straightforward to write generators that mimic Prolog predicates.

\begin{itemize}
    \item A Pylog generator \textit{succeeds} when it performs a \textbf{yield}. 
    \item A Pylog generator \textit{fails} when it \textbf{return}s without performing a \textbf{yield}. 
\end{itemize}

Generators provide a second parallel construct. Multiple-clause Prolog predicates map onto a Pylog function with multiple \textbf{yield}s in a single control path. The generic prolog structure as shown in Listing \ref{lis:prologmultipleclauses} can be implemented as shown in Listing \ref{lis:pylogmultipleyields}.

% \begin{minipage}{\linewidth}  \largev  \hrulefill  
% \begin{python}
% head :- body_1.
% head :- body_2.
% \end{python}
% \begin{lstlisting} [caption={Prolog multiple clauses},  label={lis:prologmultipleclauses}]
% \end{lstlisting}
% \end{minipage}

% can be implemented as shown in Listing \ref{}.

% \begin{minipage}{\linewidth}  \largev  \hrulefill  
% \begin{python}
% def head():
%     <some code>
%     yield
    
%     <other code>
%     yield
% \end{python}
% \begin{lstlisting} [caption={Pylog multiple sequential yields},  label={lis:pylogmultipleyields}]
% \end{lstlisting}
% \end{minipage}

Prolog's \textbf{cut} ('!') (Listing \ref{lis:prologmultipleclauseswithcut}) corresponds to a Python \textbf{if}-\textbf{else} structure (Listing \ref{lis:pylogmultipleclauseswithifelse}). The two \textbf{yield}s are in separate arms of an \textbf{if}-\textbf{else} construct.

% \begin{minipage}{\linewidth}  \largev  \hrulefill  
% \begin{python}
% head :- !, body_1.
% head :- body_2.
% \end{python}
% \begin{lstlisting} [caption={Prolog multiple clauses with a cut},  label={lis:prologmultipleclauseswithcut}]
% \end{lstlisting}
% \end{minipage}

% can be implemented as follows. The two \textbf{yield}s are in separate arms of an \textbf{if}-\textbf{else} construct.

% \begin{minipage}{\linewidth}  \largev  \hrulefill  
% \begin{python}
% def head():
%     if <condition>:
%       <some code>
%       yield
%     else
%       <other code>
%       yield
% \end{python}
% \begin{lstlisting} [caption={Multiple Pylog \textbf{yield}s in separate \textbf{if}-\textbf{else} arms},  label={lis:pylogmultipleclauseswithifelse}]
% \end{lstlisting}
% \end{minipage}

The control-flow functions discussed in Section \ref{subsec:controlfunctions} along with the \textit{append} function discussed in Section \ref{subsec:append} offer numerous examples.

\largev
Python's generator system has many more features than those covered above. But these are the ones on which Pylog depends. 

%%%%%%%%%%%%%%%%%%%%%%%%%%%%%%%%%%%%%%%%%%%%%%
\subsection{Control functions (Listings in Appendix \ref{appsubsec:controlfunctions})} \label{subsec:controlfunctions}
%%%%%%%%%%%%%%%%%%%%%%%%%%%%%%%%%%%%%%%%%%%%%%

Pylog offers the following control functions. (It's striking the extent to which generators make implementation straight-forward.)

\begin{itemize}
    \item \textit{fails} (Listing \ref{lis:fails}). A function that may be applied to a function. The resulting function succeeds if and only if the original fails.
    
    \item \textit{forall} (Listing \ref{lis:forall}). Succeeds if all the generators in its argument list succeed.
    
    \item \textit{forany} (Listing \ref{lis:forany}). Succeeds if any of the generators in its argument list succeed. On backtracking, tries them all.
    
    \item \textit{trace} (Listing \ref{lis:trace}). May be included in a list of generators (as in \textit{forall} and \textit{forany}) to log progress. The second argument determines whether \textit{trace} succeeds or fails. The third argument turns printing on or off. When included in a list of \textit{forall} generators, \texttt{succeed} should be set to \textbf{True} so that it doesn't prevent \textit{forany} from succeeding. When included in a list of \textit{forany} generators, \texttt{succeed} should be set to \textbf{False} so that \textit{forany} won't take \textit{trace} as an extraneous success.
    
    \item \textit{would\_succeed} (Listing \ref{lis:wouldsucceed}). Like Prolog's double negative, $\backslash\!\!+ \backslash+$. \textit{would\_succeed} is applied to a function. The resulting function succeeds/fails if and only if the original function succeeds/fails. If the original function succeeds, this also succeeds but without binding any variables. 
    
    \item \textit{Bool\_Yield\_Wrapper}. A class whose instances are generators that can be used in \textbf{while}-loops. \textit{Bool\_Yield\_Wrapper} instances may be created via a \textit{bool\_yield\_wrapper} decorator. The decorator returns a function that instantiates \textit{Bool\_Yield\_Wrapper} with the decorated function along with its desired arguments. The decorator is shown in Listing \ref{lis:boolYieldWrapper}.

% Here's the decorator. The class itself is too long to include in this paper. Along with the rest of the code, it's available on \href{https://github.com/RussAbbott/pylog}{\underline{this GitHub repository}}.

% \begin{minipage}{\linewidth}  \largev \hrulefill
% \begin{python}[numbers=left]
% def bool_yield_wrapper(gen):
%   """
%   A decorator. Produces a function that generates a Bool_Yield_Wrapper object. 
%   """
%   def wrapped_func(*args, **kwargs):
%     return Bool_Yield_Wrapper(gen(*args, **kwargs))

%   return wrapped_func
% \end{python}
% \begin{lstlisting} [caption={bool\_yield\_wrapper},  label={lis:boolYieldWrapper}]
% \end{lstlisting}
% \end{minipage}

\smallv
The example in Listing \ref{lis:boolYieldWrapperExample} uses \textit{bool\_yield\_wrapper} twice, once as a decorator and once as a function that can be applied directly to other functions. The example also uses the \textit{unify} function (see Section \ref{subsec:unify} below). 

% \begin{minipage}{\linewidth}  \largev \hrulefill
% \begin{python}[numbers=left]
%   @bool_yield_wrapper
%   def squares(n: int, X2: Var) -> Bool_Yield_Wrapper:
%     for i in range(n):
%       unify_gen = bool_yield_wrapper(unify)(X2, i**2)
%       while unify_gen.has_more():
%         yield

%   Square = Var()
%   squares_gen = squares(5, Square)
%   while squares_gen.has_more():
%     print(Square)
% \end{python}
% \begin{lstlisting} [caption={bool\_yield\_wrapper example},  label={lis:boolYieldWrapperExample}]
% \end{lstlisting}
% \end{minipage}



% \begin{minipage}{\linewidth}  \largev \hrulefill
% \begin{python}[numbers=left]
% def fails(f):
%   """
%   Applied to a function so that the resulting function succeeds if and only if the original fails.
%   Note that fails is applied to the function itself, not to a function call.
%   Similar to a decorator but applied explicitly when used.
%   """
%   def fails_wrapper(*args, **kwargs):
%     for _ in f(*args, **kwargs):
%       # Fail, i.e., don't yield, if f succeeds
%       return  
%     # Succeed if f fails.
%     yield     

%   return fails_wrapper
% \end{python}
% \begin{lstlisting} [caption={fails},  label={lis:fails}]
% \end{lstlisting}
% \end{minipage}

% \begin{minipage}{\linewidth}  \largev \hrulefill
% \begin{python}[numbers=left]
% def forall(gens):
%   """
%   Succeeds if all generators in the gens list succeed. The elements in the gens list
%   are embedded in lambda functions to avoid premature evaluation.
%   """
%   if not gens:
%     # They have all succeeded.
%     yield
%   else:
%     # Get gens[0] and evaluate the lambda expression to get a fresh iterator.
%     # The parentheses after gens[0] evaluates the lambda expression.
%     # If it succeeds, run the rest of the generators in the list.
%     for _ in gens[0]( ):
%       yield from forall(gens[1:])
% \end{python}
% \begin{lstlisting} [caption={forall},  label={lis:forall}]
% \end{lstlisting}
% \end{minipage}

% \begin{minipage}{\linewidth}  \largev \hrulefill
% \begin{python}[numbers=left]
% def forany(gens):
%   """
%   Succeeds if any of the generators in the gens list succeed. On backtracking, tries them all. 
%   The gens elements must be embedded in lambda functions.
%   """
%   for gen in gens:
%     yield from gen( )

% \end{python}
% \begin{lstlisting} [caption={forany},  label={lis:forany}]
% \end{lstlisting}
% \end{minipage}

% \begin{minipage}{\linewidth}  \largev \hrulefill
% \begin{python}[numbers=left]
% def trace(x, succeed=True, show_trace=True):
%   """
%   Can be included in a list of generators (as in forall and forany) to see where we are.
%   The second argument determines whether trace succeeds or fails. The third turns printing on or off.
%   When included in a list of forall generators, succeed should be set to True so that
%   it doesn't prevent forall from succeeding.
%   When included in a list of forany generators, succeed should be set to False so that forany
%   will go on the the next generator and won't take trace as an extraneous successes.
%   """
%   if show_trace:
%     print(x)
%   if succeed:
%     yield

% \end{python}
% \begin{lstlisting} [caption={trace},  label={lis:trace}]
% \end{lstlisting}
% \end{minipage}


% \begin{minipage}{\linewidth}  \largev \hrulefill
% \begin{python}[numbers=left]
% def would_succeed(f):
%   """
%   Applied to a function so that the resulting function succeeds/fails if and only if the original
%   function succeeds/fails. If the original function succeeds, this also succeeds but without 
%   binding any variables. Similar to a decorator but applied explicitly when used.
%   """
%   def would_succeed_wrapper(*args, **kwargs):
%     succeeded = False
%     for _ in f(*args, **kwargs):
%       succeeded = True
%       # Do not yield in the context of f succeeding.
      
%     # Exit the for-loop so that unification will be undone.
%     if succeeded:
%       # Succeed if f succeeded.
%       yield  
%     # The else clause is redundant. It is included here for clarity.
%     # else:
%     #   Fail if f failed.
%     #   pass   

%   return would_succeed_wrapper

% \end{python}
% \begin{lstlisting} [caption={would\_succeed},  label={lis:wouldsucceed}]
% \end{lstlisting}
% \end{minipage}

% \largev
% Finally, there is a more complexly implemented control structure, which is defined as an instance of the \textittt{Bool\_Yield\_Wrapper} class. Instances of the class are generators that can be used in \textbftt{while} loops.

% \textittt{Bool\_Yield\_Wrapper} instances may be created via a \textittt{bool\_yield\_wrapper} decorator. The decorator returns a function that instantiates \textittt{Bool\_Yield\_Wrapper} with the decorated function along with its desired arguments. 

% Here's the decorator. The class itself is too long to include in this paper. Along with the rest of the code, it's available on \href{https://github.com/RussAbbott/pylog}{\underline{this GitHub repository}}.

% \begin{minipage}{\linewidth}  \largev \hrulefill
% \begin{python}[numbers=left]
% def bool_yield_wrapper(gen):
%   """
%   A decorator. Produces a function that generates a Bool_Yield_Wrapper object. 
%   """
%   def wrapped_func(*args, **kwargs):
%     return Bool_Yield_Wrapper(gen(*args, **kwargs))

%   return wrapped_func
% \end{python}
% \begin{lstlisting} [caption={bool\_yield\_wrapper},  label={lis:boolYieldWrapper}]
% \end{lstlisting}
% \end{minipage}

% The following example uses \textit{bool\_yield\_wrapper} twice, once as a decorator and once as a function that can be applied directly to other functions. The example also uses the \textit{unify} function (see Section \ref{subsec:unify} below). 

% \begin{minipage}{\linewidth}  \largev \hrulefill
% \begin{python}[numbers=left]
%   @bool_yield_wrapper
%   def squares(n: int, X2: Var) -> Bool_Yield_Wrapper:
%     for i in range(n):
%       unify_gen = bool_yield_wrapper(unify)(X2, i**2)
%       while unify_gen.has_more():
%         yield

%   Square = Var()
%   squares_gen = squares(5, Square)
%   while squares_gen.has_more():
%     print(Square)
% \end{python}
% \begin{lstlisting} [caption={bool\_yield\_wrapper example},  label={lis:boolYieldWrapperExample}]
% \end{lstlisting}
% \end{minipage}

\smallv
The output, as expected, is the first five squares.

% \begin{minipage}{\linewidth}  \largev \hrulefill
% \begin{python}
% 0
% 1
% 4
% 9
% 16
% \end{python}
% \begin{lstlisting} [caption={bool\_yield\_wrapper example output},  label={lis:boolYieldWrapperExampleOutput}]
% \end{lstlisting}
% \end{minipage}

\smallv
Note that the \textbf{while}-loop on line 6 succeeds exactly once---because \textit{unify} succeeds exactly once. The \textbf{while}-loop on line 10 succeeds 5 times.

\smallv
An advantage of this approach is that it avoids the \textbf{for} loop. Notwithstanding our earlier discussion, \textbf{for}-loops don't feel like the right structure for backtracking. 

\smallv
A disadvantage is its wordiness. Extra lines of code (lines 4 and 9) to are needed to create the generator. One itches to get rid of them, but we were unable to do so. 

\smallv
Note that 
\begin{python}
    while squares(5, Square).has_more():
\end{python}

does not work. The \textbf{while}-loop uses the entire expression as its condition, thereby creating a new generator each time around the loop. 

\smallv
Caching the generator has the difficulty that one may want the same generator, with the same arguments, in multiple places. In practice, we found ourselves using the \textbf{for} construct most of the time.

\end{itemize}


%%%%%%%%%%%%%%%%%%%%%%%%%%%%%%%%%%%%%%%%%%%%%%
\section{Logic variables} \label{sec:logic-variables}
%%%%%%%%%%%%%%%%%%%%%%%%%%%%%%%%%%%%%%%%%%%%%%
This section discusses logic variables and their realization. 

\subsection{Instantiation}
Logic variables are either instantiated, i.e., have a value, or uninstantiated. The instantiation operation is called \textit{unify}.   

\textit{unify} is a \textit{generator}---even though the act of instantiation \textit{does not} \textbf{yield} a value. Activating \textit{unify} establishes a context within which unification holds. Leaving that context undoes the unification. 

Consider the following sequence of short code segments. 
\begin{center}
\begin{minipage}[c]{0.45\textwidth}
\begin{python1}
A = Var()
\end{python1}
\end{minipage}
\end{center}
\textit{A} is a standard Python identifier. We use an initial capital letter to distinguish logic variables from regular Python variables. \textit{Var} is the constructor for logic variables.

\textit{A} is now an uninstantiated logic variable. When an uninstantiated logic variables is printed, an internal value is shown to distinguish among logic variables. As the first logic variable in this program, \textit{A}'s internal value is \textit{\_1}.

\begin{center}
\begin{minipage}[c]{0.45\textwidth}
\begin{python1}
print(A)  # => _1
\end{python1}
\end{minipage}
\end{center}

Now we \textit{unify A} with \textit{abc}, i.e., instantiate \textit{A} to \textit{abc}. Since \textit{unify} does not \textbf{yield} a value, the \textbf{for}-loop variable is not used. 

The \textbf{for}-loop establishes a context for \textit{unify}. Within the \textbf{for}-loop body \textit{A} is instantiated to  \textit{abc}.

\begin{center}
\begin{minipage}[c]{0.45\textwidth}
\begin{python1}
for _ in unify(A, 'abc'):
    print(A)  # => abc
print(A)  # => _1
\end{python1}
\end{minipage}
\end{center}

Since there is only one way to \textit{unify A} with \textit{abc}, the  \textbf{for}-loop body runs only once.  Leaving the \textit{unify} context undoes the instantiation.

Within a \textit{unify} context, logic variables are immutable. Once a logic variable has a value, it cannot change within its context.

\begin{center}
\begin{minipage}[c]{0.45\textwidth}
\begin{python1}
A = Var()
print(A)  # => _1
for _ in unify(A, 'abc'):
    print(A)  # => abc
    # This unify fails. Its body never runs.
    for _ in unify(A 'def'):
      print(A)  # Never executed
    print(A)  # => abc
print(A)  # => _1
\end{python1}
\end{minipage}
\end{center}

\subsection{The power of \textit{unify}}
\textit{unify} can also identify logic variables with each other. After two uninstantiated logic variables are unified, whenever either gets a value, the other gets that same value.

Unification is surprisingly straightforward. Each \textit{Var} includes a \textit{next} field, which is initially \textbf{None}. When two \textit{Var}s are unified, the result depends on their states of instantiation.  
\begin{itemize}
    \item If both are uninstantiated the \textit{next} field of one points to the other. It makes no difference which points to which. A chain of linked  \textit{Var}s unifies all the \textit{Var}s in the chain. 
    \item If only one is uninstantiated, the uninstantiated one points to the other.  
    \item If both are instantiated to the same value, they are effectively unified. \textit{unify succeeds} but nothing changes.
    \item If both are instantiated but to different values, \textit{unify fails}.
\end{itemize}

A note on terminology. When called (as part of a \textbf{for}-loop) a generator will either \textbf{yield} or \textbf{return}. When a generator \textbf{yield}s, it is said to \textit{succeed}; the \textbf{for}-loop body runs. When a generator \textbf{return}s, it is said to fail; the \textbf{for}-loop body does not run. Instead we exit the \textbf{for}-loop.

We can trace the unifications in Listing \ref{unif-example}.  

% \begin{figure}[hbt]
% \centering
\begin{center}
\begin{minipage}[c]{0.45\textwidth}
\begin{python1}
(A, B, C, D) = (Var(), Var(), Var(), Var())
print(A, B, C, D) # => _1 _2 _3 _4
for _ in unify(A, B):
  for _ in unify(D, C):
    print(A, B, C, D) # => _2 _2 _3 _3
    for _ in unify(A, 'abc'):
      print(A, B, C, D) # => abc abc _3 _3
      for _ in unify(A, D):
        print(A, B, C, D) # => abc abc abc abc
      print(A, B, C, D) # => abc abc _3 _3
    print(A, B, C, D) # => _2 _2 _3 _3
  print(A, B, C, D) # => _2 _2 _3 _4
print(A, B, C, D) # => _1 _2 _3 _4
\end{python1}\linv
\begin{lstlisting} [caption={\textit{Unification example}},  label={unif-example}]
\end{lstlisting}
\end{minipage}
\end{center}
% \end{figure}

The first unifications, lines 3 and 4, produce the following. 
\begin{equation}\label{eq:one}
\begin{array}{c c c c c c c c }
A & \to & B \\
D & \to & C 
\end{array}
\end{equation}

Line 6 unifies \textit{A} and \textit{'abc'}. The first step is to go to the ends of the relevant unification chains. In this case, \textit{B} (the end of \textit{A}'s unification chain) is pointed to \textit{'abc'}. Since  \textit{'abc'} is instantiated, the arrow can only go from \textit{B} to \textit{'abc'}. 

\begin{equation}\label{eq:two}
\begin{array}{c c c c c c c c c c c}
A & \to & B            & \to & 'abc'    \\ 
  &     & D            & \to & C        
\end{array}
\end{equation}

Finally, line 8  unifies \textit{A} with \textit{D}. \textit{C} (the end of \textit{D}'s unification chain) is set to point to \textit{'abc'} (the end of \textit{A}'s unification chain). % The arrow can go only from \textit{C} to \textit{'abc'}.

\begin{equation}\label{eq:three}
\begin{array}{c c c c c c c c c c c}
A & \to & B            & \to & 'abc'      \\ 
  &     &              &     & \uparrow   \\ 
  &     & D            & \to & C        
\end{array}
\end{equation}


% To determine the value of a logical variable, one goes to the end of its unification chain. If the end element is instantiated, that is the (current) value of the variable. That's why all the variables have \textit{abc} as their values after line 8. 

% If the end of a unification chain is unintantiated, the internal value associated with that end element is a place-holder value.

\smallv
\subsection{A logic-variable version of \textit{tnvsl\_dfs\_gen}}
Listing \ref{lis:dfs-with-gen-and-logic-variables} adapts Listing \ref{lis:dfs-gen} for logic variables. The strategy is for \textit{trnsvl} to start as a tuple of uninstantiated \textit{Var}s, which become instantiated as the program runs.

First, an adapted \textit{uninstan\_indices\_lv} returns the indices of the uninstantiated \textit{Var}s in \textit{trnsvl}.
\begin{center}
\begin{minipage}[c]{0.45\textwidth}
\begin{python1}
def uninstan_indices_lv(tnvsl):
  return [indx for indx in range(len(tnvsl)) 
               if not tnvsl[indx].is_instantiated()]
\end{python1}
\end{minipage}
\end{center}

Note that \textit{tnvsl[indx]} retrieves the \textit{indx\textsuperscript{th}} \textit{tnvsl} element. If it's instantiated, it represents the value associated with the \textit{indx\textsuperscript{th}} set. If not, we don't yet have a value for the  \textit{indx\textsuperscript{th}} set.

\begin{figure}[htb]
\centering
\begin{minipage}[c]{0.45\textwidth}
\begin{python1}
def tnvsl_dfs_gen_lv(sets, tnvsl):
  var_indxs = uninstan_indices_lv(tnvsl)
    
  if not var_indxs: yield tnvsl
  else:
    empty_sets = [sets[indx].is_empty() 
                  for indx in var_indxs]
    if any(empty_sets): return None

    nxt_indx = min(var_indxs,
                   key=lambda indx: len(sets[indx]))
    used_values = PyList([tnvsl[i] 
                          for i in range(len(tnvsl)) 
                          if i not in var_indxs])
    T_Var = tnvsl[nxt_indx]
      for _ in member(T_Var, sets[nxt_indx]):
        for _ in fails(member)(T_Var, used_values):
          new_sets = [set.discard(T_Var) 
                      for set in sets]
          yield from tnvsl_dfs_gen_lv(new_sets, 
                                      tnvsl)
\end{python1}\linv
\begin{lstlisting} [caption={\textit{dfs-with-gen-and-logic-variables}},  label={lis:dfs-with-gen-and-logic-variables}]
\end{lstlisting}
\end{minipage}\linv
\end{figure}

Some comments on Listing \ref{lis:dfs-with-gen-and-logic-variables}. (We reformatted some of the lines and changed some of the names from \textit{tnvsl\_dfs\_gen} (Listing \ref{lis:dfs-gen}) so that the program will fit the width of a column.)

\begin{itemize}
    \item \textit{line 6}. The parameter \textit{sets} is a list of \textit{PySet}s. These are logic variable versions of sets. An \textit{is\_empty} method is defined for them.
    \item \textit{lines 12-14}. \textit{used\_values} are the values of the instantiated \textit{tnvsl} elements.
    \item \textit{line 15}. \textit{T\_Var} is the element at the \textit{nxt\_indx\textsuperscript{th}} position of \textit{tnvsl}. Since \textit{nxt\_indx} was selected from the uninstantiated variables, \textit{T\_Var} is an uninstantited \textit{Var}.
    \item \textit{line 16}. \textit{member} successively unifies its first argument with the elements of its second argument. It's equivalent to \textit{\textbf{for} T\_Var \textbf{in} sets[nxt\_indx]} but using unification.
    \item  \textit{line 17}. \textit{fails} takes a predicate as its argument. It converts the predicate to its negation. So \textit{fails(member)} succeeds if and only if \textit{member} fails.
    \item  \textit{line 18}. \textit{PySet}s have a \textit{discard} method that returns a copy of the \textit{PySet} without the argument.
\end{itemize}

When run, we get the same result as before---except that the uninstantiated transversal variables appear as we saw above.
\begin{center}
\begin{minipage}[c]{0.45\textwidth}
\begin{python1}
sets: [{1,2,3}, {1,2,4}, {1}], tnvsl: (_1, _2, _3)
  sets: [{2,3}, {2,4}, {}], tnvsl: (_1, _2, 1)
    sets: [{3}, {4}, {}], tnvsl: (2, _2, 1)
      sets: [{3}, {}, {}], tnvsl: (2, 4, 1)
=> (2, 4, 1)
    sets: [{2}, {2,4}, {}], tnvsl: (3, _2, 1)
      sets: [{}, {4}, {}], tnvsl: (3, 2, 1)
=> (3, 2, 1)
      sets: [{2}, {2}, {}], tnvsl: (3, 4, 1)
=> (3, 4, 1)
\end{python1}
\end{minipage}
\end{center}

The following logic variable version of Listing \ref{lis:dfs-gen-call} will run \textit{tnvsl\_dfs\_gen\_lv} and produce the same result.

\begin{center}
\begin{minipage}[c]{0.45\textwidth}
\begin{python1}
(A, B, C) = (Var(), Var(), Var())
Py_Sets = [PySet(set) for set in sets]
# PyValue creates a logic variable constant.
N = PyValue(6)
for _ in tnvsl_dfs_gen_lv(Py_Sets, (A, B, C)):
  sum_string = ' + '.join(str(i) for i in (A, B, C))
  equals = '==' if A + B + C == N else '!='
  print(f'{sum_string} {equals} {N}')
  if A + B + C == N: break
\end{python1}
\end{minipage}
\end{center}

Here we created three logic variables,  \textit{A}, \textit{B}, and \textit{C} and passed them to \textit{tnvsl\_dfs\_gen\_lv} on line 5. Each time a transversal is found, the body of the \textbf{for}-loop is executed with the values to which \textit{A}, \textit{B}, and \textit{C} have been instantiated. 

The preceding offers some sense of what one can do with logic variables. The next section really puts them to work.

%%%%%%%%%%%%%%%%%%%%%%%%%%%%%%%%%%%%%%%%%%%%%%
\section{The Zebra Puzzle (Listings in Appendix \ref{appsec:zebra})}\label{sec:zebra}
%%%%%%%%%%%%%%%%%%%%%%%%%%%%%%%%%%%%%%%%%%%%%%

The Zebra Puzzle is a well known logic puzzle.

\begin{quotation}
There are five houses in a row. Each has a unique color and is occupied by a family of unique nationality. Each family has a unique favorite smoke, a unique pet, and a unique favorite drink. Fourteen clues (Listing \ref{lis:zebra_prolog}) provide additional constraints. \textit{Who has a zebra and who drinks water?}
\end{quotation}

%%%%%%%%%%%%%%%%%%%%%%%%%%%%%%%%%%%%%%%%%%%%%%
\subsection{The clues and a Prolog solution (Listings in Appendix \ref{appsubsec:clues})} \label{subsec:clues}
%%%%%%%%%%%%%%%%%%%%%%%%%%%%%%%%%%%%%%%%%%%%%%

One can easily write Prolog programs to solve this and similar puzzles.
\begin{itemize}
\item Represent a house as a Prolog \textit{house} term with the parameters corresponding to the indicated properties:
\begin{python}
   house(<nationality>, <cigarette brand>, <pet>, <drink>, <house color>)
\end{python}
\item Define the world as a list of five \textit{house} terms, with all fields initially uninstantiated.

\item Write the clues (Listing \ref{lis:zebra_prolog}) as more-or-less direct translations of the English.
\end{itemize}

% \begin{minipage}{\linewidth}
% \begin{python}
% zebra_problem(Houses) :-
%     Houses = [house(_, _, _, _, _), house(_, _, _, _, _), house(_, _, _, _, _), 
%               house(_, _, _, _, _), house(_, _, _, _, _)], 

%     % 1. The English live in the red house.
%     member(house(english, _, _, _, red), Houses), 

%     % 2. The Spanish have a dog.
%     member(house(spanish, _, dog, _, _), Houses), 

%     % 3. They drink coffee in the green house.
%     member(house(_, _, _, coffee, green), Houses), 

%     % 4. The Ukrainians drink tea.
%     member(house(ukranians, _, _, tea, _), Houses), 

%     % 5. The green house is immediately to the right of the white house.
%     nextto(house(_, _, _, _, white), house(_, _, _, _, green), Houses), 

%     % 6. The Old Gold smokers have snails.
%     member(house(_, old_gold, snails, _, _), Houses), 

%     % 7. They smoke Kool in the yellow house.
%     member(house(_, kool, _, _, yellow), Houses), 

%     % 8. They drink milk in the middle house.
%     Houses = [_, _, house(_, _, _, milk, _), _, _], 

%     % 9. The Norwegians live in the first house on the left.
%     Houses = [house(norwegians, _, _, _, _) | _], 

%     % 10. The Chesterfield smokers live next to the fox.
%     next_to(house(_, chesterfield, _, _, _), house(_, _, fox, _, _), Houses), 

%     % 11. They smoke Kool in the house next to the horse.
%     next_to(house(_, kool, _, _, _), house(_, _, horse, _, _), Houses), 

%     % 12. The Lucky smokers drink juice.
%     member(house(_, lucky, _, juice, _), Houses), 

%     % 13. The Japanese smoke Parliament.
%     member(house(japanese, parliament, _, _, _), Houses), 

%     % 14. The Norwegians live next to the blue house.
%     next_to(house(norwegians, _, _, _, _), house(_, _, _, _, blue), Houses), 
% \end{python}
% \begin{lstlisting} [caption={Zebra puzzle in Prolog},  label={lis:zebra_prolog}]
% \end{lstlisting}
% \end{minipage}

\largev
After the following adjustments, we can run this program online using SWI-Prolog. 
\begin{itemize}
    \item SWI-Prolog includes \textit{member} and \textit{nextto} predicates. SWI-Prolog's \textit{nextto} means in the order given, as in clue 5.

    \item SWI-Prolog does not include a predicate for \textit{next to} in the sense of clues 10, 11, and 14 in which the order is unspecified. But we can write our own, say, \textit{next\_to}.
    
\begin{python}
next_to(A, B, List) :- nextto(A, B, List).
next_to(A, B, List) :- nextto(B, A, List).
\end{python}

    \item Since none of the clues mentions either a zebra or water, we add the following.

\begin{minipage}{\linewidth}
\begin{python}

    % 15. (implicit). 
    member(house(_, _, zebra, _, _), Houses), 
    member(house(_, _, _, water, _), Houses).
\end{python}
\end{minipage}

\end{itemize}

When this program is run, we get an almost instantaneous answer---shown manually formatted in Listing \ref{lis:zebra_solution}. We can conclude that 
\begin{quote} 
\begin{quote} 
\textit{The Japanese have a zebra, and the Norwegians drink water}.
\end{quote} 
\end{quote} 

% \begin{minipage}{\linewidth}
% \begin{python}
% ?- zebra_problem(Houses).
% [    
%     house(norwegians, kool, fox, water, yellow), 
%     house(ukranians, chesterfield, horse, tea, blue), 
%     house(english, old_gold, snails, milk, red), 
%     house(spanish, lucky, dog, juice, white), 
%     house(japanese, parliament, zebra, coffee, green)     
% ]
% \end{python}
% \begin{lstlisting} [caption={Zebra puzzle in Prolog},  label={lis:zebra_solution}]
% \end{lstlisting}
% \end{minipage}

\smallv

% We can conclude that 
% \begin{quote} 
% \begin{quote} 
% \textit{The Japanese have a zebra, and the Norwegians drink water}.
% \end{quote} 
% \end{quote} 

%%%%%%%%%%%%%%%%%%%%%%%%%%%%%%%%%%%%%%%%%%%%%%
\subsection{A Pylog solution (Listings in Appendix \ref{appsubsec:pylog_solution})} \label{subsec:pylog_solution}
%%%%%%%%%%%%%%%%%%%%%%%%%%%%%%%%%%%%%%%%%%%%%%

To write and run the Zebra problem in Pylog we built the following framework.
\begin{itemize}
    \item \sloppy We created a \textit{House} class as a subclass of \textit{Structure}. Users may select a house property as a pseudo-functor for displaying houses. We selected \textit{nationality}.

    \item Each clue is expressed as a Pylog function. (See Listing \ref{lis:clues_as_pylog_functions}.) 

        % \begin{minipage}{\linewidth}
        % \begin{python}
        %   def clue_1(self, Houses: SuperSequence):
        %     """ 1. The English live in the red house.  """
        %     yield from member(House(nationality='English', color='red'), Houses)

        %   ...
  
        %   def clue_8(self, Houses: SuperSequence):
        %     """ 8. They drink milk in the middle house. """
        %     yield from unify(House(drink='milk'), Houses[2])

        %   ...
        % \end{python}
        % \begin{lstlisting} [caption={Clues as Pylog functions},  label={lis:clues_as_pylog_functions}]
        % \end{lstlisting}
        % \end{minipage}

    \item The \textit{Houses} list may be any form of \textit{SuperSequence}.
    
    \item We added some simple constraint checking.
\end{itemize}
When run, the answer is the same as in the Prolog version. (See listing \ref{lis:pylog_solution}.)

% \begin{minipage}{\linewidth}
% \begin{python}
% After 1392 rule applications, 
% 	1. Norwegians(Kool, fox, water, yellow)
% 	2. Ukrainians(Chesterfield, horse, tea, blue)
% 	3. English(Old Gold, snails, milk, red)
% 	4. Spanish(Lucky, dog, juice, white)
% 	5. Japanese(Parliament, zebra, coffee, green)
% The Japanese own a zebra, and the Norwegians drink water.
% \end{python}
% \begin{lstlisting} [caption={Pylog solution},  label={lis:pylog_solution}]
% \end{lstlisting}
% \end{minipage}

\smallv
Let's compare the underlying Prolog and Pylog mechanisms. 
\smallv

\textbf{Prolog}. It's trivial to write a Prolog interpreter in Prolog. See Listing \ref{lis:prologInterpreter} \cite{Bartak1998}.

\smallv
\textbf{Pylog}. We developed \textit{three} Pylog approaches to rule interpretation. 
\begin{enumerate}

\item \textit{forall}. Use the \textit{forall} construct as in Listing \ref{lis:zebra_forall}.

% \begin{minipage}{\linewidth}
% \begin{python}
% def zebra_problem(Houses) :-
%     for _ in forall{[
%         # 1. The English live in the red house.
%         lambda: member(house(english, _, _, _, red), Houses), 
%         # 2. The Spanish have a dog.
%         lambda: member(house(spanish, _, dog, _, _), Houses), 
%         # ...
%         ]}
% \end{python}
% \begin{lstlisting} [caption={Pylog solution},  label={lis:zebra_forall}]
% \end{lstlisting}
% \end{minipage}

\textit{forall} succeeds if and only if all members of the list it is passed succeed. Each list element is protected within a \textbf{lambda} construct to prevent evaluation.

\item \textit{run\_all\_rules}. We developed a Python function that accepts a list, e.g., of houses, reflecting the state of the world, along with a list of functions. It succeeds if and only if the functions all succeed. Listing \ref{lis:zebra_ run_all_clues} is a somewhat simplified version.

% \begin{minipage}{\linewidth}
% \begin{python}
% def run_all_clues(World_List: List[Term], clues: List[Callable]):
%     if not clues:
%       # Ran all the clues. Succeed.
%       yield
%     else:
%       # Run the current clue and then the rest of the clues.
%       for _ in clues[0](World_List):
%         yield from run_all_clues(World_List, clues[1:])
% \end{python}
% \begin{lstlisting} [caption={Pylog solution},  label={lis:zebra_ run_all_clues}]
% \end{lstlisting}
% \end{minipage}

\item \textit{Embed rule chaining in the rules.} For example, see Listing \ref{lis:zebra_rule_chaining}.

% \begin{minipage}{\linewidth}
% \begin{python}
%   def clue_1(Houses: SuperSequence):
%     """ 1. The English live in the red house.  """
%     for _ in member(House(nationality='English', color='red'), Houses):
%       yield from clue_2(Houses)

%   def clue_2(Houses: SuperSequence):
%     """ 2. The Spanish have a dog. """
%     for _ in member(House(nationality='Spanish', pet='dog'), Houses):
%       yield from clue_3(Houses)
%   ...
% \end{python}
% \begin{lstlisting} [caption={Pylog solution},  label={lis:zebra_rule_chaining}]
% \end{lstlisting}
% \end{minipage}

% In this organization, unification propagates forward and backward, and backtracking occurs naturally.

Call \textit{clue\_1} with a list of uninstantiated houses, and the problem runs itself.
\end{enumerate}
\smallv 

The three approaches produce the same solution.


\section{Conclusion}\label{sec:conclusion}
Embedding rule chaining in the clues as in the previous section suggests a general template.

\begin{minipage}{\linewidth}
\begin{python}
   def some_clause(...):
     for _ in <generate options>:
       <local conditions>
       yield from next_clause(...)
\end{python}
\end{minipage}

More generally, Pylog offers a way to integrate logic programming features into a Python environment.

\begin{itemize}
  \item The magic of unification requires little more than linked chains.
  \item Prolog's control structures, including "backtracking," can be implemented as nested \textbftt{for}-loops (for both choicepoints and scope setting), with \textbftt{yield} and \textbftt{yield from} gluing the pieces together.
\end{itemize}

\appendix 
\section{The Trace decorator}

The \textit{Trace} decorator is defined as a class rather than a function. 

@Trace logs parameter values for both regular functions and generators. 

@Trace does not handle keyword parameters.

\begin{minipage}{\linewidth} \largev \hrulefill
\begin{python}[numbers=left]
from inspect import isgeneratorfunction, signature

class Trace:

    def __init__(self, f):
        self.param_names = [param.name for param in signature(f).parameters.values()]
        self.f = f
        self.depth = 0

    def __call__(self, *args):
        print(self.trace_line(args))
        self.depth += 1
        if isgeneratorfunction(self.f):
            return self.yield_from(*args)
        else:
            f_return = self.f(*args)
            self.depth -= 1
            return f_return

    def yield_from(self, *args):
        yield from self.f(*args)
        self.depth -= 1

    @staticmethod
    def to_str(xs):
        xs_string = f'[{", ".join(Trace.to_str(x) for x in xs)}]' if isinstance(xs, list) else str(xs)
        return xs_string

    def trace_line(self, args):
        # The quoted string on the next line is two spaces.
        prefix = "  " * self.depth
        params = ", ".join([f'{param_name}: {Trace.to_str(arg)}'
                            for (param_name, arg) in zip(self.param_names, args)])
        # Special case for the transversal functions
        termination = ' <=' if not args[0] else ''
        return prefix + params + termination

\end{python}

\begin{lstlisting} [caption={The Trace decorator},  label={lis:Trace}]
\end{lstlisting}
\end{minipage}

% \newpage
\printbibliography 
% \newpage
\appendix 

\newpage
%%%%%%%%%%%%%%%%%%%%%%%%%%%%%%%%%%%%%%%%%%%%%%
\section{Introduction} 
%%%%%%%%%%%%%%%%%%%%%%%%%%%%%%%%%%%%%%%%%%%%%%
There are no listings from the \textit{Introduction}.

%%%%%%%%%%%%%%%%%%%%%%%%%%%%%%%%%%%%%%%%%%%%%%
\section{Related work} 
%%%%%%%%%%%%%%%%%%%%%%%%%%%%%%%%%%%%%%%%%%%%%%
There are no listings from \textit{Related work}.

%%%%%%%%%%%%%%%%%%%%%%%%%%%%%%%%%%%%%%%%%%%%%%
\section{From Python to Prolog (Listings from Section \ref{sec:pylog})} \label{appsec:pylog}
%%%%%%%%%%%%%%%%%%%%%%%%%%%%%%%%%%%%%%%%%%%%%%

%%%%%%%%%%%%%%%%%%%%%%%%%%%%%%%%%%%%%%%%%%%%%%
\subsection{tvsl\_dfs\_first (Listings from Section \ref{subsec:tvsl_dfs_first})} \label{appsubsec:tvsl_dfs_first}
%%%%%%%%%%%%%%%%%%%%%%%%%%%%%%%%%%%%%%%%%%%%%%

\begin{minipage}{\linewidth}   \hrulefill
\begin{python}[numbers=left]
@Trace
def tvsl_dfs_first(sets: List[List[int]], partial_transversal: Tuple = ()) -> Optional[Tuple]:
  if not sets:
    return partial_transversal
  else:
    for element in sets[0]:
      if element not in partial_transversal:
        complete_transversal = tvsl_dfs_first(sets[1:], partial_transversal + (element, ))
        if complete_transversal is not None:
          return complete_transversal 
    return None
\end{python}
\begin{lstlisting} [caption={\textit{tvsl\_dfs\_first}}, label={lis:dfsfirst}]
\end{lstlisting}
\end{minipage}

\noindent
\begin{minipage}{\linewidth}
\largev   
\begin{python}[numbers=left]
sets: [[1, 2, 3], [2, 4], [1]]
  sets: [[2, 4], [1]], partial_transversal: (1,)
    sets: [[1]], partial_transversal: (1, 2)
    sets: [[1]], partial_transversal: (1, 4)
  sets: [[2, 4], [1]], partial_transversal: (2,)
    sets: [[1]], partial_transversal: (2, 4)
      sets: [], partial_transversal: (2, 4, 1) <=
\end{python}
\begin{lstlisting} [caption={\textit{transversal\_dfs\_first trace}},  label={lis:dfs_first_trace}]
\end{lstlisting}
\end{minipage}

%%%%%%%%%%%%%%%%%%%%%%%%%%%%%%%%%%%%%%%%%%%%%%
\subsection{\textbf{for}-loops as choice points and as computational aggregators (Listings from Section \ref{subsec:forloops})} \label{appsubsec:forloops}
%%%%%%%%%%%%%%%%%%%%%%%%%%%%%%%%%%%%%%%%%%%%%%

\begin{minipage}{\linewidth}  \largev \begin{python}[numbers=left]
def find_largest(lst):
    largest = lst[0]
    for element in lst[1:]:
        largest = max(largest, element)
    return largest

a_list = [3, 5, 2, 7, 4]
print(f'Largest of {a_list} is {find_largest(a_list)}.')
\end{python}
\begin{lstlisting} [caption={find largest},  label={lis:find_largest}]
\end{lstlisting}
\end{minipage}

%%%%%%%%%%%%%%%%%%%%%%%%%%%%%%%%%%%%%%%%%%%%%%
\subsection{tvsl\_dfs\_all (Listings from Section \ref{subsec:tvsl_dfs_all})} \label{appsubsec:tvsl_dfs_all}
%%%%%%%%%%%%%%%%%%%%%%%%%%%%%%%%%%%%%%%%%%%%%%

\begin{minipage}{\linewidth} \largev   \hrulefill
\begin{python}[numbers=left]
@Trace
def tvsl_dfs_all(sets: List[List[int]], partial_transversal: Tuple = ()) -> List[Tuple]:
  if not sets:
    return [partial_transversal]
  else:
    all_transversals = []
    for element in sets[0]:
      if element not in partial_transversal:
        all_transversals += tvsl_dfs_all(sets[1:], partial_transversal + (element, ))
    return all_transversals
\end{python}
\begin{lstlisting} [caption={transversal\_dfs\_all},  label={lis:dfsall}]
\end{lstlisting}
\end{minipage}

\noindent
\begin{minipage}{\linewidth}   \hrulefill  
\begin{python}
sets: [[1, 2, 3], [2, 4], [1]]
  sets: [[2, 4], [1]], partial_transversal: (1,)
    sets: [[1]], partial_transversal: (1, 2)
    sets: [[1]], partial_transversal: (1, 4)
  sets: [[2, 4], [1]], partial_transversal: (2,)
    sets: [[1]], partial_transversal: (2, 4)
      sets: [], partial_transversal: (2, 4, 1) <=
  sets: [[2, 4], [1]], partial_transversal: (3,)
    sets: [[1]], partial_transversal: (3, 2)
      sets: [], partial_transversal: (3, 2, 1) <=
    sets: [[1]], partial_transversal: (3, 4)
      sets: [], partial_transversal: (3, 4, 1) <=
\end{python}
\begin{lstlisting} [caption={\textit{transversal\_dfs\_all trace}},  label={lis:dfsalltrace}]
\end{lstlisting}
\end{minipage}


%%%%%%%%%%%%%%%%%%%%%%%%%%%%%%%%%%%%%%%%%%%%%%
\subsection{tvsl\_yield (Listings from Section \ref{subsec:tvsl_yield})} \label{appsubsec:tvsl_yield}
%%%%%%%%%%%%%%%%%%%%%%%%%%%%%%%%%%%%%%%%%%%%%%

\begin{minipage}{\linewidth}   \hrulefill
\begin{python}[numbers=left]
@Trace
def tvsl_yield(sets: List[List[int]], partial_transversal: Tuple = ()) -> Generator[Tuple, None, None]:
  if not sets:
    yield partial_transversal
  else:
    for element in sets[0]:
      if element not in partial_transversal:
        yield from tvsl_yield(sets[1:], partial_transversal + (element, ))
\end{python}
\begin{lstlisting} [caption={transversal\_dfs\_yield},  label={lis:dfsyield}]
\end{lstlisting}
\end{minipage}

\noindent
\begin{minipage}{\linewidth} \largev
\begin{python}
sets: [[1, 2, 3], [2, 4], [1]]
  sets: [[2, 4], [1]], partial_transversal: (1,)
    sets: [[1]], partial_transversal: (1, 2)
    sets: [[1]], partial_transversal: (1, 4)
  sets: [[2, 4], [1]], partial_transversal: (2,)
    sets: [[1]], partial_transversal: (2, 4)
      sets: [], partial_transversal: (2, 4, 1) <=
Transversal: (2, 4, 1)
  sets: [[2, 4], [1]], partial_transversal: (3,)
    sets: [[1]], partial_transversal: (3, 2)
      sets: [], partial_transversal: (3, 2, 1) <=
Transversal: (3, 2, 1)
    sets: [[1]], partial_transversal: (3, 4)
      sets: [], partial_transversal: (3, 4, 1) <=
Transversal: (3, 4, 1)
\end{python}
\begin{lstlisting} [caption={tvrsl\_yield trace},  label={lis:tvrsl_yield_trace}]
\end{lstlisting}
\end{minipage}

%%%%%%%%%%%%%%%%%%%%%%%%%%%%%%%%%%%%%%%%%%%%%%
\subsection{tvsl\_yield\_lv (Listings from Section \ref{subsec:tvsl_yield_lv})} \label{appsubsec:tvsl_yield_lv}
%%%%%%%%%%%%%%%%%%%%%%%%%%%%%%%%%%%%%%%%%%%%%%

\begin{minipage}{\linewidth}   \hrulefill
\begin{python}[numbers=left]
@Trace
def tvsl_yield_lv(Sets: List[PyList], Partial_Transversal: PyTuple, Complete_Tvsl: Var):
  if not Sets:
    yield from unify(Partial_Transversal, Complete_Tvsl)
  else:
    Element = Var()
    for _ in member(Element, Sets[0]):
      for _ in fails(member)(Element, Partial_Transversal):
        yield from tvsl_yield_lv(Sets[1:], Partial_Transversal + PyList([Element]), Complete_Tvsl)
\end{python}
\begin{lstlisting} [caption={tvsl\_yield\_lv},  label={lis:yieldlv}]
\end{lstlisting}
\end{minipage}

\noindent
\begin{minipage}{\linewidth} \largev 
\begin{python}
Sets: [[1, 2, 3], [2, 4], [1]], Partial_Transversal: (), Complete_Transversal: _10
  Sets: [[2, 4], [1]], Partial_Transversal: (1, ), Complete_Transversal: _10
    Sets: [[1]], Partial_Transversal: (1, 2), Complete_Transversal: _10
    Sets: [[1]], Partial_Transversal: (1, 4), Complete_Transversal: _10
  Sets: [[2, 4], [1]], Partial_Transversal: (2, ), Complete_Transversal: _10
    Sets: [[1]], Partial_Transversal: (2, 4), Complete_Transversal: _10
      Sets: [], Partial_Transversal: (2, 4, 1), Complete_Transversal: _10 <=
Transversal: (2, 4, 1)

  Sets: [[2, 4], [1]], Partial_Transversal: (3, ), Complete_Transversal: _10
    Sets: [[1]], Partial_Transversal: (3, 2), Complete_Transversal: _10
      Sets: [], Partial_Transversal: (3, 2, 1), Complete_Transversal: _10 <=
Transversal: (3, 2, 1)

    Sets: [[1]], Partial_Transversal: (3, 4), Complete_Transversal: _10
      Sets: [], Partial_Transversal: (3, 4, 1), Complete_Transversal: _10 <=
Transversal: (3, 4, 1)
\end{python}
\begin{lstlisting} [caption={Trace of tvsl\_yield\_lv},  label={lis:tvsl_yield_lv_output}]
\end{lstlisting}
\end{minipage}


%%%%%%%%%%%%%%%%%%%%%%%%%%%%%%%%%%%%%%%%%%%%%%
\subsection{tvsl\_prolog (Listings from Section \ref{subsec:tvsl_prolog})} \label{appsubsec:tvsl_prolog}
%%%%%%%%%%%%%%%%%%%%%%%%%%%%%%%%%%%%%%%%%%%%%%

\begin{minipage}{\linewidth} \largev
\begin{python}[numbers=left]
tvsl_prolog(Sets, Partial_Transversal, _Complete_Transversal) :-
    writeln('Sets': Sets;'  Partial_Transversal': Partial_Transversal), 
    fail.

tvsl_prolog([], Complete_Transversal, Complete_Transversal) :-
    format(' => '),
    writeln(Complete_Transversal).

tvsl_prolog([S|Ss], Partial_Transversal, Complete_Transversal_X) :-
    member(X, S),
    \+ member(X, Partial_Transversal),
    append(Partial_Transversal, [X], Partial_Transversal_X),
    tvsl_prolog(Ss, Partial_Transversal_X, Complete_Transversal_X).

\end{python}
\begin{lstlisting} [caption={transversal\_prolog},  label={lis:transversalprolog1}]
\end{lstlisting}
\end{minipage}


\noindent
\begin{minipage}{\linewidth} \largev 
\begin{python}
?- tvsl_prolog([[1, 2, 3], [2, 4], [1]], [], Complete_Transversal).

Sets:[[1, 2, 3], [2, 4], [1]]; Partial_Transversal:[]
Sets:[[2, 4], [1]]; Partial_Transversal:[1]
Sets:[[1]]; Partial_Transversal:[1, 2]
Sets:[[1]]; Partial_Transversal:[1, 4]
Sets:[[2, 4], [1]]; Partial_Transversal:[2]
Sets:[[1]]; Partial_Transversal:[2, 4]
Sets:[]; Partial_Transversal:[2, 4, 1]
 => [2, 4, 1]
 Complete_Transversal = [2, 4, 1]

Sets:[[2, 4], [1]]; Partial_Transversal:[3]
Sets:[[1]]; Partial_Transversal:[3, 2]
Sets:[]; Partial_Transversal:[3, 2, 1]
 => [3, 2, 1]
 Complete_Transversal = [3, 2, 1]

Sets:[[1]]; Partial_Transversal:[3, 4]
Sets:[]; Partial_Transversal:[3, 4, 1]
 => [3, 4, 1]
 Complete_Transversal = [3, 4, 1]
\end{python}
\begin{lstlisting} [caption={transversal\_prolog trace},  label={lis:transversal_prolog_trace}]
\end{lstlisting}
\end{minipage}


%%%%%%%%%%%%%%%%%%%%%%%%%%%%%%%%%%%%%%%%%%%%%%
\section{Control Functions (Listings From Section \ref{sec:control_functions})} \label{appsec:control_functions}
%%%%%%%%%%%%%%%%%%%%%%%%%%%%%%%%%%%%%%%%%%%%%%

%%%%%%%%%%%%%%%%%%%%%%%%%%%%%%%%%%%%%%%%%%%%%%
\subsection{Control flow in Prolog (Listings from Section \ref{subsec:control_flow_prolog})} \label{appsubsec:control_flow_prolog}
%%%%%%%%%%%%%%%%%%%%%%%%%%%%%%%%%%%%%%%%%%%%%%

\begin{minipage}{\linewidth} \hrulefill
\begin{python}[numbers=left]
solve([]).
solve([Term|Terms]):-
  clause(Term, Body), 
  append(Body, Terms, New_Terms), 
  solve(New_Terms).
\end{python}
\begin{lstlisting} [caption={A prolog interpreter in prolog},  label={lis:prologInterpreter}]
\end{lstlisting}
\end{minipage}


%%%%%%%%%%%%%%%%%%%%%%%%%%%%%%%%%%%%%%%%%%%%%%
\subsection{Prolog control flow in Pylog (Listings from Section \ref{subsec:control_flow_pylog})} \label{appsubsec:control_flow_pylog}
%%%%%%%%%%%%%%%%%%%%%%%%%%%%%%%%%%%%%%%%%%%%%%

\begin{minipage}{\linewidth}   \hrulefill
\begin{python}[numbers=left]
    for element in sets[0]:
      if element not in partial_transversal:
        complete_transversal = tvsl_dfs_first(sets[1:], partial_transversal + (element, ))
        if complete_transversal is not None:
          return complete_transversal 
    return None
\end{python}
\begin{lstlisting} [caption={The \textbf{else} branch of \textittt{tvsl\_dfs\_first}}, label={lis:dfsfirstelse}]
\end{lstlisting}
\end{minipage}



\noindent
\begin{minipage}{\linewidth} \largev  
\begin{python}[numbers=left]
    for element in sets[0]:
      if element not in partial_transversal:
        yield from tvsl_yield(sets[1:], partial_transversal + (element, ))
\end{python}
\begin{lstlisting} [caption={The \textbf{else} branch of \textittt{tvsl\_yield}}, label={lis:yieldelse}]
\end{lstlisting}
\end{minipage}

%%%%%%%%%%%%%%%%%%%%%%%%%%%%%%%%%%%%%%%%%%%%%%
\subsection{A review of Python generators (Listings from Section \ref{subsec:generators})} \label{appsubsec:generators}
%%%%%%%%%%%%%%%%%%%%%%%%%%%%%%%%%%%%%%%%%%%%%%


\begin{minipage}{\linewidth}  \hrulefill  
\begin{python}[numbers=left]
def find_number(search_number):
    i = 0
    while True:
        i += 1
        if i == search_number:
            print("\nFound the number:", search_number)
            return
        else:
            yield i

search_number = 5
find_number_object = find_number(search_number)
while True:
    k = next(find_number_object)
    print(f'{k} is not {search_number}')
\end{python}
\begin{lstlisting} [caption={\textittt{Generator example}},  label={lis:generatorExample1}]
\end{lstlisting}
\end{minipage}


\noindent
\begin{minipage}{\linewidth}  \largev   
\begin{verbatim}
1 is not 5
2 is not 5
3 is not 5
4 is not 5

Found the number: 5

Traceback (most recent call last):
  <line number where error occurred> 
    k = next(find_number_object)
StopIteration

Process finished with exit code 1
\end{verbatim}
\begin{lstlisting} [caption={\textittt{Generator example output}},  label={lis:generatorExample2}]
\end{lstlisting}
\end{minipage}

\noindent
\begin{minipage}{\linewidth}  \largev   
\begin{python}
def use_yield_from():
    yield from find_number_object
    print('find_number failed, but "yield from" caught the Stop Iteration exception.')
    return

for k in use_yield_from():
    print(f'{k} is not 5')
\end{python}
\begin{lstlisting} [caption={\textittt{yield from example}},  label={lis:yieldfromExample}]
\end{lstlisting}
\end{minipage}

\noindent
\begin{minipage}{\linewidth}  \largev   
\begin{verbatim}
1 is not 5
2 is not 5
3 is not 5
4 is not 5
Found the number: 5
find_number failed, but "yield from" caught the Stop Iteration exception.

Process finished with exit code 0
\end{verbatim}
\begin{lstlisting} [caption={\textittt{yield from example output}},  label={lis:yieldFromExampleOutput}]
\end{lstlisting}
\end{minipage}

%%%%%%%%%%%%%%%%%%%%%%%%%%%%%%%%%%%%%%%%%%%%%%
\subsection{\textbf{yield} : \textit{succeed} :: \textit{return} : \textit{fail} (Listings from Section \ref{subsec:yield_succeed})} \label{appsubsec:yield_succeed}
%%%%%%%%%%%%%%%%%%%%%%%%%%%%%%%%%%%%%%%%%%%%%%

\begin{minipage}{\linewidth}   \hrulefill  
\begin{python}
head :- body_1.
head :- body_2.
\end{python}
\begin{lstlisting} [caption={Prolog multiple clauses},  label={lis:prologmultipleclauses}]
\end{lstlisting}
\end{minipage}

\noindent
\begin{minipage}{\linewidth}  \largev   
\begin{python}
def head():
    <some code>
    yield
    
    <other code>
    yield
\end{python}
\begin{lstlisting} [caption={Pylog multiple sequential yields},  label={lis:pylogmultipleyields}]
\end{lstlisting}
\end{minipage}

\noindent
\begin{minipage}{\linewidth}  \largev   
\begin{python}
head :- !, body_1.
head :- body_2.
\end{python}
\begin{lstlisting} [caption={Prolog multiple clauses with a cut},  label={lis:prologmultipleclauseswithcut}]
\end{lstlisting}
\end{minipage}

\noindent
\begin{minipage}{\linewidth}  \largev   
\begin{python}
def head():
    if <condition>:
      <some code>
      yield
    else
      <other code>
      yield
\end{python}
\begin{lstlisting} [caption={Multiple Pylog \textbf{yield}s in separate \textbf{if}-\textbf{else} arms},  label={lis:pylogmultipleclauseswithifelse}]
\end{lstlisting}
\end{minipage}

%%%%%%%%%%%%%%%%%%%%%%%%%%%%%%%%%%%%%%%%%%%%%%
\subsection{Control functions (Listings from Section \ref{subsec:controlfunctions})} \label{appsubsec:controlfunctions}
%%%%%%%%%%%%%%%%%%%%%%%%%%%%%%%%%%%%%%%%%%%%%%

\begin{minipage}{\linewidth}  \largev 
\begin{python}[numbers=left]
def fails(f):
  """
  Applied to a function so that the resulting function succeeds if and only if the original fails.
  Note that fails is applied to the function itself, not to a function call.
  Similar to a decorator but applied explicitly when used.
  """
  def fails_wrapper(*args, **kwargs):
    for _ in f(*args, **kwargs):
      # Fail, i.e., don't yield, if f succeeds
      return  
    # Succeed if f fails.
    yield     

  return fails_wrapper
\end{python}
\begin{lstlisting} [caption={fails},  label={lis:fails}]
\end{lstlisting}
\end{minipage}

\noindent
\begin{minipage}{\linewidth}  \largev 
\begin{python}[numbers=left]
def forall(gens):
  """
  Succeeds if all generators in the gens list succeed. The elements in the gens list
  are embedded in lambda functions to avoid premature evaluation.
  """
  if not gens:
    # They have all succeeded.
    yield
  else:
    # Get gens[0] and evaluate the lambda expression to get a fresh iterator.
    # The parentheses after gens[0] evaluates the lambda expression.
    # If it succeeds, run the rest of the generators in the list.
    for _ in gens[0]( ):
      yield from forall(gens[1:])
\end{python}
\begin{lstlisting} [caption={forall},  label={lis:forall}]
\end{lstlisting}
\end{minipage}


\noindent
\begin{minipage}{\linewidth}  \largev 
\begin{python}[numbers=left]
def forany(gens):
  """
  Succeeds if any of the generators in the gens list succeed. On backtracking, tries them all. 
  The gens elements must be embedded in lambda functions.
  """
  for gen in gens:
    yield from gen( )

\end{python}
\begin{lstlisting} [caption={forany},  label={lis:forany}]
\end{lstlisting}
\end{minipage}

\noindent
\begin{minipage}{\linewidth}  \largev 
\begin{python}[numbers=left]
def trace(x, succeed=True, show_trace=True):
  """
  Can be included in a list of generators (as in forall and forany) to see where we are.
  The second argument determines whether trace succeeds or fails. The third turns printing on or off.
  When included in a list of forall generators, succeed should be set to True so that
  it doesn't prevent forall from succeeding.
  When included in a list of forany generators, succeed should be set to False so that forany
  will go on the the next generator and won't take trace as an extraneous successes.
  """
  if show_trace:
    print(x)
  if succeed:
    yield

\end{python}
\begin{lstlisting} [caption={trace},  label={lis:trace}]
\end{lstlisting}
\end{minipage}

\noindent
\begin{minipage}{\linewidth}  \largev 
\begin{python}[numbers=left]
def would_succeed(f):
  """
  Applied to a function so that the resulting function succeeds/fails if and only if the original
  function succeeds/fails. If the original function succeeds, this also succeeds but without 
  binding any variables. Similar to a decorator but applied explicitly when used.
  """
  def would_succeed_wrapper(*args, **kwargs):
    succeeded = False
    for _ in f(*args, **kwargs):
      succeeded = True
      # Do not yield in the context of f succeeding.
      
    # Exit the for-loop so that unification will be undone.
    if succeeded:
      # Succeed if f succeeded.
      yield  
    # The else clause is redundant. It is included here for clarity.
    # else:
    #   Fail if f failed.
    #   pass   

  return would_succeed_wrapper

\end{python}
\begin{lstlisting} [caption={would\_succeed},  label={lis:wouldsucceed}]
\end{lstlisting}
\end{minipage}

\noindent
\begin{minipage}{\linewidth}  \largev 
\begin{python}[numbers=left]
def bool_yield_wrapper(gen):
  """
  A decorator. Produces a function that generates a Bool_Yield_Wrapper object. 
  """
  def wrapped_func(*args, **kwargs):
    return Bool_Yield_Wrapper(gen(*args, **kwargs))

  return wrapped_func
\end{python}
\begin{lstlisting} [caption={bool\_yield\_wrapper},  label={lis:boolYieldWrapper}]
\end{lstlisting}
\end{minipage}


\noindent
\begin{minipage}{\linewidth}  \largev 
\begin{python}[numbers=left]
  @bool_yield_wrapper
  def squares(n: int, X2: Var) -> Bool_Yield_Wrapper:
    for i in range(n):
      unify_gen = bool_yield_wrapper(unify)(X2, i**2)
      while unify_gen.has_more():
        yield

  Square = Var()
  squares_gen = squares(5, Square)
  while squares_gen.has_more():
    print(Square)
\end{python}
\begin{lstlisting} [caption={bool\_yield\_wrapper example},  label={lis:boolYieldWrapperExample}]
\end{lstlisting}
\end{minipage}


%%%%%%%%%%%%%%%%%%%%%%%%%%%%%%%%%%%%%%%%%%%%%%
\section{Logic variables (Listings from Section \ref{sec:logic_variables})} \label{appsec:logic_variables} 
%%%%%%%%%%%%%%%%%%%%%%%%%%%%%%%%%%%%%%%%%%%%%%

%%%%%%%%%%%%%%%%%%%%%%%%%%%%%%%%%%%%%%%%%%%%%%
\subsection{PyValue (Listings from Section \ref{subsec:pyvalue})} \label{appsubsec:pyvalue}
%%%%%%%%%%%%%%%%%%%%%%%%%%%%%%%%%%%%%%%%%%%%%%
No listings from this section.

%%%%%%%%%%%%%%%%%%%%%%%%%%%%%%%%%%%%%%%%%%%%%%
\subsection{Var (Listings from Section \ref{subsec:var})} \label{appsubsec:var}
%%%%%%%%%%%%%%%%%%%%%%%%%%%%%%%%%%%%%%%%%%%%%%

\begin{minipage}{\linewidth}   \hrulefill
\begin{python}[numbers=left]
def print_ABCDE(A, B, C, D, E):
    print(f'A: {A}, B: {B}, C: {C}, D: {D}, E: {E}')

(A, B, C, D, E) = (Var(), Var(), Var(), Var(), 'abc')
print_ABCDE(A, B, C, D, E) 
for _ in unify(A, B):
  print_ABCDE(A, B, C, D, E) 
  for _ in unify(D, C):
    print_ABCDE(A, B, C, D, E) 
    for _ in unify(A, C):
      print_ABCDE(A, B, C, D, E) 
      for _ in unify(E, D):
        print_ABCDE(A, B, C, D, E) 
      print_ABCDE(A, B, C, D, E) 
    print_ABCDE(A, B, C, D, E) 
  print_ABCDE(A, B, C, D, E) 
print_ABCDE(A, B, C, D, E) 
\end{python}
\begin{lstlisting} [caption={Unifying logic variables},  label={lis:unifylogicvars1}]
\end{lstlisting}
\end{minipage}

\noindent
\begin{minipage}{\linewidth} \largev  
\begin{python}[numbers=left]
A: _195, B: _196, C: _197, D: _198, E: abc
A: _196, B: _196, C: _197, D: _198, E: abc
A: _196, B: _196, C: _197, D: _197, E: abc
A: _197, B: _197, C: _197, D: _197, E: abc
A: abc, B: abc, C: abc, D: abc, E: abc
A: _197, B: _197, C: _197, D: _197, E: abc
A: _196, B: _196, C: _197, D: _197, E: abc
A: _196, B: _196, C: _197, D: _198, E: abc
A: _195, B: _196, C: _197, D: _198, E: abc
\end{python}
\begin{lstlisting} [caption={Unifying logic variables},  label={lis:unifylogicvars2}]
\end{lstlisting}
\end{minipage}

\noindent
\begin{minipage}{\linewidth} \largev  
\begin{python}
(A, B, C, D, E) = (*n_Vars(4), 'abc')
for _ in unify_pairs([(A, B), (D, C), (A, C), (E, D)]):
\end{python}
\begin{lstlisting} [caption={Unifying logic variables shortened},  label={lis:unifylogicvarsshortened}]
\end{lstlisting}
\end{minipage}

%%%%%%%%%%%%%%%%%%%%%%%%%%%%%%%%%%%%%%%%%%%%%%
\subsection{Structure (Listings from Section \ref{subsec:structure})} \label{appsubsec:structure}
%%%%%%%%%%%%%%%%%%%%%%%%%%%%%%%%%%%%%%%%%%%%%%
No listings from this section.

%%%%%%%%%%%%%%%%%%%%%%%%%%%%%%%%%%%%%%%%%%%%%%
\subsection{Lists (Listings from Section \ref{subsec:lists})} \label{appsubsec:lists}
%%%%%%%%%%%%%%%%%%%%%%%%%%%%%%%%%%%%%%%%%%%%%%
No listings from this section.

%%%%%%%%%%%%%%%%%%%%%%%%%%%%%%%%%%%%%%%%%%%%%%
\subsection{\textit{append} (Listings from Section \ref{subsec:append})} \label{appsubsec:append}
%%%%%%%%%%%%%%%%%%%%%%%%%%%%%%%%%%%%%%%%%%%%%%

\begin{minipage}{\linewidth} \hrulefill
\begin{python}
(Xs, Ys, Zs) = (Var(), Var(), LinkedList([1, 2, 3]))
for _ in append(Xs, Ys, Zs):
  print(f'Xs = {Xs}\nYs = {Ys}\n')
\end{python}
\begin{lstlisting} [caption={append},  label={lis:append}]
\end{lstlisting}
\end{minipage}

\noindent
\begin{minipage}{\linewidth}  \largev 
\begin{python}
Xs = []
Ys = [1, 2, 3]

Xs = [1]
Ys = [2, 3]

Xs = [1, 2]
Ys = [3]

Xs = [1, 2, 3]
Ys = []
\end{python}
\begin{lstlisting} [caption={append output},  label={lis:append_output}]
\end{lstlisting}
\end{minipage}
\smallv

\noindent
\begin{minipage}{\linewidth}  \largev 
\begin{python}
append([], Ys, Ys).
append([XZ|Xs], Ys, [XZ|Zs]) :- append(Xs, Ys, Zs).
\end{python}
\begin{lstlisting} [caption={prolog append code},  label={lis:prolog_append_code}]
\end{lstlisting}
\end{minipage}

% Now the wordier but isomorphic Pylog version.


\noindent
\begin{minipage}{\linewidth}  \largev 
\begin{python}[numbers=left]
# For a cleaner presentation, declarations are dropped. All variables are Union[LinkedList, Var].
def append(Xs, Ys, Zs):

  # Corresponds to: append([], Ys, Ys).
  yield from unify_pairs([(Xs, LinkedList([])), (Ys, Zs)])

  # Corresponds to: append([XZ|Xs], Ys, [XZ|Zs]) :- append(Xs, Ys, Zs).
  (XZ_Head, Xs_Tail, Zs_Tail) = n_Vars(3)
  for _ in unify_pairs([(Xs, LinkedList(XZ_Head, Xs_Tail)),
                       (Zs, LinkedList(XZ_Head, Zs_Tail))]):
    yield from append(Xs_Tail, Ys, Zs_Tail)

\end{python}
\begin{lstlisting} [caption={Pylog append code},  label={lis:append_code}]
\end{lstlisting}
\end{minipage}

%%%%%%%%%%%%%%%%%%%%%%%%%%%%%%%%%%%%%%%%%%%%%%
\subsection{Unification (Listings from Section \ref{subsec:unify})} \label{appsubsec:unify}
%%%%%%%%%%%%%%%%%%%%%%%%%%%%%%%%%%%%%%%%%%%%%%

\begin{minipage}{\linewidth} \hrulefill
\begin{python}[numbers=left]
@euc
def unify(Left: Any, Right: Any):

  (Left, Right) = map(ensure_is_logic_variable, (Left, Right))

  # Case 1.
  if Left == Right:
    yield

  # Case 2.
  elif isinstance(Left, PyValue) and isinstance(Right, PyValue) and \
       (not Left.is_instantiated( ) or not Right.is_instantiated( )) and \
       (Left.is_instantiated( ) or Right.is_instantiated( )):
    (assignedTo, assignedFrom) = (Left, Right) if Right.is_instantiated( ) else (Right, Left)
    assignedTo._set_py_value(assignedFrom.get_py_value( ))
    yield

    assignedTo._set_py_value(None)

  # Case 3.
  elif isinstance(Left, Structure) and isinstance(Right, Structure) and Left.functor == Right.functor:
    yield from unify_sequences(Left.args, Right.args)

  # Case 4.
  elif isinstance(Left, Var) or isinstance(Right, Var):
    (pointsFrom, pointsTo) = (Left, Right) if isinstance(Left, Var) else (Right, Left)
    pointsFrom.unification_chain_next = pointsTo
    yield

    pointsFrom.unification_chain_next = None

\end{python}
\begin{lstlisting} [caption={unify},  label={lis:unify}]
\end{lstlisting}
\end{minipage}

%%%%%%%%%%%%%%%%%%%%%%%%%%%%%%%%%%%%%%%%%%%%%%
\subsection{Back to tvsl\_yield\_lv (Listings from Section \ref{subsec:more_tvsl_yield_lv})} \label{appsubsec:more_tvsl_yield_lv}
%%%%%%%%%%%%%%%%%%%%%%%%%%%%%%%%%%%%%%%%%%%%%%

No listing from this section.

%%%%%%%%%%%%%%%%%%%%%%%%%%%%%%%%%%%%%%%%%%%%%%
\section{The Zebra Puzzle (Listings From Section \ref{sec:zebra})} \label{appsec:zebra}
%%%%%%%%%%%%%%%%%%%%%%%%%%%%%%%%%%%%%%%%%%%%%%

%%%%%%%%%%%%%%%%%%%%%%%%%%%%%%%%%%%%%%%%%%%%%%
\subsection{The clues and a Prolog solution (Listings from Section \ref{subsec:clues})} \label{appsubsec:clues}
%%%%%%%%%%%%%%%%%%%%%%%%%%%%%%%%%%%%%%%%%%%%%%

\begin{minipage}{\linewidth}
\hrulefill
\begin{python}
zebra_problem(Houses) :-
    Houses = [house(_, _, _, _, _), house(_, _, _, _, _), house(_, _, _, _, _), 
              house(_, _, _, _, _), house(_, _, _, _, _)], 

    % 1. The English live in the red house.
    member(house(english, _, _, _, red), Houses), 

    % 2. The Spanish have a dog.
    member(house(spanish, _, dog, _, _), Houses), 

    % 3. They drink coffee in the green house.
    member(house(_, _, _, coffee, green), Houses), 

    % 4. The Ukrainians drink tea.
    member(house(ukranians, _, _, tea, _), Houses), 

    % 5. The green house is immediately to the right of the white house.
    nextto(house(_, _, _, _, white), house(_, _, _, _, green), Houses), 

    % 6. The Old Gold smokers have snails.
    member(house(_, old_gold, snails, _, _), Houses), 

    % 7. They smoke Kool in the yellow house.
    member(house(_, kool, _, _, yellow), Houses), 

    % 8. They drink milk in the middle house.
    Houses = [_, _, house(_, _, _, milk, _), _, _], 

    % 9. The Norwegians live in the first house on the left.
    Houses = [house(norwegians, _, _, _, _) | _], 

    % 10. The Chesterfield smokers live next to the fox.
    next_to(house(_, chesterfield, _, _, _), house(_, _, fox, _, _), Houses), 

    % 11. They smoke Kool in the house next to the horse.
    next_to(house(_, kool, _, _, _), house(_, _, horse, _, _), Houses), 

    % 12. The Lucky smokers drink juice.
    member(house(_, lucky, _, juice, _), Houses), 

    % 13. The Japanese smoke Parliament.
    member(house(japanese, parliament, _, _, _), Houses), 

    % 14. The Norwegians live next to the blue house.
    next_to(house(norwegians, _, _, _, _), house(_, _, _, _, blue), Houses), 
\end{python}
\begin{lstlisting} [caption={Zebra puzzle in Prolog},  label={lis:zebra_prolog}]
\end{lstlisting}
\end{minipage}

\noindent
\begin{minipage}{\linewidth}
\begin{python}
?- zebra_problem(Houses).
[    
    house(norwegians, kool, fox, water, yellow), 
    house(ukranians, chesterfield, horse, tea, blue), 
    house(english, old_gold, snails, milk, red), 
    house(spanish, lucky, dog, juice, white), 
    house(japanese, parliament, zebra, coffee, green)     
]
\end{python}
\begin{lstlisting} [caption={Zebra puzzle in Prolog},  label={lis:zebra_solution}]
\end{lstlisting}
\end{minipage}

%%%%%%%%%%%%%%%%%%%%%%%%%%%%%%%%%%%%%%%%%%%%%%
\subsection{A Pylog solution (Listings from Section \ref{subsec:pylog_solution})} \label{appsubsec:pylog_solution}
%%%%%%%%%%%%%%%%%%%%%%%%%%%%%%%%%%%%%%%%%%%%%%

\begin{minipage}{\linewidth}
\hrulefill
\begin{python}
  def clue_1(self, Houses: SuperSequence):
    """ 1. The English live in the red house.  """
    yield from member(House(nationality='English', color='red'), Houses)

  ...

  def clue_8(self, Houses: SuperSequence):
    """ 8. They drink milk in the middle house. """
    yield from unify(House(drink='milk'), Houses[2])

  ...
\end{python}
\begin{lstlisting} [caption={Clues as Pylog functions},  label={lis:clues_as_pylog_functions}]
\end{lstlisting}
\end{minipage}

\noindent
\begin{minipage}{\linewidth} \largev
\begin{python}
After 1392 rule applications, 
	1. Norwegians(Kool, fox, water, yellow)
	2. Ukrainians(Chesterfield, horse, tea, blue)
	3. English(Old Gold, snails, milk, red)
	4. Spanish(Lucky, dog, juice, white)
	5. Japanese(Parliament, zebra, coffee, green)
The Japanese own a zebra, and the Norwegians drink water.
\end{python}
\begin{lstlisting} [caption={Pylog solution},  label={lis:pylog_solution}]
\end{lstlisting}
\end{minipage}

\noindent
\begin{minipage}{\linewidth} \largev
\begin{python}
def zebra_problem(Houses) :-
    for _ in forall{[
        # 1. The English live in the red house.
        lambda: member(house(english, _, _, _, red), Houses), 
        # 2. The Spanish have a dog.
        lambda: member(house(spanish, _, dog, _, _), Houses), 
        # ...
        ]}
\end{python}
\begin{lstlisting} [caption={Pylog solution},  label={lis:zebra_forall}]
\end{lstlisting}
\end{minipage}

\noindent
\begin{minipage}{\linewidth} \largev
\begin{python}
def run_all_clues(World_List: List[Term], clues: List[Callable]):
    if not clues:
      # Ran all the clues. Succeed.
      yield
    else:
      # Run the current clue and then the rest of the clues.
      for _ in clues[0](World_List):
        yield from run_all_clues(World_List, clues[1:])
\end{python}
\begin{lstlisting} [caption={Pylog solution},  label={lis:zebra_ run_all_clues}]
\end{lstlisting}
\end{minipage}

\noindent
\begin{minipage}{\linewidth} \largev
\begin{python}
  def clue_1(Houses: SuperSequence):
    """ 1. The English live in the red house.  """
    for _ in member(House(nationality='English', color='red'), Houses):
      yield from clue_2(Houses)

  def clue_2(Houses: SuperSequence):
    """ 2. The Spanish have a dog. """
    for _ in member(House(nationality='Spanish', pet='dog'), Houses):
      yield from clue_3(Houses)
      
  ...
\end{python}
\begin{lstlisting} [caption={Pylog solution},  label={lis:zebra_rule_chaining}]
\end{lstlisting}
\end{minipage}

%%%%%%%%%%%%%%%%%%%%%%%%%%%%%%%%%%%%%%%%%%%%%%
\section{Conclusion (Listings from Section \ref{sec:conclusion})} \label{appsec:conclusion}
%%%%%%%%%%%%%%%%%%%%%%%%%%%%%%%%%%%%%%%%%%%%%%

\begin{minipage}{\linewidth}
\hrulefill
\begin{python}
   def some_clause(...):
     for _ in <generate options>:
       <local conditions>
       yield from next_clause(...)
\end{python}
\begin{lstlisting} [caption={A Pylog/Prolog template},  label={lis:template}]
\end{lstlisting}
\end{minipage}

% \newpage
\section{The Trace decorator}\label{app:Trace}

The \textit{Trace} decorator is defined as a class rather than a function. \textit{Trace} logs parameter values for both regular functions and generators, but \textit{Trace} does not handle keyword parameters.

\begin{minipage}{\linewidth}  \hrulefill
\begin{python}[numbers=left]
from inspect import isgeneratorfunction, signature

class Trace:

    def __init__(self, f):
        self.param_names = [param.name for param in signature(f).parameters.values()]
        self.f = f
        self.depth = 0

    def __call__(self, *args):
        print(self.trace_line(args))
        self.depth += 1
        if isgeneratorfunction(self.f):
            return self.yield_from(*args)
        else:
            f_return = self.f(*args)
            self.depth -= 1
            return f_return

    def yield_from(self, *args):
        yield from self.f(*args)
        self.depth -= 1

    @staticmethod
    def to_str(xs):
        xs_string = f'[{", ".join(Trace.to_str(x) for x in xs)}]' if isinstance(xs, list) else str(xs)
        return xs_string

    def trace_line(self, args):
        # The quoted string on the next line is two spaces.
        prefix = "  " * self.depth
        params = ", ".join([f'{param_name}: {Trace.to_str(arg)}'
                            for (param_name, arg) in zip(self.param_names, args)])
        # Special case for the transversal functions
        termination = ' <=' if not args[0] else ''
        return prefix + params + termination
\end{python}

\begin{lstlisting} [caption={The Trace decorator},  label={lis:Trace}]
\end{lstlisting}
\end{minipage}

\end{document}

