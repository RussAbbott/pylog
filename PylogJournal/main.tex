% -*- coding: utf-8; -*-
% vim: set fileencoding=utf-8 :
\documentclass[english,submission]{programming}
\usepackage{mdframed, framed}
\usepackage[backend=biber]{biblatex}
\addbibresource{example.bib}
\addbibresource{pylog.bib}
\usepackage{pythonhl}


\lstdefinelanguage[programming]{TeX}[AlLaTeX]{TeX}{%
  deletetexcs={title,author,bibliography},%
  deletekeywords={tabular},
  morekeywords={abstract},%
  moretexcs={chapter},%
  moretexcs=[2]{title,author,subtitle,keywords,maketitle,titlerunning,authorinfo,affiliation,authorrunning,paperdetails,acks,email},
  moretexcs=[3]{addbibresource,printbibliography,bibliography},%
}%
\lstset{%
  language={[programming]TeX},%
  keywordstyle=\firamedium,
  stringstyle=\color{RosyBrown},%
  texcsstyle=*{\color{Purple}\mdseries},%
  texcsstyle=*[2]{\color{Blue1}},%
  texcsstyle=*[3]{\color{ForestGreen}},%
  commentstyle={\color{FireBrick}},%
  escapechar=`,}
\newcommand*{\CTAN}[1]{\href{http://ctan.org/tex-archive/#1}{\nolinkurl{CTAN:#1}}}
%%


\def\inv{\vspace*{-6pt}}
\def\sinv{\vspace*{-3pt}}
\def\smallinv{\vspace*{-3pt}}
\def\smallinh{\hspace*{-3pt}}
\def\smallv{\vspace*{2pt}}
\def\largev{\vspace*{6pt}}
\def\smallh{\hspace*{2pt}}


\newcommand{\textbftt}[1]{\textbf{\texttt{#1}}}
\newcommand{\textittt}[1]{\textit{\texttt{#1}}}


\begin{document}

\title{Pylog$\colon$ Prolog in Python}
%\subtitle{Preparing Articles for Programming}% optional
%\titlerunning{Preparing Articles for Programming} %optional, in case that the title is too long; the running title should fit into the top page column

\author[a]{Russ Abbott}
\authorinfo{is the author of this paper. Contact him at \email{rabbott@calstatela.edu}.}
\affiliation[a]{California State University, Los Angeles, USA}

\author[a]{Jungsoo Lim}
\authorinfo{is a co-author of this paper. Contact her at \email{jlim34@calstatela.edu}.}
% \affiliation[a]{California State University, Los Angeles, USA}

\author[b]{Jay Patel}
\authorinfo{is a co-author of this paper. Contact him at \email{imjaypatel12@gmail.com}.}
\affiliation[b]{Visa, San Francisco, USA}

\keywords{backtracking, logic programming, logic variables, programming paradigms, Prolog, Pylog, Python, unification} 
% please provide 1--5 keywords


%%%%%%%%%%%%%%%%%%
%% These data MUST be filled for your submission. (see 5.3)
\paperdetails{
  %% perspective options are: art, sciencetheoretical, scienceempirical, engineering.
  %% Choose exactly the one that best describes this work. (see 2.1)
  perspective=art,
  %% State one or more areas, separated by a comma. (see 2.2)
  %% Please see list of areas in http://programming-journal.org/cfp/
  %% The list is open-ended, so use other areas if yours is/are not listed.
  area={Social Coding, General-purpose programming},
  %% You may choose the license for your paper (see 3.)
  %% License options include: cc-by (default), cc-by-nc
  %% license=cc-by,
}
%%%%%%%%%%%%%%%%%%

%%%%%%%%%%%%%%%%%%%%%%%%%%%%%
% Please go to https://dl.acm.org/ccs/ccs.cfm and generate your Classification
% System [view CCS TeX Code] stanz and copy _all of it_ to this place.
%% From HERE
\begin{CCSXML}
<ccs2012>
<concept>
<concept_id>10011007.10010940</concept_id>
<concept_desc>Software and its engineering~Software organization and properties</concept_desc>
<concept_significance>500</concept_significance>
</concept>
</ccs2012>
\end{CCSXML}

\ccsdesc[500]{Software and its engineering~Software organization and properties}
% To HERE
%%%%%%%%%%%%%%%%%%%%%%%

\maketitle
\begin{abstract}
Pylog explores the integration of two distinct programming language paradigms: (i) the modern, general purpose programming paradigm which, in its Python implementation, has been broadened to include procedural, object-oriented, and functional programming and (ii) the logic programming paradigm, most notably including logic variables (with unification) and depth-first, backtracking search. 
          
Pylog illustrates how the core logic programming features can be implemented in and integrated into Python. 

Simultaneously, Pylog demonstrates Python's breadth. Python is used in situations ranging introductory programming classes to the development of very sophisticated software. Pylog demonstrates two interestingly distinct uses for Python's \textbf{for}-loop construct.

Pylog exemplifies programming at its best, using Python features in innovative yet clear ways to integrate features of a non-Python programming paradigm into its range of capabilities. The overall result is software worth reading.
          
          The Pylog code is available at \href{https://github.com/RussAbbott/pylog}{\underline{this GitHub repository}}.
\end{abstract}


%%%%%%%%%%%%%%%%%%%%%%%%%%%%%%%%%%%%%%%%%%%%%%
\section{Introduction}
%%%%%%%%%%%%%%%%%%%%%%%%%%%%%%%%%%%%%%%%%%%%%%

\noindent\textbf{Symbolic artificial intelligence}. The birth announcement for Artificial Intelligence took the form of a workshop proposal. The proposal predicted that \textit{every aspect of learning---or any other feature of intelligence---can in principle be so precisely described that a machine can be made to simulate it.}\cite{mccarthy2006proposal}  

At the workshop, held in 1956, Newell and Simon claimed that their Logic Theorist not only took a giant step toward that goal but even \textit{solved the mind-body problem}.\cite{russell2010artificial} A year later Simon doubled-down.
\begin{quote}
    [T]here are now machines that can think, that can learn, and that can create. Moreover, their ability to do these things is going to increase rapidly until---in a visible future---the range of problems they can handle will be coextensive with the range to which the human mind has been applied.\cite{simon1957models}
\end{quote}

Perhaps not unexpectedly, such extreme optimism about the power of symbolic AI, as this work was (and is still) known, faded into the gloom of what has been labelled the AI winter. 

%%%%%%%%%%%%%%%%%%%%%%%%%%%%%%%%%%%%%%%%%%%%%%
\smallv\noindent\textbf{Deep learning}. All was not lost. Winter was followed by spring and the green shoots of (non-symbolic) deep neural networks sprang forth. Andrew Ng said of that development, 
\begin{quote}
Just as electricity transformed almost everything 100 years ago, today I actually have a hard time thinking of an industry that I don’t think AI will transform in the next several years.\cite{ng2018ai}
\end{quote}

But another disappointment followed. Deep neural nets 
\begin{quote}
are surprisingly susceptible to what are known as \textit{adversarial} attacks. Small perturbations to images that are (almost) imperceptible to human vision can cause a neural network to completely change its prediction. When minimally modified, a correctly classified image of a school bus is reclassified as an ostrich. Even worse, the classifiers report high confidence in this wrong prediction.\cite{akhtar2018threat}
\end{quote}
% (Adversarial images did not kill off deep learning. They have been co-opted, and their use is now built into  deep neural network training methodologies.\cite{shrivastava2017learning})

%%%%%%%%%%%%%%%%%%%%%%%%%%%%%%%%%%%%%%%%%%%%%%
\smallv\noindent\textbf{Deep learning: the current state}. Deep learning has achieved extraordinary success in fields such as image captioning and natural language translation.\cite{garnelo2019reconciling} But other than its remarkable achievements in game-playing via reinforcement learning\cite{silver2018general}, it's triumphs have often been superficial. 

By that we don't mean that the work is trivial. We suggest that many deep learning systems learn little more than surface patterns. The patterns may be both subtle and complex, but they are surface patterns nevertheless.

Lacker\cite{lacker-gpt3} elicits many examples of such superficial (but very sophisticated) patterns from GPT-3\cite{brown2020language}, a highly acclaimed natural language system. In one, GPT-3, acting as a personal assistant, offers to read to its interlocutor his latest email. The problem is that GPT-3 has no access to that person's email---and doesn't "know" that without access it can't read the email. (Much of the excitement surrounding GPT-3 derives from its skill as a fiction author.) 

Both the conversational interaction and the made-up email sound plausible and natural. In reality, each consists of words strung together based simply on co-occurrences that GPT-3 found in the billions upon billions of word sequences it had scanned. Although what GPT-3 produces sounds like coherent English, it's all surface patterns with no underlying semantics.

Recent work\cite{geirhos2018imagenet} (see \cite{Cepelewicz-textures-2020} for a popular discussion) suggests that much of the success of deep learning, at least when applied to image categorization, derives from the tendency of deep learning systems to focus on textures---the ultimate surface feature---rather than shapes.  This insight offers an explanation for some of deep learning's brittleness and superficiality as well as a possible mitigation strategy.

%%%%%%%%%%%%%%%%%%%%%%%%%%%%%%%%%%%%%%%%%%%%%%
\smallv\noindent\textbf{The Holy Grail: constraint programming}. In the mean time, work on symbolic AI continued. Constraint programming was born in the 1980s as an outgrowth of the interest in logic programming triggered by the Japanese Fifth Generation initiative.\cite{shapiro1983fifth} Logic Programming led to Constraint Logic Programming, which evolved into Constraint Programming. (A familiar constraint programming example is the well-known n-queens problem: how can you place n queens on an n \textit{x} n chess board so that no queen threatens any other queen? There are, of course, many practical constraint programming applications as well.)

In 1997, Eugene Frueder characterized constraint programming as \textit{the Holy Grail of computer science: the user simply states the problem and the computer solves it.}\cite{freuder1997pursuit}  Software that solves constraint programming problems is known as a solver. Constraint programming has many desirable properties. 
\begin{itemize}
    \item Solutions found by constraint programming solvers actually solve the given problem. There is no issue of how ``confident'' the solver is in the solutions it finds.
    
    \item One can understand how the solver arrived at the solution. This contrasts with the frustrating feature of neural nets that the solutions they find are generally hidden within a maze of parameters, unintelligible to human beings. 

    \item  The structure and limits of constraint programming are well understood: there will be no grand disappointments similar to those that followed the birth of artificial intelligence---unless quantum computing, once implemented, turns out to be a bust. 
    
    \item Constraint programming is closely related to computational complexity, which provides a well-studied theoretical framework for it. 
    
    \item There will be no surprises such as adversarial images. 

    \item Solver technology is easy to characterize. It is an exercise in search: find values for uninstantiated variables that satisfy the constraints.

    \item Improvements are generally incremental and consist primarily of new heuristics and better search strategies. For example, in the n-queens problem one can propagate solution steps by marking as unavailable board squares that are threatened by newly placed pieces. This reduces search times. We will see  example heuristics below.

\end{itemize}

Constraint programming solvers are now available in multiple forms. MiniZinc\cite{wallace2020problem} allows users to express constraints in what is essentially executable predicate calculus.

Solvers are also available as package add-ons to many programming languages: Choco\cite{prud2019choco} and JaCoP\cite{kuchcinski2013jacop} (two Java libraries), OscaR/CBLS\cite{Oscar} and Yuck\cite{Yuck} (two Scala libraries), and Google's OR-tools\cite{Google-OR-tools} (a collection of C++ libraries, which sport Python, Java, and .NET front ends).

In the systems just mentioned, the solver is a black box. One sets up a problem, either directly in predicate calculus or in the host language, and then calls on the solver to solve it. 

This can be frustrating for those who want more insight into the internal workings of the solvers they use. Significantly more insight is available when working either (a) in a system like Picat\cite{zhou2015constraint}, a language that combines features of logic programming and imperative programming, or (b) with Prolog (say either SICStus Prolog\cite{carlsson2014sicstus} or SWI Prolog\cite{swi-prolog}) to which a Finite Domain package has been added. But neither option helps those without a logic programming background. 

%%%%%%%%%%%%%%%%%%%%%%%%%%%%%%%%%%%%%%%%%%%%%%
\smallv\noindent\textbf{Shallow embeddings}.  Solver capabilities may be implemented directly in a host language and made available to programs in that language.\cite{hoare1998unifying, gibbons2014folding} Recent examples include Kanren\cite{Rocklin2019}, a Python embedding, and Muli\cite{dageforde2018constraint}, a Java embedding.

Most shallow embeddings have well-defined APIs; but like libraries, their inner workings are not visible. This is the case with both Kanren and Muli. Kanren is open source, but it offers no implementation documentation. Dagef{\"o}rde and Kuchen describe the Muli virtual machine\cite{dageforde2019compiler}, but the documentation is quite technical. Many who would like to understand its internal functioning may find it difficult going. 

%%%%%%%%%%%%%%%%%%%%%%%%%%%%%%%%%%%%%%%%%%%%%%
\smallv\noindent\textbf{Back to basics}. This brings us to our goal for the rest of this paper: to offer an under-the-covers tutorial about how a fully functioning embedded solver works. 

One can think of Prolog as the skeleton of a constraint satisfaction solver. Consequently, we focus on Prolog as a basic paradigmatic solver. We describe Pylog, a Python shallow embedding of Prolog's core capabilities. 

Our primary focus will be on helping readers understand how Prolog's two fundamental features, backtracking and logic variables, can be implemented \textit{simply and cleanly}. We also show how two of the heuristics common to Finite Domain packages can be added. 

Pylog should be accessible to anyone reasonable fluent in Python. In addition, the techniques used in the implementation are easily transferred to many other languages. 

We stress \textit{simply and cleanly}. There are many ways to implement backtracking and logic variables, some quite complex. Our approach is straightforward and easy to understand. 

An advantage we have over earlier Prolog embeddings is Python generators. Without generators, one is pushed to use more complex backtracking implementations, such as continuation passing\cite{amin2019lightweight} or monads\cite{seres1999embedding}. Generators, which are now widespread\cite{wikipedia-generators}, eliminate the need for such complexity. 

To be clear, we did not invent the use of generators for implementing backtracking. It has a nearly two-decade history: \cite{berger2004, Bolz2007, Delford2009, Frederiksen2011, Meyers2015, Thompson2017, Santini2018, Cesar2019, Miljkovic2019}. We would like especially to thank Ian Piumarta\cite{Piumarta2017}; Pylog began as a fork of his efforts. 

The preceding are sketches and prototypes. We offer a cleanly coded, well-explained, and fully operational solver.


\section{From Python to Prolog and back}\label{sec:Pylog}
This section offers a reasonably detailed overview of Pylog and how it relates to Prolog. Our strategy is to show how a standard Python program can be transformed, step-by-step, into a structurally similar Prolog program. Listings of these programs are gathered together in section \ref{sec:listings}.

As an example problem, we use the computation of a transversal. Given a sequence of sets (in our case lists without repetition), a transversal is a non-repeating sequence of elements with the property that the \textit{n\textsuperscript{th}} element of the traversal belongs to the \textit{n\textsuperscript{th}} set in the sequence.\footnote{From here on, we refer informally to the lists in our example as \textit{sets}.}  For example, the collection of sets [[1, 2, 3], [2, 4], [1]] has three transversals: [2, 4, 1], [3, 2, 1], and [3, 4, 1]. We use the transversal problem because it lends itself to depth-first search, the default Prolog control structure.\footnote{We use traditional, i.e., naive, depth-first search. Most modern Prologs include a constraint processing package such as CLP(FD)\cite{Triska2016}, which makes search much more efficient.

Instead of scanning the sets in the order given, one can select the next set to scanned based on how constrained the sets are. Given [[1, 2, 3], [2, 4], [1]], the third set would be scanned first, with 1 selected as its representative---thereby precluding the selection of 1 for the first set.

Another efficiency measure involves propagating constraints. Suppose our example sets are scanned from left to right. If 1 is selected from the first set, that choice would be propagated forward, eliminating 1 from the final set. Since the final set would then have no choices left, one can conclude that selecting 1 from the first set does not lead to a solution. 

Application of such constraint rules eliminate much of the backtracking inherent in naive depth-first search. Powerful as they are, we do not use such constraint techniques in this example.}

We will discuss five functions for finding transversals---the first four in Python, the final one in standard Prolog. As we discuss these programs we will introduce various Pylog features. Here is a road-map for the programs to be discussed and the Pylog features they illustrate. (To simplify formatting, in naming the programs we use \textittt{tvsl} in place of \textittt{transversal})

\begin{enumerate}
\item \textittt{tvsl\_dfs\_first} is a standard Python program that performs a depth-first search. It returns the first transversal it finds. It contains no Pylog features, but it illustrates the overall structure the others follow. 
\item \textittt{tvsl\_dfs\_all}. In contrast to \textittt{tvsl\_dfs\_first}, the program \textittt{tvsl\_dfs\_all} finds and returns \textit{all} transversals. A very common strategy, and the one \textittt{tvsl\_dfs\_all} uses, is to gather all transversals into a collection as they are found and return that collection at the end.

\item \textittt{tvsl\_dfs\_yield} also finds and returns all transversals, but it returns them one at a time as requested, as in Prolog. \textittt{tvsl\_dfs\_all} does this through the use of the Python generator structure, i.e., the \textbftt{\textbf{yield}} statement. This moves us an important step toward a Prolog-like control structure.
\smallv
\item \textittt{tvsl\_dfs\_yield\_lv} introduces logic variables, one of the most important features of Prolog.  
\item \textittt{tvsl\_prolog} is a straight Prolog program. It is operationally identical to \textittt{tvsl\_dfs\_yield\_lv}, but syntactically very different. 
\end{enumerate}

The first three Python programs have similar signatures. 

\begin{python}
def tvsl_python_1_2_3(sets: List[List[int]], 
                      partial_transversal: List[int])
                             -> <some return type>: 
\end{python}
(The return types differ from one program to an other.)

Both the fourth Python program and the Prolog program have a third parameter. Their return type, if any, is not meaningful for our purposes. In these programs, transversals, when found, are returned through the third parameter---as one does in Prolog.

\begin{python}
def tvsl_python_4(sets: List[List[int]], 
                  partial_transversal: List[int],
                  Complete_Transversal: Var)
\end{python}

\begin{python}
tvsl_prolog(+Sets, +Partial_Transversal, -Complete_Transversal)
\end{python}

The signatures have the following in common. 
\begin{enumerate}
\item The first argument lists the sets for which a transversal is desired. Initially this is the full list of sets. The programs recursively step through the list, selecting an element from each set. At each recursive call, the first argument lists the remaining sets. 

\item The second argument is a partial transversal consisting of elements selected from sets that have already been scanned. Initially, this argument is the empty list.

\item The third parameter and the returned transversal.
    \begin{enumerate}
    \item The first two programs have no third parameter. They return a single transversal, a set of transversals, or \textbftt{None} through the normal \textbftt{return} mechanism.
    
    \item The final Python function and the Prolog predicate both have a third parameter. Neither returns a value through a normal Python \textbftt{return} mechanism. In both, the third argument is initially an uninstantiated logic variable, which will be unified with a transversal that is being returned.
    \end{enumerate}
\end{enumerate}

We now turn to the details of the programs. For each program, we first introduce the relevant Python/Pylog constructs and then discuss how they are used in the example program.

\subsection{\textittt{tvsl\_dfs\_first}}

% \begin{itemize}
% \item \textittt{tvsl\_dfs\_first}. 

\begin{minipage}{\linewidth} \largev   \hrulefill
\begin{python}[numbers=left]
def tvsl_dfs_first(sets: List[List[int]], partial_transversal: List[int]) -> Optional[List[int]]:
  print(f'sets: {sets}; partial_transversal: {partial_transversal}')  
  if not sets:
    return partial_transversal
  else:
    for element in sets[0]:
      if element not in partial_transversal:
        complete_transversal = tvsl_dfs_first(sets[1:], partial_transversal + [element])
        if complete_transversal is not None:
          return complete_transversal 
    return None
\end{python}
\begin{lstlisting} [caption={\textittt{tvsl\_dfs\_first}}, label={lis:dfsfirst}]
\end{lstlisting}
\end{minipage}

\smallv

\textittt{tvsl\_dfs\_first} uses standard depth-first recursive search to find a single transversal. As the listing above shows, when we reach the end of the list of sets, we are done. At that point we return  \textit{partial\_transversal}, which is then known to be a complete transversal, if there is one. 

The return type is \textittt{Optional[List[int]]}, i.e., either a list of \textittt{int}s, or \textbftt{None} for the case in which no transversal is found. The latter situation occurs when, after considering all elements of the current set (\textittt{sets[0]}) (line 6), we have not found a complete transversal.  

It may be instructive to look at the log (below) created by the print statement (line 2).\footnote{All the programs in this section produce a log. This is the only log included in this paper.} It shows the value of the parameters at the start of each execution of the function. When \textittt{sets} is the empty list (line 3), we have found a transversal. On the other hand, when the function reaches a dead-end, it "backtracks" to the next element in the current set and tries again. 
\smallv
\smallv
\smallv

\begin{minipage}{\linewidth}  \largev   \hrulefill
\begin{python}[numbers=left]
sets: [[1, 2, 3], [2, 4], [1]]; partial_transversal: []
sets: [[2, 4], [1]]; partial_transversal: [1]
sets: [[1]]; partial_transversal: [1, 2]
sets: [[1]]; partial_transversal: [1, 4]
sets: [[2, 4], [1]]; partial_transversal: [2]
sets: [[1]]; partial_transversal: [2, 4]
sets: []; partial_transversal: [2, 4, 1]
                                =>  [2, 4, 1]
\end{python}
\begin{lstlisting} [caption={\textittt{transversal\_dfs\_first trace}},  label={lis:dfsfirsttrace}]
\end{lstlisting}
\end{minipage}

The first three lines of the log show that we have selected \textittt{[1, 2]} as the \textittt{partial\_transversal} and must now select an element of \textittt{[1]}, the remaining set. Since \textittt{1} is already in the \textittt{partial\_transversal}, it can't be selected to represent the final set. So we (blindly, as is the case with naive depth-first search) backtrack to the selection from the second set. We had initially selected \textittt{2}. Line 4 shows that we have now selected \textittt{4}. Of course that doesn't help. Having exhausted all elements of the second set, we backtrack all the way to our selection from the first set. Line 5 of the log shows that we have now selected \textittt{2} from the first set and are about to make a selection from the second set. We cannot select \textittt{2} from the second set since it is already in the \textittt{partial\_transversal}. Instead, we select \textittt{4} from the second set. We are then able to select \textittt{1} from the final set to complete the transversal. 

Even though this is a simple depth-first search, it incorporates (what appears to be) backtracking, one of the mainstays of Prolog. What implements the backtracking? In fact, there is no backtracking. The nested \textbftt{for} loops produce a backtracking effect. Although this program uses recursion to produce the nesting, recursion is not a requirement. 

Prolog, uses the term \textit{choicepoint} for places in the program at which (a) multiple choices are possible and (b) one wants to try them all if necessary. Pylog implements choicepoints by means of such nested \textbf{for} loops and related mechanisms.

\subsection{\textittt{tvsl\_dfs\_all}}

% \item \textittt{tvsl\_dfs\_all}. 

\begin{minipage}{\linewidth} \largev   \hrulefill
\begin{python}[numbers=left]
def tvsl_dfs_all(sets: List[List[int]], partial_transversal: List[int]) -> List[List[int]]:
  print(f'sets: {sets}; partial_transversal: {list(partial_transversal)}')
  if not sets:
    return [partial_transversal]
  else:
    all_transversals = []
    for element in sets[0]:
      if element not in partial_transversal:
        all_transversals += tvsl_dfs_all(sets[1:], partial_transversal + [element])
    return all_transversals
\end{python}
\begin{lstlisting} [caption={transversal\_dfs\_all},  label={lis:dfsall}]
\end{lstlisting}
\end{minipage}

\smallv

\textittt{tvsl\_dfs\_all} finds and returns \textit{all} transversals. It has the same structure as \textittt{tvsl\_dfs\_first} except that instead of returning a single transversal, each transversal is added to \textittt{all\_transversals} (line 9), which is returned when the program terminates. 

Note that \textittt{tvsl\_dfs\_first} returns \textbftt{None} if no transversal is found; \textittt{tvsl\_dfs\_all} returns an empty list. 


\subsection{\textittt{tvsl\_yield}}

\begin{minipage}{\linewidth} \largev   \hrulefill
\begin{python}[numbers=left]
def tvsl_yield(sets: List[List[int]], partial_transversal: List[int]) -> Generator[List[int], None, None]:
  print(f'sets/{sets}; '
        f'partial_transversal/{partial_transversal}')
  if not sets:
    yield partial_transversal
  else:
    for element in sets[0]:
      if element not in partial_transversal:
        yield from tvsl_yield(sets[1:], partial_transversal + [element])
            
\end{python}
\begin{lstlisting} [caption={transversal\_dfs\_yield},  label={lis:dfsyield}]
\end{lstlisting}
\end{minipage}

\smallv
% \item  \textittt{tvsl\_yield}. 

\textittt{tvsl\_yield}, although quite similar to \textittt{tvsl\_dfs\_first}, takes a significant step toward mimicking Prolog. Whereas \textittt{tvsl\_dfs\_first} \textbftt{return}s the first transversal it finds, \textittt{tvsl\_yield} \textbftt{yield}s \textit{all} the transversals it finds---but one at a time.  Instead of looking for a single transversal on lines 8 - 10 with:
\begin{python}
complete_transversal = tvsl_dfs_first(ss, partial_transversal + [element])
\end{python}
and then \textbftt{return}ing those that are not \textbftt{None}, \textittt{tvsl\_yield} uses \textbftt{yield from} (line 9) to search for and \textbftt{yield} \textit{all} transversals---but one at a time.

\begin{python}
yield from tvsl_yield(ss, partial_transversal + [element])
\end{python}

We discuss Python's \textbftt{yield} and \textbftt{yield from} in more detail below. The mechanism it provides allows us to mimic Prolog's choicepoints.

\subsection{\textittt{tvsl\_yield\_lv}}

\begin{minipage}{\linewidth} \largev   \hrulefill
\begin{python}[numbers=left]
def tvsl_yield_lv(Sets: List[PyList], 
                  Partial_Transversal: PyList,
                  Complete_Transversal: Var):
  print(f'Sets/[{", ".join([str(S) for S in Sets])}]; '
        f'Partial_Transversal/{Partial_Transversal}')
  if not Sets:
    yield from unify(Partial_Transversal,Complete_Transversal)
  else:
    Element = Var( )
    for _ in member(Element, Sets[0]):
      for _ in fails(member)(Element, Partial_Transversal):
        yield from tvsl_yield_lv(Sets[1:], 
                                 Partial_Transversal + PyList([Element]), 
                                 Complete_Transversal)
\end{python}
\begin{lstlisting} [caption={transversal\_dfs\_yield\_lv},  label={lis:dfsyieldlv}]
\end{lstlisting}
\end{minipage}

\smallv

\textittt{tvsl\_yield\_lv} moves toward Prolog along a second dimension---the use of logic variables.
% \smallv

One of Prolog's defining features is its logic variables. A logic variable is similar to a variable in mathematics. It may or may not have a value, but once it gets a value, its value never changes---i.e., logic variables are immutable.

The primary operation on logic variables is known as \textit{unification}. When a logic variable is \textit{unified} with what is known as a \textit{ground term}, e.g., a number, a string, etc., it acquires that term as its value. For example, if \textittt{X} is a logic variable,\footnote{A note about identifiers. The Python convention is to use only lower case letters in identifiers other than class names. The Prolog convention is that the first letter of an identifier determines whether it's a constant term or a variable: variables begin with upper case letters. 
\smallv
    
In the first three programs we have used strictly lower case letters in identifiers. In \textittt{tvsl\_yield\_lv}, and of course in the Prolog program to follow, we use upper case letters to begin identifiers that refer to Prolog-like logic variables. Thus the \textittt{X} and  \textittt{Y} in this discussion begin with upper case letters. In \textittt{tvsl\_yield\_lv}, the identifiers \textittt{Partial\_Transversal} and \textittt{Complete\_Transversal} begin with upper case letters. Even though they are Python variables, they are used as Pylog logic variables.} then after \textittt{unify(3, X)},\footnote{or \textittt{unify(X, 3)}, the order of the arguments is not relevant} \textittt{X} has the value \textittt{3}. 

Like mathematical variables, logic variables may be set equal to each other---even if neither has a value. So if \textittt{X} and \textittt{Y} are two logic variables, then after \textittt{unify(X, Y)} whatever value either eventually gets will be considered the value of the other. 

Consider the following example.(The function \textittt{Var} returns a new logic variable.) \footnote{In this example, as in many Pylog programs, operations, such as \textittt{unify(X, Y)} are in the body of a \textbftt{for} loop. The \textbftt{for} loop defines the scope of the operation.  The \textbftt{for} loop (deliberately) lacks an index variable since it serves no purpose in this context.}

\smallv

\begin{minipage}{\linewidth} \largev   \hrulefill
\begin{python}
(A, B, C, D, E) = (Var(), Var(), Var(), Var(), 'abc')
for _ in unify(A, B):
  for _ in unify(D, C):
    for _ in unify(A, C):
      for _ in unify(E, D):
\end{python}
\begin{lstlisting} [caption={Unifying logic variables},  label={lis:unifylogicvars}]
\end{lstlisting}
\end{minipage}
Within the body of the final loop, \textittt{A}, \textittt{B}, \textittt{C}, \textittt{D}, and \textittt{E} all have the value \textittt{'abc'}.
\smallv

The following convenience methods make it possible to write the preceding code more concisely.
\begin{itemize}
    \item \textittt{n\_Vars} takes an integer argument and generates that many \textittt{Var} objects.
    \item \textittt{unify\_pairs} takes a list of pairs (as tuples) and unifies the elements of each pair.
\end{itemize}

\begin{minipage}{\linewidth} \largev   \hrulefill
\begin{python}
(A, B, C, D, E) = (*n_Vars(4), 'abc')
for _ in unify_pairs([(A, B), (D, C), (A, C), (E, D)]):
\end{python}
\begin{lstlisting} [caption={Unifying logic variables shortened},  label={lis:unifylogicvarsshortened}]
\end{lstlisting}
\end{minipage}

\smallv

% Since the Python functions \textit{unify} and \textit{unify\_pairs} are both Python generators, they must be called from something like a \textbftt{for} loop as shown---rather than as  standard function calls.
\smallv

Here are a few additional considerations about  \textit{tvsl\_yield\_lv}.
\begin{itemize}
\item  \textit{tvsl\_yield\_lv} has a third parameter, \textit{Complete\_Transversal}, which is declared as a \textit{Var}, i.e., a logic variable. When \textit{tvsl\_yield\_lv} is called, \textit{Complete\_Transversal} is passed an uninstantiated \textittt{Var}. If \textit{Sets} is empty, we perform \textit{unify(Partial\_Transversal, Complete\_Transversal)} (line 7), which gives \textit{Complete\_Transversal} the same value as \textit{Partial\_Transversal}. This is typical of how Prolog programs return values: unify an argument with the value to be returned.
\item The \textbftt{else} clause (line 8) defines \textit{Element} to be a \textit{Var}. The line 
    \begin{python}
        for _ in member(Element, S):
    \end{python}
    unifies \textit{Element} with a different member of \textit{S} on each iteration. 
\item The syntax of the \textbftt{for} iterator/generator loop is worth a remark. The function \textit{member} is written to perform its own \textit{unify} operation and then to perform a \textit{yield}. This makes it suitable for use in a \textbf{for} iterator/generator loop as shown. 

\item The \textbftt{for} loop itself does not produce a value in the normal way. (Note the underscore.) Instead, after \textit{member} unifies its first argument with an element of its second, it \texttt{\textbf{yield}}s to indicate that the unification is complete. For example, 
%\begin{minipage}{\linewidth} \largev   \hrulefill
\begin{python}
Element = Var()
for _ in member(Element, S):
  print(Element)
\end{python}
%\begin{lstlisting} [caption={Unifying logic vars in for-loop},  label={lis:unifyinforloop}]
%\end{lstlisting}
%\end{minipage}
prints the elements of \textit{S}.
\smallv

\item In earlier functions we had written 
\begin{python}
if element not in partial_transversal:
\end{python}
In \texttt{tvsl\_yield\_lv} we write (line 11)
\begin{python}
for _ in fails(member)(Element, Partial_Transversal):
\end{python}
The Pylog \textit{fails} function does the same job as \textit{\textbackslash+}, i.e., negation, in Prolog. \textit{fails} takes another function as an argument---much like a Python decorator---and returns a function that succeeds or fails when its argument function fails or succeeds. Thus, although it's not boolean,
\begin{python}
for _ in fails(member)(Element, Partial_Transversal):
\end{python} 
plays a similar role as
\begin{python}
if element not in partial_transversal:
\end{python}
\smallv

Pylog offers \textit{would\_succeed} for double negation, Prolog's \texttt{\textbackslash+\textbackslash+}.
\begin{python}
for _ in would_succeed(member)(Element, S):
\end{python} 
succeeds if and only if
\begin{python}
for _ in member(Element, S):
\end{python} 
would succeed. The only (but very important) difference is that, as in  Prolog's double negation, \textit{would\_succeed} does not unify any variables.
\smallv

\item Consider lines 12-15 of the listing. They use Python's \textbftt{yield} \textbftt{from} construct.\footnote{Python's \textbftt{yield} \textbftt{from} has additional uses, which Pylog does not exploit.} \texttt{\textbftt{yield from} <something>} can be considered shorthand for
\begin{python}
for X in <something>:
  yield X
\end{python}

As before, \textittt{X} may be an underscore, in which case nothing is \textbftt{yield}ed, and the \textbftt{yield} statement is simply \textbftt{yield} with no argument.

\item Consider how \textbftt{yield from} is used in this program. It performs four functions.
\begin{enumerate}
    \item It calls the remaining program to be executed. In this case, it's a recursive call, but that need not be the case.
    \item It passes values from previously executed code to the called function. 
    \item It also passes on an uninstantiated variables---the third argument---which the remainder of the program will (presumably) instantiate.
    \item Since it functions as a \textbftt{yield}, it returns what will be the newly instantiated argument back up the \textbftt{yield} chain.
\end{enumerate}
\smallv

We can put this into a Prolog context. Consider a standard Prolog clause.

\begin{minipage}{\linewidth} \largev \hrulefill
\begin{python}[numbers=left]
    head(<args>) :-
        term_1(<args_1>),
        term_2(<args_2>), 
        ...
        term_n(<args_n>).
\end{python}
\begin{lstlisting} [caption={A prolog clause}, label={lis:prolog_clause}]
\end{lstlisting}
\end{minipage}

The relationship between a clause head and its body as well as that between each term and the rest of the body is exactly a \textbftt{yield from} relationship. 
% \smallv

This observation leads to our proposed Python template for Prolog, discussed at the end of section \ref{subsec:zebra}. 
\end{itemize}

\subsection{\textittt{tvsl\_prolog}}

\begin{minipage}{\linewidth} \largev \hrulefill
\begin{python}[numbers=left]
tvsl_prolog(Sets, Partial_Transversal, _Complete_Transversal) :-
    writeln('Sets'/Sets;'  Partial_Transversal'/Partial_Transversal), 
    fail.

tvsl_prolog([], Complete_Transversal, Complete_Transversal) :-
    format('                                  '),
    writeln('Complete_Transversal '=Complete_Transversal), nl.

tvsl_prolog([S|Ss], Partial_Transversal, Complete_Transversal_X) :-
    member(X, S),
    \+ member(X, Partial_Transversal),
    append(Partial_Transversal, [X], Partial_Transversal_X),
    tvsl_prolog(Ss, Partial_Transversal_X, Complete_Transversal_X).

\end{python}
\begin{lstlisting} [caption={transversal\_prolog},  label={lis:transversalprolog}]
\end{lstlisting}
\end{minipage}

\smallv
% \item 
\textittt{tvsl\_prolog}, the final program, is straight Prolog. \textit{tvsl\_prolog} and \textit{tvsl\_yield\_lv} are the same program expressed in different languages.

% \end{itemize}

% \subsection{Listings}\label{sec:listings}

\begin{center}

\begin{minipage}{\linewidth}
\begin{python}[numbers=left]
def transversal_dfs_first(sets: List[List[int]],
                          partial_transversal: List[int]) \
                          -> Optional[List[int]]:
  print(f'sets/{sets}; '
        f'partial_transversal/{partial_transversal}')  
  if not sets:
    return partial_transversal
  else:
    for element in sets[0]:
      if element not in partial_transversal:
        complete_transversal = transversal_dfs_first(sets[1:],
                                                     partial_transversal 
                                                     + [element])
        if complete_transversal is not None:
          return complete_transversal    
\end{python}
\begin{lstlisting} [caption={transversal\_dfs\_first},  label={lis:dfsfirst}]
\end{lstlisting}
\end{minipage}


\begin{minipage}{\linewidth}

\begin{python}[numbers=left]
sets/[[1, 2, 3], [2, 4], [1]]; partial_transversal/[]
sets/[[2, 4], [1]]; partial_transversal/[1]
sets/[[1]]; partial_transversal/[1, 2]
sets/[[1]]; partial_transversal/[1, 4]
sets/[[2, 4], [1]]; partial_transversal/[2]
sets/[[1]]; partial_transversal/[2, 4]
sets/[]; partial_transversal/[2, 4, 1]
                                =>  [2, 4, 1]
\end{python}
\begin{lstlisting} [caption={transversal\_dfs\_first trace},  label={lis:dfsfirsttrace}]
\end{lstlisting}
\end{minipage}


\begin{minipage}{\linewidth}
\begin{python}[numbers=left]
def transversal_dfs_all(sets: List[List[int]],
                        partial_transversal: List[int]) \
                        -> List[List[int]]:
  print(f'sets/{sets}; partial_transversal/{list(partial_transversal)}')
  if not sets:
    return [partial_transversal]
  else:
    all_transversals = []
    for element in sets[0]:
      if element not in partial_transversal:
        all_transversals += \
          transversal_dfs_all(sets[1:], partial_transversal + [element])
    return all_transversals
\end{python}
\begin{lstlisting} [caption={transversal\_dfs\_all},  label={lis:dfsall}]
\end{lstlisting}
\end{minipage}


\begin{minipage}{\linewidth}
\begin{python}[numbers=left]
def transversal_yield(sets: List[List[int]], 
                      partial_transversal: List[int]) \
                      -> Generator[List[int], None, None]:
  print(f'sets/{sets}; '
        f'partial_transversal/{partial_transversal}')
  if not sets:
    yield partial_transversal
  else:
    for element in sets[0]:
      if element not in partial_transversal:
        yield from \ 
          transversal_yield(sets[1:], partial_transversal + [element])
            
\end{python}
\begin{lstlisting} [caption={transversal\_dfs\_yield},  label={lis:dfsyield}]
\end{lstlisting}
\end{minipage}


\begin{minipage}{\linewidth}
\begin{python}[numbers=left]
def transversal_yield_lv(Sets: List[PyList], 
                         Partial_Transversal: PyList,
                         Complete_Transversal: Var):
  print(f'Sets/[{", ".join([str(S) for S in Sets])}]; '
        f'Partial_Transversal/{Partial_Transversal}')
  if not Sets:
    yield from unify(Partial_Transversal,Complete_Transversal)
  else:
    Element = Var( )
    for _ in member(Element, Sets[0]):
      for _ in fails(member)(Element, Partial_Transversal):
        yield from \
          transversal_yield_lv(Sets[1:], 
                               Partial_Transversal + PyList([Element]), 
                               Complete_Transversal)
    \end{python}
    \begin{lstlisting} [caption={transversal\_dfs\_yield\_lv},  label={lis:dfsyieldlv}]
    \end{lstlisting}
\end{minipage}


\begin{minipage}{\linewidth}
\begin{python}[numbers=left]
transversal_prolog(Sets, Partial_Transversal, _Complete_Transversal) :-
    writeln('Sets'/Sets;'  Partial_Transversal'/Partial_Transversal), 
    fail.

transversal_prolog([], Complete_Transversal, Complete_Transversal) :-
    format('                                  '),
    writeln('Complete_Transversal '=Complete_Transversal), nl.

transversal_prolog([S|Ss], Partial_Transversal, Complete_Transversal_X) :-
    member(X, S),
    \+ member(X, Partial_Transversal),
    append(Partial_Transversal, [X], Partial_Transversal_X),
    transversal_prolog(Ss, Partial_Transversal_X, Complete_Transversal_X).

\end{python}
\begin{lstlisting} [caption={transversal\_prolog},  label={lis:transversalprolog}]
\end{lstlisting}
\end{minipage}

\end{center}

% \begin{minipage}{\linewidth}
% \begin{python}
%     <your code goes here>
% \end{python}
% \begin{lstlisting}[caption={caption_goes_here},  label={label_goes_here}]
% \end{lstlisting}
% \end{minipage}




%%%%%%%%%%%%%%%%%%%%%%%%%%%%%%%%%%%%%%%%%%%%%%
\section{Logic variables} \label{sec:logic-variables}
%%%%%%%%%%%%%%%%%%%%%%%%%%%%%%%%%%%%%%%%%%%%%%
This section discusses logic variables and their realization. 

\subsection{Instantiation}
Logic variables are either instantiated, i.e., have a value, or uninstantiated. The instantiation operation is called \textit{unify}.   

\textit{unify} is a \textit{generator}---even though the act of instantiation \textit{does not} \textbf{yield} a value. Activating \textit{unify} establishes a context within which unification holds. Leaving that context undoes the unification. 

Consider the following sequence of short code segments. 
\begin{center}
\begin{minipage}[c]{0.45\textwidth}
\begin{python1}
A = Var()
\end{python1}
\end{minipage}
\end{center}
\textit{A} is a standard Python identifier. We use an initial capital letter to distinguish logic variables from regular Python variables. \textit{Var} is the constructor for logic variables.

\textit{A} is now an uninstantiated logic variable. When an uninstantiated logic variables is printed, an internal value is shown to distinguish among logic variables. As the first logic variable in this program, \textit{A}'s internal value is \textit{\_1}.

\begin{center}
\begin{minipage}[c]{0.45\textwidth}
\begin{python1}
print(A)  # => _1
\end{python1}
\end{minipage}
\end{center}

Now we \textit{unify A} with \textit{abc}, i.e., instantiate \textit{A} to \textit{abc}. Since \textit{unify} does not \textbf{yield} a value, the \textbf{for}-loop variable is not used. 

The \textbf{for}-loop establishes a context for \textit{unify}. Within the \textbf{for}-loop body \textit{A} is instantiated to  \textit{abc}.

\begin{center}
\begin{minipage}[c]{0.45\textwidth}
\begin{python1}
for _ in unify(A, 'abc'):
    print(A)  # => abc
print(A)  # => _1
\end{python1}
\end{minipage}
\end{center}

Since there is only one way to \textit{unify A} with \textit{abc}, the  \textbf{for}-loop body runs only once.  Leaving the \textit{unify} context undoes the instantiation.

Within a \textit{unify} context, logic variables are immutable. Once a logic variable has a value, it cannot change within its context.

\begin{center}
\begin{minipage}[c]{0.45\textwidth}
\begin{python1}
A = Var()
print(A)  # => _1
for _ in unify(A, 'abc'):
    print(A)  # => abc
    # This unify fails. Its body never runs.
    for _ in unify(A 'def'):
      print(A)  # Never executed
    print(A)  # => abc
print(A)  # => _1
\end{python1}
\end{minipage}
\end{center}

\subsection{The power of \textit{unify}}
\textit{unify} can also identify logic variables with each other. After two uninstantiated logic variables are unified, whenever either gets a value, the other gets that same value.

Unification is surprisingly straightforward. Each \textit{Var} includes a \textit{next} field, which is initially \textbf{None}. When two \textit{Var}s are unified, the result depends on their states of instantiation.  
\begin{itemize}
    \item If both are uninstantiated the \textit{next} field of one points to the other. It makes no difference which points to which. A chain of linked  \textit{Var}s unifies all the \textit{Var}s in the chain. 
    \item If only one is uninstantiated, the uninstantiated one points to the other.  
    \item If both are instantiated to the same value, they are effectively unified. \textit{unify succeeds} but nothing changes.
    \item If both are instantiated but to different values, \textit{unify fails}.
\end{itemize}

A note on terminology. When called (as part of a \textbf{for}-loop) a generator will either \textbf{yield} or \textbf{return}. When a generator \textbf{yield}s, it is said to \textit{succeed}; the \textbf{for}-loop body runs. When a generator \textbf{return}s, it is said to fail; the \textbf{for}-loop body does not run. Instead we exit the \textbf{for}-loop.

We can trace the unifications in Listing \ref{unif-example}.  

% \begin{figure}[hbt]
% \centering
\begin{center}
\begin{minipage}[c]{0.45\textwidth}
\begin{python1}
(A, B, C, D) = (Var(), Var(), Var(), Var())
print(A, B, C, D) # => _1 _2 _3 _4
for _ in unify(A, B):
  for _ in unify(D, C):
    print(A, B, C, D) # => _2 _2 _3 _3
    for _ in unify(A, 'abc'):
      print(A, B, C, D) # => abc abc _3 _3
      for _ in unify(A, D):
        print(A, B, C, D) # => abc abc abc abc
      print(A, B, C, D) # => abc abc _3 _3
    print(A, B, C, D) # => _2 _2 _3 _3
  print(A, B, C, D) # => _2 _2 _3 _4
print(A, B, C, D) # => _1 _2 _3 _4
\end{python1}\linv
\begin{lstlisting} [caption={\textit{Unification example}},  label={unif-example}]
\end{lstlisting}
\end{minipage}
\end{center}
% \end{figure}

The first unifications, lines 3 and 4, produce the following. 
\begin{equation}\label{eq:one}
\begin{array}{c c c c c c c c }
A & \to & B \\
D & \to & C 
\end{array}
\end{equation}

Line 6 unifies \textit{A} and \textit{'abc'}. The first step is to go to the ends of the relevant unification chains. In this case, \textit{B} (the end of \textit{A}'s unification chain) is pointed to \textit{'abc'}. Since  \textit{'abc'} is instantiated, the arrow can only go from \textit{B} to \textit{'abc'}. 

\begin{equation}\label{eq:two}
\begin{array}{c c c c c c c c c c c}
A & \to & B            & \to & 'abc'    \\ 
  &     & D            & \to & C        
\end{array}
\end{equation}

Finally, line 8  unifies \textit{A} with \textit{D}. \textit{C} (the end of \textit{D}'s unification chain) is set to point to \textit{'abc'} (the end of \textit{A}'s unification chain). % The arrow can go only from \textit{C} to \textit{'abc'}.

\begin{equation}\label{eq:three}
\begin{array}{c c c c c c c c c c c}
A & \to & B            & \to & 'abc'      \\ 
  &     &              &     & \uparrow   \\ 
  &     & D            & \to & C        
\end{array}
\end{equation}


% To determine the value of a logical variable, one goes to the end of its unification chain. If the end element is instantiated, that is the (current) value of the variable. That's why all the variables have \textit{abc} as their values after line 8. 

% If the end of a unification chain is unintantiated, the internal value associated with that end element is a place-holder value.

\smallv
\subsection{A logic-variable version of \textit{tnvsl\_dfs\_gen}}
Listing \ref{lis:dfs-with-gen-and-logic-variables} adapts Listing \ref{lis:dfs-gen} for logic variables. The strategy is for \textit{trnsvl} to start as a tuple of uninstantiated \textit{Var}s, which become instantiated as the program runs.

First, an adapted \textit{uninstan\_indices\_lv} returns the indices of the uninstantiated \textit{Var}s in \textit{trnsvl}.
\begin{center}
\begin{minipage}[c]{0.45\textwidth}
\begin{python1}
def uninstan_indices_lv(tnvsl):
  return [indx for indx in range(len(tnvsl)) 
               if not tnvsl[indx].is_instantiated()]
\end{python1}
\end{minipage}
\end{center}

Note that \textit{tnvsl[indx]} retrieves the \textit{indx\textsuperscript{th}} \textit{tnvsl} element. If it's instantiated, it represents the value associated with the \textit{indx\textsuperscript{th}} set. If not, we don't yet have a value for the  \textit{indx\textsuperscript{th}} set.

\begin{figure}[htb]
\centering
\begin{minipage}[c]{0.45\textwidth}
\begin{python1}
def tnvsl_dfs_gen_lv(sets, tnvsl):
  var_indxs = uninstan_indices_lv(tnvsl)
    
  if not var_indxs: yield tnvsl
  else:
    empty_sets = [sets[indx].is_empty() 
                  for indx in var_indxs]
    if any(empty_sets): return None

    nxt_indx = min(var_indxs,
                   key=lambda indx: len(sets[indx]))
    used_values = PyList([tnvsl[i] 
                          for i in range(len(tnvsl)) 
                          if i not in var_indxs])
    T_Var = tnvsl[nxt_indx]
      for _ in member(T_Var, sets[nxt_indx]):
        for _ in fails(member)(T_Var, used_values):
          new_sets = [set.discard(T_Var) 
                      for set in sets]
          yield from tnvsl_dfs_gen_lv(new_sets, 
                                      tnvsl)
\end{python1}\linv
\begin{lstlisting} [caption={\textit{dfs-with-gen-and-logic-variables}},  label={lis:dfs-with-gen-and-logic-variables}]
\end{lstlisting}
\end{minipage}\linv
\end{figure}

Some comments on Listing \ref{lis:dfs-with-gen-and-logic-variables}. (We reformatted some of the lines and changed some of the names from \textit{tnvsl\_dfs\_gen} (Listing \ref{lis:dfs-gen}) so that the program will fit the width of a column.)

\begin{itemize}
    \item \textit{line 6}. The parameter \textit{sets} is a list of \textit{PySet}s. These are logic variable versions of sets. An \textit{is\_empty} method is defined for them.
    \item \textit{lines 12-14}. \textit{used\_values} are the values of the instantiated \textit{tnvsl} elements.
    \item \textit{line 15}. \textit{T\_Var} is the element at the \textit{nxt\_indx\textsuperscript{th}} position of \textit{tnvsl}. Since \textit{nxt\_indx} was selected from the uninstantiated variables, \textit{T\_Var} is an uninstantited \textit{Var}.
    \item \textit{line 16}. \textit{member} successively unifies its first argument with the elements of its second argument. It's equivalent to \textit{\textbf{for} T\_Var \textbf{in} sets[nxt\_indx]} but using unification.
    \item  \textit{line 17}. \textit{fails} takes a predicate as its argument. It converts the predicate to its negation. So \textit{fails(member)} succeeds if and only if \textit{member} fails.
    \item  \textit{line 18}. \textit{PySet}s have a \textit{discard} method that returns a copy of the \textit{PySet} without the argument.
\end{itemize}

When run, we get the same result as before---except that the uninstantiated transversal variables appear as we saw above.
\begin{center}
\begin{minipage}[c]{0.45\textwidth}
\begin{python1}
sets: [{1,2,3}, {1,2,4}, {1}], tnvsl: (_1, _2, _3)
  sets: [{2,3}, {2,4}, {}], tnvsl: (_1, _2, 1)
    sets: [{3}, {4}, {}], tnvsl: (2, _2, 1)
      sets: [{3}, {}, {}], tnvsl: (2, 4, 1)
=> (2, 4, 1)
    sets: [{2}, {2,4}, {}], tnvsl: (3, _2, 1)
      sets: [{}, {4}, {}], tnvsl: (3, 2, 1)
=> (3, 2, 1)
      sets: [{2}, {2}, {}], tnvsl: (3, 4, 1)
=> (3, 4, 1)
\end{python1}
\end{minipage}
\end{center}

The following logic variable version of Listing \ref{lis:dfs-gen-call} will run \textit{tnvsl\_dfs\_gen\_lv} and produce the same result.

\begin{center}
\begin{minipage}[c]{0.45\textwidth}
\begin{python1}
(A, B, C) = (Var(), Var(), Var())
Py_Sets = [PySet(set) for set in sets]
# PyValue creates a logic variable constant.
N = PyValue(6)
for _ in tnvsl_dfs_gen_lv(Py_Sets, (A, B, C)):
  sum_string = ' + '.join(str(i) for i in (A, B, C))
  equals = '==' if A + B + C == N else '!='
  print(f'{sum_string} {equals} {N}')
  if A + B + C == N: break
\end{python1}
\end{minipage}
\end{center}

Here we created three logic variables,  \textit{A}, \textit{B}, and \textit{C} and passed them to \textit{tnvsl\_dfs\_gen\_lv} on line 5. Each time a transversal is found, the body of the \textbf{for}-loop is executed with the values to which \textit{A}, \textit{B}, and \textit{C} have been instantiated. 

The preceding offers some sense of what one can do with logic variables. The next section really puts them to work.

\section{Unification and control}\label{subsec:unification}
This section examines the Pylog implementation of unification along with a number of control functions.

\subsection{\textittt{Unification}}
As it turns out, unification is relatively straightforward. 

The \textittt{unify} function is called, \textittt{unify(Left, Right)}, where \textittt{Left} and \textittt{Right} are the Pylog objects to be unified. (Argument order is immaterial.) 

The first step (line 4) ensures that the arguments are Pylog objects. If either is an immutable Python element, such as a string or int, it is wrapped in a \textittt{PyValue}. This allows us to call, e.g, \textittt{unify(X, 'abc')}.
   
There are four \textittt{unify} cases.

\begin{enumerate}
    \item \textittt{Left} and \textittt{Right} are already the same. Since Pylog objects are immutable, neither can change, and there's nothing to do. Succeed quietly via \textbftt{yield}.

    \item \textittt{Left} and \textittt{Right} are both \textittt{PyValue}s, and exactly one of them has a value. Assign the uninstantiated \textittt{PyValue} the value of the instantiated one.
    \smallv \\
    An important step is to set the assignment back to \textbftt{None} after the \textbftt{yield} statement. (line 18) This undoes the unification on backtracking.

    \item \textittt{Left} and \textittt{Right} are both \textittt{Structure}s, and they have the same functor. Unification consists of unifying the respective arguments. 

    \item Either \textittt{Left} or \textittt{Right} is a \textittt{Var}. Point the \textittt{Var} to the other element at the end of the \textittt{Var} element's unification chain. As line 1 shows \textittt{unify} has a decorator. \textittt{euc} ensures that if either argument is a \textittt{Var} it is replaced by the element at the end of its unification chain. ("euc" stands for "end of unification chain".) Again, the unification must be undone on backtracking. (line 30) 

\end{enumerate} 

\begin{minipage}{\linewidth}  \largev \hrulefill
\begin{python}[numbers=left]
@euc
def unify(Left: Any, Right: Any):

  (Left, Right) = map(ensure_is_logic_variable, (Left, Right))

  # Case 1.
  if Left == Right:
    yield

  # Case 2.
  elif isinstance(Left, PyValue) and isinstance(Right, PyValue) and \
       (not Left.is_instantiated( ) or not Right.is_instantiated( )) and \
       (Left.is_instantiated( ) or Right.is_instantiated( )):
    (assignedTo, assignedFrom) = (Left, Right) if Right.is_instantiated( ) else (Right, Left)
    assignedTo._set_py_value(assignedFrom.get_py_value( ))
    yield

    assignedTo._set_py_value(None)

  # Case 3.
  elif isinstance(Left, Structure) and isinstance(Right, Structure) and Left.functor == Right.functor:
    yield from unify_sequences(Left.args, Right.args)

  # Case 4.
  elif isinstance(Left, Var) or isinstance(Right, Var):
    (pointsFrom, pointsTo) = (Left, Right) if isinstance(Left, Var) else (Right, Left)
    pointsFrom.unification_chain_next = pointsTo
    yield

    pointsFrom.unification_chain_next = None

\end{python}
\begin{lstlisting} [caption={unify},  label={lis:unify}]
\end{lstlisting}
\end{minipage}

\subsection{\textittt{Control functions}}
The following control functions are defined. We leave it to the doc-strings to explain what they do. It's striking how straight-forward they are.

\begin{minipage}{\linewidth}  \largev \hrulefill
\begin{python}[numbers=left]
def fails(f):
  """
  Applied to a function so that the resulting function succeeds iff the original fails.
  Note that it is applied to the function itself, not to a function call.
  Similar to a decorator but applied explicitly when used.
  """
  def fails_wrapper(*args, **kwargs):
    for _ in f(*args, **kwargs):
      # Fail, i.e., don't yield, if f succeeds
      return  
    # Succeed if f fails.
    yield     

  return fails_wrapper
\end{python}
\begin{lstlisting} [caption={Control functions},  label={lis:control}]
\end{lstlisting}
\end{minipage}

\begin{minipage}{\linewidth}  \largev \hrulefill
\begin{python}[numbers=left]
def forall(gens):
  """
  Succeeds if all generators in the gens list succeed. The elements in the gens list
  are embedded in lambda functions to avoid premature evaluation.
  """
  if not gens:
    # They have all succeeded.
    yield
  else:
    # Get gens[0] and evaluate the lambda expression to get a fresh iterator.
    # If it succeeds, run the rest of the generators in the list.
    for _ in gens[0]( ):
      yield from forall(gens[1:])
\end{python}
\begin{lstlisting} [caption={Control functions},  label={lis:control}]
\end{lstlisting}
\end{minipage}

\begin{minipage}{\linewidth}  \largev \hrulefill
\begin{python}[numbers=left]
def forany(gens):
  """
  Succeeds if any of the generators in the gens list succeed. On "back-up," tries them all. 
  The gen elements must be embedded in lambda functions.
  """
  for gen in gens:
    yield from gen( )

\end{python}
\begin{lstlisting} [caption={Control functions},  label={lis:control}]
\end{lstlisting}
\end{minipage}

\begin{minipage}{\linewidth}  \largev \hrulefill
\begin{python}[numbers=left]
def trace(x, succeed=True, show_trace=True):
  """
  Can be included in a list of generators (as in forall and forany) to see where we are.
  The second argument determines whether it succeeds or fails.
  When included in a list of forall generators, succeed should be set to True so that
  it doesn't prevent forall from succeeding.
  When included in a list of forany generators, succeed should be set to False so that forany
  will go on the the next generator and won't take this as an extraneous successes.
  """
  if show_trace:
    print(x)
  if succeed:
    yield

\end{python}
\begin{lstlisting} [caption={Control functions},  label={lis:control}]
\end{lstlisting}
\end{minipage}

\begin{minipage}{\linewidth}  \largev \hrulefill
\begin{python}[numbers=left]
def would_succeed(f):
  """
  Applied to a function so that the resulting function succeeds/fails if and only if 
  the original function succeeds/fails.
  If the original function succeeds, this also succeeds but without binding any variables.
  Similar to a decorator but applied explicitly when used.
  """
  def would_succeed_wrapper(*args, **kwargs):
    succeeded = False
    for _ in f(*args, **kwargs):
      succeeded = True
      
    # Do not yield in the context of f succeeding.
    # Un-unify any unifications that occurred in f.
    if succeeded:
      # Succeed if f succeeded.
      yield  
    # else:
    #   Fail if f failed.
    #   pass   

  return would_succeed_wrapper

\end{python}
\begin{lstlisting} [caption={Control functions},  label={lis:control}]
\end{lstlisting}
\end{minipage}


Finally, there is a more complex control structure, which is defined as an instance of the \textittt{Bool\_Yield\_Wrapper} class. It effectively converts \textbftt{yield} to a boolean. \textittt{Bool\_Yield\_Wrapper} instances are created via a \textittt{bool\_yield\_wrapper} decorator function. The decorator returns a function that instantiates \textittt{Bool\_Yield\_Wrapper} with the decorted function.

\begin{minipage}{\linewidth}  \largev \hrulefill
\begin{python}[numbers=left]
def bool_yield_wrapper(gen):
  """
  A decorator. Produces a function that generates a Bool_Yield_Wrapper object. 
  """
  def wrapped_func(*args, **kwargs):
    return Bool_Yield_Wrapper(gen(*args, **kwargs))

  return wrapped_func
\end{python}
\begin{lstlisting} [caption={Control functions},  label={lis:control}]
\end{lstlisting}
\end{minipage}

Suppose the \textittt{append} function discussed earlier is decorated with \textittt{@bool\_yield\_wrapper}. The resulting function may be used as follows. (Note that line 1 is required to generate the \textittt{append}-based instance of \textittt{@bool\_yield\_wrapper}, which is then used in the \textbftt{while} loop.)

\begin{minipage}{\linewidth}  \largev \hrulefill
\begin{python}[numbers=left]
with append(Xs, Ys, Zs) as gen:
  while gen.has_more():
    print(f'Xs = {Xs}\nYs = {Ys}\nZs = {Zs}\n')
\end{python}
\begin{lstlisting} [caption={Control functions},  label={lis:control}]
\end{lstlisting}
\end{minipage}
The output will be all the solutions to the append operation, depending on the initial values and state of instantiation of \textittt{Xs}, \textittt{Ys}, and \textittt{Zs}.

An advantage of this approach is that it avoids the \textbftt{for} loop. \textbftt{for} loops just don't feel like the right structure to use for Prolog backtracking. The disadvantage of this approach is its additional wordiness and increased indentation, i.e., two nested lines, \textbftt{with} and \textbftt{while}, instead of the simpler \textbftt{for} construct. In practice, we found ourselves using the \textbftt{for} construct most of the time.
%%%%%%%%%%%%%%%%%%%%%%%%%%%%%%%%%%%%%%%%%%%%%%
\section{The Zebra Puzzle (Listings in Appendix \ref{appsec:zebra})}\label{sec:zebra}
%%%%%%%%%%%%%%%%%%%%%%%%%%%%%%%%%%%%%%%%%%%%%%

The Zebra Puzzle is a well known logic puzzle.

\begin{quotation}
There are five houses in a row. Each has a unique color and is occupied by a family of unique nationality. Each family has a unique favorite smoke, a unique pet, and a unique favorite drink. Fourteen clues (Listing \ref{lis:zebra_prolog}) provide additional constraints. \textit{Who has a zebra and who drinks water?}
\end{quotation}

%%%%%%%%%%%%%%%%%%%%%%%%%%%%%%%%%%%%%%%%%%%%%%
\subsection{The clues and a Prolog solution (Listings in Appendix \ref{appsubsec:clues})} \label{subsec:clues}
%%%%%%%%%%%%%%%%%%%%%%%%%%%%%%%%%%%%%%%%%%%%%%

One can easily write Prolog programs to solve this and similar puzzles.
\begin{itemize}
\item Represent a house as a Prolog \textit{house} term with the parameters corresponding to the indicated properties:
\begin{python}
   house(<nationality>, <cigarette brand>, <pet>, <drink>, <house color>)
\end{python}
\item Define the world as a list of five \textit{house} terms, with all fields initially uninstantiated.

\item Write the clues (Listing \ref{lis:zebra_prolog}) as more-or-less direct translations of the English.
\end{itemize}

% \begin{minipage}{\linewidth}
% \begin{python}
% zebra_problem(Houses) :-
%     Houses = [house(_, _, _, _, _), house(_, _, _, _, _), house(_, _, _, _, _), 
%               house(_, _, _, _, _), house(_, _, _, _, _)], 

%     % 1. The English live in the red house.
%     member(house(english, _, _, _, red), Houses), 

%     % 2. The Spanish have a dog.
%     member(house(spanish, _, dog, _, _), Houses), 

%     % 3. They drink coffee in the green house.
%     member(house(_, _, _, coffee, green), Houses), 

%     % 4. The Ukrainians drink tea.
%     member(house(ukranians, _, _, tea, _), Houses), 

%     % 5. The green house is immediately to the right of the white house.
%     nextto(house(_, _, _, _, white), house(_, _, _, _, green), Houses), 

%     % 6. The Old Gold smokers have snails.
%     member(house(_, old_gold, snails, _, _), Houses), 

%     % 7. They smoke Kool in the yellow house.
%     member(house(_, kool, _, _, yellow), Houses), 

%     % 8. They drink milk in the middle house.
%     Houses = [_, _, house(_, _, _, milk, _), _, _], 

%     % 9. The Norwegians live in the first house on the left.
%     Houses = [house(norwegians, _, _, _, _) | _], 

%     % 10. The Chesterfield smokers live next to the fox.
%     next_to(house(_, chesterfield, _, _, _), house(_, _, fox, _, _), Houses), 

%     % 11. They smoke Kool in the house next to the horse.
%     next_to(house(_, kool, _, _, _), house(_, _, horse, _, _), Houses), 

%     % 12. The Lucky smokers drink juice.
%     member(house(_, lucky, _, juice, _), Houses), 

%     % 13. The Japanese smoke Parliament.
%     member(house(japanese, parliament, _, _, _), Houses), 

%     % 14. The Norwegians live next to the blue house.
%     next_to(house(norwegians, _, _, _, _), house(_, _, _, _, blue), Houses), 
% \end{python}
% \begin{lstlisting} [caption={Zebra puzzle in Prolog},  label={lis:zebra_prolog}]
% \end{lstlisting}
% \end{minipage}

\largev
After the following adjustments, we can run this program online using SWI-Prolog. 
\begin{itemize}
    \item SWI-Prolog includes \textit{member} and \textit{nextto} predicates. SWI-Prolog's \textit{nextto} means in the order given, as in clue 5.

    \item SWI-Prolog does not include a predicate for \textit{next to} in the sense of clues 10, 11, and 14 in which the order is unspecified. But we can write our own, say, \textit{next\_to}.
    
\begin{python}
next_to(A, B, List) :- nextto(A, B, List).
next_to(A, B, List) :- nextto(B, A, List).
\end{python}

    \item Since none of the clues mentions either a zebra or water, we add the following.

\begin{minipage}{\linewidth}
\begin{python}

    % 15. (implicit). 
    member(house(_, _, zebra, _, _), Houses), 
    member(house(_, _, _, water, _), Houses).
\end{python}
\end{minipage}

\end{itemize}

When this program is run, we get an almost instantaneous answer---shown manually formatted in Listing \ref{lis:zebra_solution}. We can conclude that 
\begin{quote} 
\begin{quote} 
\textit{The Japanese have a zebra, and the Norwegians drink water}.
\end{quote} 
\end{quote} 

% \begin{minipage}{\linewidth}
% \begin{python}
% ?- zebra_problem(Houses).
% [    
%     house(norwegians, kool, fox, water, yellow), 
%     house(ukranians, chesterfield, horse, tea, blue), 
%     house(english, old_gold, snails, milk, red), 
%     house(spanish, lucky, dog, juice, white), 
%     house(japanese, parliament, zebra, coffee, green)     
% ]
% \end{python}
% \begin{lstlisting} [caption={Zebra puzzle in Prolog},  label={lis:zebra_solution}]
% \end{lstlisting}
% \end{minipage}

\smallv

% We can conclude that 
% \begin{quote} 
% \begin{quote} 
% \textit{The Japanese have a zebra, and the Norwegians drink water}.
% \end{quote} 
% \end{quote} 

%%%%%%%%%%%%%%%%%%%%%%%%%%%%%%%%%%%%%%%%%%%%%%
\subsection{A Pylog solution (Listings in Appendix \ref{appsubsec:pylog_solution})} \label{subsec:pylog_solution}
%%%%%%%%%%%%%%%%%%%%%%%%%%%%%%%%%%%%%%%%%%%%%%

To write and run the Zebra problem in Pylog we built the following framework.
\begin{itemize}
    \item \sloppy We created a \textit{House} class as a subclass of \textit{Structure}. Users may select a house property as a pseudo-functor for displaying houses. We selected \textit{nationality}.

    \item Each clue is expressed as a Pylog function. (See Listing \ref{lis:clues_as_pylog_functions}.) 

        % \begin{minipage}{\linewidth}
        % \begin{python}
        %   def clue_1(self, Houses: SuperSequence):
        %     """ 1. The English live in the red house.  """
        %     yield from member(House(nationality='English', color='red'), Houses)

        %   ...
  
        %   def clue_8(self, Houses: SuperSequence):
        %     """ 8. They drink milk in the middle house. """
        %     yield from unify(House(drink='milk'), Houses[2])

        %   ...
        % \end{python}
        % \begin{lstlisting} [caption={Clues as Pylog functions},  label={lis:clues_as_pylog_functions}]
        % \end{lstlisting}
        % \end{minipage}

    \item The \textit{Houses} list may be any form of \textit{SuperSequence}.
    
    \item We added some simple constraint checking.
\end{itemize}
When run, the answer is the same as in the Prolog version. (See listing \ref{lis:pylog_solution}.)

% \begin{minipage}{\linewidth}
% \begin{python}
% After 1392 rule applications, 
% 	1. Norwegians(Kool, fox, water, yellow)
% 	2. Ukrainians(Chesterfield, horse, tea, blue)
% 	3. English(Old Gold, snails, milk, red)
% 	4. Spanish(Lucky, dog, juice, white)
% 	5. Japanese(Parliament, zebra, coffee, green)
% The Japanese own a zebra, and the Norwegians drink water.
% \end{python}
% \begin{lstlisting} [caption={Pylog solution},  label={lis:pylog_solution}]
% \end{lstlisting}
% \end{minipage}

\smallv
Let's compare the underlying Prolog and Pylog mechanisms. 
\smallv

\textbf{Prolog}. It's trivial to write a Prolog interpreter in Prolog. See Listing \ref{lis:prologInterpreter} \cite{Bartak1998}.

\smallv
\textbf{Pylog}. We developed \textit{three} Pylog approaches to rule interpretation. 
\begin{enumerate}

\item \textit{forall}. Use the \textit{forall} construct as in Listing \ref{lis:zebra_forall}.

% \begin{minipage}{\linewidth}
% \begin{python}
% def zebra_problem(Houses) :-
%     for _ in forall{[
%         # 1. The English live in the red house.
%         lambda: member(house(english, _, _, _, red), Houses), 
%         # 2. The Spanish have a dog.
%         lambda: member(house(spanish, _, dog, _, _), Houses), 
%         # ...
%         ]}
% \end{python}
% \begin{lstlisting} [caption={Pylog solution},  label={lis:zebra_forall}]
% \end{lstlisting}
% \end{minipage}

\textit{forall} succeeds if and only if all members of the list it is passed succeed. Each list element is protected within a \textbf{lambda} construct to prevent evaluation.

\item \textit{run\_all\_rules}. We developed a Python function that accepts a list, e.g., of houses, reflecting the state of the world, along with a list of functions. It succeeds if and only if the functions all succeed. Listing \ref{lis:zebra_ run_all_clues} is a somewhat simplified version.

% \begin{minipage}{\linewidth}
% \begin{python}
% def run_all_clues(World_List: List[Term], clues: List[Callable]):
%     if not clues:
%       # Ran all the clues. Succeed.
%       yield
%     else:
%       # Run the current clue and then the rest of the clues.
%       for _ in clues[0](World_List):
%         yield from run_all_clues(World_List, clues[1:])
% \end{python}
% \begin{lstlisting} [caption={Pylog solution},  label={lis:zebra_ run_all_clues}]
% \end{lstlisting}
% \end{minipage}

\item \textit{Embed rule chaining in the rules.} For example, see Listing \ref{lis:zebra_rule_chaining}.

% \begin{minipage}{\linewidth}
% \begin{python}
%   def clue_1(Houses: SuperSequence):
%     """ 1. The English live in the red house.  """
%     for _ in member(House(nationality='English', color='red'), Houses):
%       yield from clue_2(Houses)

%   def clue_2(Houses: SuperSequence):
%     """ 2. The Spanish have a dog. """
%     for _ in member(House(nationality='Spanish', pet='dog'), Houses):
%       yield from clue_3(Houses)
%   ...
% \end{python}
% \begin{lstlisting} [caption={Pylog solution},  label={lis:zebra_rule_chaining}]
% \end{lstlisting}
% \end{minipage}

% In this organization, unification propagates forward and backward, and backtracking occurs naturally.

Call \textit{clue\_1} with a list of uninstantiated houses, and the problem runs itself.
\end{enumerate}
\smallv 

The three approaches produce the same solution.


\section{Conclusion}\label{sec:conclusion}
Embedding rule chaining in the clues as in the previous section suggests a general template.

\begin{minipage}{\linewidth}
\begin{python}
   def some_clause(...):
     for _ in <generate options>:
       <local conditions>
       yield from next_clause(...)
\end{python}
\end{minipage}

More generally, Pylog offers a way to integrate logic programming features into a Python environment.

\begin{itemize}
  \item The magic of unification requires little more than linked chains.
  \item Prolog's control structures, including "backtracking," can be implemented as nested \textbftt{for}-loops (for both choicepoints and scope setting), with \textbftt{yield} and \textbftt{yield from} gluing the pieces together.
\end{itemize}

\appendix 
\section{The Trace decorator}

The \textit{Trace} decorator is defined as a class rather than a function. 

@Trace logs parameter values for both regular functions and generators. 

@Trace does not handle keyword parameters.

\begin{minipage}{\linewidth} \largev \hrulefill
\begin{python}[numbers=left]
from inspect import isgeneratorfunction, signature

class Trace:

    def __init__(self, f):
        self.param_names = [param.name for param in signature(f).parameters.values()]
        self.f = f
        self.depth = 0

    def __call__(self, *args):
        print(self.trace_line(args))
        self.depth += 1
        if isgeneratorfunction(self.f):
            return self.yield_from(*args)
        else:
            f_return = self.f(*args)
            self.depth -= 1
            return f_return

    def yield_from(self, *args):
        yield from self.f(*args)
        self.depth -= 1

    @staticmethod
    def to_str(xs):
        xs_string = f'[{", ".join(Trace.to_str(x) for x in xs)}]' if isinstance(xs, list) else str(xs)
        return xs_string

    def trace_line(self, args):
        # The quoted string on the next line is two spaces.
        prefix = "  " * self.depth
        params = ", ".join([f'{param_name}: {Trace.to_str(arg)}'
                            for (param_name, arg) in zip(self.param_names, args)])
        # Special case for the transversal functions
        termination = ' <=' if not args[0] else ''
        return prefix + params + termination

\end{python}

\begin{lstlisting} [caption={The Trace decorator},  label={lis:Trace}]
\end{lstlisting}
\end{minipage}

\end{document}

